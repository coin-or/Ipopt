%% Copyright (C) 2005, 2006 Carnegie Mellon University and others.
%%
%% The first version of this file was contributed to the Ipopt project
%% on Aug 1, 2005, by Yoshiaki Kawajiri
%%                    Department of Chemical Engineering
%%                    Carnegie Mellon University
%%                    Pittsburgh, PA 15213
%%
%% Since then, the content of this file has been updated significantly by
%%     Carl Laird and Andreas Waechter        IBM
%%
%%
%% $Id$
%%
\documentclass[10pt]{article}
\setlength{\textwidth}{6.3in}       % Text width
\setlength{\textheight}{9.4in}      % Text height
\setlength{\oddsidemargin}{0.1in}     % Left margin for even-numbered pages
\setlength{\evensidemargin}{0.1in}    % Left margin for odd-numbered pages
\setlength{\topmargin}{-0.5in}         % Top margin
\renewcommand{\baselinestretch}{1.1}
\usepackage{amsfonts}
\usepackage{amsmath}

%\usepackage{showlabels}

\newcommand{\RR}{{\mathbb{R}}}
\newcommand{\Ipopt}{{\sc Ipopt}}


\begin{document}
\title{Introduction to \Ipopt:\\
A tutorial for downloading, installing, and using \Ipopt.}

\author{Revision number of this document: $Revision$}

%\date{\today}
\maketitle

\begin{abstract}
  This document is a guide to using \Ipopt\ 3.2.1 (the C++ version
  of \Ipopt).  It includes instructions on how to obtain and compile
  \Ipopt, a description of the interface, user options, etc., as
  well as a tutorial on how to solve a nonlinear optimization problem
  with \Ipopt.

  The initial version of this document was created by
  Yoshiaki Kawajir\footnote{Department of Chemical Engineering,
    Carnegie Mellon University, Pittsburgh PA} as a course project for
  \textit{47852 Open Source Software for Optimization}, taught by
  Prof. Fran\c cois Margot at Tepper School of Business, Carnegie
  Mellon University.  The current version is maintained by Carl
  Laird\footnote{Department of Chemical Engineering, Carnegie Mellon
    University, Pittsburgh PA} and Andreas
  W\"achter\footnote{Department of Mathematical Sciences, IBM T.J.\
    Watson Research Center, Yorktown Heights, NY}.
\end{abstract}

\tableofcontents

\vspace{\baselineskip}
\begin{small}
\noindent
The following names used in this document are trademarks or registered
trademarks: AMPL, IBM, Intel, Matlab, Microsoft, MKL, Visual Studio C++,
Visual Studio C++ .NET
\end{small}

\section{Introduction}
\Ipopt\ (\underline{I}nterior \underline{P}oint \underline{Opt}imizer,
pronounced ``I--P--Opt'') is an open source software package for
large-scale nonlinear optimization. It can be used to solve general
nonlinear programming problems of the form
%\begin{subequations}\label{NLP}
\begin{eqnarray}
\min_{x\in\RR^n} &&f(x) \label{eq:obj} \\
\mbox{s.t.} \;  &&g^L \leq g(x) \leq g^U \\
                &&x^L \leq x \leq x^U, \label{eq:bounds}
\end{eqnarray}
%\end{subequations}
where $x \in \RR^n$ are the optimization variables (possibly with
lower and upper bounds, $x^L\in(\RR\cup\{-\infty\})^n$ and
$x^U\in(\RR\cup\{+\infty\})^n$), $f:\RR^n\longrightarrow\RR$ is the
objective function, and $g:\RR^n\longrightarrow \RR^m$ are the general
nonlinear constraints.  The functions $f(x)$ and $g(x)$ can be linear
or nonlinear and convex or non-convex (but should be twice
continuously differentiable). The constraints, $g(x)$, have lower and
upper bounds, $g^L\in(\RR\cup\{-\infty\})^n$ and
$g^U\in(\RR\cup\{+\infty\})^m$. Note that equality constraints of the
form $g_i(x)=\bar g_i$ can be specified by setting
$g^L_{i}=g^U_{i}=\bar g_i$.

\subsection{Mathematical Background}
\Ipopt\ implements an interior point line search filter method that
aims to find a local solution of (\ref{eq:obj})-(\ref{eq:bounds}).  The
mathematical details of the algorithm can be found in several
publications
\cite{NocWaeWal:adaptive,WaechterPhD,WaecBieg06:mp,WaeBie05:filterglobal,WaeBie05:filterlocal}.

\subsection{Availability}
The \Ipopt\ package is available from COIN-OR
(\texttt{www.coin-or.org}) under the CPL (Common Public License)
open-source license and includes the source code for \Ipopt.  This
means, it is available free of charge, also for commercial purposes.
However, if you give away software including \Ipopt\ code (in source
code or binary form) and you made changes to the \Ipopt\ source code,
you are required to make those changes public and to clearly indicate
which modifications you made.  After all, the goal of open source
software is the continuous development and improvement of software.
For details, please refer to the Common Public License.

Also, if you are using \Ipopt\ to obtain results for a publication, we
politely ask you to point out in your paper that you used \Ipopt, and
to cite the publication \cite{WaecBieg06:mp}.  Writing high-quality
numerical software takes a lot of time and effort, and does usually
not translate into a large number of publications, therefore we
believe this request is only fair :).  We also have space at the
\Ipopt\ project home page where we list publications, projects, etc.,
in which \Ipopt\ has been used.  We would be very happy to hear about
your experiences

\subsection{Prerequisites}
In order to build \Ipopt, some third party components are required:
\begin{itemize}
\item BLAS (Basic Linear Algebra Subroutines).  Many vendors of
  compilers and operating systems provide precompiled and optimized
  libraries for these dense linear algebra subroutines.  But you can
  also get the source code from {\tt www.netlib.org} and have the
  \Ipopt\ distribution compile it automatically.
\item LAPACK (Linear Algebra PACKage).  Also for LAPACK, some vendors
  offer precompiled and optimized libraries.  But like with BLAS, you
  can get the source code from {\tt www.netlib.org} and have the
  \Ipopt\ distribution compile it automatically.

  Note that currently LAPACK is only required if you intend to use the
  quasi-Newton options in \Ipopt.  You can compile the code without
  LAPACK, but an error message will then occur if you try to run the
  code with an option that requires LAPACK.  Currently, the LAPACK
  routines that are used by \Ipopt\ are only {\tt DPOTRF}, {\tt
    DPOTRS}, and {\tt DSYEV}.
\item A sparse symmetric indefinite linear solver. The \Ipopt\ needs
  to obtain the solution of sparse, symmetric, indefinite linear
  systems, and for this it relies on third-party code.  

  Currently, the following linear solvers can be used:
  \begin{itemize}
  \item MA27 from the Harwell Subroutine Library\\ (see {\tt
      http://www.cse.clrc.ac.uk/nag/hsl/}).
  \item MA57 from the Harwell Subroutine Library\\ (see {\tt
      http://www.cse.clrc.ac.uk/nag/hsl/}).
  \item The Parallel Sparse Direct Solver (PARDISO)\\ (see {\tt
      http://www.computational.unibas.ch/cs/scicomp/software/pardiso/}).\\
    Note: The Pardiso version in Intel's MKL library does not yet
    support the features necessary for \Ipopt.
  \item The Watson Sparse Matrix Package (WSMP)\\ (see {\tt
       http://www-users.cs.umn.edu/\verb|~|agupta/wsmp.html})
  \end{itemize}
  You need to include at least one of the linear solvers above in
  order to run \Ipopt, and if you want to be able to switch easily
  between different options, you can compile \Ipopt\ with all of them.
  Currently, there is development by contributors on integrating also
  MUMPS and TAUCS, but this work has not yet been completed.

  Interfaces to other linear solvers might be added in the future; if
  you are interested in contributing such an interface please contact
  us!  Note that \Ipopt\ requires that the linear solver is able to
  provide the inertia (number of positive and negative eigenvalues) of
  the symmetric matrix that is factorized.

\item Furthermore, \Ipopt\ can also use the Harwell Subroutine MC19
  for scaling of the linear systems before they are passed to the
  linear solver.  This may be particularly useful if \Ipopt\ is used
  with MA27 or MA57.  However, it is not required to have MC19 to
  compile \Ipopt; if this routine is missing, the scaling is never
  performed\footnote{There are more recent scaling routines in the
    HSL, but they have not (yet) been integrated.  Contributions are
    welcome!}.
\item ASL (AMPL Solver Library).  The source code is available at {\tt
    www.netlib.org}, and the \Ipopt\ makefiles will automatically
  compile it for you if you put the source code into a designated
  space.  NOTE: This is only required if you want to use \Ipopt\ from
  AMPL and want to compile the \Ipopt\ AMPL solver executable.
\end{itemize}
For more information on third-party components and how to obtain them,
see Section~\ref{ExternalCode}.

Since the \Ipopt\ code is written in C++, you will need a C++ compiler
to build the \Ipopt\ library.  We tried very hard to write the code as
platform and compiler independent as possible.

In addition, the configuration script currently also searches for a
Fortran, since some of the dependencies above are written in Fortran.
If all third party dependencies are available as self-contained
libraries, those compilers are in principle not necessary.  Also, it
is possible to use the Fortran-to-C compiler {\tt f2c} from {\tt
  www.netlib.org} to convert Fortran code to C, and compile the
resulting C files with a C compiler and create a library containing
the required third party dependencies.  But so far we have not tested
this ourselves, and currently the configuration script for \Ipopt\
looks for a Fortran compiler.

\subsection{How to use \Ipopt}
If desired, the \Ipopt\ distribution generates an executable for the
modeling environment AMPL. As well, you can link your problem
statement with \Ipopt\ using interfaces for C++, C, or Fortran.
\Ipopt\ can be used with most Linux/Unix environments, and on Windows
using Visual Studio .NET or Cygwin.  Below in
Section~\ref{sec:tutorial-example} this document demonstrates how to
solve problems using \Ipopt. This includes installation and
compilation of \Ipopt\ for use with AMPL as well as linking with your
own code.

Finally, the \Ipopt\ distribution includes an interface for {\tt
  CUTEr}\footnote{see {\tt http://cuter.rl.ac.uk/cuter-www/}}, if you
want to use \Ipopt\ to solve problems modeled in SIF.

The old (Fortran 2.x) version of \Ipopt\ has been interfaced with
Matlab, the NLPAPI on COIN, and is also available on NEOS.  The new
version might be available through similar means in the future.
Please check the \Ipopt\ homepage for updates.

\subsection{More Information and Contributions}
More and up-to-date information can be found at the \Ipopt\ homepage,

\begin{center}
\texttt{http://projects.coin-or.org/Ipopt}.
\end{center}

Here, you can find FAQs, some (hopefully useful) hints, a bug report
system etc.  The website is managed with Wiki, which means that every
user can edit the webpages from the regular web browser.  {\bf In
  particular, we encourage \Ipopt\ users to share their experiences
  and usage hints on the ``Success Stories'' and ``Hints and Tricks''
  pages, or to list the publications discussing applications of
  \Ipopt\ in the ``Papers related to Ipopt'' page}\footnote{Since we
  had some malicious hacker attacks destroying the content of the web
  pages in the past, you are now required to enter a user name and
  password; simply follow the instructions on top of the main project
  page.}.  In particular, if you have trouble getting \Ipopt\ work
well for your optimization problem, you might find some ideas here.
Also, if you had some difficulties to solve a problem and found a way
around it (e.g., by reformulating your problem or by using certain
\Ipopt\ options), it would be very nice if you help other users by
sharing your experience at the ``Hints and Tricks'' page.

\Ipopt\ is an open source project, and we encourage people to
contribute code (such as interfaces to appropriate linear solvers,
modeling environments, or even algorithmic features).  If you are
interested in contributing code, please have a look at the COIN
contributions webpage\footnote{see \tt
  http://www.coin-or.org/contributions.html}, and contact the \Ipopt\
project leader.

There is also a mailing list for \Ipopt, available from the webpage
\begin{center}
\texttt{http://list.coin-or.org/mailman/listinfo/coin-ipopt},
\end{center}
where you can subscribe to get notified of updates, to ask general
questions regarding installation and usage, or to share your
experience with \Ipopt. (You might want to look at the archives before
posting a question.)

We try to answer questions posted to the mailing list in a reasonable
manner.  Please understand that we cannot answer all questions in
detail, and because of time constraints, we may not be able to help
you model and debug your particular optimization problem.  However, if
you have a challenging optimization problem and are interested in
consulting services by IBM Research, please contact the \Ipopt\
project leader, Andreas W\"achter.

\subsection{History of \Ipopt}
The original \Ipopt\ (Fortran version) was a product of the dissertation
research of Andreas W\"achter \cite{WaechterPhD}, under Lorenz
T. Biegler at the Chemical Engineering Department at Carnegie Mellon
University. The code was made open source and distributed by the
COIN-OR initiative, which is now a non-profit corporation.  \Ipopt\ has
been actively developed under COIN-OR since 2002.

To continue natural extension of the code and allow easy addition of
new features, IBM Research decided to invest in an open source
re-write of \Ipopt\ in C++.  The new C++ version of the \Ipopt\
optimization code (\Ipopt\ 3.0.0 and beyond) is currently developed at
IBM Research and remains part of the COIN-OR initiative. Future
development on the Fortran version will cease with the exception of
occasional bug fix releases.

\section{Installing \Ipopt}\label{Installing}

The following sections describe the installation procedures on
UNIX/Linux systems.  For installation instructions on Windows
see Section~\ref{WindowsInstall}.

\subsection{Getting the \Ipopt\ Code}
\Ipopt\ is available from the COIN-OR subversion repository. You can
either download the code using \texttt{svn} (the
\textit{subversion}\footnote{see
  \texttt{http://subversion.tigris.org/}} client similar to CVS) or
simply retrieve a tarball (compressed archive file).  While the
tarball is an easy method to retrieve the code, using the
\textit{subversion} system allows users the benefits of the version
control system, including easy updates and revision control.

\subsubsection{Getting the \Ipopt\ code via subversion}

Of course, the \textit{subversion} client must be installed on your
system if you want to obtain the code this way (the executable is
called \texttt{svn}); it is already installed by default for many
recent Linux distributions.  Information about \textit{subversion} and
how to download it can be found at
\texttt{http://subversion.tigris.org/}.\\

To obtain the \Ipopt\ source code via subversion, change into the
directory in which you want to create a subdirectory {\tt Ipopt} with
the \Ipopt\ source code.  Then follow the steps below:
\begin{enumerate}
\item{Download the code from the repository}\\
{\tt \$ svn co https://projects.coin-or.org/svn/Ipopt/trunk CoinIpopt} \\
Note: The {\tt \$} indicates the command line
prompt, do not type {\tt \$}, only the text following it.
\item Change into the root directory of the \Ipopt\ distribution\\
{\tt \$ cd CoinIpopt}
\end{enumerate}

In the following, ``\texttt{\$IPOPTDIR}'' will refer to the directory in
which you are right now (output of \texttt{pwd}).

\subsubsection{Getting the \Ipopt\ code as a tarball}

To use the tarball, follow the steps below:
\begin{enumerate}
\item Download the latest tarball from
  \texttt{http://www.coin-or.org/Tarballs/Ipopt}.  The file you should look
  for has the form {\tt Ipopt\_{\em date}.tgz}.  It contains the daily
  snapshot of the official ``trunk'' release of \Ipopt.  Note that the
  file {\tt Ipopt\_2.2.1\_Fortran.tgz} contains the latest Fortran version of
  Ipopt, and is not described in this document.
\item Issue the following commands to unpack the archive file: \\
\texttt{\$ gunzip Ipopt\_{\em date}.tgz} \\
\texttt{\$ tar xvf Ipopt\_{\em date}.tar} \\
Note: The {\tt \$} indicates the command line
prompt, do not type {\tt \$}, only the text following it.
\item Rename the directory you just extracted:\\
\texttt{\$ mv Ipopt\_{\em date} CoinIpopt}
\item Change into the root directory of the \Ipopt\ distribution\\
{\tt \$ cd CoinIpopt}
\end{enumerate}

In the following, ``\texttt{\$IPOPTDIR}'' will refer to the directory in
which you are right now (output of \texttt{pwd}).

\subsection{Download External Code}\label{ExternalCode}
\Ipopt\ uses a few external packages that are not included in the
\Ipopt\ source code distribution, namely ASL (the AMPL Solver
Library), BLAS, LAPACK.  It also requires at least one linear solver
for symmetric indefinite matrices.

Since this third party software released under different licenses than
\Ipopt, we cannot distribute that code together with the \Ipopt\
packages and have to ask you to go through the hassle of obtaining it
yourself (even though we tried to make it as easy for you as we
could).  Keep in mind that it is still your responsibility to ensure
that your downloading and usage if the third party components conforms
with their licenses.

Note that you only need to obtain the ASL if you intend to use \Ipopt\
from AMPL.  It is not required if you want to specify your
optimization problem in a programming language (C++, C, or Fortran).
Also, currently, LAPACK is only required if you intend to use the
quasi-Newton options implemented in \Ipopt.

\subsubsection{Download BLAS, LAPACK and ASL}
If you have the download utility \texttt{wget} installed on your
system, retrieving BLAS, LAPACK, and ASL is straightforward using
scripts included with the ipopt distribution. These scripts download
the required files from the Netlib Repository
(\texttt{www.netlib.org}).\\

\noindent
{\tt \$ cd \$IPOPTDIR/ThirdParty/Blas}\\
{\tt \$ ./get.Blas}\\
{\tt \$ cd ../Lapack}\\
{\tt \$ ./get.Lapack}\\
{\tt \$ cd ../ASL}\\
{\tt \$ ./get.ASL}\\

\noindent
If you do not have \texttt{wget} installed on your system, please read
the \texttt{INSTALL.*} files in the
\texttt{\$IPOPTDIR/ThirdParty/Blas},
\texttt{\$IPOPTDIR/ThirdParty/Lapack} and
\texttt{\$IPOPTDIR/ThirdParty/ASL} directories for alternative
instructions.

\subsubsection{Download HSL Subroutines}
\Ipopt\ requires a sparse symmetric linear solver.  There are
different possibilities.  In this section we describe how to obtain
the source code for MA27 (and MC19) from the Harwell Subroutine
Library (HSL).  Those routines are freely available for
non-commercial, academic use, but it is your responsibility to
investigate the licensing of all third party code.

The use of alternative linear solvers is described in
Sections~\ref{sec:Pardiso}-\ref{sec:WSMP}.  You do not necessarily
have to use MA27 as described in this section, but at least one linear
solver is required for \Ipopt\ to function.

\begin{enumerate}
\item Go to {\tt http://hsl.rl.ac.uk/archive/hslarchive.html}
\item Follow the instruction on the website, read the license, and
  submit the registration form.
\item Go to \textit{HSL Archive Programs}, and find the package list.
\item In your browser window, click on \textit{MA27}.
\item Make sure that \textit{Double precision:} is checked. 
  Click \textit{Download package (comments removed)}
\item Save the file as {\tt ma27ad.f} in {\tt \$IPOPTDIR/ThirdParty/HSL/}\\
  Note: Some browsers append a file extension ({\tt .txt}) when you save
  the file, in which case you have to rename it.
\item Go back to the package list using the back button of your browser.
\item In your browser window, click on \textit{MC19}.
\item Make sure \textit{Double precision:} is checked. Click 
  \textit{Download package (comments removed)}
\item Save the file as {\tt mc19ad.f} in {\tt
    \$IPOPTDIR/ThirdParty/HSL/}\\
  Note: Some browsers append a file extension ({\tt .txt}) when you save
  the file, so you may have to rename it.
\end{enumerate}

Note: Whereas it is essential to have at least one linear solver, the
package MC19 could be omitted (with the consequence that you cannot
use this method for scaling the linear systems arising inside the
\Ipopt\ algorithm).  By default, MC19 is only used to scale the linear
system when using one of the Harwell solvers, but it can also be
switched on for other linear solvers (which usually have internal
scaling mechanisms).

Note: If you have the source code for the linear solver MA57
(successor of MA27) in a file called ma57ad.f (including all
dependencies), you can simply put it into the {\tt
  \$IPOPTDIR/ThirdParty/HSL/} directory.  The \Ipopt\ configuration script
will then find this file and compile it into the \Ipopt\ library (just
as is would compile MA27).

Yet another note: If you have a precompiled library containing the
Harwell codes, you can specify the location of this library with the
\verb|--with-hsl| flag for the {\tt configure} script described in
Section~\ref{sec.comp_and_inst}.

\subsubsection{Obtaining the Linear Solver Pardiso}\label{sec:Pardiso}

If you would like to compile \Ipopt\ with the Parallel Sparse Direct
Linear Solver (Pardiso), you need to obtain the Pardiso library for
your operating system.  Information about Pardiso can be found at

\texttt{http://www.computational.unibas.ch/cs/scicomp/software/pardiso}

You can obtain a free download for Pardiso if you want to use it for
``non-commercial and non-profit internal research purposes'' and are
an ``academic, non-profit, or government agency'' (taken from the
license agreement).  Instructions for this are on the above mentioned
website; make sure you read the license agreement before filling out
the download form.

Note: Pardiso is included in Intel's MKL library.  However, that
version does not yet include the changes done by the Pardiso
developers to make the linear solver work smoothly with \Ipopt.

Please consult Appendix~\ref{ExpertInstall} to find out how to
configure your \Ipopt\ installation to work with Pardiso.

\subsubsection{Obtaining the Linear Solver WSMP}\label{sec:WSMP}

If you would like to compile \Ipopt\ with the Watson Sparse Matrix
Package (WSMP), you need to obtain the WSMP library for your operating
system.  Information about WSMP can be found at

\texttt{http://www.alphaworks.ibm.com/tech/wsmp}

At this website you can obtain a ``complimentary 90-day evaluation
license'' and download the library for several operating systems; make
sure you read the license agreement before downloading the code.  Once
you obtained the code and the license, please check if the version
number of the library matches the one on the WSMP website at

{\tt http://www-users.cs.umn.edu/\verb|~|agupta/wsmp}

If a newer version is announced on that website, you can (and
probably should) request the current version by sending a message to
\verb|wsmp@watson.ibm.com|.  Please include the operating system and
other details to describe which particular version of WSMP you need.

You can use the bugfix releases with the license you obtained from
alphaWorks.

Note: Only the interface to the shared-memory version of WSMP is
currently supported.

Please consult Appendix~\ref{ExpertInstall} to find out how to
configure your \Ipopt\ installation to work with WSMP.

\subsection{Compiling and Installing \Ipopt} \label{sec.comp_and_inst}

\Ipopt\ can be easily compiled and installed with the usual {\tt
  configure}, {\tt make}, {\tt make install} commands.  We follow the
precedure that is used for most of the COIN-OR projects, based on the
GNU autotools.  At \texttt{https://projects.coin-or.org/BuildTools}
you can find a general description of the tools.

Below are the basic steps for the \Ipopt\ compilation that should work
on most systems.  For special compilations and for some
troubleshooting see Appendix~\ref{ExpertInstall} and consult the
\Ipopt\ homepage before submitting a ticket or sending a message to
the mailing list.
\begin{enumerate}
\item Go to the main directory of \Ipopt:\\
  {\tt \$ cd \$IPOPTDIR} 
\item Run the configure script\\
  {\tt \$ ./configure}

  If the last output line of the script reads ``\texttt{configure:
    Main configuration of Ipopt successful}'' then everything worked
  fine.  Otherwise, look at the screen output, have a look at the
  \texttt{config.log} output files and/or consult
  Appendix~\ref{ExpertInstall}.

  The default configure (without any options) is sufficient for most
  users. If you want to see the configure options, consult
  Appendix~\ref{ExpertInstall}, for example, if you want to use a
  non-HSL linear solver.
\item Build the code \\
{\tt \$ make}
\item If you want, you can run a short test to verify that the
  compilation was successful.  For this, you just
  enter\\
  {\tt \$ make test}\\
  This will test if the AMPL solver executable works (if you got the
  ASL code), and if the included C++, C, and Fortran examples work.

  Note: The {\tt configure} script is not able to automatically
  determine the C++ runtime libraries for the C++ compiler.  For
  certain compilers we enabled default values for this, but those
  might not exist or be wrong for your compiler.  In that case, the C
  and Fortran example in the test will most probably fail to compile.
  If you don't want to hook up the compiled \Ipopt\ library to some
  Fortran or C code that you wrote you don't need to worry about this.
  If you do want to link the \Ipopt\ library with a C or Fortran
  compiler, you need to find out the C++ runtime libraries (e.g., by
  running the C++ compiler in verbose mode for a simple example
  program) and run {\tt configure} again, and this time specify all
  C++ runtime libraries with the {\tt CXXLIBS} variable (see also
  Appendix~\ref{ExpertInstall}).
\item Install \Ipopt \\
  {\tt \$ make install}\\
  This installs
  \begin{itemize}
  \item the \Ipopt\ AMPL solver executable (if ASL source was
    downloaded) in \texttt{\$IPOPTDIR/bin},
  \item the \Ipopt\ library (\texttt{libipopt.a}) in
    \texttt{\$IPOPTDIR/lib},
  \item text files {\tt ipopt\_addlibs\_cpp.txt} and {\tt
      ipopt\_addlibs\_f.txt} in \texttt{\$IPOPTDIR/lib} that contain a
    line each with additional linking flags that are required for
    linking code with the \Ipopt\ library, for C++ and Fortran main
    programs, respectively. (This is only for convenience if you want
    to find out what additional flags are required, for example, to
    include the Fortran runtime libraries with a C++ compiler.)
  \item the necessary header files in
    \texttt{\$IPOPTDIR/include/ipopt}.
  \end{itemize}
  You can change the default installation directory (here
  \texttt{\$IPOPTDIR}) to something else (such as \texttt{/usr/local})
  by using the \verb|--prefix| switch for \texttt{configure}.
\item Install \Ipopt\ for use with {\tt CUTEr}\\
  If you have {\tt CUTEr} already installed on your system and you
  want to use \Ipopt\ as a solver for problems modeled in {\tt SIF},
  type\\
  {\tt \$ make cuter}\\
  This assumes that you have the environment variable {\tt MYCUTER}
  defined according to the {\tt CUTEr} instructions.  After this, you
  can use the script {\tt sdipo} as the {\tt CUTEr} script to solve a
  {\tt SIF} model.
\end{enumerate}

Note: It is possible to compile the code in directories separate from
the source files.  This comes in handy when you want to compile the
code with different compilers, compiler options, or different
operating system that share a common file system.  To use this
feature, change into the directory where you want to compile the code,
and then type {\tt \$IPOPTDIR/configure} with all the options
(replacing {\tt \$IPOPTDIR} by the path to {\tt configure}).  For
this, the directories with the \Ipopt\ source must not have any
configuration and compiled code.

\subsection{Installation on Windows}\label{WindowsInstall}

There are several ways to install \Ipopt\ on Windows systems.  The first
two option, described in Sections~\ref{CygwinInstall} and
\ref{CygwinInstallNative}, is to use Cygwin (see
\texttt{www.cygwin.com}), which offers a UNIX-like environment on
Windows and in which the installation procedure described earlier in
this section can be used.  If you want to use the (free) GNU
compilers, follow the instructions in Section~\ref{CygwinInstall}.  If
you have the Microsoft C++ compiler (executable called {\tt cl.exe})
and the Intel Fortran compiler (called {\tt ifort.exe}) and want to
use those to compile \Ipopt, please see Section~\ref{CygwinInstall}.
If you use MinGW, please consider the notes in Section~\ref{MinGWInstall}.
%
The \Ipopt\ distribution also includes projects files for the
Microsoft Visual Studio (see Section~\ref{VisualStudioInstall}).

\subsubsection{Installation with Cygwin using GNU compilers}\label{CygwinInstall}

Cygwin is a Linux-like environment for Windows; if you don't know what
it is you might want to have a look at the Cygwin homepage,
\texttt{www.cygwin.com}.

It is possible to build the \Ipopt\ AMPL solver executable in Cygwin
for general use in Windows.  You can also hook up \Ipopt\ to your own
program if you compile it in the Cygwin environment\footnote{It is
  also possible to build an \Ipopt\ DLL that can be used from
  non-cygwin compilers, but this is not (yet?) supported.}.

If you want to compile \Ipopt\ under Cygwin, you first have to install
Cygwin on your Windows system.  This is pretty straight forward; you
simply download the ``setup'' program from
\texttt{www.cygwin.com} and start it.

Then you do the following steps (assuming here that you don't have any
complications with firewall settings etc - in that case you might have
to choose some connection settings differently):

\begin{enumerate}
\item Click next
\item Select ``install from the internet'' (default) and click next
\item Select a directory where Cygwin is to be installed (you can
  leave the default) and choose all other things to your liking, then
  click next
\item Select a temp dir for Cygwin setup to store some files (if you
  put it on your desktop you will later remember to delete it)
\item Select ``direct connection'' (default) and click next
\item Select some mirror site that seems close by to you and click next
\item OK, now comes the complicated part:\\
  You need to select the packages that you want to have installed.  By
  default, there are already selections, but the compilers are usually
  not pre-chosen.  You need to make sure that you select the GNU
  compilers (for Fortran, C, and C++ --- together with the MinGW
  options), the GNU Make, and Subversion.  For this, click on the "Devel"
  branch (which opens a subtree) and select:
  \begin{itemize}
  \item gcc
  \item gcc-core
  \item gcc-g77
  \item gcc-g++
  \item gcc-mingw
  \item gcc-mingw-core
  \item gcc-mingw-g77
  \item gcc-mingw-g++
  \item make
  \item subversion
  \end{itemize}

  Then, in the ``Web'' branch, please select ``wget'' (which will make
  the installation of third party dependencies for \Ipopt\ easier)

  This will automatically also select some other packages.
\item\label{it:cyg_done} Then you click on next, and Cygwin will be
  installed (follow the rest of the instructions and choose everything
  else to your liking).  At a later point you can easily add/remove
  packages with the setup program.

\item Now that you have Cygwin, you can open a Cygwin window, which is
  like a UNIX shell window.

\item\label{it:cyg_inst} Now you just follow the instructions in the
  beginning of Sections~\ref{Installing}: You download the \Ipopt\
  code into your Cygwin home directory (from the Windows explorer that
  is usually something like
  \texttt{C:$\backslash$Cygwin$\backslash$home$\backslash$your\_user\_name}).
  After that you obtain the third party code (like on Linux/UNIX),
  type

  \texttt{./configure}

  and

  \texttt{make install}

  in the correct directories, and hopefully that will work.  The
  \Ipopt\ AMPL solver executable will be in the subdirectory
  \texttt{bin} (called ``\texttt{ipopt.exe}'').  If you want to set
  the installation, type

  \texttt{make test}
\end{enumerate}

\subsubsection{Installation with Cygwin using native compilers}\label{CygwinInstallNative}

The \Ipopt\ configure script and Makefiles have been tested with the
Microsoft Visual C++ .NET 2003 Standard compiler together with the
Intel Visual Fortran Compiler 8.1.  It might also work with other
compilers.

Here are the steps that work on my system:

\begin{enumerate}
\item Follow the instructions in Section~\ref{CygwinInstall} until
  Step~\ref{it:cyg_inst} and stop after your downloaded the third
  party code.

\item Now you need to make sure that Cygwin knows about the native
  compilers.  For this you need to edit the file {\tt cygwin.bat} in
  the Cygwin base directory (usually {\tt C:$\backslash$cygwin}).
  Here you need to add the line

  \texttt{call ``C:$\backslash$Program
    Files$\backslash$Intel$\backslash$Fortran$\backslash$compiler80$\backslash$IA32$\backslash$BIN$\backslash$ifortvars.bat''}

  or whatever the location of that batch file is on your computer.

\item Run the configuration script, and tell it that you want to use
  the native compilers:

  \texttt{./configure CC=cl CXX=cl F77=ifort}

  Make sure the last message is

  \texttt{Main Ipopt configuration successful}

\item\label{it:ASLcompile} If want to compile the AMPL solver
  executable, you need to compile the ASL library from a script.  For
  this you need to change into the ASL compilation directory, execute
  the script \texttt{compile\_MS\_ASL}, and go back to the directory
  where you were:

  \texttt{cd ThirdParty/ASL}

  \texttt{./compile\_MS\_ASL}

  \texttt{cd -}

\item Now you can compile the code with

  \texttt{make},

  test the installation with

  \texttt{make test},

  and install everything with

  \texttt{make install}
\end{enumerate}

\subsubsection{Installation with MinGW}\label{MinGWInstall}

You can compile \Ipopt\ also under MinGW, which is another UNIX-like
environment for Windows.  It can be obtained from
\texttt{http://www.mingw.org/}.

A compilation with the GNU compilers works just like with any other
UNIX system, as described in Section~\ref{sec.comp_and_inst}.  If you
want to use the native compilers (e.g., {\tt cl} and {\tt ifort}), you
need to make sure they are in the path for the MSys prompt.  Also, as
for the procedure described in Section~\ref{CygwinInstallNative}, you
need to run the \texttt{compile\_MS\_ASL} script in the
\texttt{ThirdParty/ASL} immediately after you run the configuration
script. 


\subsubsection{Using Visual Studio}\label{VisualStudioInstall}

The \Ipopt\ distribution includes project files that can be used to
compile the \Ipopt\ library and a Fortran and C++ example within the
Microsoft Visual Studio.  The project files have been created with
Microsoft Visual C++ .NET 2003 Standard, and the Intel Visual Fortran
Compiler 8.1.

In order to use those project files, download the \Ipopt\ source code,
as well as the required third party code (put it into the {\tt
ThirdParty$\backslash$Blas}, {\tt ThirdParty$\backslash$Lapack}, and {\tt
ThirdParty$\backslash$HSL} directories---ASL is not required for the
Fortran and C examples). Then open the solution file:\\

\texttt{\$IPOPTDIR$\backslash$Ipopt$\backslash$Windows$\backslash$VisualStudio\_dotNET$\backslash$Ipopt$\backslash$Ipopt.sln}\\

Please also read the README file in
\texttt{\$IPOPTDIR$\backslash$Ipopt$\backslash$Windows$\backslash$VisualStudio\_dotNET}

Note: Since the project files were created only with the Standard
edition of the C++ compiler, code optimization might be disabled; for
fast performance make sure you enable code optimization.

\section{Interfacing your NLP to \Ipopt: A tutorial example.}
\label{sec:tutorial-example}

\Ipopt\ has been designed to be flexible for a wide variety of
applications, and there are a number of ways to interface with \Ipopt\
that allow specific data structures and linear solver
techniques. Nevertheless, the authors have included a standard
representation that should meet the needs of most users.

This tutorial will discuss four interfaces to \Ipopt, namely the AMPL
modeling language\cite{FouGayKer:AMPLbook} interface, and the C++, C,
and Fortran code interfaces.  AMPL is a 3rd party modeling language
tool that allows users to write their optimization problem in a syntax
that resembles the way the problem would be written mathematically.
Once the problem has been formulated in AMPL, the problem can be
easily solved using the (already compiled) \Ipopt\ AMPL solver
executable, {\tt ipopt}. Interfacing your problem by directly linking
code requires more effort to write, but can be far more efficient for
large problems.

We will illustrate how to use each of the four interfaces using an
example problem, number 71 from the Hock-Schittkowsky test suite \cite{HS},
%\begin{subequations}\label{HS71}
  \begin{eqnarray}
    \min_{x \in \Re^4} &&x_1 x_4 (x_1 + x_2 + x_3)  +  x_3 \label{eq:ex_obj} \\
    \mbox{s.t.}  &&x_1 x_2 x_3 x_4 \ge 25 \label{eq:ex_ineq} \\
    &&x_1^2 + x_2^2 + x_3^2 + x_4^2  =  40 \label{eq:ex_equ} \\
    &&1 \leq x_1, x_2, x_3, x_4 \leq 5, \label{eq:ex_bounds}
  \end{eqnarray}
%\end{subequations}
with the starting point
\begin{equation}
x_0 = (1, 5, 5, 1) \label{eq:ex_startpt}
\end{equation}
and the optimal solution
\[
x_* = (1.00000000, 4.74299963, 3.82114998, 1.37940829). \nonumber
\]

You can find further, less documented examples for using \Ipopt\ from
your own source code in the {\tt Ipopt/examples} subdirectory.

\subsection{Using \Ipopt\ through AMPL}
Using the AMPL solver executable is by far the easiest way to
solve a problem with \Ipopt. The user must simply formulate the problem
in AMPL syntax, and solve the problem through the AMPL environment.
There are drawbacks, however. AMPL is a 3rd party package and, as
such, must be appropriately licensed (a free student version for
limited problem size is available from the AMPL website,
\texttt{www.ampl.com}). Furthermore, the AMPL environment may be prohibitive
for very large problems. Nevertheless, formulating the problem in AMPL
is straightforward and even for large problems, it is often used as a
prototyping tool before using one of the code interfaces.

This tutorial is not intended as a guide to formulating models in
AMPL. If you are not already familiar with AMPL, please consult
\cite{FouGayKer:AMPLbook}.

The problem presented in equations
(\ref{eq:ex_obj})--(\ref{eq:ex_startpt}) can be solved with \Ipopt\ with
the AMPL model file given in Figure~\ref{fig:HS71}.

\begin{figure}
  \centering
\begin{footnotesize}
\begin{verbatim}
# tell ampl to use the ipopt executable as a solver
# make sure ipopt is in the path!
option solver ipopt;

# declare the variables and their bounds, 
# set notation could be used, but this is straightforward
var x1 >= 1, <= 5; 
var x2 >= 1, <= 5; 
var x3 >= 1, <= 5; 
var x4 >= 1, <= 5;

# specify the objective function
minimize obj:
                x1 * x4 * (x1 + x2 + x3) + x3;
        
# specify the constraints
s.t.
        inequality:
                x1 * x2 * x3 * x4 >= 25;
                
        equality:
                x1^2 + x2^2 + x3^2 +x4^2 = 40;

# specify the starting point            
let x1 := 1;
let x2 := 5;
let x3 := 5;
let x4 := 1;

# solve the problem
solve;

# print the solution
display x1;
display x2;
display x3;
display x4;
\end{verbatim}
\end{footnotesize}
  
  \caption{AMPL model file hs071\_ampl.mod}
  \label{fig:HS71}
\end{figure}

The line, ``{\tt option solver ipopt;}'' tells AMPL to use \Ipopt\ as
the solver. The \Ipopt\ executable (installed in
Section~\ref{sec.comp_and_inst}) must be in the {\tt PATH} for AMPL to
find it. The remaining lines specify the problem in AMPL format. The
problem can now be solved by starting AMPL and loading the mod file:
\begin{verbatim}
$ ampl
> model hs071_ampl.mod;
.
.
.
\end{verbatim}
%$
The problem will be solved using \Ipopt\ and the solution will be
displayed.

At this point, AMPL users may wish to skip the sections about
interfacing with code, but should read Section \ref{sec:options}
concerning \Ipopt\ options, and Section \ref{sec:output} which
explains the output displayed by \Ipopt.

\subsection{Interfacing with \Ipopt\ through code}
In order to solve a problem, \Ipopt\ needs more information than just
the problem definition (for example, the derivative information). If
you are using a modeling language like AMPL, the extra information is
provided by the modeling tool and the \Ipopt\ interface. When
interfacing with \Ipopt\ through your own code, however, you must
provide this additional information.

\begin{figure}
\begin{enumerate}
\item Problem dimensions \label{it.prob_dim}
  \begin{itemize}
  \item number of variables
  \item number of constraints
  \end{itemize}
\item Problem bounds
  \begin{itemize}
  \item variable bounds
  \item constraint bounds
  \end{itemize}
\item Initial starting point
  \begin{itemize}
  \item Initial values for the primal $x$ variables
  \item Initial values for the multipliers (only
    required for a warm start option)
  \end{itemize}
\item Problem Structure \label{it.prob_struct}
  \begin{itemize}
  \item number of nonzeros in the Jacobian of the constraints
  \item number of nonzeros in the Hessian of the Lagrangian function
  \item sparsity structure of the Jacobian of the constraints
  \item sparsity structure of the Hessian of the Lagrangian function
  \end{itemize}
\item Evaluation of Problem Functions \label{it.prob_eval} \\
  Information evaluated using a given point ($x,
  \lambda, \sigma_f$ coming from \Ipopt)
  \begin{itemize}
  \item Objective function, $f(x)$
  \item Gradient of the objective $\nabla f(x)$
  \item Constraint function values, $g(x)$
  \item Jacobian of the constraints, $\nabla g(x)^T$
  \item Hessian of the Lagrangian function, 
    $\sigma_f \nabla^2 f(x) + \sum_{i=1}^m\lambda_i\nabla^2
    g_i(x)$ \\
    (this is not required if a quasi-Newton options is chosen to
    approximate the second derivatives)
  \end{itemize}
\end{enumerate}
\caption{Information required by \Ipopt}
\label{fig.required_info}
\end{figure}
%\vspace{0.1in}
The information required by \Ipopt\ is shown in Figure
\ref{fig.required_info}. The problem dimensions and bounds are
straightforward and come solely from the problem definition. The
initial starting point is used by the algorithm when it begins
iterating to solve the problem. If \Ipopt\ has difficulty converging, or
if it converges to a locally infeasible point, adjusting the starting
point may help.  Depending on the starting point, \Ipopt\ may also
converge to different local solutions.

Providing the sparsity structure of derivative matrices is a bit more
involved. \Ipopt\ is a nonlinear programming solver that is designed
for solving large-scale, sparse problems. While \Ipopt\ can be
customized for a variety of matrix formats, the triplet format is used
for the standard interfaces in this tutorial. For an overview of the
triplet format for sparse matrices, see Appendix~\ref{app.triplet}.
Before solving the problem, \Ipopt\ needs to know the number of
nonzero elements and the sparsity structure (row and column indices of
each of the nonzero entries) of the constraint Jacobian and the
Lagrangian function Hessian. Once defined, this nonzero structure MUST
remain constant for the entire optimization procedure. This means that
the structure needs to include entries for any element that could ever
be nonzero, not only those that are nonzero at the starting point.

As \Ipopt\ iterates, it will need the values for
Item~\ref{it.prob_eval}. in Figure~\ref{fig.required_info} evaluated at
particular points. Before we can begin coding the interface, however,
we need to work out the details of these equations symbolically for
example problem (\ref{eq:ex_obj})-(\ref{eq:ex_bounds}).

The gradient of the objective $f(x)$ is given by
\[%\begin{equation}
\left[
\begin{array}{c}
x_1 x_4 + x_4 (x_1 + x_2 + x_3) \\
x_1 x_4 \\
x_1 x_4 + 1 \\
x_1 (x_1 + x_2 + x_3)
\end{array}
\right],
\]%\end{equation}
and the Jacobian of the constraints $g(x)$ is
\[%\begin{equation}
\left[
\begin{array}{cccc}
x_2 x_3 x_4     & x_1 x_3 x_4   & x_1 x_2 x_4   & x_1 x_2 x_3   \\
2 x_1           & 2 x_2         & 2 x_3         & 2 x_4
\end{array}
\right].
\]%\end{equation}

We also need to determine the Hessian of the Lagrangian\footnote{If a
  quasi-Newton option is chosen to approximate the second derivatives,
  this is not required.  However, if second derivatives can be
  computed, it is often worthwhile to let \Ipopt\ use them, since the
  algorithm is then usually more robust and converges faster.  More on
  the quasi-Newton approximation in Section~\ref{sec:quasiNewton}.}.
The Lagrangian function for the NLP
(\ref{eq:ex_obj})-(\ref{eq:ex_bounds}) is defined as $f(x) + g(x)^T
\lambda$ and the Hessian of the Lagrangian function is, technically, $
\nabla^2 f(x_k) + \sum_{i=1}^m\lambda_i\nabla^2 g_i(x_k)$.  However,
so that \Ipopt\ can ask for the Hessian of the objective or the
constraints independently if required, we introduce a factor
($\sigma_f$) in front of the objective term.
%
For \Ipopt\ then, the symbolic form of the Hessian of the
Lagrangian is
\begin{equation}\label{eq:IpoptLAG}
\sigma_f \nabla^2 f(x_k) + \sum_{i=1}^m\lambda_i\nabla^2 g_i(x_k)
\end{equation}
(with the $\sigma_f$ parameter), and for the example problem this becomes
%\begin{eqnarray}
%{\cal L}(x,\lambda) &{=}& f(x) + c(x)^T \lambda \nonumber \\
%&{=}& \left(x_1 x_4 (x_1 + x_2 + x_3)  +  x_3\right) 
%+ \left(x_1 x_2 x_3 x_4\right) \lambda_1 \nonumber \\
%&& \;\;\;\;\;+ \left(x_1^2 + x_2^2 + x_3^2 + x_4^2\right) \lambda_2 
%- \displaystyle \sum_{i \in 1..4} z^L_i + \sum_{i \in 1..4} z^U_i
%\end{eqnarray}
\[%\begin{equation}
\sigma_f \left[
\begin{array}{cccc}
2 x_4           & x_4           & x_4           & 2 x_1 + x_2 + x_3     \\
x_4             & 0             & 0             & x_1                   \\
x_4             & 0             & 0             & x_1                   \\
2 x_1+x_2+x_3   & x_1           & x_1           & 0
\end{array}
\right]
+
\lambda_1
\left[
\begin{array}{cccc}
0               & x_3 x_4       & x_2 x_4       & x_2 x_3       \\
x_3 x_4         & 0             & x_1 x_4       & x_1 x_3       \\
x_2 x_4         & x_1 x_4       & 0             & x_1 x_2       \\
x_2 x_3         & x_1 x_3       & x_1 x_2       & 0 
\end{array}
\right]
+
\lambda_2
\left[
\begin{array}{cccc}
2       & 0     & 0     & 0     \\
0       & 2     & 0     & 0     \\
0       & 0     & 2     & 0     \\
0       & 0     & 0     & 2
\end{array}
\right]
\]%\end{equation}
where the first term comes from the Hessian of the objective function,
and the second and third term from the Hessian of the constraints
(\ref{eq:ex_ineq}) and (\ref{eq:ex_equ}), respectively. Therefore, the
dual variables $\lambda_1$ and $\lambda_2$ are then the multipliers
for constraints (\ref{eq:ex_ineq}) and (\ref{eq:ex_equ}), respectively.

%C =============================================================================
%C
%C     This is an example for the usage of IPOPT.
%C     It implements problem 71 from the Hock-Schittkowsky test suite:
%C
%C     min   x1*x4*(x1 + x2 + x3)  +  x3
%C     s.t.  x1*x2*x3*x4                   >=  25
%C           x1**2 + x2**2 + x3**2 + x4**2  =  40
%C           1 <=  x1,x2,x3,x4  <= 5
%C
%C     Starting point:
%C        x = (1, 5, 5, 1)
%C
%C     Optimal solution:
%C        x = (1.00000000, 4.74299963, 3.82114998, 1.37940829)
%C
%C =============================================================================
\vspace{\baselineskip}

The remaining sections of the tutorial will lead you through
the coding required to solve example problem
(\ref{eq:ex_obj})--(\ref{eq:ex_bounds}) using, first C++, then C, and finally
Fortran. Completed versions of these examples can be found in {\tt
\$IPOPTDIR/Ipopt/examples} under {\tt hs071\_cpp}, {\tt hs071\_c}, {\tt
hs071\_f}.

As a user, you are responsible for coding two sections of the program
that solves a problem using \Ipopt: the main executable (e.g., {\tt
  main}) and the problem representation.  Typically, you will write an
executable that prepares the problem, and then passes control over to
\Ipopt\ through an {\tt Optimize} or {\tt Solve} call. In this call,
you will give \Ipopt\ everything that it requires to call back to your
code whenever it needs functions evaluated (like the objective
function, the Jacobian of the constraints, etc.).  In each of the
three sections that follow (C++, C, and Fortran), we will first
discuss how to code the problem representation, and then how to code
the executable.

\subsection{The C++ Interface}
This tutorial assumes that you are familiar with the C++ programming
language, however, we will lead you through each step of the
implementation. For the problem representation, we will create a class
that inherits off of the pure virtual base class, {\tt TNLP} ({\tt
  IpTNLP.hpp}). For the executable (the {\tt main} function) we will
make the call to \Ipopt\ through the {\tt IpoptApplication} class
({\tt IpIpoptApplication.hpp}). In addition, we will also be using the
{\tt SmartPtr} class ({\tt IpSmartPtr.hpp}) which implements a reference
counting pointer that takes care of memory management (object
deletion) for you (for details, see Appendix~\ref{app.smart_ptr}).

After ``\texttt{make install}'' (see Section~\ref{sec.comp_and_inst}),
the header files are installed in \texttt{\$IPOPTDIR/include/ipopt}
(or in \texttt{\$PREFIX/include/ipopt} if the switch
\verb|--prefix=$PREFIX| was used for {\tt configure}).

\subsubsection{Coding the Problem Representation}\label{sec.cpp_problem}
We provide the information required in Figure \ref{fig.required_info}
by coding the {\tt HS071\_NLP} class, a specific implementation of the
{\tt TNLP} base class. In the executable, we will create an instance
of the {\tt HS071\_NLP} class and give this class to \Ipopt\ so it can
evaluate the problem functions through the {\tt TNLP} interface. If
you have any difficulty as the implementation proceeds, have a look at
the completed example in the {\tt Ipopt/examples/hs071\_cpp} directory.

Start by creating a new directory under {\tt examples}, called {\tt
  MyExample} and create the files {\tt hs071\_nlp.hpp} and {\tt
  hs071\_nlp.cpp}. In {\tt hs071\_nlp.hpp}, include {\tt IpTNLP.hpp}
(the base class), tell the compiler that we are using the \Ipopt\
namespace, and create the declaration of the {\tt HS071\_NLP} class,
inheriting off of {\tt TNLP}. Have a look at the {\tt TNLP} class in
{\tt IpTNLP.hpp}; you will see eight pure virtual methods that we must
implement. Declare these methods in the header file.  Implement each
of the methods in {\tt HS071\_NLP.cpp} using the descriptions given
below. In {\tt hs071\_nlp.cpp}, first include the header file for your
class and tell the compiler that you are using the \Ipopt\ namespace.
A full version of these files can be found in the {\tt
  Ipopt/examples/hs071\_cpp} directory.

It is very easy to make mistakes in the implementation of the function
evaluation methods, in particular regarding the derivatives.  \Ipopt\
has a feature that can help you to debug the derivative code, using
finite differences, see Section~\ref{sec:deriv-checker}.

Note that the return value of any {\tt bool}-valued function should be
{\tt true}, unless an error occurred, for example, because the value of
a problem function could not be evaluated at the required point.

\paragraph{Method {\texttt{get\_nlp\_info}}} with prototype
\begin{verbatim}
virtual bool get_nlp_info(Index& n, Index& m, Index& nnz_jac_g,
                          Index& nnz_h_lag, IndexStyleEnum& index_style)
\end{verbatim}
Give \Ipopt\ the information about the size of the problem (and hence,
the size of the arrays that it needs to allocate). 
\begin{itemize}
\item {\tt n}: (out), the number of variables in the problem (dimension of $x$).
\item {\tt m}: (out), the number of constraints in the problem (dimension of $g(x)$).
\item {\tt nnz\_jac\_g}: (out), the number of nonzero entries in the Jacobian.
\item {\tt nnz\_h\_lag}: (out), the number of nonzero entries in the Hessian.
\item {\tt index\_style}: (out), the numbering style used for row/col entries in the sparse matrix
format ({\tt C\_STYLE}: 0-based, {\tt FORTRAN\_STYLE}: 1-based; see
also Appendix~\ref{app.triplet}).
\end{itemize}
\Ipopt\ uses this information when allocating the arrays that
it will later ask you to fill with values. Be careful in this method
since incorrect values will cause memory bugs which may be very
difficult to find.

Our example problem has 4 variables (n), and 2 constraints (m). The
constraint Jacobian for this small problem is actually dense and has 8
nonzeros (we still need to represent this Jacobian using the sparse
matrix triplet format). The Hessian of the Lagrangian has 10
``symmetric'' nonzeros (i.e., nonzeros in the lower left triangular
part.).  Keep in mind that the number of nonzeros is the total number
of elements that may \emph{ever} be nonzero, not just those that are
nonzero at the starting point. This information is set once for the
entire problem.

\begin{footnotesize}
\begin{verbatim}
bool HS071_NLP::get_nlp_info(Index& n, Index& m, Index& nnz_jac_g, 
                             Index& nnz_h_lag, IndexStyleEnum& index_style)
{
  // The problem described in HS071_NLP.hpp has 4 variables, x[0] through x[3]
  n = 4;

  // one equality constraint and one inequality constraint
  m = 2;

  // in this example the Jacobian is dense and contains 8 nonzeros
  nnz_jac_g = 8;

  // the Hessian is also dense and has 16 total nonzeros, but we
  // only need the lower left corner (since it is symmetric)
  nnz_h_lag = 10;

  // use the C style indexing (0-based)
  index_style = TNLP::C_STYLE;

  return true;
}
\end{verbatim}
\end{footnotesize}

\paragraph{Method {\texttt{get\_bounds\_info}}} with prototype
\begin{verbatim}
virtual bool get_bounds_info(Index n, Number* x_l, Number* x_u,
                             Index m, Number* g_l, Number* g_u)
\end{verbatim}
Give \Ipopt\ the value of the bounds on the variables and constraints.
\begin{itemize}
\item {\tt n}: (in), the number of variables in the problem (dimension of $x$). 
\item {\tt x\_l}: (out) the lower bounds $x^L$ for $x$. 
\item {\tt x\_u}: (out) the upper bounds $x^U$ for $x$.
\item {\tt m}: (in), the number of constraints in the problem (dimension of $g(x)$).
\item {\tt g\_l}: (out) the lower bounds $g^L$ for $g(x)$. 
\item {\tt g\_u}: (out) the upper bounds $g^U$ for $g(x)$.
\end{itemize}
The values of {\tt n} and {\tt m} that you specified in {\tt
  get\_nlp\_info} are passed to you for debug checking.  Setting a
lower bound to a value less than or equal to the value of the option
{\tt nlp\_lower\_bound\_inf} will cause \Ipopt\ to assume no lower
bound. Likewise, specifying the upper bound above or equal to the
value of the option {\tt nlp\_upper\_bound\_inf} will cause \Ipopt\ to
assume no upper bound.  These options, {\tt nlp\_lower\_bound\_inf}
and {\tt nlp\_upper\_bound\_inf}, are set to $-10^{19}$ and $10^{19}$,
respectively, by default, but may be modified by changing the options
(see Section \ref{sec:options}).

In our example, the first constraint has a lower bound of $25$ and no upper
bound, so we set the lower bound of constraint {\tt [0]} to $25$ and
the upper bound to some number greater than $10^{19}$. The second
constraint is an equality constraint and we set both bounds to
$40$. \Ipopt\ recognizes this as an equality constraint and does not
treat it as two inequalities.

\begin{footnotesize}
\begin{verbatim}
bool HS071_NLP::get_bounds_info(Index n, Number* x_l, Number* x_u,
                                Index m, Number* g_l, Number* g_u)
{
  // here, the n and m we gave IPOPT in get_nlp_info are passed back to us.
  // If desired, we could assert to make sure they are what we think they are.
  assert(n == 4);
  assert(m == 2);

  // the variables have lower bounds of 1
  for (Index i=0; i<4; i++) {
    x_l[i] = 1.0;
  }

  // the variables have upper bounds of 5
  for (Index i=0; i<4; i++) {
    x_u[i] = 5.0;
  }

  // the first constraint g1 has a lower bound of 25
  g_l[0] = 25;
  // the first constraint g1 has NO upper bound, here we set it to 2e19.
  // Ipopt interprets any number greater than nlp_upper_bound_inf as 
  // infinity. The default value of nlp_upper_bound_inf and nlp_lower_bound_inf
  // is 1e19 and can be changed through ipopt options.
  g_u[0] = 2e19;

  // the second constraint g2 is an equality constraint, so we set the 
  // upper and lower bound to the same value
  g_l[1] = g_u[1] = 40.0;

  return true;
}
\end{verbatim}
\end{footnotesize}

\paragraph{Method {\texttt{get\_starting\_point}}} with prototype
\begin{verbatim}
virtual bool get_starting_point(Index n, bool init_x, Number* x,
                                bool init_z, Number* z_L, Number* z_U,
                                Index m, bool init_lambda, Number* lambda)
\end{verbatim}
Give \Ipopt\ the starting point before it begins iterating.
\begin{itemize}
\item {\tt n}: (in), the number of variables in the problem (dimension of $x$). 
\item {\tt init\_x}: (in), if true, this method must provide an initial value for $x$.
\item {\tt x}: (out), the initial values for the primal variables, $x$.
\item {\tt init\_z}: (in), if true, this method must provide an initial value 
        for the bound multipliers $z^L$ and $z^U$.
\item {\tt z\_L}: (out), the initial values for the bound multipliers, $z^L$.
\item {\tt z\_U}: (out), the initial values for the bound multipliers, $z^U$.
\item {\tt m}: (in), the number of constraints in the problem (dimension of $g(x)$).
\item {\tt init\_lambda}: (in), if true, this method must provide an initial value 
        for the constraint multipliers, $\lambda$.
\item {\tt lambda}: (out), the initial values for the constraint multipliers, $\lambda$.
\end{itemize}

The variables {\tt n} and {\tt m} are passed in for your convenience.
These variables will have the same values you specified in {\tt
  get\_nlp\_info}.

Depending on the options that have been set, \Ipopt\ may or may not
require bounds for the primal variables $x$, the bound multipliers
$z^L$ and $z^U$, and the constraint multipliers $\lambda$. The boolean
flags {\tt init\_x}, {\tt init\_z}, and {\tt init\_lambda} tell you
whether or not you should provide initial values for $x$, $z^L$, $z^U$, or
$\lambda$ respectively. The default options only require an initial
value for the primal variables $x$.  Note, the initial values for
bound multiplier components for ``infinity'' bounds
($x_L^{(i)}=-\infty$ or $x_U^{(i)}=\infty$) are ignored.

In our example, we provide initial values for $x$ as specified in the
example problem. We do not provide any initial values for the dual
variables, but use an assert to immediately let us know if we are ever
asked for them.

\begin{footnotesize}
\begin{verbatim}
bool HS071_NLP::get_starting_point(Index n, bool init_x, Number* x,
                                   bool init_z, Number* z_L, Number* z_U,
                                   Index m, bool init_lambda,
                                   Number* lambda)
{
  // Here, we assume we only have starting values for x, if you code
  // your own NLP, you can provide starting values for the dual variables
  // if you wish to use a warmstart option
  assert(init_x == true);
  assert(init_z == false);
  assert(init_lambda == false);

  // initialize to the given starting point
  x[0] = 1.0;
  x[1] = 5.0;
  x[2] = 5.0;
  x[3] = 1.0;

  return true;
}
\end{verbatim}
\end{footnotesize}

\paragraph{Method {\texttt{eval\_f}}} with prototype
\begin{verbatim}
virtual bool eval_f(Index n, const Number* x, 
                    bool new_x, Number& obj_value)
\end{verbatim}
Return the value of the objective function at the point $x$.
\begin{itemize}
\item {\tt n}: (in), the number of variables in the problem (dimension
  of $x$).
\item {\tt x}: (in), the values for the primal variables, $x$, at which
  $f(x)$ is to be evaluated.
\item {\tt new\_x}: (in), false if any evaluation method was
  previously called with the same values in {\tt x}, true otherwise.
\item {\tt obj\_value}: (out) the value of the objective function
  ($f(x)$).
\end{itemize}

The boolean variable {\tt new\_x} will be false if the last call to
any of the evaluation methods ({\tt eval\_*}) used the same $x$
values. This can be helpful when users have efficient implementations
that calculate multiple outputs at once. \Ipopt\ internally caches
results from the {\tt TNLP} and generally, this flag can be ignored.

The variable {\tt n} is passed in for your convenience. This variable
will have the same value you specified in {\tt get\_nlp\_info}.

For our example, we ignore the {\tt new\_x} flag and calculate the objective.

\begin{footnotesize}
\begin{verbatim}
bool HS071_NLP::eval_f(Index n, const Number* x, bool new_x, Number& obj_value)
{
  assert(n == 4);

  obj_value = x[0] * x[3] * (x[0] + x[1] + x[2]) + x[2];

  return true;
}
\end{verbatim}
\end{footnotesize}

\paragraph{Method {\texttt{eval\_grad\_f}}} with prototype
\begin{verbatim}
virtual bool eval_grad_f(Index n, const Number* x, bool new_x, 
                         Number* grad_f)
\end{verbatim}
Return the gradient of the objective function at the point $x$.
\begin{itemize}
\item {\tt n}: (in), the number of variables in the problem (dimension of $x$). 
\item {\tt x}: (in), the values for the primal variables, $x$, at which
  $\nabla f(x)$ is to be evaluated.
\item {\tt new\_x}: (in), false if any evaluation method was previously called 
        with the same values in {\tt x}, true otherwise.
\item {\tt grad\_f}: (out) the array of values for the gradient of the 
        objective function ($\nabla f(x)$).
\end{itemize}

The gradient array is in the same order as the $x$ variables (i.e., the
gradient of the objective with respect to {\tt x[2]} should be put in
{\tt grad\_f[2]}).

The boolean variable {\tt new\_x} will be false if the last call to
any of the evaluation methods ({\tt eval\_*}) used the same $x$
values. This can be helpful when users have efficient implementations
that calculate multiple outputs at once. \Ipopt\ internally caches
results from the {\tt TNLP} and generally, this flag can be ignored.

The variable {\tt n} is passed in for your convenience. This
variable will have the same value you specified in {\tt
get\_nlp\_info}.

In our example, we ignore the {\tt new\_x} flag and calculate the
values for the gradient of the objective.

\begin{footnotesize}
\begin{verbatim}
bool HS071_NLP::eval_grad_f(Index n, const Number* x, bool new_x, Number* grad_f)
{
  assert(n == 4);

  grad_f[0] = x[0] * x[3] + x[3] * (x[0] + x[1] + x[2]);
  grad_f[1] = x[0] * x[3];
  grad_f[2] = x[0] * x[3] + 1;
  grad_f[3] = x[0] * (x[0] + x[1] + x[2]);

  return true;
}
\end{verbatim}
\end{footnotesize}

\paragraph{Method {\texttt{eval\_g}}} with prototype
\begin{verbatim}
virtual bool eval_g(Index n, const Number* x, 
                    bool new_x, Index m, Number* g)
\end{verbatim}
Return the value of the constraint function at the point $x$.
\begin{itemize}
\item {\tt n}: (in), the number of variables in the problem (dimension of $x$). 
\item {\tt x}: (in), the values for the primal variables, $x$, at
  which the constraint functions,
  $g(x)$, are to be evaluated.
\item {\tt new\_x}: (in), false if any evaluation method was previously called 
        with the same values in {\tt x}, true otherwise.
\item {\tt m}: (in), the number of constraints in the problem (dimension of $g(x)$).
\item {\tt g}: (out) the array of constraint function values, $g(x)$.
\end{itemize}

The values returned in {\tt g} should be only the $g(x)$ values, 
do not add or subtract the bound values $g^L$ or $g^U$.

The boolean variable {\tt new\_x} will be false if the last call to
any of the evaluation methods ({\tt eval\_*}) used the same $x$
values. This can be helpful when users have efficient implementations
that calculate multiple outputs at once. \Ipopt\ internally caches
results from the {\tt TNLP} and generally, this flag can be ignored.

The variables {\tt n} and {\tt m} are passed in for your convenience.
These variables will have the same values you specified in {\tt
  get\_nlp\_info}.

In our example, we ignore the {\tt new\_x} flag and calculate the
values of constraint functions.

\begin{footnotesize}
\begin{verbatim}
bool HS071_NLP::eval_g(Index n, const Number* x, bool new_x, Index m, Number* g)
{
  assert(n == 4);
  assert(m == 2);

  g[0] = x[0] * x[1] * x[2] * x[3];
  g[1] = x[0]*x[0] + x[1]*x[1] + x[2]*x[2] + x[3]*x[3];

  return true;
} 
\end{verbatim}
\end{footnotesize}

\paragraph{Method {\texttt{eval\_jac\_g}}} with prototype
\begin{verbatim}
virtual bool eval_jac_g(Index n, const Number* x, bool new_x,
                        Index m, Index nele_jac, Index* iRow, 
                        Index *jCol, Number* values)
\end{verbatim}
Return either the sparsity structure of the Jacobian of the
constraints, or the values for the Jacobian of the constraints at the
point $x$.
\begin{itemize}
\item {\tt n}: (in), the number of variables in the problem (dimension of $x$). 
\item {\tt x}: (in), the values for the primal variables, $x$, at which
  the constraint Jacobian, $\nabla g(x)^T$, is to be evaluated.
\item {\tt new\_x}: (in), false if any evaluation method was previously called 
        with the same values in {\tt x}, true otherwise.
\item {\tt m}: (in), the number of constraints in the problem (dimension of $g(x)$).
\item {\tt n\_ele\_jac}: (in), the number of nonzero elements in the 
        Jacobian (dimension of {\tt iRow}, {\tt jCol}, and {\tt values}).
\item {\tt iRow}: (out), the row indices of entries in the Jacobian of the constraints.
\item {\tt jCol}: (out), the column indices of entries in the Jacobian of the constraints.
\item {\tt values}: (out), the values of the entries in the Jacobian of the constraints.
\end{itemize}

The Jacobian is the matrix of derivatives where the derivative of
constraint $g^{(i)}$ with respect to variable $x^{(j)}$ is placed in
row $i$ and column $j$. See Appendix \ref{app.triplet} for a
discussion of the sparse matrix format used in this method.

If the {\tt iRow} and {\tt jCol} arguments are not {\tt NULL}, then
\Ipopt\ wants you to fill in the sparsity structure of the Jacobian
(the row and column indices only). At this time, the {\tt x} argument
and the {\tt values} argument will be {\tt NULL}.

If the {\tt x} argument and the {\tt values} argument are not {\tt
  NULL}, then \Ipopt\ wants you to fill in the values of the Jacobian
as calculated from the array {\tt x} (using the same order as you used
when specifying the sparsity structure). At this time, the {\tt iRow}
and {\tt jCol} arguments will be {\tt NULL};

The boolean variable {\tt new\_x} will be false if the last call to
any of the evaluation methods ({\tt eval\_*}) used the same $x$
values. This can be helpful when users have efficient implementations
that calculate multiple outputs at once. \Ipopt\ internally caches
results from the {\tt TNLP} and generally, this flag can be ignored.

The variables {\tt n}, {\tt m}, and {\tt nele\_jac} are passed in for
your convenience. These arguments will have the same values you
specified in {\tt get\_nlp\_info}.

In our example, the Jacobian is actually dense, but we still
specify it using the sparse format.

\begin{footnotesize}
\begin{verbatim}
bool HS071_NLP::eval_jac_g(Index n, const Number* x, bool new_x,
                           Index m, Index nele_jac, Index* iRow, Index *jCol,
                           Number* values)
{
  if (values == NULL) {
    // return the structure of the Jacobian

    // this particular Jacobian is dense
    iRow[0] = 0; jCol[0] = 0;
    iRow[1] = 0; jCol[1] = 1;
    iRow[2] = 0; jCol[2] = 2;
    iRow[3] = 0; jCol[3] = 3;
    iRow[4] = 1; jCol[4] = 0;
    iRow[5] = 1; jCol[5] = 1;
    iRow[6] = 1; jCol[6] = 2;
    iRow[7] = 1; jCol[7] = 3;
  }
  else {
    // return the values of the Jacobian of the constraints
    
    values[0] = x[1]*x[2]*x[3]; // 0,0
    values[1] = x[0]*x[2]*x[3]; // 0,1
    values[2] = x[0]*x[1]*x[3]; // 0,2
    values[3] = x[0]*x[1]*x[2]; // 0,3

    values[4] = 2*x[0]; // 1,0
    values[5] = 2*x[1]; // 1,1
    values[6] = 2*x[2]; // 1,2
    values[7] = 2*x[3]; // 1,3
  }

  return true;
}
\end{verbatim}
\end{footnotesize}

\paragraph{Method {\texttt{eval\_h}}} with prototype
\begin{verbatim}
virtual bool eval_h(Index n, const Number* x, bool new_x,
                    Number obj_factor, Index m, const Number* lambda,
                    bool new_lambda, Index nele_hess, Index* iRow,
                    Index* jCol, Number* values)
\end{verbatim}
Return either the sparsity structure of the Hessian of the Lagrangian, or the values of the 
Hessian of the Lagrangian (\ref{eq:IpoptLAG}) for the given values for $x$,
$\sigma_f$, and $\lambda$.
\begin{itemize}
\item {\tt n}: (in), the number of variables in the problem (dimension
  of $x$).
\item {\tt x}: (in), the values for the primal variables, $x$, at which
  the Hessian is to be evaluated.
\item {\tt new\_x}: (in), false if any evaluation method was previously called 
        with the same values in {\tt x}, true otherwise.
\item {\tt obj\_factor}: (in), factor in front of the objective term
  in the Hessian, $sigma_f$.
\item {\tt m}: (in), the number of constraints in the problem (dimension of $g(x)$).
\item {\tt lambda}: (in), the values for the constraint multipliers,
  $\lambda$, at which the Hessian is to be evaluated.
\item {\tt new\_lambda}: (in), false if any evaluation method was
  previously called with the same values in {\tt lambda}, true
  otherwise.
\item {\tt nele\_hess}: (in), the number of nonzero elements in the
  Hessian (dimension of {\tt iRow}, {\tt jCol}, and {\tt values}).
\item {\tt iRow}: (out), the row indices of entries in the Hessian.
\item {\tt jCol}: (out), the column indices of entries in the Hessian.
\item {\tt values}: (out), the values of the entries in the Hessian.
\end{itemize}

The Hessian matrix that \Ipopt\ uses is defined in
Eq.~\ref{eq:IpoptLAG}.  See Appendix \ref{app.triplet} for a
discussion of the sparse symmetric matrix format used in this method.

If the {\tt iRow} and {\tt jCol} arguments are not {\tt NULL}, then
\Ipopt\ wants you to fill in the sparsity structure of the Hessian
(the row and column indices for the lower or upper triangular part
only). In this case, the {\tt x}, {\tt lambda}, and {\tt values}
arrays will be {\tt NULL}.

If the {\tt x}, {\tt lambda}, and {\tt values} arrays are not {\tt
  NULL}, then \Ipopt\ wants you to fill in the values of the Hessian
as calculated using {\tt x} and {\tt lambda} (using the same order as
you used when specifying the sparsity structure). In this case, the
{\tt iRow} and {\tt jCol} arguments will be {\tt NULL}.

The boolean variables {\tt new\_x} and {\tt new\_lambda} will both be
false if the last call to any of the evaluation methods ({\tt
  eval\_*}) used the same values. This can be helpful when users have
efficient implementations that calculate multiple outputs at once.
\Ipopt\ internally caches results from the {\tt TNLP} and generally,
this flag can be ignored.

The variables {\tt n}, {\tt m}, and {\tt nele\_hess} are passed in for
your convenience. These arguments will have the same values you
specified in {\tt get\_nlp\_info}.

In our example, the Hessian is dense, but we still specify it using the
sparse matrix format. Because the Hessian is symmetric, we only need to 
specify the lower left corner.

\begin{footnotesize}
\begin{verbatim}
bool HS071_NLP::eval_h(Index n, const Number* x, bool new_x,
                       Number obj_factor, Index m, const Number* lambda,
                       bool new_lambda, Index nele_hess, Index* iRow,
                       Index* jCol, Number* values)
{
  if (values == NULL) {
    // return the structure. This is a symmetric matrix, fill the lower left
    // triangle only.

    // the Hessian for this problem is actually dense
    Index idx=0;
    for (Index row = 0; row < 4; row++) {
      for (Index col = 0; col <= row; col++) {
        iRow[idx] = row; 
        jCol[idx] = col;
        idx++;
      }
    }
    
    assert(idx == nele_hess);
  }
  else {
    // return the values. This is a symmetric matrix, fill the lower left
    // triangle only

    // fill the objective portion
    values[0] = obj_factor * (2*x[3]); // 0,0

    values[1] = obj_factor * (x[3]);   // 1,0
    values[2] = 0;                     // 1,1

    values[3] = obj_factor * (x[3]);   // 2,0
    values[4] = 0;                     // 2,1
    values[5] = 0;                     // 2,2

    values[6] = obj_factor * (2*x[0] + x[1] + x[2]); // 3,0
    values[7] = obj_factor * (x[0]);                 // 3,1
    values[8] = obj_factor * (x[0]);                 // 3,2
    values[9] = 0;                                   // 3,3


    // add the portion for the first constraint
    values[1] += lambda[0] * (x[2] * x[3]); // 1,0
    
    values[3] += lambda[0] * (x[1] * x[3]); // 2,0
    values[4] += lambda[0] * (x[0] * x[3]); // 2,1

    values[6] += lambda[0] * (x[1] * x[2]); // 3,0
    values[7] += lambda[0] * (x[0] * x[2]); // 3,1
    values[8] += lambda[0] * (x[0] * x[1]); // 3,2

    // add the portion for the second constraint
    values[0] += lambda[1] * 2; // 0,0

    values[2] += lambda[1] * 2; // 1,1

    values[5] += lambda[1] * 2; // 2,2

    values[9] += lambda[1] * 2; // 3,3
  }

  return true;
}
\end{verbatim}
\end{footnotesize}

\paragraph{Method \texttt{finalize\_solution}} with prototype

\begin{verbatim}
virtual void finalize_solution(SolverReturn status, Index n,
                               const Number* x, const Number* z_L,
                               const Number* z_U, Index m, const Number* g,
                               const Number* lambda, Number obj_value)
\end{verbatim}
This is the only method that is not mentioned in Figure
\ref{fig.required_info}. This method is called by \Ipopt\ after the
algorithm has finished (successfully or even with most errors).
\begin{itemize}
\item {\tt status}: (in), gives the status of the algorithm as
  specified in {\tt IpAlgTypes.hpp},
  \begin{itemize}
  \item {\tt SUCCESS}: Algorithm terminated successfully at a locally
    optimal point, satisfying the convergence tolerances (can be
    specified by options).
  \item {\tt MAXITER\_EXCEEDED}: Maximum number of iterations exceeded
    (can be specified by an option).
  \item {\tt STOP\_AT\_TINY\_STEP}: Algorithm proceeds with very
    little progress.
  \item {\tt STOP\_AT\_ACCEPTABLE\_POINT}: Algorithm stopped at a
    point that was converged, not to ``desired'' tolerances, but to
    ``acceptable'' tolerances (see the {\tt acceptable-...} options).
  \item {\tt LOCAL\_INFEASIBILITY}: Algorithm converged to a point of
    local infeasibility. Problem may be infeasible.
  \item {\tt USER\_REQUESTED\_STOP}: The user call-back function {\tt
      intermediate\_callback} (see Section~\ref{sec:add_meth})
    returned {\tt false}, i.e., the user code requested a premature
    termination of the optimization.
  \item {\tt DIVERGING\_ITERATES}: It seems that the iterates diverge.
  \item {\tt RESTORATION\_FAILURE}: Restoration phase failed,
    algorithm doesn't know how to proceed.
  \item {\tt ERROR\_IN\_STEP\_COMPUTATION}: An unrecoverable error
    occurred while \Ipopt\ tried to compute the search direction.
  \item {\tt INVALID\_NUMBER\_DETECTED}:  Algorithm received an
    invalid number (such as {\tt NaN} or {\tt Inf}) from the NLP; see
    also option {\tt check\_derivatives\_for\_naninf}.
  \item {\tt
      INTERNAL\_ERROR}: An unknown internal error occurred.  Please
    contact the \Ipopt\ authors through the mailing list.
  \end{itemize}
\item {\tt n}: (in), the number of variables in the problem (dimension
  of $x$).
\item {\tt x}: (in), the final values for the primal variables, $x_*$.
\item {\tt z\_L}: (in), the final values for the lower bound
  multipliers, $z^L_*$.
\item {\tt z\_U}: (in), the final values for the upper bound
  multipliers, $z^U_*$.
\item {\tt m}: (in), the number of constraints in the problem
  (dimension of $g(x)$).
\item {\tt g}: (in), the final value of the constraint function
  values, $g(x_*)$.
\item {\tt lambda}: (in), the final values of the constraint
  multipliers, $\lambda_*$.
\item {\tt obj\_value}: (in), the final value of the objective,
  $f(x_*)$.
\end{itemize}

This method gives you the return status of the algorithm
(SolverReturn), and the values of the variables, 
the objective and constraint function values when the algorithm exited.

In our example, we will print the values of some of the variables to 
the screen.

\begin{footnotesize}
\begin{verbatim}
void HS071_NLP::finalize_solution(SolverReturn status,
                                  Index n, const Number* x, const Number* z_L,
                                  const Number* z_U, Index m, const Number* g,
                                  const Number* lambda, Number obj_value)
{
  // here is where we would store the solution to variables, or write to a file, etc
  // so we could use the solution. 

  // For this example, we write the solution to the console
  printf("\n\nSolution of the primal variables, x\n");
  for (Index i=0; i<n; i++) {
    printf("x[%d] = %e\n", i, x[i]); 
  }

  printf("\n\nSolution of the bound multipliers, z_L and z_U\n");
  for (Index i=0; i<n; i++) {
    printf("z_L[%d] = %e\n", i, z_L[i]); 
  }
  for (Index i=0; i<n; i++) {
    printf("z_U[%d] = %e\n", i, z_U[i]); 
  }

  printf("\n\nObjective value\n");
  printf("f(x*) = %e\n", obj_value); 
}
\end{verbatim}
\end{footnotesize}

This is all that is required for our {\tt HS071\_NLP} class and 
the coding of the problem representation.
 
\subsubsection{Coding the Executable (\texttt{main})}
Now that we have a problem representation, the {\tt HS071\_NLP} class,
we need to code the main function that will call \Ipopt\ and ask \Ipopt\
to find a solution.

Here, we must create an instance of our problem ({\tt HS071\_NLP}),
create an instance of the \Ipopt\ solver (\texttt{IpoptApplication}),
initialize it, and ask the solver to find a solution. We always use
the \texttt{SmartPtr} template class instead of raw C++ pointers when
creating and passing \Ipopt\ objects. To find out more information
about smart pointers and the {\tt SmartPtr} implementation used in
\Ipopt, see Appendix \ref{app.smart_ptr}.

Create the file {\tt MyExample.cpp} in the MyExample directory.
Include {\tt HS071\_NLP.hpp} and {\tt IpIpoptApplication.hpp}, tell
the compiler to use the {\tt Ipopt} namespace, and implement the {\tt
  main} function.

\begin{footnotesize}
\begin{verbatim}
#include "IpIpoptApplication.hpp"
#include "hs071_nlp.hpp"

using namespace Ipopt;

int main(int argv, char* argc[])
{
  // Create a new instance of your nlp 
  //  (use a SmartPtr, not raw)
  SmartPtr<TNLP> mynlp = new HS071_NLP();

  // Create a new instance of IpoptApplication
  //  (use a SmartPtr, not raw)
  SmartPtr<IpoptApplication> app = new IpoptApplication();

  // Change some options
  // Note: The following choices are only examples, they might not be
  //       suitable for your optimization problem.
  app->Options()->SetNumericValue("tol", 1e-9);
  app->Options()->SetStringValue("mu_strategy", "adaptive");
  app->Options()->SetStringValue("output_file", "ipopt.out");

  // Intialize the IpoptApplication and process the options
  app->Initialize();

  // Ask Ipopt to solve the problem
  ApplicationReturnStatus status = app->OptimizeTNLP(mynlp);

  if (status == Solve_Succeeded) {
    printf("\n\n*** The problem solved!\n");
  }
  else {
    printf("\n\n*** The problem FAILED!\n");
  }

  // As the SmartPtrs go out of scope, the reference count
  // will be decremented and the objects will automatically 
  // be deleted.

  return (int) status;
}
\end{verbatim} 
\end{footnotesize}

The first line of code in {\tt main} creates an instance of {\tt
  HS071\_NLP}. We then create an instance of the \Ipopt\ solver, {\tt
  IpoptApplication}. The call to {\tt app->Initialize(...)} will
initialize that object, process this options (particularly the output
related options), and the call to {\tt app->OptimizeTNLP(...)}  will
run \Ipopt\ and try to solve the problem. By default, \Ipopt\ will
write to its progress to the console, and return the {\tt
  SolverReturn} status.

\subsubsection{Compiling and Testing the Example}
Our next task is to compile and test the code. If you are familiar
with the compiler and linker used on your system, you can build the
code, including the \Ipopt\ library {\tt libipopt.a} (and other
necessary libraries, as listed in the {\tt ipopt\_addlibs\_cpp.txt}
and {\tt ipopt\_addlibs\_f.txt} files).  If you are using Linux/UNIX,
then a sample makefile exists already that was created by configure.
Copy {\tt Ipopt/examples/hs071\_cpp/Makefile} into your {\tt MyExample}
directory.  This makefile was created for the {\tt hs071\_cpp} code,
but it can be easily modified for your example problem. Edit the file,
making the following changes,

\begin{itemize}
\item change the {\tt EXE} variable \\
{\tt EXE = my\_example}
\item change the {\tt OBJS} variable \\
{\tt OBJS = HS071\_NLP.o MyExample.o}
\end{itemize}
and the problem should compile easily with, \\
{\tt \$ make} \\
Now run the executable,\\ 
{\tt \$ ./my\_example} \\
and you should see output resembling the following,

\begin{footnotesize}
\begin{verbatim}
******************************************************************************
This program contains Ipopt, a library for large-scale nonlinear optimization.
 Ipopt is released as open source code under the Common Public License (CPL).
         For more information visit http://projects.coin-or.org/Ipopt
******************************************************************************

Number of nonzeros in equality constraint Jacobian...:        4
Number of nonzeros in inequality constraint Jacobian.:        4
Number of nonzeros in Lagrangian Hessian.............:       10

Total number of variables............................:        4
                     variables with only lower bounds:        0
                variables with lower and upper bounds:        4
                     variables with only upper bounds:        0
Total number of equality constraints.................:        1
Total number of inequality constraints...............:        1
        inequality constraints with only lower bounds:        1
   inequality constraints with lower and upper bounds:        0
        inequality constraints with only upper bounds:        0

iter    objective    inf_pr   inf_du lg(mu)  ||d||  lg(rg) alpha_du alpha_pr  ls
   0  1.6109693e+01 1.12e+01 5.28e-01   0.0 0.00e+00    -  0.00e+00 0.00e+00   0
   1  1.7410406e+01 8.38e-01 2.25e+01  -0.3 7.97e-01    -  3.19e-01 1.00e+00f  1
   2  1.8001613e+01 1.06e-02 4.96e+00  -0.3 5.60e-02   2.0 9.97e-01 1.00e+00h  1
   3  1.7199482e+01 9.04e-02 4.24e-01  -1.0 9.91e-01    -  9.98e-01 1.00e+00f  1
   4  1.6940955e+01 2.09e-01 4.58e-02  -1.4 2.88e-01    -  9.66e-01 1.00e+00h  1
   5  1.7003411e+01 2.29e-02 8.42e-03  -2.9 7.03e-02    -  9.68e-01 1.00e+00h  1
   6  1.7013974e+01 2.59e-04 8.65e-05  -4.5 6.22e-03    -  1.00e+00 1.00e+00h  1
   7  1.7014017e+01 2.26e-07 5.71e-08  -8.0 1.43e-04    -  1.00e-00 1.00e+00h  1
   8  1.7014017e+01 4.62e-14 9.09e-14  -8.0 6.95e-08    -  1.00e+00 1.00e+00h  1

Number of Iterations....: 8

Number of objective function evaluations             = 9
Number of objective gradient evaluations             = 9
Number of equality constraint evaluations            = 9
Number of inequality constraint evaluations          = 9
Number of equality constraint Jacobian evaluations   = 9
Number of inequality constraint Jacobian evaluations = 9
Number of Lagrangian Hessian evaluations             = 8
Total CPU secs in IPOPT (w/o function evaluations)   =      0.220
Total CPU secs in NLP function evaluations           =      0.000

EXIT: Optimal Solution Found.


Solution of the primal variables, x
x[0] = 1.000000e+00
x[1] = 4.743000e+00
x[2] = 3.821150e+00
x[3] = 1.379408e+00


Solution of the bound multipliers, z_L and z_U
z_L[0] = 1.087871e+00
z_L[1] = 2.428776e-09
z_L[2] = 3.222413e-09
z_L[3] = 2.396076e-08
z_U[0] = 2.272727e-09
z_U[1] = 3.537314e-08
z_U[2] = 7.711676e-09
z_U[3] = 2.510890e-09


Objective value
f(x*) = 1.701402e+01


*** The problem solved!
\end{verbatim}
\end{footnotesize}

This completes the basic C++ tutorial, but see Section
\ref{sec:output} which explains the standard console output of \Ipopt
and Section \ref{sec:options} for information about the use of options
to customize the behavior of \Ipopt.

The {\tt Ipopt/examples/ScalableProblems} directory contains other NLP
problems coded in C++.

\subsubsection{Additional methods in {\tt TNLP}}\label{sec:add_meth}

The following methods are available to additional features that are
not explained in the example.  Default implementations for those
methods are provided, so that a user can safely ignore them, unless
she wants to make use of those features.  These features is not yet(?)
available from C or Fortran.

\paragraph{Method \texttt{intermediate\_callback}} with prototype
\begin{verbatim}
virtual bool intermediate_callback(AlgorithmMode mode,
                                   Index iter, Number obj_value,
                                   Number inf_pr, Number inf_du,
                                   Number mu, Number d_norm,
                                   Number regularization_size,
                                   Number alpha_du, Number alpha_pr,
                                   Index ls_trials,
                                   const IpoptData* ip_data,
                                   IpoptCalculatedQuantities* ip_cq)
\end{verbatim}
It is not required to implement (overload) this method.  This method
is called once per iteration (during the convergence check), and can
be used to obtain information about the optimization status while
\Ipopt\ solves the problem, and also to requires a premature
termination.

The information provided by the entities in the argument list
corresponds to what \Ipopt\ prints in the iteration summary (see also
Section~\ref{sec:output}).  Further information can be obtained from
the {\tt ip\_data} and {\tt ip\_cq} objects (for experts only :).

You you let this method return {\tt false}, \Ipopt\ will terminate
with the {\tt User\_Requested\_Stop} status.  If you do not implement
this method (as we do in this example), the default implementation
always returns {\tt true}.

\paragraph{Method \texttt{get\_scaling\_parameters}} with prototype
\begin{verbatim}
virtual bool get_scaling_parameters(Number& obj_scaling,
                                    bool& use_x_scaling, Index n,
                                    Number* x_scaling,
                                    bool& use_g_scaling, Index m,
                                    Number* g_scaling)
\end{verbatim}

This method is called if the {\tt nlp\_scaling\_method} is chosen as
{\tt user-scaling}.  Then the user is to provide scaling factors for
the objective function, as well as for the optimization variables
and/or constraints.  The return value should be true, unless an error
occurred, and the program is to be aborted.

The value returned in {\tt obj\_scaling} determines, how \Ipopt\
should internally scale the objective function.  For example, if this
number is chosen to be 10, then \Ipopt\ solves internally an
optimization problem that has 10 times the value of the original
objective function provided by the {\tt TNLP}.  In particular, if this
value is negative, then \Ipopt\ will maximize the objective function
instead of minimizing it.

The scaling factors for the variables can be returned in {\tt
  x\_scaling}, which has the same length as {\tt x} in the other {\tt
  TNLP} methods, and the factors are ordered like {\tt x}.  You need
to set {\tt use\_x\_scaling} to {\tt true}, if you want \Ipopt\ so scale
the variables.  If it is {\tt false}, no internal scaling of the
variables is done.  Similarly, the scaling factors for the constraints
can be returned in {\tt g\_scaling}, and this scaling is activated by
setting {\tt use\_g\_scaling} to {\tt true}.

As a guideline, we suggest to scale the optimization problem (either
directly in the original formulation, or after using scaling factors)
so that all sensitivities, i.e., all non-zero first partial
derivatives, are typically of the order $0.1-10$.

\paragraph{Method \texttt{get\_number\_of\_nonlinear\_variables}} with prototype
\begin{verbatim}
virtual Index get_number_of_nonlinear_variables()
\end{verbatim}

This method is only important if the limited-memory quasi-Newton
options is used, see Section~\ref{sec:quasiNewton}.  It is to be used
to return the number of variables that appear nonlinearly in the
objective function or in at least one constraint function.  If a
negative number is returned, \Ipopt\ assumes that all variables are
nonlinear.

If the user doesn't overload this method in her implementation of the
class derived from {\tt TNLP}, the default implementation returns -1,
i.e., then all variables are assumed to be nonlinear.

\paragraph{Method \texttt{get\_list\_of\_nonlinear\_variables}} with prototype
\begin{verbatim}
virtual bool get_list_of_nonlinear_variables(Index num_nonlin_vars,
                                             Index* pos_nonlin_vars)
\end{verbatim}

This method is called by \Ipopt\ only if the limited-memory
quasi-Newton options is used, and if the {\tt
  get\_number\_of\_nonlinear\_variables} method returns a positive
number; this number is then identical with {\tt num\_nonlin\_vars} and
the length of the array {\tt pos\_nonlin\_vars}.  In this call, you
need to list the indices of all nonlinear variables in {\tt
  pos\_nonlin\_vars}, where the numbering starts with 0 order 1,
depending on the numbering style determined in {\tt get\_nlp\_info}.


\subsection{The C Interface}\label{sec.cinterface}
The C interface for \Ipopt\ is declared in the header file {\tt
  IpStdCInterface.h}, which is found in\\
\texttt{\$IPOPTDIR/include/ipopt} (or in
\texttt{\$PREFIX/include/ipopt} if the switch
\verb|--prefix=$PREFIX| was used for {\tt configure}); while
reading this section, it will be helpful to have a look at this file.

In order to solve an optimization problem with the C interface, one
has to create an {\tt IpoptProblem}\footnote{{\tt IpoptProblem} is a
  pointer to a C structure; you should not access this structure
  directly, only through the functions provided in the C interface.}
with the function {\tt CreateIpoptProblem}, which later has to be
passed to the {\tt IpoptSolve} function.

The {\tt IpoptProblem} created by {\tt CreateIpoptProblem} contains
the problem dimensions, the variable and constraint bounds, and the
function pointers for callbacks that will be used to evaluate the NLP
problem functions and their derivatives (see also the discussion of
the C++ methods {\tt get\_nlp\_info} and {\tt get\_bounds\_info} in
Section~\ref{sec.cpp_problem} for information about the arguments of
{\tt CreateIpoptProblem}).

The prototypes for the callback functions, {\tt Eval\_F\_CB}, {\tt
  Eval\_Grad\_F\_CB}, etc., are defined in the header file {\tt
  IpStdCInterface.h}.  Their arguments correspond one-to-one to the
arguments for the C++ methods discussed in
Section~\ref{sec.cpp_problem}; for example, for the meaning of $\tt
n$, $\tt x$, $\tt new\_x$, $\tt obj\_value$ in the declaration of {\tt
  Eval\_F\_CB} see the discussion of ``{\tt eval\_f}''.  The callback
functions should return {\tt TRUE}, unless there was a problem doing
the requested function/derivative evaluation at the given point {\tt
  x} (then it should return {\tt FALSE}).

Note the additional argument of type {\tt UserDataPtr} in the callback
functions.  This pointer argument is available for you to communicate
information between the main program that calls {\tt IpoptSolve} and
any of the callback functions.  This pointer is simply passed
unmodified by \Ipopt\ among those functions.  For example, you can
use this to pass constants that define the optimization problem and
are computed before the optimization in the main C program to the
callback functions.

After an {\tt IpoptProblem} has been created, you can set algorithmic
options for \Ipopt\ (see Section~\ref{sec:options}) using the {\tt
  AddIpopt...Option} functions.  Finally, the \Ipopt\ algorithm is
called with {\tt IpoptSolve}, giving \Ipopt\ the {\tt IpoptProblem},
the starting point, and arrays to store the solution values (primal
and dual variables), if desired.  Finally, after everything is done,
you should call {\tt FreeIpoptProblem} to release internal memory that
is still allocated inside \Ipopt.

In the remainder of this section we discuss how the example problem
(\ref{eq:ex_obj})--(\ref{eq:ex_bounds}) can be solved using the C
interface.  A completed version of this example can be found in {\tt
  Ipopt/examples/hs071\_c}.

% We first create the necessary callback
% functions for evaluating the NLP. As just discussed, the \Ipopt\ C
% interface required callbacks to evaluate the objective value,
% constraints, gradient of the objective, Jacobian of the constraints,
% and the Hessian of the Lagrangian.  These callbacks are implemented
% using function pointers.  Have a look at the C++ implementation for
% {\tt eval\_f}, {\tt eval\_g}, {\tt eval\_grad\_f}, {\tt eval\_jac\_g},
% and {\tt eval\_h} in Section \ref{sec.cpp_problem}. The C
% implementations have somewhat different prototypes, but are
% implemented almost identically to the C++ code.

\vspace{\baselineskip}

In order to implement the example problem on your own, create a new
directory {\tt MyCExample} and create a new file, {\tt
  hs071\_c.c}.  Here, include the interface header file {\tt
  IpStdCInterface.h}, along with other necessary header files, such as
{\tt stdlib.h} and {\tt assert.h}.  Add the prototypes and
implementations for the five callback functions.  Have a look at the
C++ implementation for {\tt eval\_f}, {\tt eval\_g}, {\tt
  eval\_grad\_f}, {\tt eval\_jac\_g}, and {\tt eval\_h} in Section
\ref{sec.cpp_problem}. The C implementations have somewhat different
prototypes, but are implemented almost identically to the C++ code.
See the completed example in {\tt Ipopt/examples/hs071\_c/hs071\_c.c} if you
are not sure how to do this.

We now need to implement the {\tt main} function, create the {\tt
  IpoptProblem}, set options, and call {\tt IpoptSolve}. The {\tt
  CreateIpoptProblem} function requires the problem dimensions, the
variable and constraint bounds, and the function pointers to the
callback routines. The {\tt IpoptSolve} function requires the {\tt
  IpoptProblem}, the starting point, and allocated arrays for the
solution.  The {\tt main} function from the example is shown next, and
discussed below.

%in Figure~\ref{fig:cexample-main}.
%\begin{figure}
%  \centering
\begin{footnotesize}
\begin{verbatim}
int main()
{
  Index n=-1;                          /* number of variables */
  Index m=-1;                          /* number of constraints */
  Number* x_L = NULL;                  /* lower bounds on x */
  Number* x_U = NULL;                  /* upper bounds on x */
  Number* g_L = NULL;                  /* lower bounds on g */
  Number* g_U = NULL;                  /* upper bounds on g */
  IpoptProblem nlp = NULL;             /* IpoptProblem */
  enum ApplicationReturnStatus status; /* Solve return code */
  Number* x = NULL;                    /* starting point and solution vector */
  Number* mult_x_L = NULL;             /* lower bound multipliers 
					  at the solution */
  Number* mult_x_U = NULL;             /* upper bound multipliers 
					  at the solution */
  Number obj;                          /* objective value */
  Index i;                             /* generic counter */
  
  /* set the number of variables and allocate space for the bounds */
  n=4;
  x_L = (Number*)malloc(sizeof(Number)*n);
  x_U = (Number*)malloc(sizeof(Number)*n);
  /* set the values for the variable bounds */
  for (i=0; i<n; i++) {
    x_L[i] = 1.0;
    x_U[i] = 5.0;
  }

  /* set the number of constraints and allocate space for the bounds */
  m=2;
  g_L = (Number*)malloc(sizeof(Number)*m);
  g_U = (Number*)malloc(sizeof(Number)*m);
  /* set the values of the constraint bounds */
  g_L[0] = 25; g_U[0] = 2e19;
  g_L[1] = 40; g_U[1] = 40;

  /* create the IpoptProblem */
  nlp = CreateIpoptProblem(n, x_L, x_U, m, g_L, g_U, 8, 10, 0, 
			   &eval_f, &eval_g, &eval_grad_f, 
			   &eval_jac_g, &eval_h);
  
  /* We can free the memory now - the values for the bounds have been
     copied internally in CreateIpoptProblem */
  free(x_L);
  free(x_U);
  free(g_L);
  free(g_U);

  /* set some options */
  AddIpoptNumOption(nlp, "tol", 1e-9);
  AddIpoptStrOption(nlp, "mu_strategy", "adaptive");

  /* allocate space for the initial point and set the values */
  x = (Number*)malloc(sizeof(Number)*n);
  x[0] = 1.0;
  x[1] = 5.0;
  x[2] = 5.0;
  x[3] = 1.0;

  /* allocate space to store the bound multipliers at the solution */
  mult_x_L = (Number*)malloc(sizeof(Number)*n);
  mult_x_U = (Number*)malloc(sizeof(Number)*n);

  /* solve the problem */
  status = IpoptSolve(nlp, x, NULL, &obj, NULL, mult_x_L, mult_x_U, NULL);

  if (status == Solve_Succeeded) {
    printf("\n\nSolution of the primal variables, x\n");
    for (i=0; i<n; i++) {
      printf("x[%d] = %e\n", i, x[i]); 
    }

    printf("\n\nSolution of the bound multipliers, z_L and z_U\n");
    for (i=0; i<n; i++) {
      printf("z_L[%d] = %e\n", i, mult_x_L[i]); 
    }
    for (i=0; i<n; i++) {
      printf("z_U[%d] = %e\n", i, mult_x_U[i]); 
    }

    printf("\n\nObjective value\n");
    printf("f(x*) = %e\n", obj); 
  }
 
  /* free allocated memory */
  FreeIpoptProblem(nlp);
  free(x);
  free(mult_x_L);
  free(mult_x_U);

  return 0;
}
\end{verbatim}
\end{footnotesize}
%  \caption{{\tt main} function for C example}
%  \label{fig:cexample-main}
%\end{figure}

Here, we declare all the necessary variables and set the dimensions of
the problem.  The problem has 4 variables, so we set {\tt n} and
allocate space for the variable bounds (don't forget to call {\tt
  free} for each of your {\tt malloc} calls before the end of the
program). We then set the values for the variable bounds.

The problem has 2 constraints, so we set {\tt m} and allocate space
for the constraint bounds. The first constraint has a lower bound of
$25$ and no upper bound.  Here we set the upper bound to
\texttt{2e19}. \Ipopt\ interprets any number greater than or equal to
\texttt{nlp\_upper\_bound\_inf} as infinity. The default value of
\texttt{nlp\_lower\_bound\_inf} and \texttt{nlp\_upper\_bound\_inf} is
\texttt{-1e19} and \texttt{1e19}, respectively, and can be changed
through \Ipopt\ options.  The second constraint is an equality with
right hand side 40, so we set both the upper and the lower bound to
40.

We next create an instance of the {\tt IpoptProblem} by calling {\tt
CreateIpoptProblem}, giving it the problem dimensions and the variable
and constraint bounds. The arguments {\tt nele\_jac} and {\tt
nele\_hess} are the number of elements in Jacobian and the Hessian,
respectively. See Appendix~\ref{app.triplet} for a description of the
sparse matrix format. The {\tt index\_style} argument specifies whether
we want to use C style indexing for the row and column indices of the
matrices or Fortran style indexing. Here, we set it to {\tt 0} to
indicate C style.  We also include the references to each of our
callback functions. \Ipopt\ uses these function pointers to ask for
evaluation of the NLP when required.

After freeing the bound arrays that are no longer required, the next
two lines illustrate how you can change the value of options through
the interface.  \Ipopt\ options can also be changed by creating a {\tt
ipopt.opt} file (see Section~\ref{sec:options}). We next allocate
space for the initial point and set the values as given in the problem
definition.

The call to {\tt IpoptSolve} can provide us with information about the
solution, but most of this is optional. Here, we want values for the
bound multipliers at the solution and we allocate space for these.

We can now make the call to {\tt IpoptSolve} and find the solution of
the problem. We pass in the {\tt IpoptProblem}, the starting point
{\tt x} (\Ipopt\ will use this array to return the solution or final
point as well).  The next 5 arguments are pointers so \Ipopt\ can fill
in values at the solution.  If these pointers are set to {\tt NULL},
\Ipopt\ will ignore that entry.  For example, here, we do not want the
constraint function values at the solution or the constraint
multipliers, so we set those entries to {\tt NULL}. We do want the
value of the objective, and the multipliers for the variable bounds.
The last argument is a {\tt void*} for user data. Any pointer you give
here will also be passed to you in the callback functions.

The return code is an {\tt ApplicationReturnStatus} enumeration, see
the header file {\tt ReturnCodes\_inc.h} which is installed along {\tt
  IpStdCInterface.h} in the \Ipopt\ include directory.

After the optimizer terminates, we check the status and print the
solution if successful. Finally, we free the {\tt IpoptProblem} and
the remaining memory, and return from {\tt main}.

\subsection{The Fortran Interface}

The Fortran interface is essentially a wrapper of the C interface
discussed in Section~\ref{sec.cinterface}.  The way to hook up \Ipopt\
in a Fortran program is very similar to how it is done for the C
interface, and the functions of the Fortran interface correspond
one-to-one to the those of the C and C++ interface, including their
arguments.  You can find an implementation of the example problem
(\ref{eq:ex_obj})--(\ref{eq:ex_bounds}) in {\tt
  \$IPOPTDIR/Ipopt/examples/hs071\_f}.

The only special things to consider are:
\begin{itemize}
\item The return value of the function {\tt IPCREATE} is of an {\tt
    INTEGER} type that must be large enough to capture a pointer
  on the particular machine.  This means, that you have to declare
  the ``handle'' for the IpoptProblem as {\tt INTEGER*8} if your
  program is compiled in 64-bit mode.  All other {\tt INTEGER}-type
  variables must be of the regular type.
\item For the call of {\tt IPSOLVE} (which is the function that is to
  be called to run \Ipopt), all arrays, including those for the dual
  variables, must be given (in contrast to the C interface).  The
  return value {\tt IERR} of this function indicates the outcome of
  the optimization (see the include file {\tt IpReturnCodes.inc} in
  the \Ipopt\ include directory).
\item The return {\tt IERR} value of the remaining functions has to be
  set to zero, unless there was a problem during execution of the
  function call.
\item The callback functions ({\tt EV\_*} in the example) include the
  arguments {\tt IDAT} and {\tt DAT}, which are {\tt INTEGER} and {\tt
    DOUBLE PRECISION} arrays that are passed unmodified between the
  main program calling {\tt IPSOLVE} and the evaluation subroutines
  {\tt EV\_*} (similarly to {\tt UserDataPtr} arguments in the C
  interface).  These arrays can be used to pass ``private'' data
  between the main program and the user-provided Fortran subroutines.

  The last argument of the {\tt EV\_*} subroutines, {\tt IERR}, is to
  be set to 0 by the user on return, unless there was a problem
  during the evaluation of the optimization problem
  function/derivative for the given point {\tt X} (then it should
  return a non-zero value).
\end{itemize}

\section{Special Features}
\subsection{Derivative Checker}\label{sec:deriv-checker}
When writing code for the evaluation of derivatives it is very easy to
make mistakes (much easier than writing it correctly the first time
:).  As a convenient feature, \Ipopt\ provides the option to run a
simple derivative checker, based on finite differences, before the
optimization is started.

To use the derivative checker, you need to use the option {\tt
  derivative\_test}.  By default, this option is set to {\tt none},
i.e., no finite difference test is performed,  It is set to {\tt
  first-order}, then the first derivatives of the objective function
and the constraints are verified, and for the setting {\tt
  second-order}, the second derivatives are tested as well.

The verification is done by a simple finite differences approximation,
where each component of the user-provided starting point is perturbed
one of the other.  The relative size of the perturbation is determined
by the option {\tt derivative\_test\_perturbation}.  The default value
($10^{-8}$, about the square root of the machine precision) is
probably fine in most cases, but if you believe that you see wrong
warnings, you might want to play with this parameter.  When the test is
performed, \Ipopt\ prints out a line for every partial derivative, for
which the user-provided derivative value deviates too much from the
finite difference approximation.  The relative tolerance for deciding
when a warning should be issued, is determined by the option {\tt
  derivative\_test\_tol}.  If you want to see the user-provided and
estimated derivative values with the relative deviation for each
single partial derivative, you can switch the {\tt
  derivative\_test\_print\_all} option to {\tt yes}.

A typical output is:

\begin{footnotesize}
\begin{verbatim}
Starting derivative checker.

* grad_f[          2] = -6.5159999999999991e+02    ~ -6.5559997134793468e+02  [ 6.101e-03]
* jac_g [    4,    4] =  0.0000000000000000e+00    ~  2.2160643690464592e-02  [ 2.216e-02]
* jac_g [    4,    5] =  1.3798494268463347e+01 v  ~  1.3776333629422766e+01  [ 1.609e-03]
* jac_g [    6,    7] =  1.4776333636790881e+01 v  ~  1.3776333629422766e+01  [ 7.259e-02]

Derivative checker detected 4 error(s).
\end{verbatim}
\end{footnotesize}

The start (``\verb|*|'') in the first column indicates that this line
corresponds to some partial derivative for which the error tolerance
was exceeded.  Next, we see which partial derivative is concerned in
this output line.  For example, in the first line, it is the second
component of the objective function gradient (or the third, if the
{\tt C\_STYLE} numbering is used, i.e., when counting of indices
starts with 0 instead of 1).  The first floating point number is the
value given by the user code, and the second number (after
``\verb|~|'') is the finite differences estimation.  Finally, the
number in square brackets is the relative difference between those two
numbers.

For constraints, the first index after {\tt jac\_g} is the number of
the constraint, and the second one corresponds to the variable number
(again, the choice of the numbering style matters).

Since also the sparsity structure of the constraint Jacobian has to be
provided by the user, it can be faulty as well.  For this, the ``{\tt
  v}'' after a user-provided derivative value indicates that this
component of the Jacobian is part of the user provided sparsity
structure.  If there is no ``{\tt v}'', it means that the user did not
include this partial derivative in the list of non-zero elements.  In
the above output, the partial derivative ``{\tt jac\_g[4,4]}'' is
non-zero (based on the finite difference approximation), but it is not
included in the list of non-zero elements (missing ``{\tt v}''), so
that the user probably made a mistake in the sparsity structure.  The
other two Jacobian entries are provided in the non-zero structure but
their values seem to be off.

For second derivatives, the output lines look like:

\begin{footnotesize}
\begin{verbatim}
*             obj_hess[    1,    1] =  1.8810000000000000e+03 v  ~  1.8820000036612328e+03  [ 5.314e-04]
*     3-th constr_hess[    2,    4] =  1.0000000000000000e+00 v  ~  0.0000000000000000e+00  [ 1.000e+00]
\end{verbatim}
\end{footnotesize}

There, the first line shows the deviation of the user-provided partial
second derivative in the Hessian for the objective function, and the
second line show an error in a partial derivative for the Hessian of
the third constraint (again, the numbering style matters).

Since the second derivatives are approximates by finite differences of
the first derivatives, you should first correct errors for the first
derivatives.  Also, since the finite difference approximations are
quite expensive, you should try to debug a small instance of your
problem if you can.  Finally, it is of course also a good idea to run
your code through some memory checker, such as {\tt valgrind} on Linux.

\subsection{Quasi-Newton Approximation of Second Derivatives}
\label{sec:quasiNewton}

\Ipopt\ has an option to approximate the Hessian of the Lagrangian by
a limited-memory quasi-Newton method (L-BFGS).  You can use this
feature using the {\tt hessian\_approximation} and {\tt
  limited\_memory...} options.  In this case, it is not necessary to
implement the Hessian computation method {\tt eval\_h} in {\tt TNLP}.
If you are using the C or Fortran interface, you still need to
implement these functions, but they should return {\tt false} or {\tt
  IERR=1}, respectively, and don't need to do anything else.

In general, when second derivatives can be computed with reasonable
computational effort, it is usually a good idea to use them, since
then \Ipopt\ normally converges in fewer iterations and is more
robust.  An exception here might be the case, where your optimization
problem has a dense Hessian or a large percentage of non-zero entries
in the Hessian, and then using the quasi-Newton approximation might be
better, even if it the number of iterations increases, since the
computation time per iteration might be significantly higher due to
the very large number of non-zero elements in the linear systems that
\Ipopt\ solves in order to compute the search direction, if exact
second derivatives are used.

Since the Hessian of the Lagrangian is zero for all variables that
appear only linearly in the objective and constraint functions, the
Hessian approximation should only take place in the space of all
nonlinear variables.  By default, it is assumed that all variables are
nonlinear, but you can tell \Ipopt\ explicitly which variables are
nonlinear, using the {\tt get\_number\_of\_nonlinear\_variables} and
{\tt get\_list\_of\_nonlinear\_variables} method of the {\tt TNLP}
class, see Section~\ref{sec:add_meth}.  (Those methods have been
implemented for the AMPL interface, so you would automatically only
approximate the Hessian in the space of the nonlinear variables, if
you are using the quasi-Newton option for AMPL models.)  Currently,
those two methods are not available through the C or Fortran
interface.

\section{\Ipopt\ Options}\label{sec:options}
Ipopt has many (maybe too many) options that can be adjusted for the
algorithm.  Options are all identified by a string name, and their
values can be of one of three types: Number (real), Integer, or
String. Number options are used for things like tolerances, integer
options are used for things like maximum number of iterations, and
string options are used for setting algorithm details, like the NLP
scaling method. Options can be set through code, through the AMPL
interface if you are using AMPL, or by creating a {\tt ipopt.opt}
file in the directory you are executing \Ipopt.

The {\tt ipopt.opt} file is read line by line and each line should
contain the option name, followed by whitespace, and then the
value. Comments can be included with the {\tt \#} symbol. Don't forget
to ensure you have a newline at the end of the file. For example,
\begin{verbatim}
# This is a comment

# Turn off the NLP scaling
nlp_scaling_method none

# Change the initial barrier parameter
mu_init 1e-2

# Set the max number of iterations
max_iter 500
\end{verbatim}
is a valid {\tt ipopt.opt} file.

Options can also be set in code. Have a look at the examples to see
how this is done. 

A subset of \Ipopt\ options are available through AMPL. To set options
through AMPL, use the internal AMPL command {\tt options}.  For
example, \\
{\tt options ipopt\_options "nlp\_scaling\_method=none mu\_init=1e-2
  max\_iter=500"} \\
is a valid options command in AMPL. The most important options are
referenced in Appendix~\ref{app.options_ref}. To see which options are
available through AMPL, you can run the AMPL solver executable with
the ``{\tt -=}'' flag from the command prompt.  To specify other
options when using AMPL, you can always create {\tt ipopt.opt}.  Note,
the {\tt ipopt.opt} file is given preference when setting options.
This way, you can easily override any options set in a particular
executable or AMPL model by specifying new values in {\tt ipopt.opt}.

For a list of the most important valid options, see the Appendix
\ref{app.options_ref}. You can print the documentation for all \Ipopt\
options by using the option \\

{\tt print\_options\_documentation yes} \\

and running \Ipopt\ (like the AMPL solver executable, for
instance). This will output all of the options documentation to the
console.

\section{\Ipopt\ Output}\label{sec:output}
This section describes the standard \Ipopt\ console output with the
default setting for {\tt print\_level}. The output is designed to
provide a quick summary of each iteration as \Ipopt\ solves the problem.

Before \Ipopt\ starts to solve the problem, it displays the problem
statistics (number of nonzero-elements in the matrices, number of
variables, etc.). Note that if you have fixed variables (both upper
and lower bounds are equal), \Ipopt\ may remove these variables from
the problem internally and not include them in the problem statistics.

Following the problem statistics, \Ipopt\ will begin to solve the
problem and you will see output resembling the following,
\begin{verbatim}
iter    objective    inf_pr   inf_du lg(mu)  ||d||  lg(rg) alpha_du alpha_pr  ls
   0  1.6109693e+01 1.12e+01 5.28e-01   0.0 0.00e+00    -  0.00e+00 0.00e+00   0
   1  1.8029749e+01 9.90e-01 6.62e+01   0.1 2.05e+00    -  2.14e-01 1.00e+00f  1
   2  1.8719906e+01 1.25e-02 9.04e+00  -2.2 5.94e-02   2.0 8.04e-01 1.00e+00h  1
\end{verbatim}
and the columns of output are defined as,
\begin{description}
\item[{\tt iter}:] The current iteration count. This includes regular
  iterations and iterations while in restoration phase. If the
  algorithm is in the restoration phase, the letter {\tt r'} will be
  appended to the iteration number.
\item[{\tt objective}:] The unscaled objective value at the current
  point. During the restoration phase, this value remains the unscaled
  objective value for the original problem.
\item[{\tt inf\_pr}:] The scaled primal infeasibility at the current
  point. During the restoration phase, this value is the primal
  infeasibility of the original problem at the current point.
\item[{\tt inf\_du}:] The scaled dual infeasibility at the current
  point. During the restoration phase, this is the value of the dual
  infeasibility for the restoration phase problem.
\item[{\tt lg(mu)}:] $\log_{10}$ of the value of the barrier parameter mu.
\item[{\tt ||d||}:] The infinity norm (max) of the primal step (for
  the original variables $x$ and the internal slack variables $s$).
  During the restoration phase, this value includes the values of
  additional variables, $p$ and $n$ (see Eq.~(30) in
  \cite{WaecBieg06:mp}).
\item[{\tt lg(rg)}:] $\log_{10}$ of the value of the regularization
  term for the Hessian of the Lagrangian in the augmented system.
\item[{\tt alpha\_du}:] The stepsize for the dual variables.
\item[{\tt alpha\_pr}:] The stepsize for the primal variables.
\item[{\tt ls}:] The number of backtracking line search steps.
\end{description}

When the algorithm terminates, \Ipopt\ will output a message to the
screen based on the return status of the call to {\tt Optimize}. The following
is a list of the possible return codes, their corresponding output message
to the console, and a brief description.
\begin{description}
\item[{\tt Solve\_Succeeded}:] $\;$ \\
  Console Message: {\tt EXIT: Optimal Solution Found.} \\
  This message indicates that \Ipopt\ found a (locally) optimal point
  within the desired tolerances.
\item[{\tt Solved\_To\_Acceptable\_Level}:]  $\;$ \\
  Console Message: {\tt EXIT: Solved To Acceptable Level.} \\
  This indicates that the algorithm did not converge to the
  ``desired'' tolerances, but that it was able to obtain a point
  satisfying the ``acceptable'' tolerance level as specified by {\tt
    acceptable-*} options. This may happen if the desired tolerances
  are too small for the current problem.
\item[{\tt Infeasible\_Problem\_Detected}:]  $\;$ \\
  Console Message: {\tt EXIT: Converged to a point of
    local infeasibility. Problem may be infeasible.} \\
  The restoration phase converged to a point that is a minimizer for
  the constraint violation (in the $\ell_1$-norm), but is not feasible
  for the original problem. This indicates that the problem may be
  infeasible (or at least that the algorithm is stuck at a locally
  infeasible point).  The returned point (the minimizer of the
  constraint violation) might help you to find which constraint is
  causing the problem.  If you believe that the NLP is feasible,
  it might help to start the optimization from a different point.
\item[{\tt Search\_Direction\_Becomes\_Too\_Small}:]  $\;$ \\
  Console Message: {\tt EXIT: Search Direction is becoming Too Small.} \\
  This indicates that \Ipopt\ is calculating very small step sizes and
  making very little progress.  This could happen if the problem has
  been solved to the best numerical accuracy possible given the
  current scaling.
\item[{\tt Diverging\_Iterates}:]  $\;$ \\
  Console Message: {\tt EXIT: Iterates divering; problem might be
    unbounded.} \\
  This message is printed if the max-norm of the iterates becomes
  larger than the value of the option {\tt diverging\_iterates\_tol}.
  This can happen if the problem is unbounded below and the iterates
  are diverging.
\item[{\tt User\_Requested\_Stop}:]  $\;$ \\
  Console Message: {\tt EXIT: Stopping optimization at current point
    as requested by user.} \\
  This message is printed if the user call-back method {\tt
    intermediate\_callback} returned {\tt false} (see
  Section~\ref{sec:add_meth}).
\item[{\tt Maximum\_Iterations\_Exceeded}:]  $\;$ \\
  Console Message: {\tt EXIT: Maximum Number of Iterations Exceeded.} \\
  This indicates that \Ipopt\ has exceeded the maximum number of
  iterations as specified by the option {\tt max\_iter}.
\item[{\tt Restoration\_Failed}:]  $\;$ \\
  Console Message: {\tt EXIT: Restoration Failed!} \\
  This indicates that the restoration phase failed to find a feasible
  point that was acceptable to the filter line search for the original
  problem. This could happen if the problem is highly degenerate, does
  not satisfy the constraint qualification, or if your NLP code
  provides incorrect derivative information.
\item[{\tt Error\_In\_Step\_Computation}:]  $\;$ \\
  Console Message: {\tt EXIT: Error in step computation (regularization becomes too large?)!} \\
  This messages is printed if \Ipopt\ is unable to compute a search
  direction, despite several attempts to modify the iteration matrix.
  Usually, the value of the regularization parameter then becomes too
  large.  One situation where this can happen is when values in the
  Hessian are invalid ({\tt NaN} or {\tt Inf}).  You can check whether this
  is true by using the {\tt check\_derivatives\_for\_naninf} option.
\item[{\tt Invalid\_Option}:]  $\;$ \\
  Console Message: (details about the particular error
  will be output to the console) \\
  This indicates that there was some problem specifying the options.
  See the specific message for details.
\item[{\tt Not\_Enough\_Degrees\_Of\_Freedom}:]  $\;$ \\
  Console Message: {\tt EXIT: Problem has too few degrees of freedom.} \\
  This indicates that your problem, as specified, has too few degrees
  of freedom. This can happen if you have too many equality
  constraints, or if you fix too many variables (\Ipopt\ removes fixed
  variables).
\item[{\tt Invalid\_Problem\_Definition}:]  $\;$ \\
  Console Message: (no console message, this is a return code for the
  C and Fortran interfaces only.) \\
  This indicates that there was an exception of some sort when
  building the {\tt IpoptProblem} structure in the C or Fortran
  interface. Likely there is an error in your model or the {\tt main}
  routine.
\item[{\tt Unrecoverable\_Exception}:]  $\;$ \\
  Console Message: (details about the particular error
  will be output to the console) \\
  This indicates that \Ipopt\ has thrown an exception that does not
  have an internal return code. See the specific message for details.
\item[{\tt NonIpopt\_Exception\_Thrown}:]  $\;$ \\
  Console Message: {\tt Unknown Exception caught in Ipopt} \\
  An unknown exception was caught in \Ipopt. This exception could have
  originated from your model or any linked in third party code.
\item[{\tt Insufficient\_Memory}:]  $\;$ \\
  Console Message: {\tt EXIT: Not enough memory.} \\
  An error occurred while trying to allocate memory. The problem may
  be too large for your current memory and swap configuration.
\item[{\tt Internal\_Error}:]  $\;$ \\
  Console Message: {\tt EXIT: INTERNAL ERROR: Unknown SolverReturn
    value - Notify IPOPT Authors.} \\
  An unknown internal error has occurred. Please notify the authors of
  \Ipopt\ via the mailing list.

\end{description}

\appendix
\newpage
\section{Triplet Format for Sparse Matrices}\label{app.triplet}
\Ipopt\ was designed for optimizing large sparse nonlinear programs.
Because of problem sparsity, the required matrices (like the
constraints Jacobian or Lagrangian Hessian) are not stored as dense
matrices, but rather in a sparse matrix format. For the tutorials in
this document, we use the triplet format.  Consider the matrix
\begin{equation}
\label{eqn.ex_matrix}
\left[
\begin{array}{ccccccc}
1.1     & 0             & 0             & 0             & 0             & 0             & 0.5 \\
0       & 1.9   & 0             & 0             & 0             & 0             & 0.5 \\
0       & 0             & 2.6   & 0             & 0             & 0             & 0.5 \\
0       & 0             & 7.8   & 0.6   & 0             & 0             & 0    \\
0       & 0             & 0             & 1.5   & 2.7   & 0             & 0     \\
1.6     & 0             & 0             & 0             & 0.4   & 0             & 0     \\
0       & 0             & 0             & 0             & 0             & 0.9   & 1.7 \\
\end{array}
\right]
\end{equation}

A standard dense matrix representation would need to store $7 \cdot
7{=} 49$ floating point numbers, where many entries would be zero. In
triplet format, however, only the nonzero entries are stored. The
triplet format records the row number, the column number, and the
value of all nonzero entries in the matrix. For the matrix above, this
means storing $14$ integers for the rows, $14$ integers for the
columns, and $14$ floating point numbers for the values. While this
does not seem like a huge space savings over the $49$ floating point
numbers stored in the dense representation, for larger matrices, the
space savings are very dramatic\footnote{For an $n \times n$ matrix,
the dense representation grows with the the square of $n$, while the
sparse representation grows linearly in the number of nonzeros.}.

The option {\tt index\_style} in {\tt get\_nlp\_info} tells \Ipopt\ if
you prefer to use C style indexing (0-based, i.e., starting the
counting at 0) for the row and column indices or Fortran style
(1-based). Tables \ref{tab.fortran_triplet} and \ref{tab.c_triplet}
below show the triplet format for both indexing styles, using the
example matrix (\ref{eqn.ex_matrix}).

\begin{footnotesize}
\begin{table}[ht]%[!h]
\begin{center}
\begin{tabular}{c c c}
row     		&       col     	&       value 			    \\
\hline
{\tt iRow[0] = 1}       &       {\tt jCol[0] = 1}       & {\tt values[0] = 1.1}     \\
{\tt iRow[1] = 1}       &       {\tt jCol[1] = 7}       & {\tt values[1] = 0.5}     \\
{\tt iRow[2] = 2}       &       {\tt jCol[2] = 2}       & {\tt values[2] = 1.9}     \\
{\tt iRow[3] = 2}       &       {\tt jCol[3] = 7}       & {\tt values[3] = 0.5}     \\
{\tt iRow[4] = 3}       &       {\tt jCol[4] = 3}       & {\tt values[4] = 2.6}     \\
{\tt iRow[5] = 3}       &       {\tt jCol[5] = 7}       & {\tt values[5] = 0.5}     \\
{\tt iRow[6] = 4}       &       {\tt jCol[6] = 3}       & {\tt values[6] = 7.8}     \\
{\tt iRow[7] = 4}       &       {\tt jCol[7] = 4}       & {\tt values[7] = 0.6}     \\
{\tt iRow[8] = 5}       &       {\tt jCol[8] = 4}       & {\tt values[8] = 1.5}     \\
{\tt iRow[9] = 5}       &       {\tt jCol[9] = 5}       & {\tt values[9] = 2.7}     \\
{\tt iRow[10] = 6}      &       {\tt jCol[10] = 1}      & {\tt values[10] = 1.6}     \\
{\tt iRow[11] = 6}      &       {\tt jCol[11] = 5}      & {\tt values[11] = 0.4}     \\
{\tt iRow[12] = 7}      &       {\tt jCol[12] = 6}      & {\tt values[12] = 0.9}     \\
{\tt iRow[13] = 7}      &       {\tt jCol[13] = 7}      & {\tt values[13] = 1.7}
\end{tabular}
\caption{Triplet Format of Matrix (\ref{eqn.ex_matrix}) 
with {\tt index\_style=FORTRAN\_STYLE}}
\label{tab.fortran_triplet}
\end{center}
\end{table}
\begin{table}[ht]%[!h]
\begin{center}
\begin{tabular}{c c c}
row     		&       col     	&       value 			    \\
\hline
{\tt iRow[0] = 0}       &       {\tt jCol[0] = 0}       & {\tt values[0] = 1.1}     \\
{\tt iRow[1] = 0}       &       {\tt jCol[1] = 6}       & {\tt values[1] = 0.5}     \\
{\tt iRow[2] = 1}       &       {\tt jCol[2] = 1}       & {\tt values[2] = 1.9}     \\
{\tt iRow[3] = 1}       &       {\tt jCol[3] = 6}       & {\tt values[3] = 0.5}     \\
{\tt iRow[4] = 2}       &       {\tt jCol[4] = 2}       & {\tt values[4] = 2.6}     \\
{\tt iRow[5] = 2}       &       {\tt jCol[5] = 6}       & {\tt values[5] = 0.5}     \\
{\tt iRow[6] = 3}       &       {\tt jCol[6] = 2}       & {\tt values[6] = 7.8}     \\
{\tt iRow[7] = 3}       &       {\tt jCol[7] = 3}       & {\tt values[7] = 0.6}     \\
{\tt iRow[8] = 4}       &       {\tt jCol[8] = 3}       & {\tt values[8] = 1.5}     \\
{\tt iRow[9] = 4}       &       {\tt jCol[9] = 4}       & {\tt values[9] = 2.7}     \\
{\tt iRow[10] = 5}      &       {\tt jCol[10] = 0}      & {\tt values[10] = 1.6}     \\
{\tt iRow[11] = 5}      &       {\tt jCol[11] = 4}      & {\tt values[11] = 0.4}     \\
{\tt iRow[12] = 6}      &       {\tt jCol[12] = 5}      & {\tt values[12] = 0.9}     \\
{\tt iRow[13] = 6}      &       {\tt jCol[13] = 6}      & {\tt values[13] = 1.7}
\end{tabular}
\caption{Triplet Format of Matrix (\ref{eqn.ex_matrix}) 
with {\tt index\_style=C\_STYLE}}
\label{tab.c_triplet}
\end{center}
\end{table}
\end{footnotesize}
The individual elements of the matrix can be listed in any order, and
if there are multiple items for the same nonzero position, the values
provided for those positions are added.

The Hessian of the Lagrangian is a symmetric matrix. In the case of a
symmetric matrix, you only need to specify the lower left triangular
part (or, alternatively, the upper right triangular part). For
example, given the matrix,
\begin{equation}
\label{eqn.ex_sym_matrix}
\left[
\begin{array}{ccccccc}
1.0	& 0	& 3.0	& 0	& 2.0 	\\
0	& 1.1	& 0	& 0	& 5.0	\\
3.0	& 0	& 1.2	& 6.0	& 0	\\
0	& 0	& 6.0	& 1.3	& 9.0	\\
2.0	& 5.0	& 0	& 9.0	& 1.4
\end{array}
\right]
\end{equation}
the triplet format is shown in Tables \ref{tab.sym_fortran_triplet}
and \ref{tab.sym_c_triplet}.

\begin{footnotesize}
\begin{table}[ht]%[!h]
\begin{center}
\caption{Triplet Format of Matrix (\ref{eqn.ex_matrix}) 
with {\tt index\_style=FORTRAN\_STYLE}}
\label{tab.sym_fortran_triplet}
\begin{tabular}{c c c}
row     		&       col     	&       value 			    \\
\hline
{\tt iRow[0] = 1}       &       {\tt jCol[0] = 1}       & {\tt values[0] = 1.0}     \\
{\tt iRow[1] = 2}       &       {\tt jCol[1] = 1}       & {\tt values[1] = 1.1}     \\
{\tt iRow[2] = 3}       &       {\tt jCol[2] = 1}       & {\tt values[2] = 3.0}     \\
{\tt iRow[3] = 3}       &       {\tt jCol[3] = 3}       & {\tt values[3] = 1.2}     \\
{\tt iRow[4] = 4}       &       {\tt jCol[4] = 3}       & {\tt values[4] = 6.0}     \\
{\tt iRow[5] = 4}       &       {\tt jCol[5] = 4}       & {\tt values[5] = 1.3}     \\
{\tt iRow[6] = 5}       &       {\tt jCol[6] = 1}       & {\tt values[6] = 2.0}     \\
{\tt iRow[7] = 5}       &       {\tt jCol[7] = 2}       & {\tt values[7] = 5.0}     \\
{\tt iRow[8] = 5}       &       {\tt jCol[8] = 4}       & {\tt values[8] = 9.0}     \\
{\tt iRow[9] = 5}       &       {\tt jCol[9] = 5}       & {\tt values[9] = 1.4}
\end{tabular}
\end{center}
\end{table}
\begin{table}[ht]%[!h]
\begin{center}
\caption{Triplet Format of Matrix (\ref{eqn.ex_matrix}) 
with {\tt index\_style=C\_STYLE}}
\label{tab.sym_c_triplet}
\begin{tabular}{c c c}
row     		&       col     	&       value 			    \\
\hline
{\tt iRow[0] = 0}       &       {\tt jCol[0] = 0}       & {\tt values[0] = 1.0}     \\
{\tt iRow[1] = 1}       &       {\tt jCol[1] = 0}       & {\tt values[1] = 1.1}     \\
{\tt iRow[2] = 2}       &       {\tt jCol[2] = 0}       & {\tt values[2] = 3.0}     \\
{\tt iRow[3] = 2}       &       {\tt jCol[3] = 2}       & {\tt values[3] = 1.2}     \\
{\tt iRow[4] = 3}       &       {\tt jCol[4] = 2}       & {\tt values[4] = 6.0}     \\
{\tt iRow[5] = 3}       &       {\tt jCol[5] = 3}       & {\tt values[5] = 1.3}     \\
{\tt iRow[6] = 4}       &       {\tt jCol[6] = 0}       & {\tt values[6] = 2.0}     \\
{\tt iRow[7] = 4}       &       {\tt jCol[7] = 1}       & {\tt values[7] = 5.0}     \\
{\tt iRow[8] = 4}       &       {\tt jCol[8] = 3}       & {\tt values[8] = 9.0}     \\
{\tt iRow[9] = 4}       &       {\tt jCol[9] = 4}       & {\tt values[9] = 1.4}
\end{tabular}
\end{center}
\end{table}
\end{footnotesize}
\newpage
\section{The Smart Pointer Implementation: {\tt SmartPtr<T>}} \label{app.smart_ptr}

The {\tt SmartPtr} class is described in {\tt IpSmartPtr.hpp}. It is a
template class that takes care of deleting objects for us so we need
not be concerned about memory leaks. Instead of pointing to an object
with a raw C++ pointer (e.g. {\tt HS071\_NLP*}), we use a {\tt
  SmartPtr}.  Every time a {\tt SmartPtr} is set to reference an
object, it increments a counter in that object (see the {\tt
  ReferencedObject} base class if you are interested). If a {\tt
  SmartPtr} is done with the object, either by leaving scope or being
set to point to another object, the counter is decremented. When the
count of the object goes to zero, the object is automatically deleted.
{\tt SmartPtr}'s are very simple, just use them as you would a
standard pointer.

It is very important to use {\tt SmartPtr}'s instead of raw pointers
when passing objects to \Ipopt. Internally, \Ipopt\ uses smart
pointers for referencing objects. If you use a raw pointer in your
executable, the object's counter will NOT get incremented. Then, when
\Ipopt\ uses smart pointers inside its own code, the counter will get
incremented. However, before \Ipopt\ returns control to your code, it
will decrement as many times as it incremented earlier, and the
counter will return to zero. Therefore, \Ipopt\ will delete the
object. When control returns to you, you now have a raw pointer that
points to a deleted object.

This might sound difficult to anyone not familiar with the use of
smart pointers, but just follow one simple rule; always use a SmartPtr
when creating or passing an \Ipopt\ object.

\newpage
\section{Options Reference} \label{app.options_ref}
Options can be set using {\tt ipopt.opt}, through your own code, or through the 
AMPL {\tt ipopt\_options} command. See Section \ref{sec:options} for an explanation of
how to use these commands.
Shown here is a list of the most important options for Ipopt. To view
the full list of options, run the ipopt executable with the option,
\begin{verbatim}
print_options_documentation yes
\end{verbatim}

The most common options are:


\paragraph{print\_level:} Output verbosity level. $\;$ \\
 Sets the default verbosity level for console
output. The larger this value the more detailed
is the output. The valid range for this integer option is
$0 \le {\tt print\_level } \le 11$
and its default value is $4$.


\paragraph{print\_user\_options:} Print all options set by the user. $\;$ \\
 If selected, the algorithm will print the list of
all options set by the user including their
values and whether they have been used.
The default value for this string option is "no".
\\ 
Possible values:
\begin{itemize}
   \item no: don't print options
   \item yes: print options
\end{itemize}

\paragraph{print\_options\_documentation:} Switch to print all algorithmic options. $\;$ \\
 If selected, the algorithm will print the list of
all available algorithmic options with some
documentation before solving the optimization
problem.
The default value for this string option is "no".
\\ 
Possible values:
\begin{itemize}
   \item no: don't print list
   \item yes: print list
\end{itemize}

\paragraph{output\_file:} File name of desired output file (leave unset for no file output). $\;$ \\
 NOTE: This option only works when read from the
ipopt.opt options file! An output file with this
name will be written (leave unset for no file
output).  The verbosity level is by default set
to "print\_level", but can be overridden with
"file\_print\_level".  The file name is changed
to use only small letters.
The default value for this string option is "".
\\ 
Possible values:
\begin{itemize}
   \item *: Any acceptable standard file name
\end{itemize}

\paragraph{file\_print\_level:} Verbosity level for output file. $\;$ \\
 NOTE: This option only works when read from the
ipopt.opt options file! Determines the verbosity
level for the file specified by "output\_file". 
By default it is the same as "print\_level". The valid range for this integer option is
$0 \le {\tt file\_print\_level } \le 11$
and its default value is $4$.


\paragraph{tol:} Desired convergence tolerance (relative). $\;$ \\
 Determines the convergence tolerance for the
algorithm.  The algorithm terminates
successfully, if the (scaled) NLP error becomes
smaller than this value, and if the (absolute)
criteria according to "dual\_inf\_tol",
"primal\_inf\_tol", and "cmpl\_inf\_tol" are met.
 (This is epsilon\_tol in Eqn. (6) in
implementation paper).  See also
"acceptable\_tol" as a second termination
criterion.  Note, some other algorithmic features
also use this quantity to determine thresholds
etc. The valid range for this real option is 
$0 <  {\tt tol } <  {\tt +inf}$
and its default value is $1 \cdot 10^{-08}$.


\paragraph{max\_iter:} Maximum number of iterations. $\;$ \\
 The algorithm terminates with an error message if
the number of iterations exceeded this number. The valid range for this integer option is
$0 \le {\tt max\_iter } <  {\tt +inf}$
and its default value is $3000$.


\paragraph{compl\_inf\_tol:} Desired threshold for the complementarity conditions. $\;$ \\
 Absolute tolerance on the complementarity.
Successful termination requires that the max-norm
of the (unscaled) complementarity is less than
this threshold. The valid range for this real option is 
$0 <  {\tt compl\_inf\_tol } <  {\tt +inf}$
and its default value is $0.0001$.


\paragraph{dual\_inf\_tol:} Desired threshold for the dual infeasibility. $\;$ \\
 Absolute tolerance on the dual infeasibility.
Successful termination requires that the max-norm
of the (unscaled) dual infeasibility is less than
this threshold. The valid range for this real option is 
$0 <  {\tt dual\_inf\_tol } <  {\tt +inf}$
and its default value is $0.0001$.


\paragraph{constr\_viol\_tol:} Desired threshold for the constraint violation. $\;$ \\
 Absolute tolerance on the constraint violation.
Successful termination requires that the max-norm
of the (unscaled) constraint violation is less
than this threshold. The valid range for this real option is 
$0 <  {\tt constr\_viol\_tol } <  {\tt +inf}$
and its default value is $0.0001$.


\paragraph{acceptable\_tol:} "Acceptable" convergence tolerance (relative). $\;$ \\
 Determines which (scaled) overall optimality
error is considered to be "acceptable." There are
two levels of termination criteria.  If the usual
"desired" tolerances (see tol, dual\_inf\_tol
etc) are satisfied at an iteration, the algorithm
immediately terminates with a success message. 
On the other hand, if the algorithm encounters
"acceptable\_iter" many iterations in a row that
are considered "acceptable", it will terminate
before the desired convergence tolerance is met.
This is useful in cases where the algorithm might
not be able to achieve the "desired" level of
accuracy. The valid range for this real option is 
$0 <  {\tt acceptable\_tol } <  {\tt +inf}$
and its default value is $1 \cdot 10^{-06}$.


\paragraph{acceptable\_compl\_inf\_tol:} "Acceptance" threshold for the complementarity conditions. $\;$ \\
 Absolute tolerance on the complementarity.
"Acceptable" termination requires that the
max-norm of the (unscaled) complementarity is
less than this threshold; see also
acceptable\_tol. The valid range for this real option is 
$0 <  {\tt acceptable\_compl\_inf\_tol } <  {\tt +inf}$
and its default value is $0.01$.


\paragraph{acceptable\_constr\_viol\_tol:} "Acceptance" threshold for the constraint violation. $\;$ \\
 Absolute tolerance on the constraint violation.
"Acceptable" termination requires that the
max-norm of the (unscaled) constraint violation
is less than this threshold; see also
acceptable\_tol. The valid range for this real option is 
$0 <  {\tt acceptable\_constr\_viol\_tol } <  {\tt +inf}$
and its default value is $0.01$.


\paragraph{acceptable\_dual\_inf\_tol:} "Acceptance" threshold for the dual infeasibility. $\;$ \\
 Absolute tolerance on the dual infeasibility.
"Acceptable" termination requires that the
(max-norm of the unscaled) dual infeasibility is
less than this threshold; see also
acceptable\_tol. The valid range for this real option is 
$0 <  {\tt acceptable\_dual\_inf\_tol } <  {\tt +inf}$
and its default value is $0.01$.


\paragraph{diverging\_iterates\_tol:} Threshold for maximal value of primal iterates. $\;$ \\
 If any component of the primal iterates exceeded
this value (in absolute terms), the optimization
is aborted with the exit message that the
iterates seem to be diverging. The valid range for this real option is 
$0 <  {\tt diverging\_iterates\_tol } <  {\tt +inf}$
and its default value is $1 \cdot 10^{+20}$.


\paragraph{barrier\_tol\_factor:} Factor for mu in barrier stop test. $\;$ \\
 The convergence tolerance for each barrier
problem in the monotone mode is the value of the
barrier parameter times "barrier\_tol\_factor".
This option is also used in the adaptive mu
strategy during the monotone mode. (This is
kappa\_epsilon in implementation paper). The valid range for this real option is 
$0 <  {\tt barrier\_tol\_factor } <  {\tt +inf}$
and its default value is $10$.


\paragraph{obj\_scaling\_factor:} Scaling factor for the objective function. $\;$ \\
 This option sets a scaling factor for the
objective function. The scaling is seen
internally by Ipopt but the unscaled objective is
reported in the console output. If additional
scaling parameters are computed (e.g.
user-scaling or gradient-based), both factors are
multiplied. If this value is chosen to be
negative, Ipopt will maximize the objective
function instead of minimizing it. The valid range for this real option is 
${\tt -inf} <  {\tt obj\_scaling\_factor } <  {\tt +inf}$
and its default value is $1$.


\paragraph{nlp\_scaling\_method:} Select the technique used for scaling the NLP. $\;$ \\
 Selects the technique used for scaling the
problem internally before it is solved. For
user-scaling, the parameters come from the NLP.
If you are using AMPL, they can be specified
through suffixes ("scaling\_factor")
The default value for this string option is "gradient-based".
\\ 
Possible values:
\begin{itemize}
   \item none: no problem scaling will be performed
   \item user-scaling: scaling parameters will come from the user
   \item gradient-based: scale the problem so the maximum gradient at
the starting point is scaling\_max\_gradient
\end{itemize}

\paragraph{nlp\_scaling\_max\_gradient:} Maximum gradient after NLP scaling. $\;$ \\
 This is the gradient scaling cut-off. If the
maximum gradient is above this value, then
gradient based scaling will be performed. Scaling
parameters are calculated to scale the maximum
gradient back to this value. (This is g\_max in
Section 3.8 of the implementation paper.) Note:
This option is only used if
"nlp\_scaling\_method" is chosen as
"gradient-based". The valid range for this real option is 
$0 <  {\tt nlp\_scaling\_max\_gradient } <  {\tt +inf}$
and its default value is $100$.


\paragraph{bound\_relax\_factor:} Factor for initial relaxation of the bounds. $\;$ \\
 Before start of the optimization, the bounds
given by the user are relaxed.  This option sets
the factor for this relaxation.  If it is set to
zero, then then bounds relaxation is disabled.
(See Eqn.(35) in implementation paper.) The valid range for this real option is 
$0 \le {\tt bound\_relax\_factor } <  {\tt +inf}$
and its default value is $1 \cdot 10^{-08}$.


\paragraph{honor\_original\_bounds:} Indicates whether final points should be projected into original bounds. $\;$ \\
 Ipopt might relax the bounds during the
optimization (see, e.g., option
"bound\_relax\_factor").  This option determines
whether the final point should be projected back
into the user-provide original bounds after the
optimization.
The default value for this string option is "yes".
\\ 
Possible values:
\begin{itemize}
   \item no: Leave final point unchanged
   \item yes: Project final point back into original bounds
\end{itemize}

\paragraph{check\_derivatives\_for\_naninf:} Indicates whether it is desired to check for Nan/Inf in derivative matrices $\;$ \\
 Activating this option will cause an error if an
invalid number is detected in the constraint
Jacobians or the Lagrangian Hessian.  If this is
not activated, the test is skipped, and the
algorithm might proceed with invalid numbers and
fail.
The default value for this string option is "no".
\\ 
Possible values:
\begin{itemize}
   \item no: Don't check (faster).
   \item yes: Check Jacobians and Hessian for Nan and Inf.
\end{itemize}

\paragraph{mu\_strategy:} Update strategy for barrier parameter. $\;$ \\
 Determines which barrier parameter update
strategy is to be used.
The default value for this string option is "monotone".
\\ 
Possible values:
\begin{itemize}
   \item monotone: use the monotone (Fiacco-McCormick) strategy
   \item adaptive: use the adaptive update strategy
\end{itemize}

\paragraph{mu\_oracle:} Oracle for a new barrier parameter in the adaptive strategy. $\;$ \\
 Determines how a new barrier parameter is
computed in each "free-mode" iteration of the
adaptive barrier parameter strategy. (Only
considered if "adaptive" is selected for option
"mu\_strategy").
The default value for this string option is "quality-function".
\\ 
Possible values:
\begin{itemize}
   \item probing: Mehrotra's probing heuristic
   \item loqo: LOQO's centrality rule
   \item quality-function: minimize a quality function
\end{itemize}

\paragraph{quality\_function\_max\_section\_steps:} Maximum number of search steps during direct search procedure determining the optimal centering parameter. $\;$ \\
 The golden section search is performed for the
quality function based mu oracle. (Only used if
option "mu\_oracle" is set to "quality-function".) The valid range for this integer option is
$0 \le {\tt quality\_function\_max\_section\_steps } <  {\tt +inf}$
and its default value is $8$.


\paragraph{fixed\_mu\_oracle:} Oracle for the barrier parameter when switching to fixed mode. $\;$ \\
 Determines how the first value of the barrier
parameter should be computed when switching to
the "monotone mode" in the adaptive strategy.
(Only considered if "adaptive" is selected for
option "mu\_strategy".)
The default value for this string option is "average\_compl".
\\ 
Possible values:
\begin{itemize}
   \item probing: Mehrotra's probing heuristic
   \item loqo: LOQO's centrality rule
   \item quality-function: minimize a quality function
   \item average\_compl: base on current average complementarity
\end{itemize}

\paragraph{mu\_init:} Initial value for the barrier parameter. $\;$ \\
 This option determines the initial value for the
barrier parameter (mu).  It is only relevant in
the monotone, Fiacco-McCormick version of the
algorithm. (i.e., if "mu\_strategy" is chosen as
"monotone") The valid range for this real option is 
$0 <  {\tt mu\_init } <  {\tt +inf}$
and its default value is $0.1$.


\paragraph{mu\_max\_fact:} Factor for initialization of maximum value for barrier parameter. $\;$ \\
 This option determines the upper bound on the
barrier parameter.  This upper bound is computed
as the average complementarity at the initial
point times the value of this option. (Only used
if option "mu\_strategy" is chosen as "adaptive".) The valid range for this real option is 
$0 <  {\tt mu\_max\_fact } <  {\tt +inf}$
and its default value is $1000$.


\paragraph{mu\_max:} Maximum value for barrier parameter. $\;$ \\
 This option specifies an upper bound on the
barrier parameter in the adaptive mu selection
mode.  If this option is set, it overwrites the
effect of mu\_max\_fact. (Only used if option
"mu\_strategy" is chosen as "adaptive".) The valid range for this real option is 
$0 <  {\tt mu\_max } <  {\tt +inf}$
and its default value is $100000$.


\paragraph{mu\_min:} Minimum value for barrier parameter. $\;$ \\
 This option specifies the lower bound on the
barrier parameter in the adaptive mu selection
mode. By default, it is set to
min("tol","compl\_inf\_tol")/("barrier\_tol\_fact-
or"+1), which should be a reasonable value. (Only
used if option "mu\_strategy" is chosen as
"adaptive".) The valid range for this real option is 
$0 <  {\tt mu\_min } <  {\tt +inf}$
and its default value is $1 \cdot 10^{-09}$.


\paragraph{mu\_linear\_decrease\_factor:} Determines linear decrease rate of barrier parameter. $\;$ \\
 For the Fiacco-McCormick update procedure the new
barrier parameter mu is obtained by taking the
minimum of mu*"mu\_linear\_decrease\_factor" and
mu\^"superlinear\_decrease\_power".  (This is
kappa\_mu in implementation paper.) This option
is also used in the adaptive mu strategy during
the monotone mode. The valid range for this real option is 
$0 <  {\tt mu\_linear\_decrease\_factor } <  1$
and its default value is $0.2$.


\paragraph{mu\_superlinear\_decrease\_power:} Determines superlinear decrease rate of barrier parameter. $\;$ \\
 For the Fiacco-McCormick update procedure the new
barrier parameter mu is obtained by taking the
minimum of mu*"mu\_linear\_decrease\_factor" and
mu\^"superlinear\_decrease\_power".  (This is
theta\_mu in implementation paper.) This option
is also used in the adaptive mu strategy during
the monotone mode. The valid range for this real option is 
$1 <  {\tt mu\_superlinear\_decrease\_power } <  2$
and its default value is $1.5$.


\paragraph{bound\_frac:} Desired minimum relative distance from the initial point to bound. $\;$ \\
 Determines how much the initial point might have
to be modified in order to be sufficiently inside
the bounds (together with "bound\_push").  (This
is kappa\_2 in Section 3.6 of implementation
paper.) The valid range for this real option is 
$0 <  {\tt bound\_frac } \le 0.5$
and its default value is $0.01$.


\paragraph{bound\_push:} Desired minimum absolute distance from the initial point to bound. $\;$ \\
 Determines how much the initial point might have
to be modified in order to be sufficiently inside
the bounds (together with "bound\_frac").  (This
is kappa\_1 in Section 3.6 of implementation
paper.) The valid range for this real option is 
$0 <  {\tt bound\_push } <  {\tt +inf}$
and its default value is $0.01$.


\paragraph{bound\_mult\_init\_val:} Initial value for the bound multipliers. $\;$ \\
 All dual variables corresponding to bound
constraints are initialized to this value. The valid range for this real option is 
$0 <  {\tt bound\_mult\_init\_val } <  {\tt +inf}$
and its default value is $1$.


\paragraph{constr\_mult\_init\_max:} Maximum allowed least-square guess of constraint multipliers. $\;$ \\
 Determines how large the initial least-square
guesses of the constraint multipliers are allowed
to be (in max-norm). If the guess is larger than
this value, it is discarded and all constraint
multipliers are set to zero.  This options is
also used when initializing the restoration
phase. By default,
"resto.constr\_mult\_init\_max" (the one used in
RestoIterateInitializer) is set to zero. The valid range for this real option is 
$0 \le {\tt constr\_mult\_init\_max } <  {\tt +inf}$
and its default value is $1000$.


\paragraph{bound\_mult\_init\_val:} Initial value for the bound multipliers. $\;$ \\
 All dual variables corresponding to bound
constraints are initialized to this value. The valid range for this real option is 
$0 <  {\tt bound\_mult\_init\_val } <  {\tt +inf}$
and its default value is $1$.


\paragraph{warm\_start\_init\_point:} Warm-start for initial point $\;$ \\
 Indicates whether this optimization should use a
warm start initialization, where values of primal
and dual variables are given (e.g., from a
previous optimization of a related problem.)
The default value for this string option is "no".
\\ 
Possible values:
\begin{itemize}
   \item no: do not use the warm start initialization
   \item yes: use the warm start initialization
\end{itemize}

\paragraph{warm\_start\_bound\_push:} same as bound\_push for the regular initializer. $\;$ \\
 The valid range for this real option is 
$0 <  {\tt warm\_start\_bound\_push } <  {\tt +inf}$
and its default value is $0.001$.


\paragraph{warm\_start\_bound\_frac:} same as bound\_frac for the regular initializer. $\;$ \\
 The valid range for this real option is 
$0 <  {\tt warm\_start\_bound\_frac } \le 0.5$
and its default value is $0.001$.


\paragraph{warm\_start\_mult\_bound\_push:} same as mult\_bound\_push for the regular initializer. $\;$ \\
 The valid range for this real option is 
$0 <  {\tt warm\_start\_mult\_bound\_push } <  {\tt +inf}$
and its default value is $0.001$.


\paragraph{warm\_start\_mult\_init\_max:} Maximum initial value for the equality multipliers. $\;$ \\
 The valid range for this real option is 
${\tt -inf} <  {\tt warm\_start\_mult\_init\_max } <  {\tt +inf}$
and its default value is $1 \cdot 10^{+06}$.


\paragraph{alpha\_for\_y:} Method to determine the step size for constraint multipliers. $\;$ \\
 This option determines how the step size
(alpha\_y) will be calculated when updating the
constraint multipliers.
The default value for this string option is "primal".
\\ 
Possible values:
\begin{itemize}
   \item primal: use primal step size
   \item bound\_mult: use step size for the bound multipliers (good
for LPs)
   \item min: use the min of primal and bound multipliers
   \item max: use the max of primal and bound multipliers
   \item full: take a full step of size one
   \item min\_dual\_infeas: choose step size minimizing new dual
infeasibility
   \item safe\_min\_dual\_infeas: like "min\_dual\_infeas", but safeguarded by
"min" and "max"
\end{itemize}

\paragraph{recalc\_y:} Tells the algorithm to recalculate the equality and inequality multipliers as least square estimates. $\;$ \\
 This asks the algorithm to recompute the
multipliers, whenever the current infeasibility
is less than recalc\_y\_feas\_tol. Choosing yes
might be helpful in the quasi-Newton option. 
However, each recalculation requires an extra
factorization of the linear system.  If a limited
memory quasi-Newton option is chosen, this is
used by default.
The default value for this string option is "no".
\\ 
Possible values:
\begin{itemize}
   \item no: use the Newton step to update the multipliers
   \item yes: use least-square multiplier estimates
\end{itemize}

\paragraph{recalc\_y\_feas\_tol:} Feasibility threshold for recomputation of multipliers. $\;$ \\
 If recalc\_y is chosen and the current
infeasibility is less than this value, then the
multipliers are recomputed. The valid range for this real option is 
$0 <  {\tt recalc\_y\_feas\_tol } <  {\tt +inf}$
and its default value is $1 \cdot 10^{-06}$.


\paragraph{max\_soc:} Maximum number of second order correction trial steps at each iteration. $\;$ \\
 Choosing 0 disables the second order corrections.
(This is p\^{max} of Step A-5.9 of Algorithm A in
implementation paper.) The valid range for this integer option is
$0 \le {\tt max\_soc } <  {\tt +inf}$
and its default value is $4$.


\paragraph{watchdog\_shortened\_iter\_trigger:} Number of shortened iterations that trigger the watchdog. $\;$ \\
 If the number of successive iterations in which
the backtracking line search did not accept the
first trial point exceeds this number, the
watchdog procedure is activated.  Choosing "0"
here disables the watchdog procedure. The valid range for this integer option is
$0 \le {\tt watchdog\_shortened\_iter\_trigger } <  {\tt +inf}$
and its default value is $10$.


\paragraph{watchdog\_trial\_iter\_max:} Maximum number of watchdog iterations. $\;$ \\
 This option determines the number of trial
iterations allowed before the watchdog procedure
is aborted and the algorithm returns to the
stored point. The valid range for this integer option is
$1 \le {\tt watchdog\_trial\_iter\_max } <  {\tt +inf}$
and its default value is $3$.


\paragraph{expect\_infeasible\_problem:} Enable heuristics to quickly detect an infeasible problem. $\;$ \\
 This options is meant to activate heuristics that
may speed up the infeasibility determination if
you expect that there is a good chance for the
problem to be infeasible.  In the filter line
search procedure, the restoration phase is called
more quickly than usually, and more reduction in
the constraint violation is enforced before the
restoration phase is left. If the problem is
square, this option is enabled automatically.
The default value for this string option is "no".
\\ 
Possible values:
\begin{itemize}
   \item no: the problem probably be feasible
   \item yes: the problem has a good chance to be infeasible
\end{itemize}

\paragraph{expect\_infeasible\_problem\_ctol:} Threshold for disabling "expect\_infeasible\_problem" option. $\;$ \\
 If the constraint violation becomes smaller than
this threshold, the "expect\_infeasible\_problem"
heuristics in the filter line search are
disabled. If the problem is square, this options
is set to 0. The valid range for this real option is 
$0 \le {\tt expect\_infeasible\_problem\_ctol } <  {\tt +inf}$
and its default value is $0.001$.


\paragraph{start\_with\_resto:} Tells algorithm to switch to restoration phase in first iteration. $\;$ \\
 Setting this option to "yes" forces the algorithm
to switch to the feasibility restoration phase in
the first iteration. If the initial point is
feasible, the algorithm will abort with a failure.
The default value for this string option is "no".
\\ 
Possible values:
\begin{itemize}
   \item no: don't force start in restoration phase
   \item yes: force start in restoration phase
\end{itemize}

\paragraph{soft\_resto\_pderror\_reduction\_factor:} Required reduction in primal-dual error in the soft restoration phase. $\;$ \\
 The soft restoration phase attempts to reduce the
primal-dual error with regular steps. If the
damped primal-dual step (damped only to satisfy
the fraction-to-the-boundary rule) is not
decreasing the primal-dual error by at least this
factor, then the regular restoration phase is
called. Choosing "0" here disables the soft
restoration phase. The valid range for this real option is 
$0 \le {\tt soft\_resto\_pderror\_reduction\_factor } <  {\tt +inf}$
and its default value is $0.9999$.


\paragraph{required\_infeasibility\_reduction:} Required reduction of infeasibility before leaving restoration phase. $\;$ \\
 The restoration phase algorithm is performed,
until a point is found that is acceptable to the
filter and the infeasibility has been reduced by
at least the fraction given by this option. The valid range for this real option is 
$0 \le {\tt required\_infeasibility\_reduction } <  1$
and its default value is $0.9$.


\paragraph{bound\_mult\_reset\_threshold:} Threshold for resetting bound multipliers after the restoration phase. $\;$ \\
 After returning from the restoration phase, the
bound multipliers are updated with a Newton step
for complementarity.  Here, the change in the
primal variables during the entire restoration
phase is taken to be the corresponding primal
Newton step. However, if after the update the
largest bound multiplier exceeds the threshold
specified by this option, the multipliers are all
reset to 1. The valid range for this real option is 
$0 \le {\tt bound\_mult\_reset\_threshold } <  {\tt +inf}$
and its default value is $1000$.


\paragraph{constr\_mult\_reset\_threshold:} Threshold for resetting equality and inequality multipliers after restoration phase. $\;$ \\
 After returning from the restoration phase, the
constraint multipliers are recomputed by a least
square estimate.  This option triggers when those
least-square estimates should be ignored. The valid range for this real option is 
$0 \le {\tt constr\_mult\_reset\_threshold } <  {\tt +inf}$
and its default value is $0$.


\paragraph{evaluate\_orig\_obj\_at\_resto\_trial:} Determines if the original objective function should be evaluated at restoration phase trial points. $\;$ \\
 Setting this option to "yes" makes the
restoration phase algorithm evaluate the
objective function of the original problem at
every trial point encountered during the
restoration phase, even if this value is not
required.  In this way, it is guaranteed that the
original objective function can be evaluated
without error at all accepted iterates; otherwise
the algorithm might fail at a point where the
restoration phase accepts an iterate that is good
for the restoration phase problem, but not the
original problem.  On the other hand, if the
evaluation of the original objective is
expensive, this might be costly.
The default value for this string option is "yes".
\\ 
Possible values:
\begin{itemize}
   \item no: skip evaluation
   \item yes: evaluate at every trial point
\end{itemize}

\paragraph{linear\_solver:} Linear solver used for step computations. $\;$ \\
 Determines which linear algebra package is to be
used for the solution of the augmented linear
system (for obtaining the search directions).
Note, the code must have been compiled with the
linear solver you want to choose. Depending on
your Ipopt installation, not all options are
available.
The default value for this string option is "ma27".
\\ 
Possible values:
\begin{itemize}
   \item ma27: use the Harwell routine MA27
   \item ma57: use the Harwell routine MA57
   \item pardiso: use the Pardiso package
   \item wsmp: use WSMP package
   \item taucs: use TAUCS package (not yet working)
   \item mumps: use MUMPS package (not yet working)
\end{itemize}

\paragraph{linear\_system\_scaling:} Method for scaling the linear system. $\;$ \\
 Determines the method used to compute symmetric
scaling factors for the augmented system (see
also the "linear\_scaling\_on\_demand" option). 
This scaling is independentof the NLP problem
scaling.  By default, MC19 is only used if MA27
or MA57 are selected as linear solvers. This
option is only available if Ipopt has been
compiled with MC19.
The default value for this string option is "mc19".
\\ 
Possible values:
\begin{itemize}
   \item none: no scaling will be performed
   \item mc19: use the Harwell routine MC19
\end{itemize}

\paragraph{linear\_scaling\_on\_demand:} Flag indicating that linear scaling is only done if it seems required. $\;$ \\
 This option is only important if a linear scaling
method (e.g., mc19) is used.  If you choose "no",
then the scaling factors are computed for every
linear system from the start.  This can be quite
expensive. Choosing "yes" means that the
algorithm will start the scaling method only when
the solutions to the linear system seem not good,
and then use it until the end.
The default value for this string option is "yes".
\\ 
Possible values:
\begin{itemize}
   \item no: Always scale the linear system.
   \item yes: Start using linear system scaling if solutions
seem not good.
\end{itemize}

\paragraph{max\_refinement\_steps:} Maximum number of iterative refinement steps per linear system solve. $\;$ \\
 Iterative refinement (on the full unsymmetric
system) is performed for each right hand side. 
This option determines the maximum number of
iterative refinement steps. The valid range for this integer option is
$0 \le {\tt max\_refinement\_steps } <  {\tt +inf}$
and its default value is $10$.


\paragraph{min\_refinement\_steps:} Minimum number of iterative refinement steps per linear system solve. $\;$ \\
 Iterative refinement (on the full unsymmetric
system) is performed for each right hand side. 
This option determines the minimum number of
iterative refinements (i.e. at least
"min\_refinement\_steps" iterative refinement
steps are enforced per right hand side.) The valid range for this integer option is
$0 \le {\tt min\_refinement\_steps } <  {\tt +inf}$
and its default value is $1$.


\paragraph{max\_hessian\_perturbation:} Maximum value of regularization parameter for handling negative curvature. $\;$ \\
 In order to guarantee that the search directions
are indeed proper descent directions, Ipopt
requires that the inertia of the (augmented)
linear system for the step computation has the
correct number of negative and positive
eigenvalues. The idea is that this guides the
algorithm away from maximizers and makes Ipopt
more likely converge to first order optimal
points that are minimizers. If the inertia is not
correct, a multiple of the identity matrix is
added to the Hessian of the Lagrangian in the
augmented system. This parameter gives the
maximum value of the regularization parameter. If
a regularization of that size is not enough, the
algorithm skips this iteration and goes to the
restoration phase. (This is delta\_w\^max in the
implementation paper.) The valid range for this real option is 
$0 <  {\tt max\_hessian\_perturbation } <  {\tt +inf}$
and its default value is $1 \cdot 10^{+20}$.


\paragraph{min\_hessian\_perturbation:} Smallest perturbation of the Hessian block. $\;$ \\
 The size of the perturbation of the Hessian block
is never selected smaller than this value, unless
no perturbation is necessary. (This is
delta\_w\^min in implementation paper.) The valid range for this real option is 
$0 \le {\tt min\_hessian\_perturbation } <  {\tt +inf}$
and its default value is $1 \cdot 10^{-20}$.


\paragraph{first\_hessian\_perturbation:} Size of first x-s perturbation tried. $\;$ \\
 The first value tried for the x-s perturbation in
the inertia correction scheme.(This is delta\_0
in the implementation paper.) The valid range for this real option is 
$0 <  {\tt first\_hessian\_perturbation } <  {\tt +inf}$
and its default value is $0.0001$.


\paragraph{perturb\_inc\_fact\_first:} Increase factor for x-s perturbation for very first perturbation. $\;$ \\
 The factor by which the perturbation is increased
when a trial value was not sufficient - this
value is used for the computation of the very
first perturbation and allows a different value
for for the first perturbation than that used for
the remaining perturbations. (This is
bar\_kappa\_w\^+ in the implementation paper.) The valid range for this real option is 
$1 <  {\tt perturb\_inc\_fact\_first } <  {\tt +inf}$
and its default value is $100$.


\paragraph{perturb\_inc\_fact:} Increase factor for x-s perturbation. $\;$ \\
 The factor by which the perturbation is increased
when a trial value was not sufficient - this
value is used for the computation of all
perturbations except for the first. (This is
kappa\_w\^+ in the implementation paper.) The valid range for this real option is 
$1 <  {\tt perturb\_inc\_fact } <  {\tt +inf}$
and its default value is $8$.


\paragraph{perturb\_dec\_fact:} Decrease factor for x-s perturbation. $\;$ \\
 The factor by which the perturbation is decreased
when a trial value is deduced from the size of
the most recent successful perturbation. (This is
kappa\_w\^- in the implementation paper.) The valid range for this real option is 
$0 <  {\tt perturb\_dec\_fact } <  1$
and its default value is $0.333333$.


\paragraph{jacobian\_regularization\_value:} Size of the regularization for rank-deficient constraint Jacobians. $\;$ \\
 (This is bar delta\_c in the implementation
paper.) The valid range for this real option is 
$0 \le {\tt jacobian\_regularization\_value } <  {\tt +inf}$
and its default value is $1 \cdot 10^{-08}$.


\paragraph{hessian\_approximation:} Indicates what Hessian information is to be used. $\;$ \\
 This determines which kind of information for the
Hessian of the Lagrangian function is used by the
algorithm.
The default value for this string option is "exact".
\\ 
Possible values:
\begin{itemize}
   \item exact: Use second derivatives provided by the NLP.
   \item limited-memory: Perform a limited-memory quasi-Newton 
approximation
\end{itemize}

\paragraph{limited\_memory\_max\_history:} Maximum size of the history for the limited quasi-Newton Hessian approximation. $\;$ \\
 This option determines the number of most recent
iterations that are taken into account for the
limited-memory quasi-Newton approximation. The valid range for this integer option is
$0 \le {\tt limited\_memory\_max\_history } <  {\tt +inf}$
and its default value is $6$.


\paragraph{limited\_memory\_max\_skipping:} Threshold for successive iterations where update is skipped. $\;$ \\
 If the update is skipped more than this number of
successive iterations, we quasi-Newton
approximation is reset. The valid range for this integer option is
$1 \le {\tt limited\_memory\_max\_skipping } <  {\tt +inf}$
and its default value is $2$.


\paragraph{derivative\_test:} Enable derivative checker $\;$ \\
 If this option is enabled, a (slow) derivative
test will be performed before the optimization. 
The test is performed at the user provided
starting point and marks derivative values that
seem suspicious
The default value for this string option is "none".
\\ 
Possible values:
\begin{itemize}
   \item none: do not perform derivative test
   \item first-order: perform test of first derivatives at starting
point
   \item second-order: perform test of first and second derivatives at
starting point
\end{itemize}

\paragraph{derivative\_test\_perturbation:} Size of the finite difference perturbation in derivative test. $\;$ \\
 This determines the relative perturbation of the
variable entries. The valid range for this real option is 
$0 <  {\tt derivative\_test\_perturbation } <  {\tt +inf}$
and its default value is $1 \cdot 10^{-08}$.


\paragraph{derivative\_test\_tol:} Threshold for indicating wrong derivative. $\;$ \\
 If the relative deviation of the estimated
derivative from the given one is larger than this
value, the corresponding derivative is marked as
wrong. The valid range for this real option is 
$0 <  {\tt derivative\_test\_tol } <  {\tt +inf}$
and its default value is $0.0001$.


\paragraph{derivative\_test\_print\_all:} Indicates whether information for all estimated derivatives should be printed. $\;$ \\
 Determines verbosity of derivative checker.
The default value for this string option is "no".
\\ 
Possible values:
\begin{itemize}
   \item no: Print only suspect derivatives
   \item yes: Print all derivatives
\end{itemize}

\paragraph{ma27\_pivtol:} Pivot tolerance for the linear solver MA27. $\;$ \\
 A smaller number pivots for sparsity, a larger
number pivots for stability.  This option is only
available if Ipopt has been compiled with MA27. The valid range for this real option is 
$0 <  {\tt ma27\_pivtol } <  1$
and its default value is $1 \cdot 10^{-08}$.


\paragraph{ma27\_pivtolmax:} Maximum pivot tolerance for the linear solver MA27. $\;$ \\
 Ipopt may increase pivtol as high as pivtolmax to
get a more accurate solution to the linear
system.  This option is only available if Ipopt
has been compiled with MA27. The valid range for this real option is 
$0 <  {\tt ma27\_pivtolmax } <  1$
and its default value is $0.0001$.


\paragraph{ma27\_liw\_init\_factor:} Integer workspace memory for MA27. $\;$ \\
 The initial integer workspace memory =
liw\_init\_factor * memory required by unfactored
system. Ipopt will increase the workspace size by
meminc\_factor if required.  This option is only
available if Ipopt has been compiled with MA27. The valid range for this real option is 
$1 \le {\tt ma27\_liw\_init\_factor } <  {\tt +inf}$
and its default value is $5$.


\paragraph{ma27\_la\_init\_factor:} Real workspace memory for MA27. $\;$ \\
 The initial real workspace memory =
la\_init\_factor * memory required by unfactored
system. Ipopt will increase the workspace size by
meminc\_factor if required.  This option is only
available if  Ipopt has been compiled with MA27. The valid range for this real option is 
$1 \le {\tt ma27\_la\_init\_factor } <  {\tt +inf}$
and its default value is $5$.


\paragraph{ma27\_meminc\_factor:} Increment factor for workspace size for MA27. $\;$ \\
 If the integer or real workspace is not large
enough, Ipopt will increase its size by this
factor.  This option is only available if Ipopt
has been compiled with MA27. The valid range for this real option is 
$1 \le {\tt ma27\_meminc\_factor } <  {\tt +inf}$
and its default value is $10$.


\paragraph{ma57\_pivtol:} Pivot tolerance for the linear solver MA57. $\;$ \\
 A smaller number pivots for sparsity, a larger
number pivots for stability. This option is only
available if Ipopt has been compiled with MA57. The valid range for this real option is 
$0 <  {\tt ma57\_pivtol } <  1$
and its default value is $1 \cdot 10^{-08}$.


\paragraph{ma57\_pivtolmax:} Maximum pivot tolerance for the linear solver MA57. $\;$ \\
 Ipopt may increase pivtol as high as
ma57\_pivtolmax to get a more accurate solution
to the linear system.  This option is only
available if Ipopt has been compiled with MA57. The valid range for this real option is 
$0 <  {\tt ma57\_pivtolmax } <  1$
and its default value is $0.0001$.


\paragraph{ma57\_pre\_alloc:} Safety factor for work space memory allocation for the linear solver MA57. $\;$ \\
 If 1 is chosen, the suggested amount of work
space is used.  However, choosing a larger number
might avoid reallocation if the suggest values do
not suffice.  This option is only available if
Ipopt has been compiled with MA57. The valid range for this real option is 
$1 \le {\tt ma57\_pre\_alloc } <  {\tt +inf}$
and its default value is $3$.


\paragraph{pardiso\_matching\_strategy:} Matching strategy to be used by Pardiso $\;$ \\
 This is IPAR(13) in Pardiso manual.  This option
is only available if Ipopt has been compiled with
Pardiso.
The default value for this string option is "complete+2x2".
\\ 
Possible values:
\begin{itemize}
   \item complete: Match complete (IPAR(13)=1)
   \item complete+2x2: Match complete+2x2 (IPAR(13)=2)
   \item constraints: Match constraints (IPAR(13)=3)
\end{itemize}

\paragraph{pardiso\_out\_of\_core\_power:} Enables out-of-core variant of Pardiso $\;$ \\
 Setting this option to a positive integer k makes
Pardiso work in the out-of-core variant where the
factor is split in 2\^k subdomains.  This is
IPARM(50) in the Pardiso manual.  This option is
only available if Ipopt has been compiled with
Pardiso. The valid range for this integer option is
$0 \le {\tt pardiso\_out\_of\_core\_power } <  {\tt +inf}$
and its default value is $0$.


\paragraph{wsmp\_num\_threads:} Number of threads to be used in WSMP $\;$ \\
 This determines on how many processors WSMP is
running on.  This option is only available if
Ipopt has been compiled with WSMP. The valid range for this integer option is
$1 \le {\tt wsmp\_num\_threads } <  {\tt +inf}$
and its default value is $1$.


\paragraph{wsmp\_ordering\_option:} Determines how ordering is done in WSMP $\;$ \\
 This corresponds to the value of WSSMP's
IPARM(16).  This option is only available if
Ipopt has been compiled with WSMP. The valid range for this integer option is
$-2 \le {\tt wsmp\_ordering\_option } \le 3$
and its default value is $1$.


\paragraph{wsmp\_pivtol:} Pivot tolerance for the linear solver WSMP. $\;$ \\
 A smaller number pivots for sparsity, a larger
number pivots for stability.  This option is only
available if Ipopt has been compiled with WSMP. The valid range for this real option is 
$0 <  {\tt wsmp\_pivtol } <  1$
and its default value is $0.0001$.


\paragraph{wsmp\_pivtolmax:} Maximum pivot tolerance for the linear solver WSMP. $\;$ \\
 Ipopt may increase pivtol as high as pivtolmax to
get a more accurate solution to the linear
system.  This option is only available if Ipopt
has been compiled with WSMP. The valid range for this real option is 
$0 <  {\tt wsmp\_pivtolmax } <  1$
and its default value is $0.1$.


\paragraph{wsmp\_scaling:} Determines how the matrix is scaled by WSMP. $\;$ \\
 This corresponds to the value of WSSMP's
IPARM(10). This option is only available if Ipopt
has been compiled with WSMP. The valid range for this integer option is
$0 \le {\tt wsmp\_scaling } \le 3$
and its default value is $0$.


\newpage
\section{Detailed Installation Information}\label{ExpertInstall}

The configuration script and Makefiles in the \Ipopt\ distribution
have been created using GNU's {\tt autoconf} and {\tt automake}.  They
attempt to automatically adapt the compiler settings etc.\ to the
system they are running on.  We tested the provided scripts for a
number of different machines, operating systems and compilers, but you
might run into a situation where the default setting does not work, or
where you need to change the settings to fit your particular
environment.

In general, you can see the list of options and variables that can be
set for the {\tt configure} script by typing \verb/configure --help/.
Below a few particular options are discussed:

\begin{itemize}
\item The {\tt configure} script tries to determine automatically, if
  you have BLAS and/or LAPACK already installed on your system (trying
  a few default libraries), and if it does not find them, it makes
  sure that you put the source code in the required place.

  However, you can specify a BLAS library (such as your local ATLAS
  library\footnote{see {\tt http://math-atlas.sourceforge.net/}})
  explicitly, using the \verb/--with-blas/ flag for {\tt configure}.
  For example,

  \verb|./configure --with-blas="-L$HOME/lib -latlas"|

  To tell the configure script to compile and use the downloaded BLAS
  source files even if a BLAS library is found on your system, specify
  \verb|--with-blas=BUILD|.

  Similarly, you can use the \verb/--with-lapack/ switch to specify
  the location of your LAPACK library, or use the keyword {\tt BUILD}
  to force the \Ipopt\ makefiles to compile LAPACK together with
  \Ipopt.

\item Similarly, if you have a precompiled library containing the
  Harwell Subroutines, you can specify its location with the
  \verb|--with-hsl| flag.  And the location of the directory with the
  AMPL solver library {\tt amplsolver.a} and the ASL header files can
  be specified with \verb|--with-asldir|.

\item If you want to compile \Ipopt\ with the linear solver Pardiso
  (see Section~\ref{sec:Pardiso}), you need to specify the location of
  the library with the \verb|--with-pardiso| flag, including required
  additional libraries and flags.  For example, if you want to compile
  \Ipopt\ with the parallel version of Pardiso (located in {\tt
    \$HOME/lib}) on an AIX system in 64bit mode, you should add the
  flag

  \verb|--with-pardiso="-qsmp=omp $HOME/lib/libpardiso_P4AIX51_64_P.so"|

  If you are using the parallel version of Pardiso, you need to
  specify the number of processors it should run on with the
  environment variable \verb|OMP_NUM_THREADS|, as described in the
  Pardiso manual.

\item If you want to compile \Ipopt\ with the linear solver WSMP (see
  Section~\ref{sec:WSMP}), you need to specify the location of the
  library with the \verb|--with-wsmp| flag, including required
  additional libraries and flags.  For example, if you want to compile
  \Ipopt\ with WSMP (located in {\tt \$HOME/lib}) on an Intel IA32
  Linux system, you should add the flag

  \verb|--with-wsmp="$HOME/lib/libwsmp.a"|

\item If you want to specify that you want to use particular
  compilers, you can do so by adding the variables definitions for
  {\tt CXX}, {\tt CC}, and {\tt F77} to the {\tt ./configure} command
  line, to specify the C++, C, and Fortran compiler, respectively.
  For example,

  {\tt ./configure CXX=g++ CC=gcc F77=g77}

  In order to set the compiler flags, you should use the variables
  {\tt CXXFLAGS}, {\tt CFLAGS}, {\tt FFLAGS}.  Note, that the \Ipopt\
  code uses ``{\tt dynamic\_cast}''.  Therefore it is necessary that
  the C++ code is compiled including RTTI (Run-Time Type Information).
  Some compilers need to be given special flags to do that (e.g.,
  ``{\tt -qrtti=dyna}'' for the AIX {\tt xlC} compiler).

\item By default, the \Ipopt\ library is compiled as a shared library,
  on systems where this is supported.  If you want to generate a
  static library, you need to specify the {\tt --disable-shared}
  flag.  If you want to compile both shared and static libraries, you
  should specify the {\tt --enable-static} flag.

\item If you want to link the \Ipopt\ library with a main program
  written in C or Fortran, the C and Fortran compiler doing the
  linking of the executable needs to be told about the C++ runtime
  libraries.  Unfortunately, the current version of {\tt autoconf}
  does not provide the automatic detection of those libraries.  We
  have hard-coded some default values for some systems and compilers,
  but this might not work all the time.

  If you have problems linking your Fortran or C code with the \Ipopt\
  library {\tt libipopt.a} and the linker complains about missing
  symbols from C++ (e.g., the standard template library), you should
  specify the C++ libraries with the {\tt CXXLIBS} variable.  To find out
  what those libraries are, it is probably helpful to link a  simple C++
  program with verbose compiler output.

  For example, for the Intel compilers on a Linux system, you
  might need to specify something like

  {\tt ./configure CC=icc F77=ifort CXX=icpc $\backslash$\\ \hspace*{14ex} CXXLIBS='-L/usr/lib/gcc-lib/i386-redhat-linux/3.2.3 -lstdc++'}

\item Compilation in 64bit mode sometimes requires some special
  consideration.  For example, for compilation of 64bit code on AIX,
  we recommend the following configuration

  {\tt ./configure AR='ar -X64' NM='nm -X64' $\backslash$\\
    \hspace*{14ex} CC='xlc -q64' F77='xlf -q64' CXX='xlC
    -q64'$\backslash$\\ \hspace*{14ex} CFLAGS='-O3
    -bmaxdata:0x3f0000000'
    $\backslash$\\ \hspace*{14ex} FFLAGS='-O3 -bmaxdata:0x3f0000000' $\backslash$\\
    \hspace*{14ex} CXXFLAGS='-qrtti=dyna -O3 -bmaxdata:0x3f0000000'}

% \item To build library/archive files (with the ending {\tt .a})
%   including C++ code in some environments, it is necessary to use the
%   C++ compiler instead of {\tt ar} to build the archive.  This is for
%   example the case for some older compilers on SGI and SUN.  For this,
%   the {\tt configure} variables {\tt AR}, {\tt ARFLAGS}, and {\tt
%     AR\_X} are provided.  Here, {\tt AR} specifies the command for the
%   archiver for creating an archive, and {\tt ARFLAGS} specifies
%   additional flags.  {\tt AR\_X} contains the command for extracting
%   all files from an archive.  For example, the default setting for SUN
%   compilers for our configure script is

%   {\tt AR='CC -xar' ARFLAGS='-o' AR\_X='ar x'}

\item It is possible to compile the \Ipopt\ library in a debug
  configuration, by specifying \verb|--enable-debug|.  Then the
  compilers will use the debug flags (unless the compilation flag
  variables are overwritten in the {\tt configure} command line), and
  additional debug checks are compiled into the code (see {\tt
    IpDebug.hpp}).  This usually leads to a significant slowdown of
  the code, but might be helpful when debugging something.

\item It is not necessary to produce the binary files in the
  directories where the source files are.  If you want to compile the
  code on different systems or with different compilers/options on a
  shared file system, you can keep one single copy of the source files
  in one directory, and the binary files for each configuration in
  separate directories.  For this, simply run the configure script in
  the directory where you want the base directory for the \Ipopt\
  binary files.  For example:

  {\tt \$ mkdir \$HOME/Ipopt-objects}\\
  {\tt \$ cd \$HOME/Ipopt-objects}\\
  {\tt \$ \$HOME/CoinIpopt/configure}

\end{itemize}

%\bibliographystyle{plain}
%\bibliography{/home/andreasw/tex/andreas}
%% Copyright (C) 2005, 2006 Carnegie Mellon University and others.
%%
%% The first version of this file was contributed to the Ipopt project
%% on Aug 1, 2005, by Yoshiaki Kawajiri
%%                    Department of Chemical Engineering
%%                    Carnegie Mellon University
%%                    Pittsburgh, PA 15213
%%
%% Since then, the content of this file has been updated significantly by
%%     Carl Laird and Andreas Waechter        IBM
%%
%%
%% $Id$
%%
\documentclass[10pt]{article}
\setlength{\textwidth}{6.3in}       % Text width
\setlength{\textheight}{9.4in}      % Text height
\setlength{\oddsidemargin}{0.1in}     % Left margin for even-numbered pages
\setlength{\evensidemargin}{0.1in}    % Left margin for odd-numbered pages
\setlength{\topmargin}{-0.5in}         % Top margin
\renewcommand{\baselinestretch}{1.1}
\usepackage{amsfonts}
\usepackage{amsmath}

\newcommand{\RR}{{\mathbb{R}}}
\newcommand{\Ipopt}{{\sc Ipopt}}


\begin{document}
\title{Introduction to \Ipopt:\\
A tutorial for downloading, installing, and using \Ipopt.}

\author{Revision number of this document: $Revision$}

%\date{\today}
\maketitle

\begin{abstract}
  This document is a guide to using \Ipopt\ 3.1 (the new C++ version
  of \Ipopt).  It includes instructions on how to obtain and compile
  \Ipopt, a description of the interface, user options, etc.,, as
  well as a tutorial on how to solve a nonlinear optimization problem
  with \Ipopt.

  The initial version of this document was created by
  Yoshiaki Kawajir\footnote{Department of Chemical Engineering,
    Carnegie Mellon University, Pittsburgh PA} as a course project for
  \textit{47852 Open Source Software for Optimization}, taught by
  Prof. Fran\c cois Margot at Tepper School of Business, Carnegie
  Mellon University.  The current version is maintained by Carl
  Laird\footnote{Department of Chemical Engineering, Carnegie Mellon
    University, Pittsburgh PA} and Andreas
  W\"achter\footnote{Department of Mathematical Sciences, IBM T.J.\
    Watson Research Center, Yorktown Heights, NY}.
\end{abstract}

\tableofcontents

\vspace{\baselineskip}
\begin{small}
\noindent
The following names used in this document are trademarks or registered
trademarks: AMPL, IBM, Intel, Microsoft, Visual Studio C++, Visual
Studio C++ .NET
\end{small}

\section{Introduction}
\Ipopt\ (\underline{I}nterior \underline{P}oint \underline{Opt}imizer,
pronounced ``I--P--Opt'') is an open source software package for
large-scale nonlinear optimization. It can be used to solve general
nonlinear programming problems of the form
%\begin{subequations}\label{NLP}
\begin{eqnarray}
\min_{x\in\RR^n} &&f(x) \label{eq:obj} \\
\mbox{s.t.} \;  &&g^L \leq g(x) \leq g^U \\
                &&x^L \leq x \leq x^U, \label{eq:bounds}
\end{eqnarray}
%\end{subequations}
where $x \in \RR^n$ are the optimization variables (possibly with
lower and upper bounds, $x^L\in(\RR\cup\{-\infty\})^n$ and
$x^U\in(\RR\cup\{+\infty\})^n$), $f:\RR^n\longrightarrow\RR$ is the
objective function, and $g:\RR^n\longrightarrow \RR^m$ are the general
nonlinear constraints.  The functions $f(x)$ and $g(x)$ can be linear
or nonlinear and convex or non-convex (but should be twice
continuously differentiable). The constraints, $g(x)$, have lower and
upper bounds, $g^L\in(\RR\cup\{-\infty\})^n$ and
$g^U\in(\RR\cup\{+\infty\})^m$. Note that equality constraints of the
form $g_i(x)=\bar g_i$ can be specified by setting
$g^L_{i}=g^U_{i}=\bar g_i$.

\subsection{Mathematical Background}
\Ipopt\ implements an interior point line search filter method that
aims to find a local solution of (\ref{eq:obj})-(\ref{eq:bounds}).  The
mathematical details of the algorithm can be found in several
publications
\cite{NocWaeWal:adaptive,WaechterPhD,WaecBieg06:mp,WaeBie05:filterglobal,WaeBie05:filterlocal}.

\subsection{Availability}
The \Ipopt\ package is available from COIN-OR
(\texttt{www.coin-or.org}) under the CPL (Common Public License)
open-source license and includes the source code for \Ipopt.  This
means, it is available free of charge, also for commercial purposes.
However, if you give away software including \Ipopt\ code (in source
code or binary form) and you made changes to the \Ipopt\ source code,
you are required to make those changes public and to clearly indicate
which modifications you made.  After all, the goal of open source
software is the continuous development and improvement of software.
For details, please refer to the Common Public License.

Also, if you are using \Ipopt\ to obtain results for a publication, we
politely ask you to point out in your paper that you used \Ipopt, and
to cite the publication \cite{WaecBieg06:mp}.  Writing high-quality
numerical software takes a lot of time and effort, and does usually
not translate into a large number of publications, therefore we believe
this request is only fair :).

\subsection{Prerequisites}
In order to build \Ipopt, some third party components are required:
\begin{itemize}
\item BLAS (Basic Linear Algebra Subroutines).  Many vendors of
  compilers and operating systems provide precompiled and optimized
  libraries for these dense linear algebra subroutines.  But you can
  also get the source code from {\tt www.netlib.org} and have the
  \Ipopt\ distribution compile it automatically.
\item LAPACK (Linear Algebra PACKage).  Also for LAPACK, some vendors
  offer precompiled and optimized libraries.  But like with BLAS, you
  can get the source code from {\tt www.netlib.org} and have the
  \Ipopt\ distribution compile it automatically.

  Note that currently LAPACK is only required if you intend to use the
  quasi-Newton options in \Ipopt.  You can compile the code without
  LAPACK, but an error message will then occur if you try to run the
  code with an option that requires LAPACK.  Currently, the LAPACK
  routines that are used by \Ipopt\ are only {\tt DPOTRF}, {\tt
    DPOTRS}, and {\tt DSYEV}.
\item A sparse symmetric indefinite linear solver. The \Ipopt\ needs
  to obtain the solution of sparse, symmetric, indefinite linear
  systems, and for this it relies on third-party code.  

  Currently, the following linear solvers can be used:
  \begin{itemize}
  \item MA27 from the Harwell Subroutine Library\\ (see {\tt
      http://www.cse.clrc.ac.uk/nag/hsl/}).
  \item MA57 from the Harwell Subroutine Library\\ (see {\tt
      http://www.cse.clrc.ac.uk/nag/hsl/}).
  \item The Watson Sparse Matrix Package (WSMP)\\ (see {\tt
      http://www-users.cs.umn.edu/\~agupta/wsmp.html})
  \item The Parallel Sparse Direct Linear Solver (PARDISO)\\ (see {\tt
      http://www.computational.unibas.ch/cs/scicomp/software/pardiso/}).
  \end{itemize}
  You need to include at least one of the linear solvers above in
  order to run \Ipopt.

  Interfaces to other linear solvers might be added in the future; if
  you are interested in contributing such an interface please contact
  us!  Note that \Ipopt\ requires that the linear solver is able to
  provide the inertia (number of positive and negative eigenvalues) of
  the symmetric matrix that is factorized.

\item Furthermore, \Ipopt\ can also use the Harwell Subroutine MC19
  for scaling of the linear systems before they are passed to the
  linear solver.  This may be particularly useful if \Ipopt\ is used
  with MA27 or MA57.  However, it is not required to have MC19 to
  compile \Ipopt; if this routine is missing, the scaling is never
  performed.
\item ASL (AMPL Solver Library).  The source code is available at {\tt
    www.netlib.org}, and the \Ipopt\ makefiles will automatically
  compile it for you if you put the source code into a designated
  space.  NOTE: This is only required if you want to use \Ipopt\ from
  AMPL and want to compile the \Ipopt\ AMPL solver executable.
\end{itemize}
For more information on third-party components and how to obtain them,
see Section~\ref{ExternalCode}.

Since the \Ipopt\ code is written in C++, you will need a C++ compiler
to build the \Ipopt\ library.  We tried very hard to write the code as
platform and compiler independent as possible.

In addition, the configuration script currently also searches for a
Fortran, since some of the dependencies above are written in Fortran.
If all third party dependencies are available as self-contained
libraries, those compilers are in principle not necessary.  Also, it
is possible to use the Fortran-to-C compiler {\tt f2c} from {\tt
  www.netlib.org} to convert Fortran code to C, and compile the
resulting C files with a C compiler and create a library containing
the required third party dependencies.  But so far we have not tested
this ourselves, and currently the configuration script for \Ipopt\
looks for a Fortran compiler.

\subsection{How to use \Ipopt}
If desired, the \Ipopt\ distribution generates an executable for the
modeling environment AMPL. As well, you can link your problem
statement with \Ipopt\ using interfaces for C++, C, or Fortran.
\Ipopt\ can be used with most Linux/Unix environments, and on Windows
using Visual Studio .NET or Cygwin.  Below in
Section~\ref{sec:tutorial-example} this document demonstrates how to
solve problems using \Ipopt. This includes installation and
compilation of \Ipopt\ for use with AMPL as well as linking with your
own code.

Finally, the \Ipopt\ distribution includes an interface for {\tt
  CUTEr}\footnote{see {\tt http://cuter.rl.ac.uk/cuter-www/}}, if you
want to use \Ipopt\ to solve problems modeled in SIF.

The old (Fortran 2.x) version of \Ipopt\ has been interface with
Matlab, and is also available from NEOS, and the new version will be
available through similar means in the future.  Please check the
\Ipopt\ homepage for updates.

\subsection{More Information and Contributions}
More and up-to-date information can be found at the \Ipopt\ homepage,

\begin{center}
\texttt{http://projects.coin-or.org/Ipopt}.
\end{center}

Here, you can find FAQs, some (hopefully useful) hints, a bug report
system etc.  The website is managed with Wiki, which means that every
user can edit the webpages from the regular web browser.  In
particular, we encourage \Ipopt\ users to share their experiences and
usage hints on the ``Success Stories'' and ``Hints and Tricks''
pages\footnote{Since we had some malicious hacker attacks destroying
  the content of the web pages in the past, you are now required to
  enter a user name and password; simply follow the instructions in
  the last paragraph of the Documentation section on the main project
  page.}

\Ipopt\ is an open source project, and we encourage people to
contribute code (such as interfaces to appropriate linear solvers,
modeling environments, or even algorithmic features).  If you are
interested in contributing code, please have a look at the COIN
constributions webpage\footnote{see \tt
  http://www.coin-or.org/contributions.html}, and contact the \Ipopt\
project leader.

There is also a mailing list for \Ipopt, available from the webpage
\begin{center}
\texttt{http://list.coin-or.org/mailman/listinfo/coin-ipopt},
\end{center}
where you can
subscribe to get notified of updates, and to ask general questions
regarding installation and usage. (You might want to look at the
archives before posting a question.)

We try to answer questions posted to the mailing list in a reasonable
manner.  Please understand that we cannot answer all questions in
detail, and because of time constraints, we may not be able to help
you model and debug your particular optimization problem.  However, if
you have a challenging optimization problem and are interested in
consulting services by IBM Research, please contact the \Ipopt\
project leader, Andreas W\"achter.

\subsection{History of \Ipopt}
The original \Ipopt\ (Fortran version) was a product of the dissertation
research of Andreas W\"achter \cite{WaechterPhD}, under Lorenz
T. Biegler at the Chemical Engineering Department at Carnegie Mellon
University. The code was made open source and distributed by the
COIN-OR initiative, which is now a non-profit corporation.  \Ipopt\ has
been actively developed under COIN-OR since 2002.

To continue natural extension of the code and allow easy addition of
new features, IBM Research decided to invest in an open source
re-write of \Ipopt\ in C++.  The new C++ version of the \Ipopt\
optimization code (\Ipopt\ 3.0 and beyond) is currently developed at IBM
Research and remains part of the COIN-OR initiative. Future
development on the Fortran version will cease with the exception of
occasional bug fix releases.

\section{Installing \Ipopt}\label{Installing}

The following sections describe the installation procedures on
UNIX/Linux systems.  For installation instructions on Windows
see Section~\ref{WindowsInstall}.

\subsection{Getting the \Ipopt\ Code}
\Ipopt\ is available from the COIN-OR subversion repository. You can
either download the code using \texttt{svn} (the
\textit{subversion}\footnote{see
  \texttt{http://subversion.tigris.org/}} client similar to CVS) or
simply retrieve a tarball (compressed archive file).  While the
tarball is an easy method to retrieve the code, using the
\textit{subversion} system allows users the benefits of the version
control system, including easy updates and revision control.

\subsubsection{Getting the \Ipopt\ code via subversion}

Of course, the \textit{subversion} client must be installed on your
system if you want to obtain the code this way (the executable is
called \texttt{svn}); it is already installed by default for many
recent Linux distributions.  Information about \textit{subversion} and
how to download it can be found at
\texttt{http://subversion.tigris.org/}.\\

To obtain the \Ipopt\ source code via subversion, change into the
directory in which you want to create a subdirectory {\tt Ipopt} with
the \Ipopt\ source code.  Then follow the steps below:
\begin{enumerate}
\item{Download the code from the repository}\\
{\tt \$ svn co https://www.coin-or.org/svn/Ipopt/trunk Ipopt} \\
Note: The {\tt \$} indicates the command line
prompt, do not type {\tt \$}, only the text following it.
\item Change into the root directory of the \Ipopt\ distribution\\
{\tt \$ cd Ipopt}
\end{enumerate}

In the following, ``\texttt{\$IPOPTDIR}'' will refer to the directory in
which you are right now (output of \texttt{pwd}).

\subsubsection{Getting the \Ipopt\ code as a tarball}

To use the tarball, follow the steps below:
\begin{enumerate}
\item Download the latest tarball from
\texttt{http://www.coin-or.org/Tarballs}.  The file you should look
for has the form \texttt{ipopt-3.x.x.tar.gz} (where
``\texttt{3.x.x.}'' is the version number).  Put this file in a
directory under which you want to put the \Ipopt\ installation.
\item Issue the following commands to unpack the archive file: \\
\texttt{\$ gunzip ipopt-3.x.x.tar.gz} \\
\texttt{\$ tar xvf ipopt-3.x.x.tar} \\
Note: The {\tt \$} indicates the command line
prompt, do not type {\tt \$}, only the text following it.
\item Change into the root directory of the \Ipopt\ distribution\\
{\tt \$ cd ipopt-3.x.x}
\end{enumerate}

In the following, ``\texttt{\$IPOPTDIR}'' will refer to the directory in
which you are right now (output of \texttt{pwd}).

\subsection{Download External Code}\label{ExternalCode}
\Ipopt\ uses a few external packages that are not included in the
\Ipopt\ source code distribution, namely ASL (the AMPL Solver
Library), BLAS, LAPACK.  It also requires a sparse symmetric linear
solver.

Since this third party software released under different licenses than
\Ipopt, we cannot distribute that code together with the \Ipopt\
packages and have to ask you to go through the hassle of obtaining it
yourself (even though we tried to make it as easy for you as we
could).  Keep in mind that it is still your responsibility to ensure
that your downloading and usage if the third party components conforms
with their licenses.

Note that you only need to obtain the ASL if you intend to use \Ipopt\
from AMPL.  It is not required if you want to specify your
optimization problem in a programming language (C++, C, or Fortran).
Also, currently, LAPACK is only required if you intend to use the
quasi-Newton options implemented in \Ipopt.

\subsubsection{Download BLAS, LAPACK and ASL}
If you have the download utility \texttt{wget} installed on your
system, retrieving BLAS, LAPACK, and ASL is straightforward using
scripts included with the ipopt distribution. These scripts download
the required files from the Netlib Repository
(\texttt{www.netlib.org}).\\

\noindent
{\tt \$ cd \$IPOPTDIR/Extern/blas}\\
{\tt \$ ./get.blas}\\
{\tt \$ cd ../lapack}\\
{\tt \$ ./get.lapack}\\
{\tt \$ cd ../ASL}\\
{\tt \$ ./get.ASL}\\

\noindent
If you do not have \texttt{wget} installed on your system, please read
the \texttt{INSTALL.*} files in the \texttt{\$IPOPTDIR/Extern/blas},
\texttt{\$IPOPTDIR/Extern/lapack} and \texttt{\$IPOPTDIR/Extern/ASL}
directories for alternative instructions.

\subsubsection{Download HSL Subroutines}
\Ipopt\ requires a sparse symmetric linear solver.  There are
different possibilities.  In this section we describe how to obtain
the source code for MA27 (and MC19) from the Harwell Subroutine
Library (HSL).  Those routines are freely available for
non-commercial, academic use, but it is your responsibility to
investigate the licensing of all third party code.

The use of alternative linear solvers is described in
Appendix~\ref{ExpertInstall}.  You do not necessarily have to use MA27
as described in this section, but at least one linear solver is
required for \Ipopt\ to function.

\begin{enumerate}
\item Go to {\tt http://hsl.rl.ac.uk/archive/hslarchive.html}
\item Follow the instruction on the website, read the license, and
  submit the registration form.
\item Go to \textit{HSL Archive Programs}, and find the package list.
\item In your browser window, click on \textit{MA27}.
\item Make sure that \textit{Double precision:} is checked. 
  Click \textit{Download package (comments removed)}
\item Save the file as {\tt ma27ad.f} in {\tt \$IPOPTDIR/Extern/HSL/}\\
  Note: Some browsers append a file extension ({\tt .txt}) when you save
  the file, in which case you have to rename it.
\item Go back to the package list using the back button of your browser.
\item In your browser window, click on \textit{MC19}.
\item Make sure \textit{Double precision:} is checked. Click 
  \textit{Download package (comments removed)}
\item Save the file as {\tt mc19ad.f} in {\tt
    \$IPOPTDIR/Extern/HSL/}\\
  Note: Some browsers append a file extension ({\tt .txt}) when you save
  the file, so you may have to rename it.
\end{enumerate}

Note: Whereas currently obtaining MA27 is essential for using \Ipopt,
MC19 could be omitted (with the consequence that you cannot use this
method for scaling the linear systems arising inside the \Ipopt\
algorithm).

Note: If you have the source code for the linear solver MA57
(successor of MA27) in a file called ma57ad.f (including all
dependencies), you can simply put it into the {\tt
  \$IPOPTDIR/Extern/HSL/} directory.  The \Ipopt\ configuration script
will then find this file and compile it into the \Ipopt\ library (just
as is would compile MA27).

\subsection{Compiling and Installing \Ipopt} \label{sec.comp_and_inst}

\Ipopt\ can be easily compiled and installed with the usual {\tt
  configure}, {\tt make}, {\tt make install} commands.  Below are the
basic steps that should work on most systems.  For special
compilations and for some troubleshooting see
Appendix~\ref{ExpertInstall} and consult the \Ipopt\ homepage before
submitting a ticket or sending a message to the mailing list.
\begin{enumerate}
\item Go to the main directory of \Ipopt:\\
  {\tt \$ cd \$IPOPTDIR} 
\item Run the configure script\\
  {\tt \$ ./configure}

  If the last output line of the script reads ``\texttt{configure:
    Configuration successful}'' then everything worked fine.
  Otherwise, look at the screen output, have a look at the
  \texttt{config.log} output file and/or consult
  Appendix~\ref{ExpertInstall}.

  The default configure (without any options) is sufficient for most
  users. If you want to see the configure options, consult
  Appendix~\ref{ExpertInstall}.
\item Build the code \\
{\tt \$ make}
\item Install \Ipopt \\
  {\tt \$ make install}\\
  This installs
  \begin{itemize}
  \item the \Ipopt\ AMPL solver executable (if ASL source was
    downloaded) in \texttt{\$IPOPTDIR/bin},
  \item the \Ipopt\ library (\texttt{libipopt.a}) in
    \texttt{\$IPOPTDIR/lib},
  \item text files {\tt ipopt\_addlibs\_cpp.txt} and {\tt
      ipopt\_addlibs\_f.txt} in \texttt{\$IPOPTDIR/lib} that contain a
    line each with additional linking flags that are required for
    linking code with the ipopt library, for C++ and Fortran main
    programs, respectively. (This is only for convenience if you want
    to find out what additional flags are required, for example, to
    include the Fortran runtime libraries with a C++ compiler.)
  \item the necessary header files in
    \texttt{\$IPOPTDIR/include/ipopt}.
  \end{itemize}
  You can change the default installation directory (here
  \texttt{\$IPOPTDIR}) to something else (such as \texttt{/usr/local})
  by using the \verb|--prefix| switch for \texttt{configure}.
%\item Test the installation \\
%  {\tt \$ make test}\\
%  This should ?...?
\item Install \Ipopt\ for use with {\tt CUTEr}\\
  If you have {\tt CUTEr} already installed on your system and you
  want to use \Ipopt\ as a solver for problems modeled in {\tt SIF},
  type\\
  {\tt \$ make cuter}\\
  This assumes that you have the environment variable {\tt MYCUTER}
  defined according to the {\tt CUTEr} instructions.  After this, you
  can use the script {\tt sdipo} as the {\tt CUTEr} script to solve a
  {\tt SIF} model.
\end{enumerate}

\subsection{Installation on Windows}\label{WindowsInstall}

There are two ways to install \Ipopt\ on Windows systems.  The first
option, described in Section~\ref{CygwinInstall}, is to use Cygwin (see
\texttt{www.cygwin.com}), which offers a UNIX-like environment
on Windows and in which the installation procedure described earlier
in this section can be used.  The \Ipopt\ distribution also includes
projects files for the Microsoft Visual Studio (see
Section~\ref{VisualStudioInstall}).

\subsubsection{Installation with Cygwin}\label{CygwinInstall}

Cygwin is a Linux-like environment for Windows; if you don't know what
it is you might want to have a look at the Cygwin homepage,
\texttt{www.cygwin.com}.

It is possible to build the \Ipopt\ AMPL solver executable in Cygwin
for general use in Windows.  You can also hook up \Ipopt\ to your own
program if you compile it in the Cygwin environment\footnote{It is
  also possible to build an \Ipopt\ DLL that can be used from
  non-cygwin compilers, but this is not (yet?) supported.}.

If you want to compile \Ipopt\ under Cygwin, you first have to install
Cygwin on your Windows system.  This is pretty straight forward; you
simply download the ``setup'' program from
\texttt{www.cygwin.com} and start it.

Then you do the following steps (assuming here that you don't have any
complications with firewall settings etc - in that case you might have
to choose some connection settings differently):

\begin{enumerate}
\item Click next
\item Select ``install from the internet'' (default) and click next
\item Select a directory where Cygwin is to be installed (you can
  leave the default) and choose all other things to your liking, then
  click next
\item Select a temp dir for Cygwin setup to store some files (if you
  put it on your desktop you will later remember to delete it)
\item Select ``direct connection'' (default) and click next
\item Select some mirror site that seems close by to you and click next
\item OK, now comes the complicated part:\\
  You need to select the packages that you want to have installed.  By
  default, there are already selections, but the compilers are usually
  not pre-chosen.  You need to make sure that you select the GNU
  compilers (for Fortran, C, and C++ --- together with the MinGW
  options), the GNU Make, and Subversion.  For this, click on the "Devel"
  branch (which opens a subtree) and select:
  \begin{itemize}
  \item gcc
  \item gcc-core
  \item gcc-g77
  \item gcc-g++
  \item gcc-mingw
  \item gcc-mingw-core
  \item gcc-mingw-g77
  \item gcc-mingw-g++
  \item make
  \item subversion
  \end{itemize}

  Then, in the ``Web'' branch, please select ``wget'' (which will make
  the installation of third party dependencies for \Ipopt\ easier)

  This will automatically also select some other packages.
\item Then you click on next, and Cygwin will be installed (follow the
  rest of the instructions and choose everything else to your liking).
  At a later point you can easily add/remove packages with the setup
  program.

\item Now that you have Cygwin, you can open a Cygwin window, which is
  like a UNIX shell window.

\item Now you just follow the instructions in the beginning of
  Sections~\ref{Installing}:  You download the \Ipopt\ code into
  your Cygwin home directory (from the Windows explorer that is
  usually something like
  \texttt{C:$\backslash$Cygwin$\backslash$home$\backslash$your\_user\_name}).
  After that you obtain the third party code (like on Linux/UNIX),
  type

  \texttt{./configure}

  and

  \texttt{make install}

  in the correct directories, and hopefully that will work.  The
  \Ipopt\ AMPL solver executable will be in the subdirectory
  \texttt{bin} (called ``\texttt{ipopt.exe}'').
\end{enumerate}

\subsubsection{Using Visual Studio}\label{VisualStudioInstall}

The \Ipopt\ distribution includes project files that can be used to
compile the \Ipopt\ library and a Fortran and C++ example within the
Microsoft Visual Studio.  The project files have been created with
Microsoft Visual C++ .NET 2003 Standard, and the Intel Visual Fortran
Compiler 8.1.

In order to use those project files, download the \Ipopt\ source code,
as well as the required third party code (put it into the {\tt
  Extern/blas}, {\tt Extern/lapack}, and {\tt Extern/HSL}
directories---ASL is not required for the Fortran and C
examples). Then open the solution file\\

\texttt{\$IPOPTDIR$\backslash$Windows$\backslash$VisualStudio\_dotNET$\backslash$Ipopt$\backslash$Ipopt.sln}\\

Note: Since the project files were created only with the Standard
edition of the C++ compiler, code optimization might be disabled; for
fast performance make sure you enable code optimization.

\section{Interfacing your NLP to \Ipopt: A tutorial example.}
\label{sec:tutorial-example}

\Ipopt\ has been designed to be flexible for a wide variety of
applications, and there are a number of ways to interface with \Ipopt\
that allow specific data structures and linear solver
techniques. Nevertheless, the authors have included a standard
representation that should meet the needs of most users.

This tutorial will discuss four interfaces to \Ipopt, namely the AMPL
modeling language\cite{FouGayKer:AMPLbook} interface, and the C++, C,
and Fortran code interfaces.  AMPL is a 3rd party modeling language
tool that allows users to write their optimization problem in a syntax
that resembles the way the problem would be written mathematically.
Once the problem has been formulated in AMPL, the problem can be
easily solved using the (already compiled) \Ipopt\ AMPL solver
executable, {\tt ipopt}. Interfacing your problem by directly linking
code requires more effort to write, but can be far more efficient for
large problems.

We will illustrate how to use each of the four interfaces using an
example problem, number 71 from the Hock-Schittkowsky test suite \cite{HS},
%\begin{subequations}\label{HS71}
  \begin{eqnarray}
    \min_{x \in \Re^4} &&x_1 x_4 (x_1 + x_2 + x_3)  +  x_3 \label{eq:ex_obj} \\
    \mbox{s.t.}  &&x_1 x_2 x_3 x_4 \ge 25 \label{eq:ex_ineq} \\
    &&x_1^2 + x_2^2 + x_3^2 + x_4^2  =  40 \label{eq:ex_equ} \\
    &&1 \leq x_1, x_2, x_3, x_4 \leq 5, \label{eq:ex_bounds}
  \end{eqnarray}
%\end{subequations}
with the starting point
\begin{equation}
x_0 = (1, 5, 5, 1) \label{eq:ex_startpt}
\end{equation}
and the optimal solution
\[
x_* = (1.00000000, 4.74299963, 3.82114998, 1.37940829). \nonumber
\]

\subsection{Using \Ipopt\ through AMPL}
Using the AMPL solver executable is by far the easiest way to
solve a problem with \Ipopt. The user must simply formulate the problem
in AMPL syntax, and solve the problem through the AMPL environment.
There are drawbacks, however. AMPL is a 3rd party package and, as
such, must be appropriately licensed (a free student version for
limited problem size is available from the AMPL website,
\texttt{www.ampl.com}). Furthermore, the AMPL environment may be prohibitive
for very large problems. Nevertheless, formulating the problem in AMPL
is straightforward and even for large problems, it is often used as a
prototyping tool before using one of the code interfaces.

This tutorial is not intended as a guide to formulating models in
AMPL. If you are not already familiar with AMPL, please consult
\cite{FouGayKer:AMPLbook}.

The problem presented in equations
(\ref{eq:ex_obj})--(\ref{eq:ex_startpt}) can be solved with \Ipopt\ with
the AMPL model file given in Figure~\ref{fig:HS71}.

\begin{figure}
  \centering
\begin{footnotesize}
\begin{verbatim}
# tell ampl to use the ipopt executable as a solver
# make sure ipopt is in the path!
option solver ipopt;

# declare the variables and their bounds, 
# set notation could be used, but this is straightforward
var x1 >= 1, <= 5; 
var x2 >= 1, <= 5; 
var x3 >= 1, <= 5; 
var x4 >= 1, <= 5;

# specify the objective function
minimize obj:
                x1 * x4 * (x1 + x2 + x3) + x3;
        
# specify the constraints
s.t.
        inequality:
                x1 * x2 * x3 * x4 >= 25;
                
        equality:
                x1^2 + x2^2 + x3^2 +x4^2 = 40;

# specify the starting point            
let x1 := 1;
let x2 := 5;
let x3 := 5;
let x4 := 1;

# solve the problem
solve;

# print the solution
display x1;
display x2;
display x3;
display x4;
\end{verbatim}
\end{footnotesize}
  
  \caption{AMPL model file hs071\_ampl.mod}
  \label{fig:HS71}
\end{figure}

The line, ``{\tt option solver ipopt;}'' tells AMPL to use \Ipopt\ as
the solver. The \Ipopt\ executable (installed in
Section~\ref{sec.comp_and_inst}) must be in the {\tt PATH} for AMPL to
find it. The remaining lines specify the problem in AMPL format. The
problem can now be solved by starting AMPL and loading the mod file:
\begin{verbatim}
$ ampl
> model hs071_ampl.mod;
.
.
.
\end{verbatim}
%$
The problem will be solved using \Ipopt\ and the solution will be
displayed.

At this point, AMPL users may wish to skip the sections about
interfacing with code, but should read Section \ref{sec.options}
concerning \Ipopt\ options, and Section \ref{sec.output} which
explains the output displayed by \Ipopt.

\subsection{Interfacing with \Ipopt\ through code}
In order to solve a problem, \Ipopt\ needs more information than just
the problem definition (for example, the derivative information). If
you are using a modeling language like AMPL, the extra information is
provided by the modeling tool and the \Ipopt\ interface. When
interfacing with \Ipopt\ through your own code, however, you must
provide this additional information.

\begin{figure}
\begin{enumerate}
\item Problem dimensions \label{it.prob_dim}
  \begin{itemize}
  \item number of variables
  \item number of constraints
  \end{itemize}
\item Problem bounds
  \begin{itemize}
  \item variable bounds
  \item constraint bounds
  \end{itemize}
\item Initial starting point
  \begin{itemize}
  \item Initial values for the primal $x$ variables
  \item Initial values for the multipliers (only
    required for a warm start option)
  \end{itemize}
\item Problem Structure \label{it.prob_struct}
  \begin{itemize}
  \item number of nonzeros in the Jacobian of the constraints
  \item number of nonzeros in the Hessian of the Lagrangian function
  \item sparsity structure of the Jacobian of the constraints
  \item sparsity structure of the Hessian of the Lagrangian function
  \end{itemize}
\item Evaluation of Problem Functions \label{it.prob_eval} \\
  Information evaluated using a given point ($x,
  \lambda, \sigma_f$ coming from \Ipopt)
  \begin{itemize}
  \item Objective function, $f(x)$
  \item Gradient of the objective $\nabla f(x)$
  \item Constraint function values, $g(x)$
  \item Jacobian of the constraints, $\nabla g(x)^T$
  \item Hessian of the Lagrangian function, 
    $\sigma_f \nabla^2 f(x) + \sum_{i=1}^m\lambda_i\nabla^2
    g_i(x)$ \\
    (this is not required if a quasi-Newton options is chosen to
    approximate the second derivatives)
  \end{itemize}
\end{enumerate}
\caption{Information required by \Ipopt}
\label{fig.required_info}
\end{figure}
%\vspace{0.1in}
The information required by \Ipopt\ is shown in Figure
\ref{fig.required_info}. The problem dimensions and bounds are
straightforward and come solely from the problem definition. The
initial starting point is used by the algorithm when it begins
iterating to solve the problem. If \Ipopt\ has difficulty converging, or
if it converges to a locally infeasible point, adjusting the starting
point may help.  Depending on the starting point, \Ipopt\ may also
converge to different local solutions.

Providing the sparsity structure of derivative matrices is a bit more
involved. \Ipopt\ is a nonlinear programming solver that is designed
for solving large-scale, sparse problems. While \Ipopt\ can be
customized for a variety of matrix formats, the triplet format is used
for the standard interfaces in this tutorial. For an overview of the
triplet format for sparse matrices, see Appendix~\ref{app.triplet}.
Before solving the problem, \Ipopt\ needs to know the number of
nonzero elements and the sparsity structure (row and column indices of
each of the nonzero entries) of the constraint Jacobian and the
Lagrangian function Hessian. Once defined, this nonzero structure MUST
remain constant for the entire optimization procedure. This means that
the structure needs to include entries for any element that could ever
be nonzero, not only those that are nonzero at the starting point.

As \Ipopt\ iterates, it will need the values for
Item~\ref{it.prob_eval}. in Figure~\ref{fig.required_info} evaluated at
particular points. Before we can begin coding the interface, however,
we need to work out the details of these equations symbolically for
example problem (\ref{eq:ex_obj})-(\ref{eq:ex_bounds}).

The gradient of the objective $f(x)$ is given by
\[%\begin{equation}
\left[
\begin{array}{c}
x_1 x_4 + x_4 (x_1 + x_2 + x_3) \\
x_1 x_4 \\
x_1 x_4 + 1 \\
x_1 (x_1 + x_2 + x_3)
\end{array}
\right],
\]%\end{equation}
and the Jacobian of the constraints $g(x)$ is
\[%\begin{equation}
\left[
\begin{array}{cccc}
x_2 x_3 x_4     & x_1 x_3 x_4   & x_1 x_2 x_4   & x_1 x_2 x_3   \\
2 x_1           & 2 x_2         & 2 x_3         & 2 x_4
\end{array}
\right].
\]%\end{equation}

We also need to determine the Hessian of the Lagrangian\footnote{If a
  quasi-Newton option is chosen to approximate the second derivatives,
  this is not required.  However, if second derivatives can be
  computed, it is often worthwhile to let \Ipopt\ use them, since the
  algorithm is then usually more robust and converges faster.  More on
  the quasi-Newton approximation in Section~\ref{sec:quasiNewton}.}.
The Lagrangian function for the NLP
(\ref{eq:ex_obj})-(\ref{eq:ex_bounds}) is defined as $f(x) + g(x)^T
\lambda$ and the Hessian of the Lagrangian function is, technically, $
\nabla^2 f(x_k) + \sum_{i=1}^m\lambda_i\nabla^2 g_i(x_k)$.  However,
so that \Ipopt\ can ask for the Hessian of the objective or the
constraints independently if required, we introduce a factor
($\sigma_f$) in front of the objective term.
%
For \Ipopt\ then, the symbolic form of the Hessian of the
Lagrangian is
\begin{equation}\label{eq:IpoptLAG}
\sigma_f \nabla^2 f(x_k) + \sum_{i=1}^m\lambda_i\nabla^2 g_i(x_k)
\end{equation}
(with the $\sigma_f$ parameter), and for the example problem this becomes
%\begin{eqnarray}
%{\cal L}(x,\lambda) &{=}& f(x) + c(x)^T \lambda \nonumber \\
%&{=}& \left(x_1 x_4 (x_1 + x_2 + x_3)  +  x_3\right) 
%+ \left(x_1 x_2 x_3 x_4\right) \lambda_1 \nonumber \\
%&& \;\;\;\;\;+ \left(x_1^2 + x_2^2 + x_3^2 + x_4^2\right) \lambda_2 
%- \displaystyle \sum_{i \in 1..4} z^L_i + \sum_{i \in 1..4} z^U_i
%\end{eqnarray}
\[%\begin{equation}
\sigma_f \left[
\begin{array}{cccc}
2 x_4           & x_4           & x_4           & 2 x_1 + x_2 + x_3     \\
x_4             & 0             & 0             & x_1                   \\
x_4             & 0             & 0             & x_1                   \\
2 x_1+x_2+x_3   & x_1           & x_1           & 0
\end{array}
\right]
+
\lambda_1
\left[
\begin{array}{cccc}
0               & x_3 x_4       & x_2 x_4       & x_2 x_3       \\
x_3 x_4         & 0             & x_1 x_4       & x_1 x_3       \\
x_2 x_4         & x_1 x_4       & 0             & x_1 x_2       \\
x_2 x_3         & x_1 x_3       & x_1 x_2       & 0 
\end{array}
\right]
+
\lambda_2
\left[
\begin{array}{cccc}
2       & 0     & 0     & 0     \\
0       & 2     & 0     & 0     \\
0       & 0     & 2     & 0     \\
0       & 0     & 0     & 2
\end{array}
\right]
\]%\end{equation}
where the first term comes from the Hessian of the objective function,
and the second and third term from the Hessian of the constraints
(\ref{eq:ex_ineq}) and (\ref{eq:ex_equ}), respectively. Therefore, the
dual variables $\lambda_1$ and $\lambda_2$ are then the multipliers
for constraints (\ref{eq:ex_ineq}) and (\ref{eq:ex_equ}), respectively.

%C =============================================================================
%C
%C     This is an example for the usage of IPOPT.
%C     It implements problem 71 from the Hock-Schittkowsky test suite:
%C
%C     min   x1*x4*(x1 + x2 + x3)  +  x3
%C     s.t.  x1*x2*x3*x4                   >=  25
%C           x1**2 + x2**2 + x3**2 + x4**2  =  40
%C           1 <=  x1,x2,x3,x4  <= 5
%C
%C     Starting point:
%C        x = (1, 5, 5, 1)
%C
%C     Optimal solution:
%C        x = (1.00000000, 4.74299963, 3.82114998, 1.37940829)
%C
%C =============================================================================
\vspace{\baselineskip}

The remaining sections of the tutorial will lead you through
the coding required to solve example problem
(\ref{eq:ex_obj})--(\ref{eq:ex_bounds}) using, first C++, then C, and finally
Fortran. Completed versions of these examples can be found in {\tt
\$IPOPTDIR/Examples} under {\tt hs071\_cpp}, {\tt hs071\_c}, {\tt
hs071\_f}.

As a user, you are responsible for coding two sections of the program
that solves a problem using \Ipopt: the main executable (e.g., {\tt
  main}) and the problem representation.  Typically, you will write an
executable that prepares the problem, and then passes control over to
\Ipopt\ through an {\tt Optimize} or {\tt Solve} call. In this call,
you will give \Ipopt\ everything that it requires to call back to your
code whenever it needs functions evaluated (like the objective
function, the Jacobian of the constraints, etc.).  In each of the
three sections that follow (C++, C, and Fortran), we will first
discuss how to code the problem representation, and then how to code
the executable.

\subsection{The C++ Interface}
This tutorial assumes that you are familiar with the C++ programming
language, however, we will lead you through each step of the
implementation. For the problem representation, we will create a class
that inherits off of the pure virtual base class, {\tt TNLP} ({\tt
  IpTNLP.hpp}). For the executable (the {\tt main} function) we will
make the call to \Ipopt\ through the {\tt IpoptApplication} class
({\tt IpIpoptApplication.hpp}). In addition, we will also be using the
{\tt SmartPtr} class ({\tt IpSmartPtr.hpp}) which implements a reference
counting pointer that takes care of memory management (object
deletion) for you (for details, see Appendix~\ref{app.smart_ptr}).

After ``\texttt{make install}'' (see Section~\ref{sec.comp_and_inst}),
the header files are installed in \texttt{\$IPOPTDIR/include/ipopt}
(or in \texttt{\$PREFIX/include/ipopt} if the switch
\verb|--prefix=$PREFIX| was used for {\tt configure}).

\subsubsection{Coding the Problem Representation}\label{sec.cpp_problem}
We provide the information required in Figure \ref{fig.required_info}
by coding the {\tt HS071\_NLP} class, a specific implementation of the
{\tt TNLP} base class. In the executable, we will create an instance
of the {\tt HS071\_NLP} class and give this class to \Ipopt\ so it can
evaluate the problem functions through the {\tt TNLP} interface. If
you have any difficulty as the implementation proceeds, have a look at
the completed example in the {\tt Examples/hs071\_cpp} directory.

Start by creating a new directory under Examples, called {\tt
  MyExample} and create the files {\tt hs071\_nlp.hpp} and {\tt
  hs071\_nlp.cpp}. In {\tt hs071\_nlp.hpp}, include {\tt IpTNLP.hpp}
(the base class), tell the compiler that we are using the \Ipopt\
namespace, and create the declaration of the {\tt HS071\_NLP} class,
inheriting off of {\tt TNLP}. Have a look at the {\tt TNLP} class in
{\tt IpTNLP.hpp}; you will see eight pure virtual methods that we must
implement. Declare these methods in the header file.  Implement each
of the methods in {\tt HS071\_NLP.cpp} using the descriptions given
below. In {\tt hs071\_nlp.cpp}, first include the header file for your
class and tell the compiler that you are using the \Ipopt\ namespace.
A full version of these files can be found in the {\tt
  Examples/hs071\_cpp} directory.

It is very easy to make mistakes in the implementation of the function
evaluation methods, in particular regarding the derivatives.  \Ipopt\
has a feature that can help you to debug the derivative code, using
finite differences, see Section~\ref{sec:deriv-checker}.

Note that the return value of any {\tt bool}-valued function should be
{\tt true}, unless an error occured, for example, because the value of
a problem function could not be evaluated at the required point.

\paragraph{Method {\texttt{get\_nlp\_info}}} with prototype
\begin{verbatim}
virtual bool get_nlp_info(Index& n, Index& m, Index& nnz_jac_g,
                          Index& nnz_h_lag, IndexStyleEnum& index_style)
\end{verbatim}
Give \Ipopt\ the information about the size of the problem (and hence,
the size of the arrays that it needs to allocate). 
\begin{itemize}
\item {\tt n}: (out), the number of variables in the problem (dimension of $x$).
\item {\tt m}: (out), the number of constraints in the problem (dimension of $g(x)$).
\item {\tt nnz\_jac\_g}: (out), the number of nonzero entries in the Jacobian.
\item {\tt nnz\_h\_lag}: (out), the number of nonzero entries in the Hessian.
\item {\tt index\_style}: (out), the numbering style used for row/col entries in the sparse matrix
format ({\tt C\_STYLE}: 0-based, {\tt FORTRAN\_STYLE}: 1-based; see
also Appendix~\ref{app.triplet}).
\end{itemize}
\Ipopt\ uses this information when allocating the arrays that
it will later ask you to fill with values. Be careful in this method
since incorrect values will cause memory bugs which may be very
difficult to find.

Our example problem has 4 variables (n), and 2 constraints (m). The
constraint Jacobian for this small problem is actually dense and has 8
nonzeros (we still need to represent this Jacobian using the sparse
matrix triplet format). The Hessian of the Lagrangian has 10
``symmetric'' nonzeros (i.e., nonzeros in the lower left triangular
part.).  Keep in mind that the number of nonzeros is the total number
of elements that may \emph{ever} be nonzero, not just those that are
nonzero at the starting point. This information is set once for the
entire problem.

\begin{footnotesize}
\begin{verbatim}
bool HS071_NLP::get_nlp_info(Index& n, Index& m, Index& nnz_jac_g, 
                             Index& nnz_h_lag, IndexStyleEnum& index_style)
{
  // The problem described in HS071_NLP.hpp has 4 variables, x[0] through x[3]
  n = 4;

  // one equality constraint and one inequality constraint
  m = 2;

  // in this example the Jacobian is dense and contains 8 nonzeros
  nnz_jac_g = 8;

  // the Hessian is also dense and has 16 total nonzeros, but we
  // only need the lower left corner (since it is symmetric)
  nnz_h_lag = 10;

  // use the C style indexing (0-based)
  index_style = TNLP::C_STYLE;

  return true;
}
\end{verbatim}
\end{footnotesize}

\paragraph{Method {\texttt{get\_bounds\_info}}} with prototype
\begin{verbatim}
virtual bool get_bounds_info(Index n, Number* x_l, Number* x_u,
                             Index m, Number* g_l, Number* g_u)
\end{verbatim}
Give \Ipopt\ the value of the bounds on the variables and constraints.
\begin{itemize}
\item {\tt n}: (in), the number of variables in the problem (dimension of $x$). 
\item {\tt x\_l}: (out) the lower bounds $x^L$ for $x$. 
\item {\tt x\_u}: (out) the upper bounds $x^U$ for $x$.
\item {\tt m}: (in), the number of constraints in the problem (dimension of $g(x)$).
\item {\tt g\_l}: (out) the lower bounds $g^L$ for $g(x)$. 
\item {\tt g\_u}: (out) the upper bounds $g^U$ for $g(x)$.
\end{itemize}
The values of {\tt n} and {\tt m} that you specified in {\tt
  get\_nlp\_info} are passed to you for debug checking.  Setting a
lower bound to a value less than or equal to the value of the option
{\tt nlp\_lower\_bound\_inf} will cause \Ipopt\ to assume no lower
bound. Likewise, specifying the upper bound above or equal to the
value of the option {\tt nlp\_upper\_bound\_inf} will cause \Ipopt\ to
assume no upper bound.  These options, {\tt nlp\_lower\_bound\_inf}
and {\tt nlp\_upper\_bound\_inf}, are set to $-10^{19}$ and $10^{19}$,
respectively, by default, but may be modified by changing the options
(see Section \ref{sec.options}).

In our example, the first constraint has a lower bound of $25$ and no upper
bound, so we set the lower bound of constraint {\tt [0]} to $25$ and
the upper bound to some number greater than $10^{19}$. The second
constraint is an equality constraint and we set both bounds to
$40$. \Ipopt\ recognizes this as an equality constraint and does not
treat it as two inequalities.

\begin{footnotesize}
\begin{verbatim}
bool HS071_NLP::get_bounds_info(Index n, Number* x_l, Number* x_u,
                                Index m, Number* g_l, Number* g_u)
{
  // here, the n and m we gave IPOPT in get_nlp_info are passed back to us.
  // If desired, we could assert to make sure they are what we think they are.
  assert(n == 4);
  assert(m == 2);

  // the variables have lower bounds of 1
  for (Index i=0; i<4; i++) {
    x_l[i] = 1.0;
  }

  // the variables have upper bounds of 5
  for (Index i=0; i<4; i++) {
    x_u[i] = 5.0;
  }

  // the first constraint g1 has a lower bound of 25
  g_l[0] = 25;
  // the first constraint g1 has NO upper bound, here we set it to 2e19.
  // Ipopt interprets any number greater than nlp_upper_bound_inf as 
  // infinity. The default value of nlp_upper_bound_inf and nlp_lower_bound_inf
  // is 1e19 and can be changed through ipopt options.
  g_u[0] = 2e19;

  // the second constraint g2 is an equality constraint, so we set the 
  // upper and lower bound to the same value
  g_l[1] = g_u[1] = 40.0;

  return true;
}
\end{verbatim}
\end{footnotesize}

\paragraph{Method {\texttt{get\_starting\_point}}} with prototype
\begin{verbatim}
virtual bool get_starting_point(Index n, bool init_x, Number* x,
                                bool init_z, Number* z_L, Number* z_U,
                                Index m, bool init_lambda, Number* lambda)
\end{verbatim}
Give \Ipopt\ the starting point before it begins iterating.
\begin{itemize}
\item {\tt n}: (in), the number of variables in the problem (dimension of $x$). 
\item {\tt init\_x}: (in), if true, this method must provide an initial value for $x$.
\item {\tt x}: (out), the initial values for the primal variables, $x$.
\item {\tt init\_z}: (in), if true, this method must provide an initial value 
        for the bound multipliers $z^L$ and $z^U$.
\item {\tt z\_L}: (out), the initial values for the bound multipliers, $z^L$.
\item {\tt z\_U}: (out), the initial values for the bound multipliers, $z^U$.
\item {\tt m}: (in), the number of constraints in the problem (dimension of $g(x)$).
\item {\tt init\_lambda}: (in), if true, this method must provide an initial value 
        for the constraint multipliers, $\lambda$.
\item {\tt lambda}: (out), the initial values for the constraint multipliers, $\lambda$.
\end{itemize}

The variables {\tt n} and {\tt m} are passed in for your convenience.
These variables will have the same values you specified in {\tt
  get\_nlp\_info}.

Depending on the options that have been set, \Ipopt\ may or may not
require bounds for the primal variables $x$, the bound multipliers
$z^L$ and $z^U$, and the constraint multipliers $\lambda$. The boolean
flags {\tt init\_x}, {\tt init\_z}, and {\tt init\_lambda} tell you
whether or not you should provide initial values for $x$, $z^L$, $z^U$, or
$\lambda$ respectively. The default options only require an initial
value for the primal variables $x$.  Note, the initial values for
bound multiplier components for ``infinity'' bounds
($x_L^{(i)}=-\infty$ or $x_U^{(i)}=\infty$) are ignored.

In our example, we provide initial values for $x$ as specified in the
example problem. We do not provide any initial values for the dual
variables, but use an assert to immediately let us know if we are ever
asked for them.

\begin{footnotesize}
\begin{verbatim}
bool HS071_NLP::get_starting_point(Index n, bool init_x, Number* x,
                                   bool init_z, Number* z_L, Number* z_U,
                                   Index m, bool init_lambda,
                                   Number* lambda)
{
  // Here, we assume we only have starting values for x, if you code
  // your own NLP, you can provide starting values for the dual variables
  // if you wish to use a warmstart option
  assert(init_x == true);
  assert(init_z == false);
  assert(init_lambda == false);

  // initialize to the given starting point
  x[0] = 1.0;
  x[1] = 5.0;
  x[2] = 5.0;
  x[3] = 1.0;

  return true;
}
\end{verbatim}
\end{footnotesize}

\paragraph{Method {\texttt{eval\_f}}} with prototype
\begin{verbatim}
virtual bool eval_f(Index n, const Number* x, 
                    bool new_x, Number& obj_value)
\end{verbatim}
Return the value of the objective function at the point $x$.
\begin{itemize}
\item {\tt n}: (in), the number of variables in the problem (dimension
  of $x$).
\item {\tt x}: (in), the values for the primal variables, $x$, at which
  $f(x)$ is to be evaluated.
\item {\tt new\_x}: (in), false if any evaluation method was
  previously called with the same values in {\tt x}, true otherwise.
\item {\tt obj\_value}: (out) the value of the objective function
  ($f(x)$).
\end{itemize}

The boolean variable {\tt new\_x} will be false if the last call to
any of the evaluation methods ({\tt eval\_*}) used the same $x$
values. This can be helpful when users have efficient implementations
that calculate multiple outputs at once. \Ipopt\ internally caches
results from the {\tt TNLP} and generally, this flag can be ignored.

The variable {\tt n} is passed in for your convenience. This variable
will have the same value you specified in {\tt get\_nlp\_info}.

For our example, we ignore the {\tt new\_x} flag and calculate the objective.

\begin{footnotesize}
\begin{verbatim}
bool HS071_NLP::eval_f(Index n, const Number* x, bool new_x, Number& obj_value)
{
  assert(n == 4);

  obj_value = x[0] * x[3] * (x[0] + x[1] + x[2]) + x[2];

  return true;
}
\end{verbatim}
\end{footnotesize}

\paragraph{Method {\texttt{eval\_grad\_f}}} with prototype
\begin{verbatim}
virtual bool eval_grad_f(Index n, const Number* x, bool new_x, 
                         Number* grad_f)
\end{verbatim}
Return the gradient of the objective function at the point $x$.
\begin{itemize}
\item {\tt n}: (in), the number of variables in the problem (dimension of $x$). 
\item {\tt x}: (in), the values for the primal variables, $x$, at which
  $\nabla f(x)$ is to be evaluated.
\item {\tt new\_x}: (in), false if any evaluation method was previously called 
        with the same values in {\tt x}, true otherwise.
\item {\tt grad\_f}: (out) the array of values for the gradient of the 
        objective function ($\nabla f(x)$).
\end{itemize}

The gradient array is in the same order as the $x$ variables (i.e., the
gradient of the objective with respect to {\tt x[2]} should be put in
{\tt grad\_f[2]}).

The boolean variable {\tt new\_x} will be false if the last call to
any of the evaluation methods ({\tt eval\_*}) used the same $x$
values. This can be helpful when users have efficient implementations
that calculate multiple outputs at once. \Ipopt\ internally caches
results from the {\tt TNLP} and generally, this flag can be ignored.

The variable {\tt n} is passed in for your convenience. This
variable will have the same value you specified in {\tt
get\_nlp\_info}.

In our example, we ignore the {\tt new\_x} flag and calculate the
values for the gradient of the objective.

\begin{footnotesize}
\begin{verbatim}
bool HS071_NLP::eval_grad_f(Index n, const Number* x, bool new_x, Number* grad_f)
{
  assert(n == 4);

  grad_f[0] = x[0] * x[3] + x[3] * (x[0] + x[1] + x[2]);
  grad_f[1] = x[0] * x[3];
  grad_f[2] = x[0] * x[3] + 1;
  grad_f[3] = x[0] * (x[0] + x[1] + x[2]);

  return true;
}
\end{verbatim}
\end{footnotesize}

\paragraph{Method {\texttt{eval\_g}}} with prototype
\begin{verbatim}
virtual bool eval_g(Index n, const Number* x, 
                    bool new_x, Index m, Number* g)
\end{verbatim}
Return the value of the constraint function at the point $x$.
\begin{itemize}
\item {\tt n}: (in), the number of variables in the problem (dimension of $x$). 
\item {\tt x}: (in), the values for the primal variables, $x$, at
  which the constraint functions,
  $g(x)$, are to be evaluated.
\item {\tt new\_x}: (in), false if any evaluation method was previously called 
        with the same values in {\tt x}, true otherwise.
\item {\tt m}: (in), the number of constraints in the problem (dimension of $g(x)$).
\item {\tt g}: (out) the array of constraint function values, $g(x)$.
\end{itemize}

The values returned in {\tt g} should be only the $g(x)$ values, 
do not add or subtract the bound values $g^L$ or $g^U$.

The boolean variable {\tt new\_x} will be false if the last call to
any of the evaluation methods ({\tt eval\_*}) used the same $x$
values. This can be helpful when users have efficient implementations
that calculate multiple outputs at once. \Ipopt\ internally caches
results from the {\tt TNLP} and generally, this flag can be ignored.

The variables {\tt n} and {\tt m} are passed in for your convenience.
These variables will have the same values you specified in {\tt
  get\_nlp\_info}.

In our example, we ignore the {\tt new\_x} flag and calculate the
values of constraint functions.

\begin{footnotesize}
\begin{verbatim}
bool HS071_NLP::eval_g(Index n, const Number* x, bool new_x, Index m, Number* g)
{
  assert(n == 4);
  assert(m == 2);

  g[0] = x[0] * x[1] * x[2] * x[3];
  g[1] = x[0]*x[0] + x[1]*x[1] + x[2]*x[2] + x[3]*x[3];

  return true;
} 
\end{verbatim}
\end{footnotesize}

\paragraph{Method {\texttt{eval\_jac\_g}}} with prototype
\begin{verbatim}
virtual bool eval_jac_g(Index n, const Number* x, bool new_x,
                        Index m, Index nele_jac, Index* iRow, 
                        Index *jCol, Number* values)
\end{verbatim}
Return either the sparsity structure of the Jacobian of the
constraints, or the values for the Jacobian of the constraints at the
point $x$.
\begin{itemize}
\item {\tt n}: (in), the number of variables in the problem (dimension of $x$). 
\item {\tt x}: (in), the values for the primal variables, $x$, at which
  the constraint Jacobian, $\nabla g(x)^T$, is to be evaluated.
\item {\tt new\_x}: (in), false if any evaluation method was previously called 
        with the same values in {\tt x}, true otherwise.
\item {\tt m}: (in), the number of constraints in the problem (dimension of $g(x)$).
\item {\tt n\_ele\_jac}: (in), the number of nonzero elements in the 
        Jacobian (dimension of {\tt iRow}, {\tt jCol}, and {\tt values}).
\item {\tt iRow}: (out), the row indices of entries in the Jacobian of the constraints.
\item {\tt jCol}: (out), the column indices of entries in the Jacobian of the constraints.
\item {\tt values}: (out), the values of the entries in the Jacobian of the constraints.
\end{itemize}

The Jacobian is the matrix of derivatives where the derivative of
constraint $g^{(i)}$ with respect to variable $x^{(j)}$ is placed in
row $i$ and column $j$. See Appendix \ref{app.triplet} for a
discussion of the sparse matrix format used in this method.

If the {\tt iRow} and {\tt jCol} arguments are not {\tt NULL}, then
\Ipopt\ wants you to fill in the sparsity structure of the Jacobian
(the row and column indices only). At this time, the {\tt x} argument
and the {\tt values} argument will be {\tt NULL}.

If the {\tt x} argument and the {\tt values} argument are not {\tt
  NULL}, then \Ipopt\ wants you to fill in the values of the Jacobian
as calculated from the array {\tt x} (using the same order as you used
when specifying the sparsity structure). At this time, the {\tt iRow}
and {\tt jCol} arguments will be {\tt NULL};

The boolean variable {\tt new\_x} will be false if the last call to
any of the evaluation methods ({\tt eval\_*}) used the same $x$
values. This can be helpful when users have efficient implementations
that calculate multiple outputs at once. \Ipopt\ internally caches
results from the {\tt TNLP} and generally, this flag can be ignored.

The variables {\tt n}, {\tt m}, and {\tt nele\_jac} are passed in for
your convenience. These arguments will have the same values you
specified in {\tt get\_nlp\_info}.

In our example, the Jacobian is actually dense, but we still
specify it using the sparse format.

\begin{footnotesize}
\begin{verbatim}
bool HS071_NLP::eval_jac_g(Index n, const Number* x, bool new_x,
                           Index m, Index nele_jac, Index* iRow, Index *jCol,
                           Number* values)
{
  if (values == NULL) {
    // return the structure of the Jacobian

    // this particular Jacobian is dense
    iRow[0] = 0; jCol[0] = 0;
    iRow[1] = 0; jCol[1] = 1;
    iRow[2] = 0; jCol[2] = 2;
    iRow[3] = 0; jCol[3] = 3;
    iRow[4] = 1; jCol[4] = 0;
    iRow[5] = 1; jCol[5] = 1;
    iRow[6] = 1; jCol[6] = 2;
    iRow[7] = 1; jCol[7] = 3;
  }
  else {
    // return the values of the Jacobian of the constraints
    
    values[0] = x[1]*x[2]*x[3]; // 0,0
    values[1] = x[0]*x[2]*x[3]; // 0,1
    values[2] = x[0]*x[1]*x[3]; // 0,2
    values[3] = x[0]*x[1]*x[2]; // 0,3

    values[4] = 2*x[0]; // 1,0
    values[5] = 2*x[1]; // 1,1
    values[6] = 2*x[2]; // 1,2
    values[7] = 2*x[3]; // 1,3
  }

  return true;
}
\end{verbatim}
\end{footnotesize}

\paragraph{Method {\texttt{eval\_h}}} with prototype
\begin{verbatim}
virtual bool eval_h(Index n, const Number* x, bool new_x,
                    Number obj_factor, Index m, const Number* lambda,
                    bool new_lambda, Index nele_hess, Index* iRow,
                    Index* jCol, Number* values)
\end{verbatim}
Return either the sparsity structure of the Hessian of the Lagrangian, or the values of the 
Hessian of the Lagrangian (\ref{eq:IpoptLAG}) for the given values for $x$,
$\sigma_f$, and $\lambda$.
\begin{itemize}
\item {\tt n}: (in), the number of variables in the problem (dimension
  of $x$).
\item {\tt x}: (in), the values for the primal variables, $x$, at which
  the Hessian is to be evaluated.
\item {\tt new\_x}: (in), false if any evaluation method was previously called 
        with the same values in {\tt x}, true otherwise.
\item {\tt obj\_factor}: (in), factor in front of the objective term
  in the Hessian, $sigma_f$.
\item {\tt m}: (in), the number of constraints in the problem (dimension of $g(x)$).
\item {\tt lambda}: (in), the values for the constraint multipliers,
  $\lambda$, at which the Hessian is to be evaluated.
\item {\tt new\_lambda}: (in), false if any evaluation method was
  previously called with the same values in {\tt lambda}, true
  otherwise.
\item {\tt nele\_hess}: (in), the number of nonzero elements in the
  Hessian (dimension of {\tt iRow}, {\tt jCol}, and {\tt values}).
\item {\tt iRow}: (out), the row indices of entries in the Hessian.
\item {\tt jCol}: (out), the column indices of entries in the Hessian.
\item {\tt values}: (out), the values of the entries in the Hessian.
\end{itemize}

The Hessian matrix that \Ipopt\ uses is defined in
Eq.~\ref(eq:IpoptLAG).  See Appendix \ref{app.triplet} for a
discussion of the sparse symmetric matrix format used in this method.

If the {\tt iRow} and {\tt jCol} arguments are not {\tt NULL}, then
\Ipopt\ wants you to fill in the sparsity structure of the Hessian
(the row and column indices for the lower or upper triangular part
only). In this case, the {\tt x}, {\tt lambda}, and {\tt values}
arrays will be {\tt NULL}.

If the {\tt x}, {\tt lambda}, and {\tt values} arrays are not {\tt
  NULL}, then \Ipopt\ wants you to fill in the values of the Hessian
as calculated using {\tt x} and {\tt lambda} (using the same order as
you used when specifying the sparsity structure). In this case, the
{\tt iRow} and {\tt jCol} arguments will be {\tt NULL}.

The boolean variables {\tt new\_x} and {\tt new\_lambda} will both be
false if the last call to any of the evaluation methods ({\tt
  eval\_*}) used the same values. This can be helpful when users have
efficient implementations that calculate multiple outputs at once.
\Ipopt\ internally caches results from the {\tt TNLP} and generally,
this flag can be ignored.

The variables {\tt n}, {\tt m}, and {\tt nele\_hess} are passed in for
your convenience. These arguments will have the same values you
specified in {\tt get\_nlp\_info}.

In our example, the Hessian is dense, but we still specify it using the
sparse matrix format. Because the Hessian is symmetric, we only need to 
specify the lower left corner.

\begin{footnotesize}
\begin{verbatim}
bool HS071_NLP::eval_h(Index n, const Number* x, bool new_x,
                       Number obj_factor, Index m, const Number* lambda,
                       bool new_lambda, Index nele_hess, Index* iRow,
                       Index* jCol, Number* values)
{
  if (values == NULL) {
    // return the structure. This is a symmetric matrix, fill the lower left
    // triangle only.

    // the Hessian for this problem is actually dense
    Index idx=0;
    for (Index row = 0; row < 4; row++) {
      for (Index col = 0; col <= row; col++) {
        iRow[idx] = row; 
        jCol[idx] = col;
        idx++;
      }
    }
    
    assert(idx == nele_hess);
  }
  else {
    // return the values. This is a symmetric matrix, fill the lower left
    // triangle only

    // fill the objective portion
    values[0] = obj_factor * (2*x[3]); // 0,0

    values[1] = obj_factor * (x[3]);   // 1,0
    values[2] = 0;                     // 1,1

    values[3] = obj_factor * (x[3]);   // 2,0
    values[4] = 0;                     // 2,1
    values[5] = 0;                     // 2,2

    values[6] = obj_factor * (2*x[0] + x[1] + x[2]); // 3,0
    values[7] = obj_factor * (x[0]);                 // 3,1
    values[8] = obj_factor * (x[0]);                 // 3,2
    values[9] = 0;                                   // 3,3


    // add the portion for the first constraint
    values[1] += lambda[0] * (x[2] * x[3]); // 1,0
    
    values[3] += lambda[0] * (x[1] * x[3]); // 2,0
    values[4] += lambda[0] * (x[0] * x[3]); // 2,1

    values[6] += lambda[0] * (x[1] * x[2]); // 3,0
    values[7] += lambda[0] * (x[0] * x[2]); // 3,1
    values[8] += lambda[0] * (x[0] * x[1]); // 3,2

    // add the portion for the second constraint
    values[0] += lambda[1] * 2; // 0,0

    values[2] += lambda[1] * 2; // 1,1

    values[5] += lambda[1] * 2; // 2,2

    values[9] += lambda[1] * 2; // 3,3
  }

  return true;
}
\end{verbatim}
\end{footnotesize}

{\bf TODO: User STOP method here?}

\paragraph{Method \texttt{finalize\_solution}} with prototype
\begin{verbatim}
virtual void finalize_solution(SolverReturn status, Index n,
                               const Number* x, const Number* z_L,
                               const Number* z_U, Index m, const Number* g,
                               const Number* lambda, Number obj_value)
\end{verbatim}
This is the only method that is not mentioned in Figure
\ref{fig.required_info}. This method is called by \Ipopt\ after the
algorithm has finished (successfully or even with most errors).
\begin{itemize}
\item {\tt status}: (in), gives the status of the algorithm as
  specified in {\tt IpAlgTypes.hpp},
  \begin{itemize}
  \item {\tt SUCCESS}: Algorithm terminated successfully at a locally
    optimal point, satisfying the convergence tolerances (can be
    specified by options).
  \item {\tt MAXITER\_EXCEEDED}: Maximum number of iterations exceeded
    (can be specified by an option).
  \item {\tt STOP\_AT\_TINY\_STEP}: Algorithm proceeds with very
    little progress.
  \item {\tt STOP\_AT\_ACCEPTABLE\_POINT}: Algorithm stopped at a
    point that was converged, not to ``desired'' tolerances, but to
    ``acceptable'' tolerances (see the {\tt acceptable-...} options).
  \item {\tt LOCAL\_INFEASIBILITY}: Algorithm converged to a point of
    local infeasibility. Problem may be infeasible.
  \item {\tt DIVERGING\_ITERATES}: It seems that the iterates diverge.
  \item {\tt RESTORATION\_FAILURE}: Restoration phase failed,
    algorithm doesn't know how to proceed.
  \item {\tt
      INTERNAL\_ERROR}: An unknown internal error occurred.  Please
    contact the \Ipopt\ authors through the mailing list.
  \end{itemize}
\item {\tt n}: (in), the number of variables in the problem (dimension
  of $x$).
\item {\tt x}: (in), the final values for the primal variables, $x_*$.
\item {\tt z\_L}: (in), the final values for the lower bound
  multipliers, $z^L_*$.
\item {\tt z\_U}: (in), the final values for the upper bound
  multipliers, $z^U_*$.
\item {\tt m}: (in), the number of constraints in the problem
  (dimension of $g(x)$).
\item {\tt g}: (in), the final value of the constraint function
  values, $g(x_*)$.
\item {\tt lambda}: (in), the final values of the constraint
  multipliers, $\lambda_*$.
\item {\tt obj\_value}: (in), the final value of the objective,
  $f(x_*)$.
\end{itemize}

{\bf TODO: Should be provide IpStatistics here?}

This method gives you the return status of the algorithm
(SolverReturn), and the values of the variables, 
the objective and constraint function values when the algorithm exited.

In our example, we will print the values of some of the variables to 
the screen.

\begin{footnotesize}
\begin{verbatim}
void HS071_NLP::finalize_solution(SolverReturn status,
                                  Index n, const Number* x, const Number* z_L,
                                  const Number* z_U, Index m, const Number* g,
                                  const Number* lambda, Number obj_value)
{
  // here is where we would store the solution to variables, or write to a file, etc
  // so we could use the solution. 

  // For this example, we write the solution to the console
  printf("\n\nSolution of the primal variables, x\n");
  for (Index i=0; i<n; i++) {
    printf("x[%d] = %e\n", i, x[i]); 
  }

  printf("\n\nSolution of the bound multipliers, z_L and z_U\n");
  for (Index i=0; i<n; i++) {
    printf("z_L[%d] = %e\n", i, z_L[i]); 
  }
  for (Index i=0; i<n; i++) {
    printf("z_U[%d] = %e\n", i, z_U[i]); 
  }

  printf("\n\nObjective value\n");
  printf("f(x*) = %e\n", obj_value); 
}
\end{verbatim}
\end{footnotesize}

This is all that is required for our {\tt HS071\_NLP} class and 
the coding of the problem representation.
 
\subsubsection{Coding the Executable (\texttt{main})}
Now that we have a problem representation, the {\tt HS071\_NLP} class,
we need to code the main function that will call \Ipopt\ and ask \Ipopt\
to find a solution.

Here, we must create an instance of our problem ({\tt HS071\_NLP}),
create an instance of the \Ipopt\ solver (\texttt{IpoptApplication}),
and ask the solver to find a solution. We always use the
\texttt{SmartPtr} template class instead of raw C++ pointers when
creating and passing \Ipopt\ objects. To find out more information
about smart pointers and the {\tt SmartPtr} implementation used in
\Ipopt, see Appendix \ref{app.smart_ptr}.

Create the file {\tt MyExample.cpp} in the MyExample directory.
Include {\tt HS071\_NLP.hpp} and {\tt IpIpoptApplication.hpp}, tell
the compiler to use the {\tt Ipopt} namespace, and implement the {\tt
  main} function.

\begin{footnotesize}
\begin{verbatim}
#include "IpIpoptApplication.hpp"
#include "hs071_nlp.hpp"

using namespace Ipopt;

int main(int argv, char* argc[])
{
  // Create a new instance of your nlp 
  //  (use a SmartPtr, not raw)
  SmartPtr<TNLP> mynlp = new HS071_NLP();

  // Create a new instance of IpoptApplication
  //  (use a SmartPtr, not raw)
  SmartPtr<IpoptApplication> app = new IpoptApplication();

  // Change some options
  app->Options()->SetNumericValue("tol", 1e-9);
  app->Options()->SetStringValue("mu_strategy", "adaptive");

  // Ask Ipopt to solve the problem
  ApplicationReturnStatus status = app->OptimizeTNLP(mynlp);

  if (status == Solve_Succeeded) {
    printf("\n\n*** The problem solved!\n");
  }
  else {
    printf("\n\n*** The problem FAILED!\n");
  }

  // As the SmartPtrs go out of scope, the reference count
  // will be decremented and the objects will automatically 
  // be deleted.

  return (int) status;
}
\end{verbatim} 
\end{footnotesize}

The first line of code in {\tt main} creates an instance of {\tt
  HS071\_NLP}. We then create an instance of the \Ipopt\ solver, {\tt
  IpoptApplication}. The call to {\tt app->OptimizeTNLP(...)} will run
\Ipopt\ and try to solve the problem. By default, \Ipopt\ will write
to its progress to the console, and return the {\tt SolverReturn}
status.

\subsubsection{Compiling and Testing the Example}
Our next task is to compile and test the code. If you are familiar
with the compiler and linker used on your system, you can build the
code, including the \Ipopt\ library {\tt libipopt.a} (and other
necessary libraries, as listed in the {\tt ipopt\_addlibs\_cpp.txt}
and {\tt ipopt\_addlibs\_f.txt} files).  If you are using Linux/UNIX,
then a sample makefile exists already that was created by configure.
Copy {\tt Examples/hs071\_cpp/Makefile} into your {\tt MyExample}
directory.  This makefile was created for the {\tt hs071\_cpp} code,
but it can be easily modified for your example problem. Edit the file,
making the following changes,

\begin{itemize}
\item change the {\tt EXE} variable \\
{\tt EXE = my\_example}
\item change the {\tt OBJS} variable \\
{\tt OBJS = HS071\_NLP.o MyExample.o}
\end{itemize}
and the problem should compile easily with, \\
{\tt \$ make} \\
Now run the executable,\\ 
{\tt \$ ./my\_example} \\
and you should see output resembling the following,

\begin{footnotesize}
\begin{verbatim}
Total number of variables............................:        4
                     variables with only lower bounds:        0
                variables with lower and upper bounds:        4
                     variables with only upper bounds:        0
Total number of equality constraints.................:        1
Total number of inequality constraints...............:        1
        inequality constraints with only lower bounds:        1
   inequality constraints with lower and upper bounds:        0
        inequality constraints with only upper bounds:        0
 
 iter     objective    inf_pr   inf_du lg(mu)  ||d||  lg(rg) alpha_du alpha_pr  ls
    0   1.7159878e+01 2.01e-02 5.20e-01  -1.0 0.00e+00    -  0.00e+00 0.00e+00   0 y
    1   1.7146308e+01 1.63e-01 1.47e-01  -1.0 1.15e-01    -  9.86e-01 1.00e+00f  1
    2   1.7065508e+01 3.10e-02 8.47e-02  -1.7 1.99e-01    -  9.54e-01 1.00e+00h  1 Nhj
    3   1.7002626e+01 4.10e-02 4.81e-03  -2.5 5.52e-02    -  1.00e+00 1.00e+00h  1
    4   1.7019082e+01 1.20e-03 1.81e-04  -2.5 1.10e-02    -  1.00e+00 1.00e+00h  1
    5   1.7014253e+01 1.80e-04 4.87e-05  -3.8 4.86e-03    -  1.00e+00 1.00e+00h  1
    6   1.7014020e+01 9.25e-07 2.15e-07  -5.7 2.76e-04    -  1.00e+00 1.00e+00h  1
    7   1.7014017e+01 1.01e-10 2.60e-11  -8.6 3.32e-06    -  1.00e+00 1.00e+00h  1
 
Number of Iterations....: 7
 
                                   (scaled)                 (unscaled)
Objective...............:   1.7014017145177885e+01    1.7014017145177885e+01
Dual infeasibility......:   2.5980210027546616e-11    2.5980210027546616e-11
Constraint violation....:   1.8175683180743363e-11    1.8175683180743363e-11
Complementarity.........:   2.5282956951655172e-09    2.5282956951655172e-09
Overall NLP error.......:   2.5282956951655172e-09    2.5282956951655172e-09
 
 
Number of objective function evaluations             = 8
Number of objective gradient evaluations             = 8
Number of equality constraint evaluations            = 8
Number of inequality constraint evaluations          = 8
Number of equality constraint Jacobian evaluations   = 8
Number of inequality constraint Jacobian evaluations = 8
Number of Lagrangian Hessian evaluations             = 9
 
EXIT: Optimal Solution Found.
 
 
Solution of the primal variables, x
x[0] = 1
x[1] = 4.743
x[2] = 3.82115
x[3] = 1.37941
 
 
Solution of the bound multipliers, z_L and z_U
z_L[0] = 1.08787
z_L[1] = 6.69317e-10
z_L[2] = 8.8877e-10
z_L[3] = 6.57011e-09
z_U[0] = 6.26262e-10
z_U[1] = 9.78906e-09
z_U[2] = 2.12283e-09
z_U[3] = 6.92528e-10
 
 
Objective value
f(x*) = 17.014
 
 
*** The problem solved!
\end{verbatim}
\end{footnotesize}

This completes the basic C++ tutorial, but see Section
\ref{sec.output} which explains the standard console output of \Ipopt
and Section \ref{sec.options} for information about the use of options
to customize the behavior of \Ipopt.

The {\tt Examples/ScalableProblems} directory contains another set
of NLP problems coded in C++.

\subsection{The C Interface}\label{sec.cinterface}
The C interface for \Ipopt\ is declared in the header file {\tt
  IpStdCInterface.h}, which is found in\\
\texttt{\$IPOPTDIR/include/ipopt} (or in
\texttt{\$PREFIX/include/ipopt} if the switch
\verb|--prefix=$PREFIX| was used for {\tt configure}); while
reading this section, it will be helpful to have a look at this file.

In order to solve an optimization problem with the C interface, one
has to create an {\tt IpoptProblem}\footnote{{\tt IpoptProblem} is a
  pointer to a C structure; you should not access this structure
  directly, only through the functions provided in the C interface.}
with the function {\tt CreateIpoptProblem}, which later has to be
passed to the {\tt IpoptSolve} function.

The {\tt IpoptProblem} created by {\tt CreateIpoptProblem} contains
the problem dimensions, the variable and constraint bounds, and the
function pointers for callbacks that will be used to evaluate the NLP
problem functions and their derivatives (see also the discussion of
the C++ methods {\tt get\_nlp\_info} and {\tt get\_bounds\_info} in
Section~\ref{sec.cpp_problem} for information about the arguments of
{\tt CreateIpoptProblem}).

The prototypes for the callback functions, {\tt Eval\_F\_CB}, {\tt
  Eval\_Grad\_F\_CB}, etc., are defined in the header file {\tt
  IpStdCInterface.h}.  Their arguments correspond one-to-one to the
arguments for the C++ methods discussed in
Section~\ref{sec.cpp_problem}; for example, for the meaning of $\tt
n$, $\tt x$, $\tt new\_x$, $\tt obj\_value$ in the declaration of {\tt
  Eval\_F\_CB} see the discussion of ``{\tt eval\_f}''.  The callback
functions should return {\tt TRUE}, unless there was a problem doing
the requested function/derivative evaluation at the given point {\tt
  x} (then it should return {\tt FALSE}).

Note the additional argument of type {\tt UserDataPtr} in the callback
functions.  This pointer argument is available for you to communicate
information between the main program that calls {\tt IpoptSolve} and
any of the callback functions.  This pointer is simply passed
unmodified by \Ipopt\ among those functions.  For example, you can
use this to pass constants that define the optimization problem and
are computed before the optimization in the main C program to the
callback functions.

After an {\tt IpoptProblem} has been created, you can set algorithmic
options for \Ipopt\ (see Section~\ref{sec.options}) using the {\tt
  AddIpopt...Option} functions.  Finally, the \Ipopt\ algorithm is
called with {\tt IpoptSolve}, giving \Ipopt\ the {\tt IpoptProblem},
the starting point, and arrays to store the solution values (primal
and dual variables), if desired.  Finally, after everything is done,
you should call {\tt FreeIpoptProblem} to release internal memory that
is still allocated inside \Ipopt.

In the remainder of this section we discuss how the example problem
(\ref{eq:ex_obj})--(\ref{eq:ex_bounds}) can be solved using the C
interface.  A completed version of this example can be found in {\tt
  Examples/hs071\_c}.

% We first create the necessary callback
% functions for evaluating the NLP. As just discussed, the \Ipopt\ C
% interface required callbacks to evaluate the objective value,
% constraints, gradient of the objective, Jacobian of the constraints,
% and the Hessian of the Lagrangian.  These callbacks are implemented
% using function pointers.  Have a look at the C++ implementation for
% {\tt eval\_f}, {\tt eval\_g}, {\tt eval\_grad\_f}, {\tt eval\_jac\_g},
% and {\tt eval\_h} in Section \ref{sec.cpp_problem}. The C
% implementations have somewhat different prototypes, but are
% implemented almost identically to the C++ code.

\vspace{\baselineskip}

In order to implement the example problem on your own, create a new
directory {\tt MyCExample} and create a new file, {\tt
  hs071\_c.c}.  Here, include the interface header file {\tt
  IpStdCInterface.h}, along with other necessary header files, such as
{\tt stdlib.h} and {\tt assert.h}.  Add the prototypes and
implementations for the five callback functions.  Have a look at the
C++ implementation for {\tt eval\_f}, {\tt eval\_g}, {\tt
  eval\_grad\_f}, {\tt eval\_jac\_g}, and {\tt eval\_h} in Section
\ref{sec.cpp_problem}. The C implementations have somewhat different
prototypes, but are implemented almost identically to the C++ code.
See the completed example in {\tt Examples/hs071\_c/hs071\_c.c} if you
are not sure how to do this.

We now need to implement the {\tt main} function, create the {\tt
  IpoptProblem}, set options, and call {\tt IpoptSolve}. The {\tt
  CreateIpoptProblem} function requires the problem dimensions, the
variable and constraint bounds, and the function pointers to the
callback routines. The {\tt IpoptSolve} function requires the {\tt
  IpoptProblem}, the starting point, and allocated arrays for the
solution.  The {\tt main} function from the example is shown next, and
discussed below.

%in Figure~\ref{fig:cexample-main}.
%\begin{figure}
%  \centering
\begin{footnotesize}
\begin{verbatim}
int main()
{
  Index n=-1;                          /* number of variables */
  Index m=-1;                          /* number of constraints */
  Number* x_L = NULL;                  /* lower bounds on x */
  Number* x_U = NULL;                  /* upper bounds on x */
  Number* g_L = NULL;                  /* lower bounds on g */
  Number* g_U = NULL;                  /* upper bounds on g */
  IpoptProblem nlp = NULL;             /* IpoptProblem */
  enum ApplicationReturnStatus status; /* Solve return code */
  Number* x = NULL;                    /* starting point and solution vector */
  Number* mult_x_L = NULL;             /* lower bound multipliers 
					  at the solution */
  Number* mult_x_U = NULL;             /* upper bound multipliers 
					  at the solution */
  Number obj;                          /* objective value */
  Index i;                             /* generic counter */
  
  /* set the number of variables and allocate space for the bounds */
  n=4;
  x_L = (Number*)malloc(sizeof(Number)*n);
  x_U = (Number*)malloc(sizeof(Number)*n);
  /* set the values for the variable bounds */
  for (i=0; i<n; i++) {
    x_L[i] = 1.0;
    x_U[i] = 5.0;
  }

  /* set the number of constraints and allocate space for the bounds */
  m=2;
  g_L = (Number*)malloc(sizeof(Number)*m);
  g_U = (Number*)malloc(sizeof(Number)*m);
  /* set the values of the constraint bounds */
  g_L[0] = 25; g_U[0] = 2e19;
  g_L[1] = 40; g_U[1] = 40;

  /* create the IpoptProblem */
  nlp = CreateIpoptProblem(n, x_L, x_U, m, g_L, g_U, 8, 10, 0, 
			   &eval_f, &eval_g, &eval_grad_f, 
			   &eval_jac_g, &eval_h);
  
  /* We can free the memory now - the values for the bounds have been
     copied internally in CreateIpoptProblem */
  free(x_L);
  free(x_U);
  free(g_L);
  free(g_U);

  /* set some options */
  AddIpoptNumOption(nlp, "tol", 1e-9);
  AddIpoptStrOption(nlp, "mu_strategy", "adaptive");

  /* allocate space for the initial point and set the values */
  x = (Number*)malloc(sizeof(Number)*n);
  x[0] = 1.0;
  x[1] = 5.0;
  x[2] = 5.0;
  x[3] = 1.0;

  /* allocate space to store the bound multipliers at the solution */
  mult_x_L = (Number*)malloc(sizeof(Number)*n);
  mult_x_U = (Number*)malloc(sizeof(Number)*n);

  /* solve the problem */
  status = IpoptSolve(nlp, x, NULL, &obj, NULL, mult_x_L, mult_x_U, NULL);

  if (status == Solve_Succeeded) {
    printf("\n\nSolution of the primal variables, x\n");
    for (i=0; i<n; i++) {
      printf("x[%d] = %e\n", i, x[i]); 
    }

    printf("\n\nSolution of the bound multipliers, z_L and z_U\n");
    for (i=0; i<n; i++) {
      printf("z_L[%d] = %e\n", i, mult_x_L[i]); 
    }
    for (i=0; i<n; i++) {
      printf("z_U[%d] = %e\n", i, mult_x_U[i]); 
    }

    printf("\n\nObjective value\n");
    printf("f(x*) = %e\n", obj); 
  }
 
  /* free allocated memory */
  FreeIpoptProblem(nlp);
  free(x);
  free(mult_x_L);
  free(mult_x_U);

  return 0;
}
\end{verbatim}
\end{footnotesize}
%  \caption{{\tt main} function for C example}
%  \label{fig:cexample-main}
%\end{figure}

Here, we declare all the necessary variables and set the dimensions of
the problem.  The problem has 4 variables, so we set {\tt n} and
allocate space for the variable bounds (don't forget to call {\tt
  free} for each of your {\tt malloc} calls before the end of the
program). We then set the values for the variable bounds.

The problem has 2 constraints, so we set {\tt m} and allocate space
for the constraint bounds. The first constraint has a lower bound of
$25$ and no upper bound.  Here we set the upper bound to
\texttt{2e19}. \Ipopt\ interprets any number greater than or equal to
\texttt{nlp\_upper\_bound\_inf} as infinity. The default value of
\texttt{nlp\_lower\_bound\_inf} and \texttt{nlp\_upper\_bound\_inf} is
\texttt{-1e19} and \texttt{1e19}, respectively, and can be changed
through \Ipopt\ options.  The second constraint is an equality with
right hand side 40, so we set both the upper and the lower bound to
40.

We next create an instance of the {\tt IpoptProblem} by calling {\tt
CreateIpoptProblem}, giving it the problem dimensions and the variable
and constraint bounds. The arguments {\tt nele\_jac} and {\tt
nele\_hess} are the number of elements in Jacobian and the Hessian,
respectively. See Appendix~\ref{app.triplet} for a description of the
sparse matrix format. The {\tt index\_style} argument specifies whether
we want to use C style indexing for the row and column indices of the
matrices or Fortran style indexing. Here, we set it to {\tt 0} to
indicate C style.  We also include the references to each of our
callback functions. \Ipopt\ uses these function pointers to ask for
evaluation of the NLP when required.

After freeing the bound arrays that are no longer required, the next
two lines illustrate how you can change the value of options through
the interface.  \Ipopt\ options can also be changed by creating a {\tt
PARAMS.DAT} file (see Section~\ref{sec.options}). We next allocate
space for the initial point and set the values as given in the problem
definition.

The call to {\tt IpoptSolve} can provide us with information about the
solution, but most of this is optional. Here, we want values for the
bound multipliers at the solution and we allocate space for these.

We can now make the call to {\tt IpoptSolve} and find the solution of
the problem. We pass in the {\tt IpoptProblem}, the starting point
{\tt x} (\Ipopt\ will use this array to return the solution or final
point as well).  The next 5 arguments are pointers so \Ipopt\ can fill
in values at the solution.  If these pointers are set to {\tt NULL},
\Ipopt\ will ignore that entry.  For example, here, we do not want the
constraint function values at the solution or the constraint
multipliers, so we set those entries to {\tt NULL}. We do want the
value of the objective, and the multipliers for the variable bounds.
The last argument is a {\tt void*} for user data. Any pointer you give
here will also be passed to you in the callback functions.

The return code is an {\tt ApplicationReturnStatus} enumeration, see
the header file {\tt ReturnCodes\_inc.h} which is installed along {\tt
  IpStdCInterface.h} in the \Ipopt\ include directory.

After the optimizer terminates, we check the status and print the
solution if successful. Finally, we free the {\tt IpoptProblem} and
the remaining memory, and return from {\tt main}.

\subsection{The Fortran Interface}

The Fortran interface is essentially a wrapper of the C interface
discussed in Section~\ref{sec.cinterface}.  The way to hook up \Ipopt\
in a Fortran program is very similar to how it is done for the C
interface, and the functions of the Fortran interface correspond
one-to-one to the those of the C and C++ interface, including their
arguments.  You can find an implementation of the example problem
(\ref{eq:ex_obj})--(\ref{eq:ex_bounds}) in {\tt
  \$IPOPTDIR/Examples/hs071\_f}.

The only special things to consider are:
\begin{itemize}
\item The return value of the function {\tt IPCREATE} is of an {\tt
    INTEGER} type that must be large enough to capture a pointer
  on the particular machine.  This means, that you have to declare
  the ``handle'' for the IpoptProblem as {\tt INTEGER*8} if your
  program is compiled in 64-bit mode.  All other {\tt INTEGER}-type
  variables must be of the regular type.
\item For the call of {\tt IPSOLVE} (which is the function that is to
  be called to run \Ipopt), all arrays, including those for the dual
  variables, must be given (in contrast to the C interface).  The
  return value {\tt IERR} of this function indicates the outcome of
  the optimization (see the include file {\tt IpReturnCodes.inc} in
  the \Ipopt\ include directory).
\item The return {\tt IERR} value of the remaining functions has to be
  set to zero, unless there was a problem during execution of the
  function call.
\item The callback functions ({\tt EV\_*} in the example) include the
  arguments {\tt IDAT} and {\tt DAT}, which are {\tt INTEGER} and {\tt
    DOUBLE PRECISION} arrays that are passed unmodified between the
  main program calling {\tt IPSOLVE} and the evaluation subroutines
  {\tt EV\_*} (similarly to {\tt UserDataPtr} arguments in the C
  interface).  These arrays can be used to pass ``private'' data
  between the main program and the user-provided Fortran subroutines.

  The last argument of the {\tt EV\_*} subroutines, {\tt IERR}, is to
  be set to 0 by the user on return, unless there was a problem
  during the evaluation of the optimization problem
  function/derivative for the given point {\tt X} (then it should
  return a non-zero value).
\end{itemize}

\section{Special Features}
\subsection{Derivative Checker}\label{sec:deriv-checker}
\subsection{Quasi-Newton Approximation of Second
  Derivatives}\label{sec:quasiNewton}

\section{\Ipopt\ Options}\label{sec.options}
Ipopt has many (maybe too many) options that can be adjusted for the
algorithm.  Options are all identified by a string name, and their
values can be of one of three types: Number (real), Integer, or
String. Number options are used for things like tolerances, integer
options are used for things like maximum number of iterations, and
string options are used for setting algorithm details, like the NLP
scaling method. Options can be set through code, through the AMPL
interface if you are using AMPL, or by creating a {\tt PARAMS.DAT}
file in the directory you are executing \Ipopt.

The {\tt PARAMS.DAT} file is read line by line and each line should
contain the option name, followed by whitespace, and then the
value. Comments can be included with the {\tt \#} symbol. Don't forget
to ensure you have a newline at the end of the file. For example,
\begin{verbatim}
# This is a comment

# Turn off the NLP scaling
nlp_scaling_method none

# Change the initial barrier parameter
mu_init 1e-2

# Set the max number of iterations
max_iter 500
\end{verbatim}
is a valid {\tt PARAMS.DAT} file.

Options can also be set in code. Have a look at the examples to see
how this is done. 

A subset of \Ipopt\ options are available through AMPL. To set options
through AMPL, use the internal AMPL command {\tt options}.  For
example, \\ 
{\tt options ipopt "nlp\_scaling\_method=none mu\_init=1e-2
max\_iter=500"} \\ 
is a valid options command in AMPL. The most common
options are referenced in Appendix~\ref{app.options_ref}. These are also
the options that are available through AMPL using the {\tt options}
command {\bf TODO: CHECK IF THAT IS CORRECT}. To specify other options when using AMPL, you can always
create {\tt PARAMS.DAT}.  Note, the {\tt PARAMS.DAT} file is given
preference when setting options. This way, you can easily override any
options set in a particular executable or AMPL model by specifying new
values in {\tt PARAMS.DAT}.

For a short list of the valid options, see the Appendix
\ref{app.options_ref}. You can print the documentation for all \Ipopt\
options by adding the option, \\

{\tt print\_options\_documentation yes} \\

and running \Ipopt\ (like the AMPL solver executable, for
instance). This will output all of the options documentation to the
console.

\section{\Ipopt\ Output}\label{sec.output}
This section describes the standard \Ipopt\ console output with the
default setting for {\tt print\_level}. The output is designed to
provide a quick summary of each iteration as \Ipopt\ solves the problem.

Before \Ipopt\ starts to solve the problem, it displays the problem
statistics (number of variables, etc.). Note that if you have fixed
variables (both upper and lower bounds are equal), \Ipopt\ may remove
these variables from the problem internally and not include them in
the problem statistics.

Following the problem statistics, \Ipopt\ will begin to solve the
problem and you will see output resembling the following,
\begin{verbatim}
iter    objective    inf_pr   inf_du lg(mu)  ||d||  lg(rg) alpha_du alpha_pr  ls
   0  1.6109693e+01 1.12e+01 5.28e-01   0.0 0.00e+00    -  0.00e+00 0.00e+00   0
   1  1.8029749e+01 9.90e-01 6.62e+01   0.1 2.05e+00    -  2.14e-01 1.00e+00f  1
   2  1.8719906e+01 1.25e-02 9.04e+00  -2.2 5.94e-02   2.0 8.04e-01 1.00e+00h  1
\end{verbatim}
and the columns of output are defined as,
\begin{description}
\item[{\tt iter}:] The current iteration count. This includes regular
  iterations and iterations while in restoration phase. If the
  algorithm is in the restoration phase, the letter {\tt r'} will be
  appended to the iteration number.
\item[{\tt objective}:] The unscaled objective value at the current
  point. During the restoration phase, this value remains the unscaled
  objective value for the original problem.
\item[{\tt inf\_pr}:] The scaled primal infeasibility at the current
  point. During the restoration phase, this value is the primal
  infeasibility of the original problem at the current point.
\item[{\tt inf\_du}:] The scaled dual infeasibility at the current
  point. During the restoration phase, this is the value of the dual
  infeasibility for the restoration phase problem.
\item[{\tt lg(mu)}:] $\log_{10}$ of the value of the barrier parameter mu.
\item[{\tt ||d||}:] The infinity norm (max) of the primal step (for
  the original variables $x$ and the internal slack variables $s$).
  During the restoration phase, this value includes the values of
  additional variables, $p$ and $n$ (see Eq.~(30) in
  \cite{WaecBieg06:mp}).
\item[{\tt lg(rg)}:] $\log_{10}$ of the value of the regularization
  term for the Hessian of the Lagrangian in the augmented system.
\item[{\tt alpha\_du}:] The stepsize for the dual variables.
\item[{\tt alpha\_pr}:] The stepsize for the primal variables.
\item[{\tt ls}:] The number of backtracking line search steps.
\end{description}

When the algorithm terminates, \Ipopt\ will output a message to the
screen based on the return status of the call to {\tt Optimize}. The following
is a list of the possible return codes, their corresponding output message
to the console, and a brief description.
\begin{description}
\item[{\tt Solve\_Succeeded}:] $\;$ \\
  Console Message: {\tt EXIT: Optimal Solution Found.} \\
  This message indicates that \Ipopt\ found a (locally) optimal point
  within the desired tolerances.
\item[{\tt Solved\_To\_Acceptable\_Level}:]  $\;$ \\
  Console Message: {\tt EXIT: Solved To Acceptable Level.} \\
  This indicates that the algorithm did not converge to the
  ``desired'' tolerances, but that it was able to obtain a point
  satisfying the ``acceptable'' tolerance level as specified by {\tt
    acceptable-*} options. This may happen if the desired tolerances
  are too small for the current problem.
\item[{\tt Infeasible\_Problem\_Detected}:]  $\;$ \\
  Console Message: {\tt EXIT: Converged to a point of
    local infeasibility. Problem may be infeasible.} \\
  The restoration phase converged to a point that is a minimizer for
  the constraint violation (in the $\ell_1$-norm), but is not feasible
  for the original problem. This indicates that the problem may be
  infeasible (or at least that the algorithm is stuck at a locally
  infeasible point).  The returned point (the minimizer of the
  constraint violation) might help you to find which constraint is
  causing the problem.  If you believe that the NLP is feasible,
  it might help to start the optimization from a different point.
\item[{\tt Search\_Direction\_Becomes\_Too\_Small}:]  $\;$ \\
  Console Message: {\tt EXIT: Search Direction is becoming Too Small.} \\
  This indicates that \Ipopt\ is calculating very small step sizes and
  making very little progress.  This could happen if the problem has
  been solved to the best numerical accuracy possible given the
  current scaling.
\item[{\tt Maximum\_Iterations\_Exceeded}:]  $\;$ \\
  Console Message: {\tt EXIT: Maximum Number of Iterations Exceeded.} \\
  This indicates that \Ipopt\ has exceeded the maximum number of
  iterations as specified by the option {\tt max\_iter}.
\item[{\tt Restoration\_Failed}:]  $\;$ \\
  Console Message: {\tt EXIT: Restoration Failed!} \\
  This indicates that the restoration phase failed to find a feasible
  point that was acceptable to the filter line search for the original
  problem. This could happen if the problem is highly degenerate, does
  not satisfy the constraint qualification, or if your NLP code
  provides incorrect derivative information.
\item[{\tt Invalid\_Option}:]  $\;$ \\
  Console Message: (details about the particular error
  will be output to the console) \\
  This indicates that there was some problem specifying the options.
  See the specific message for details.
\item[{\tt Not\_Enough\_Degrees\_Of\_Freedom}:]  $\;$ \\
  Console Message: {\tt EXIT: Problem has too few degrees of freedom.} \\
  This indicates that your problem, as specified, has too few degrees
  of freedom. This can happen if you have too many equality
  constraints, or if you fix too many variables (\Ipopt\ removes fixed
  variables).
\item[{\tt Invalid\_Problem\_Definition}:]  $\;$ \\
  Console Message: (no console message, this is a return code for the
  C and Fortran interfaces only.) \\
  This indicates that there was an exception of some sort when
  building the {\tt IpoptProblem} structure in the C or Fortran
  interface. Likely there is an error in your model or the {\tt main}
  routine.
\item[{\tt Unrecoverable\_Exception}:]  $\;$ \\
  Console Message: (details about the particular error
  will be output to the console) \\
  This indicates that \Ipopt\ has thrown an exception that does not
  have an internal return code. See the specific message for details.
\item[{\tt NonIpopt\_Exception\_Thrown}:]  $\;$ \\
  Console Message: {\tt Unknown Exception caught in Ipopt} \\
  An unknown exception was caught in \Ipopt. This exception could have
  originated from your model or any linked in third party code.
\item[{\tt Insufficient\_Memory}:]  $\;$ \\
  Console Message: {\tt EXIT: Not enough memory.} \\
  An error occurred while trying to allocate memory. The problem may
  be too large for your current memory and swap configuration.
\item[{\tt Internal\_Error}:]  $\;$ \\
  Console Message: {\tt EXIT: INTERNAL ERROR: Unknown SolverReturn
    value - Notify IPOPT Authors.} \\
  An unknown internal error has occurred. Please notify the authors of
  \Ipopt.

\end{description}

\appendix
\newpage
\section{Triplet Format for Sparse Matrices}\label{app.triplet}
\Ipopt\ was designed for optimizing large sparse nonlinear programs.
Because of problem sparsity, the required matrices (like the
constraints Jacobian or Lagrangian Hessian) are not stored as dense
matrices, but rather in a sparse matrix format. For the tutorials in
this document, we use the triplet format.  Consider the matrix
\begin{equation}
\label{eqn.ex_matrix}
\left[
\begin{array}{ccccccc}
1.1     & 0             & 0             & 0             & 0             & 0             & 0.5 \\
0       & 1.9   & 0             & 0             & 0             & 0             & 0.5 \\
0       & 0             & 2.6   & 0             & 0             & 0             & 0.5 \\
0       & 0             & 7.8   & 0.6   & 0             & 0             & 0    \\
0       & 0             & 0             & 1.5   & 2.7   & 0             & 0     \\
1.6     & 0             & 0             & 0             & 0.4   & 0             & 0     \\
0       & 0             & 0             & 0             & 0             & 0.9   & 1.7 \\
\end{array}
\right]
\end{equation}

A standard dense matrix representation would need to store $7 \cdot
7{=} 49$ floating point numbers, where many entries would be zero. In
triplet format, however, only the nonzero entries are stored. The
triplet format records the row number, the column number, and the
value of all nonzero entries in the matrix. For the matrix above, this
means storing $14$ integers for the rows, $14$ integers for the
columns, and $14$ floating point numbers for the values. While this
does not seem like a huge space savings over the $49$ floating point
numbers stored in the dense representation, for larger matrices, the
space savings are very dramatic\footnote{For an $n \times n$ matrix,
the dense representation grows with the the square of $n$, while the
sparse representation grows linearly in the number of nonzeros.}.

The option {\tt index\_style} in {\tt get\_nlp\_info} tells \Ipopt\ if
you prefer to use C style indexing (0-based, i.e., starting the
counting at 0) for the row and column indices or Fortran style
(1-based). Tables \ref{tab.fortran_triplet} and \ref{tab.c_triplet}
below show the triplet format for both indexing styles, using the
example matrix (\ref{eqn.ex_matrix}).

\begin{footnotesize}
\begin{table}[ht]%[!h]
\begin{center}
\begin{tabular}{c c c}
row     		&       col     	&       value 			    \\
\hline
{\tt iRow[0] = 1}       &       {\tt jCol[0] = 1}       & {\tt values[0] = 1.1}     \\
{\tt iRow[1] = 1}       &       {\tt jCol[1] = 7}       & {\tt values[1] = 0.5}     \\
{\tt iRow[2] = 2}       &       {\tt jCol[2] = 2}       & {\tt values[2] = 1.9}     \\
{\tt iRow[3] = 2}       &       {\tt jCol[3] = 7}       & {\tt values[3] = 0.5}     \\
{\tt iRow[4] = 3}       &       {\tt jCol[4] = 3}       & {\tt values[4] = 2.6}     \\
{\tt iRow[5] = 3}       &       {\tt jCol[5] = 7}       & {\tt values[5] = 0.5}     \\
{\tt iRow[6] = 4}       &       {\tt jCol[6] = 3}       & {\tt values[6] = 7.8}     \\
{\tt iRow[7] = 4}       &       {\tt jCol[7] = 4}       & {\tt values[7] = 0.6}     \\
{\tt iRow[8] = 5}       &       {\tt jCol[8] = 4}       & {\tt values[8] = 1.5}     \\
{\tt iRow[9] = 5}       &       {\tt jCol[9] = 5}       & {\tt values[9] = 2.7}     \\
{\tt iRow[10] = 6}      &       {\tt jCol[10] = 1}      & {\tt values[10] = 1.6}     \\
{\tt iRow[11] = 6}      &       {\tt jCol[11] = 5}      & {\tt values[11] = 0.4}     \\
{\tt iRow[12] = 7}      &       {\tt jCol[12] = 6}      & {\tt values[12] = 0.9}     \\
{\tt iRow[13] = 7}      &       {\tt jCol[13] = 7}      & {\tt values[13] = 1.7}
\end{tabular}
\caption{Triplet Format of Matrix (\ref{eqn.ex_matrix}) 
with {\tt index\_style=FORTRAN\_STYLE}}
\label{tab.fortran_triplet}
\end{center}
\end{table}
\begin{table}[ht]%[!h]
\begin{center}
\begin{tabular}{c c c}
row     		&       col     	&       value 			    \\
\hline
{\tt iRow[0] = 0}       &       {\tt jCol[0] = 0}       & {\tt values[0] = 1.1}     \\
{\tt iRow[1] = 0}       &       {\tt jCol[1] = 6}       & {\tt values[1] = 0.5}     \\
{\tt iRow[2] = 1}       &       {\tt jCol[2] = 1}       & {\tt values[2] = 1.9}     \\
{\tt iRow[3] = 1}       &       {\tt jCol[3] = 6}       & {\tt values[3] = 0.5}     \\
{\tt iRow[4] = 2}       &       {\tt jCol[4] = 2}       & {\tt values[4] = 2.6}     \\
{\tt iRow[5] = 2}       &       {\tt jCol[5] = 6}       & {\tt values[5] = 0.5}     \\
{\tt iRow[6] = 3}       &       {\tt jCol[6] = 2}       & {\tt values[6] = 7.8}     \\
{\tt iRow[7] = 3}       &       {\tt jCol[7] = 3}       & {\tt values[7] = 0.6}     \\
{\tt iRow[8] = 4}       &       {\tt jCol[8] = 3}       & {\tt values[8] = 1.5}     \\
{\tt iRow[9] = 4}       &       {\tt jCol[9] = 4}       & {\tt values[9] = 2.7}     \\
{\tt iRow[10] = 5}      &       {\tt jCol[10] = 0}      & {\tt values[10] = 1.6}     \\
{\tt iRow[11] = 5}      &       {\tt jCol[11] = 4}      & {\tt values[11] = 0.4}     \\
{\tt iRow[12] = 6}      &       {\tt jCol[12] = 5}      & {\tt values[12] = 0.9}     \\
{\tt iRow[13] = 6}      &       {\tt jCol[13] = 6}      & {\tt values[13] = 1.7}
\end{tabular}
\caption{Triplet Format of Matrix (\ref{eqn.ex_matrix}) 
with {\tt index\_style=C\_STYLE}}
\label{tab.c_triplet}
\end{center}
\end{table}
\end{footnotesize}
The individual elements of the matrix can be listed in any order, and
if there are multiple items for the same nonzero position, the values
provided for those positions are added.

The Hessian of the Lagrangian is a symmetric matrix. In the case of a
symmetric matrix, you only need to specify the lower left triangual
part (or, alternatively, the upper right triangular part). For
example, given the matrix,
\begin{equation}
\label{eqn.ex_sym_matrix}
\left[
\begin{array}{ccccccc}
1.0	& 0	& 3.0	& 0	& 2.0 	\\
0	& 1.1	& 0	& 0	& 5.0	\\
3.0	& 0	& 1.2	& 6.0	& 0	\\
0	& 0	& 6.0	& 1.3	& 9.0	\\
2.0	& 5.0	& 0	& 9.0	& 1.4
\end{array}
\right]
\end{equation}
the triplet format is shown in Tables \ref{tab.sym_fortran_triplet}
and \ref{tab.sym_c_triplet}.

\begin{footnotesize}
\begin{table}[ht]%[!h]
\begin{center}
\caption{Triplet Format of Matrix (\ref{eqn.ex_matrix}) 
with {\tt index\_style=FORTRAN\_STYLE}}
\label{tab.sym_fortran_triplet}
\begin{tabular}{c c c}
row     		&       col     	&       value 			    \\
\hline
{\tt iRow[0] = 1}       &       {\tt jCol[0] = 1}       & {\tt values[0] = 1.0}     \\
{\tt iRow[1] = 2}       &       {\tt jCol[1] = 1}       & {\tt values[1] = 1.1}     \\
{\tt iRow[2] = 3}       &       {\tt jCol[2] = 1}       & {\tt values[2] = 3.0}     \\
{\tt iRow[3] = 3}       &       {\tt jCol[3] = 3}       & {\tt values[3] = 1.2}     \\
{\tt iRow[4] = 4}       &       {\tt jCol[4] = 3}       & {\tt values[4] = 6.0}     \\
{\tt iRow[5] = 4}       &       {\tt jCol[5] = 4}       & {\tt values[5] = 1.3}     \\
{\tt iRow[6] = 5}       &       {\tt jCol[6] = 1}       & {\tt values[6] = 2.0}     \\
{\tt iRow[7] = 5}       &       {\tt jCol[7] = 2}       & {\tt values[7] = 5.0}     \\
{\tt iRow[8] = 5}       &       {\tt jCol[8] = 4}       & {\tt values[8] = 9.0}     \\
{\tt iRow[9] = 5}       &       {\tt jCol[9] = 5}       & {\tt values[9] = 1.4}
\end{tabular}
\end{center}
\end{table}
\begin{table}[ht]%[!h]
\begin{center}
\caption{Triplet Format of Matrix (\ref{eqn.ex_matrix}) 
with {\tt index\_style=C\_STYLE}}
\label{tab.sym_c_triplet}
\begin{tabular}{c c c}
row     		&       col     	&       value 			    \\
\hline
{\tt iRow[0] = 0}       &       {\tt jCol[0] = 0}       & {\tt values[0] = 1.0}     \\
{\tt iRow[1] = 1}       &       {\tt jCol[1] = 0}       & {\tt values[1] = 1.1}     \\
{\tt iRow[2] = 2}       &       {\tt jCol[2] = 0}       & {\tt values[2] = 3.0}     \\
{\tt iRow[3] = 2}       &       {\tt jCol[3] = 2}       & {\tt values[3] = 1.2}     \\
{\tt iRow[4] = 3}       &       {\tt jCol[4] = 2}       & {\tt values[4] = 6.0}     \\
{\tt iRow[5] = 3}       &       {\tt jCol[5] = 3}       & {\tt values[5] = 1.3}     \\
{\tt iRow[6] = 4}       &       {\tt jCol[6] = 0}       & {\tt values[6] = 2.0}     \\
{\tt iRow[7] = 4}       &       {\tt jCol[7] = 1}       & {\tt values[7] = 5.0}     \\
{\tt iRow[8] = 4}       &       {\tt jCol[8] = 3}       & {\tt values[8] = 9.0}     \\
{\tt iRow[9] = 4}       &       {\tt jCol[9] = 4}       & {\tt values[9] = 1.4}
\end{tabular}
\end{center}
\end{table}
\end{footnotesize}
\newpage
\section{The Smart Pointer Implementation: {\tt SmartPtr<T>}} \label{app.smart_ptr}

The {\tt SmartPtr} class is described in {\tt IpSmartPtr.hpp}. It is a
template class that takes care of deleting objects for us so we need
not be concerned about memory leaks. Instead of pointing to an object
with a raw C++ pointer (e.g. {\tt HS071\_NLP*}), we use a {\tt
  SmartPtr}.  Every time a {\tt SmartPtr} is set to reference an
object, it increments a counter in that object (see the {\tt
  ReferencedObject} base class if you are interested). If a {\tt
  SmartPtr} is done with the object, either by leaving scope or being
set to point to another object, the counter is decremented. When the
count of the object goes to zero, the object is automatically deleted.
{\tt SmartPtr}'s are very simple, just use them as you would a
standard pointer.

It is very important to use {\tt SmartPtr}'s instead of raw pointers
when passing objects to \Ipopt. Internally, \Ipopt\ uses smart
pointers for referencing objects. If you use a raw pointer in your
executable, the object's counter will NOT get incremented. Then, when
\Ipopt\ uses smart pointers inside its own code, the counter will get
incremented. However, before \Ipopt\ returns control to your code, it
will decrement as many times as it incremented earlier, and the
counter will return to zero. Therefore, \Ipopt\ will delete the
object. When control returns to you, you now have a raw pointer that
points to a deleted object.

This might sound difficult to anyone not familiar with the use of
smart pointers, but just follow one simple rule; always use a SmartPtr
when creating or passing an \Ipopt\ object.

\newpage
\section{Options Reference} \label{app.options_ref}
Options can be set using {\tt PARAMS.DAT}, through your own code, or through the 
AMPL {\tt options} command. See Section \ref{sec.options} for an explanation of
how to use these commands.
Shown here is a short list of the most common options for Ipopt. To view
the full list of options, run the ipopt executable with the option,
\begin{verbatim}
print_options_documentation yes
\end{verbatim}

The most common options are:


\paragraph{print\_level:} Output verbosity level. $\;$ \\
 Sets the default verbosity level for console
output. The larger this value the more detailed
is the output. The valid range for this integer option is
$0 \le {\tt print\_level } \le 11$
and its default value is $4$.


\paragraph{print\_user\_options:} Print all options set by the user. $\;$ \\
 If selected, the algorithm will print the list of
all options set by the user including their
values and whether they have been used.
The default value for this string option is "no".
\\ 
Possible values:
\begin{itemize}
   \item no: don't print options
   \item yes: print options
\end{itemize}

\paragraph{print\_options\_documentation:} Switch to print all algorithmic options. $\;$ \\
 If selected, the algorithm will print the list of
all available algorithmic options with some
documentation before solving the optimization
problem.
The default value for this string option is "no".
\\ 
Possible values:
\begin{itemize}
   \item no: don't print list
   \item yes: print list
\end{itemize}

\paragraph{output\_file:} File name of desired output file (leave unset for no file output). $\;$ \\
 NOTE: This option only works when read from the
ipopt.opt options file! An output file with this
name will be written (leave unset for no file
output).  The verbosity level is by default set
to "print\_level", but can be overridden with
"file\_print\_level".  The file name is changed
to use only small letters.
The default value for this string option is "".
\\ 
Possible values:
\begin{itemize}
   \item *: Any acceptable standard file name
\end{itemize}

\paragraph{file\_print\_level:} Verbosity level for output file. $\;$ \\
 NOTE: This option only works when read from the
ipopt.opt options file! Determines the verbosity
level for the file specified by "output\_file". 
By default it is the same as "print\_level". The valid range for this integer option is
$0 \le {\tt file\_print\_level } \le 11$
and its default value is $4$.


\paragraph{tol:} Desired convergence tolerance (relative). $\;$ \\
 Determines the convergence tolerance for the
algorithm.  The algorithm terminates
successfully, if the (scaled) NLP error becomes
smaller than this value, and if the (absolute)
criteria according to "dual\_inf\_tol",
"primal\_inf\_tol", and "cmpl\_inf\_tol" are met.
 (This is epsilon\_tol in Eqn. (6) in
implementation paper).  See also
"acceptable\_tol" as a second termination
criterion.  Note, some other algorithmic features
also use this quantity to determine thresholds
etc. The valid range for this real option is 
$0 <  {\tt tol } <  {\tt +inf}$
and its default value is $1 \cdot 10^{-08}$.


\paragraph{max\_iter:} Maximum number of iterations. $\;$ \\
 The algorithm terminates with an error message if
the number of iterations exceeded this number. The valid range for this integer option is
$0 \le {\tt max\_iter } <  {\tt +inf}$
and its default value is $3000$.


\paragraph{compl\_inf\_tol:} Desired threshold for the complementarity conditions. $\;$ \\
 Absolute tolerance on the complementarity.
Successful termination requires that the max-norm
of the (unscaled) complementarity is less than
this threshold. The valid range for this real option is 
$0 <  {\tt compl\_inf\_tol } <  {\tt +inf}$
and its default value is $0.0001$.


\paragraph{dual\_inf\_tol:} Desired threshold for the dual infeasibility. $\;$ \\
 Absolute tolerance on the dual infeasibility.
Successful termination requires that the max-norm
of the (unscaled) dual infeasibility is less than
this threshold. The valid range for this real option is 
$0 <  {\tt dual\_inf\_tol } <  {\tt +inf}$
and its default value is $0.0001$.


\paragraph{constr\_viol\_tol:} Desired threshold for the constraint violation. $\;$ \\
 Absolute tolerance on the constraint violation.
Successful termination requires that the max-norm
of the (unscaled) constraint violation is less
than this threshold. The valid range for this real option is 
$0 <  {\tt constr\_viol\_tol } <  {\tt +inf}$
and its default value is $0.0001$.


\paragraph{acceptable\_tol:} "Acceptable" convergence tolerance (relative). $\;$ \\
 Determines which (scaled) overall optimality
error is considered to be "acceptable." There are
two levels of termination criteria.  If the usual
"desired" tolerances (see tol, dual\_inf\_tol
etc) are satisfied at an iteration, the algorithm
immediately terminates with a success message. 
On the other hand, if the algorithm encounters
"acceptable\_iter" many iterations in a row that
are considered "acceptable", it will terminate
before the desired convergence tolerance is met.
This is useful in cases where the algorithm might
not be able to achieve the "desired" level of
accuracy. The valid range for this real option is 
$0 <  {\tt acceptable\_tol } <  {\tt +inf}$
and its default value is $1 \cdot 10^{-06}$.


\paragraph{acceptable\_compl\_inf\_tol:} "Acceptance" threshold for the complementarity conditions. $\;$ \\
 Absolute tolerance on the complementarity.
"Acceptable" termination requires that the
max-norm of the (unscaled) complementarity is
less than this threshold; see also
acceptable\_tol. The valid range for this real option is 
$0 <  {\tt acceptable\_compl\_inf\_tol } <  {\tt +inf}$
and its default value is $0.01$.


\paragraph{acceptable\_constr\_viol\_tol:} "Acceptance" threshold for the constraint violation. $\;$ \\
 Absolute tolerance on the constraint violation.
"Acceptable" termination requires that the
max-norm of the (unscaled) constraint violation
is less than this threshold; see also
acceptable\_tol. The valid range for this real option is 
$0 <  {\tt acceptable\_constr\_viol\_tol } <  {\tt +inf}$
and its default value is $0.01$.


\paragraph{acceptable\_dual\_inf\_tol:} "Acceptance" threshold for the dual infeasibility. $\;$ \\
 Absolute tolerance on the dual infeasibility.
"Acceptable" termination requires that the
(max-norm of the unscaled) dual infeasibility is
less than this threshold; see also
acceptable\_tol. The valid range for this real option is 
$0 <  {\tt acceptable\_dual\_inf\_tol } <  {\tt +inf}$
and its default value is $0.01$.


\paragraph{diverging\_iterates\_tol:} Threshold for maximal value of primal iterates. $\;$ \\
 If any component of the primal iterates exceeded
this value (in absolute terms), the optimization
is aborted with the exit message that the
iterates seem to be diverging. The valid range for this real option is 
$0 <  {\tt diverging\_iterates\_tol } <  {\tt +inf}$
and its default value is $1 \cdot 10^{+20}$.


\paragraph{barrier\_tol\_factor:} Factor for mu in barrier stop test. $\;$ \\
 The convergence tolerance for each barrier
problem in the monotone mode is the value of the
barrier parameter times "barrier\_tol\_factor".
This option is also used in the adaptive mu
strategy during the monotone mode. (This is
kappa\_epsilon in implementation paper). The valid range for this real option is 
$0 <  {\tt barrier\_tol\_factor } <  {\tt +inf}$
and its default value is $10$.


\paragraph{obj\_scaling\_factor:} Scaling factor for the objective function. $\;$ \\
 This option sets a scaling factor for the
objective function. The scaling is seen
internally by Ipopt but the unscaled objective is
reported in the console output. If additional
scaling parameters are computed (e.g.
user-scaling or gradient-based), both factors are
multiplied. If this value is chosen to be
negative, Ipopt will maximize the objective
function instead of minimizing it. The valid range for this real option is 
${\tt -inf} <  {\tt obj\_scaling\_factor } <  {\tt +inf}$
and its default value is $1$.


\paragraph{nlp\_scaling\_method:} Select the technique used for scaling the NLP. $\;$ \\
 Selects the technique used for scaling the
problem internally before it is solved. For
user-scaling, the parameters come from the NLP.
If you are using AMPL, they can be specified
through suffixes ("scaling\_factor")
The default value for this string option is "gradient-based".
\\ 
Possible values:
\begin{itemize}
   \item none: no problem scaling will be performed
   \item user-scaling: scaling parameters will come from the user
   \item gradient-based: scale the problem so the maximum gradient at
the starting point is scaling\_max\_gradient
\end{itemize}

\paragraph{nlp\_scaling\_max\_gradient:} Maximum gradient after NLP scaling. $\;$ \\
 This is the gradient scaling cut-off. If the
maximum gradient is above this value, then
gradient based scaling will be performed. Scaling
parameters are calculated to scale the maximum
gradient back to this value. (This is g\_max in
Section 3.8 of the implementation paper.) Note:
This option is only used if
"nlp\_scaling\_method" is chosen as
"gradient-based". The valid range for this real option is 
$0 <  {\tt nlp\_scaling\_max\_gradient } <  {\tt +inf}$
and its default value is $100$.


\paragraph{bound\_relax\_factor:} Factor for initial relaxation of the bounds. $\;$ \\
 Before start of the optimization, the bounds
given by the user are relaxed.  This option sets
the factor for this relaxation.  If it is set to
zero, then then bounds relaxation is disabled.
(See Eqn.(35) in implementation paper.) The valid range for this real option is 
$0 \le {\tt bound\_relax\_factor } <  {\tt +inf}$
and its default value is $1 \cdot 10^{-08}$.


\paragraph{honor\_original\_bounds:} Indicates whether final points should be projected into original bounds. $\;$ \\
 Ipopt might relax the bounds during the
optimization (see, e.g., option
"bound\_relax\_factor").  This option determines
whether the final point should be projected back
into the user-provide original bounds after the
optimization.
The default value for this string option is "yes".
\\ 
Possible values:
\begin{itemize}
   \item no: Leave final point unchanged
   \item yes: Project final point back into original bounds
\end{itemize}

\paragraph{check\_derivatives\_for\_naninf:} Indicates whether it is desired to check for Nan/Inf in derivative matrices $\;$ \\
 Activating this option will cause an error if an
invalid number is detected in the constraint
Jacobians or the Lagrangian Hessian.  If this is
not activated, the test is skipped, and the
algorithm might proceed with invalid numbers and
fail.
The default value for this string option is "no".
\\ 
Possible values:
\begin{itemize}
   \item no: Don't check (faster).
   \item yes: Check Jacobians and Hessian for Nan and Inf.
\end{itemize}

\paragraph{mu\_strategy:} Update strategy for barrier parameter. $\;$ \\
 Determines which barrier parameter update
strategy is to be used.
The default value for this string option is "monotone".
\\ 
Possible values:
\begin{itemize}
   \item monotone: use the monotone (Fiacco-McCormick) strategy
   \item adaptive: use the adaptive update strategy
\end{itemize}

\paragraph{mu\_oracle:} Oracle for a new barrier parameter in the adaptive strategy. $\;$ \\
 Determines how a new barrier parameter is
computed in each "free-mode" iteration of the
adaptive barrier parameter strategy. (Only
considered if "adaptive" is selected for option
"mu\_strategy").
The default value for this string option is "quality-function".
\\ 
Possible values:
\begin{itemize}
   \item probing: Mehrotra's probing heuristic
   \item loqo: LOQO's centrality rule
   \item quality-function: minimize a quality function
\end{itemize}

\paragraph{quality\_function\_max\_section\_steps:} Maximum number of search steps during direct search procedure determining the optimal centering parameter. $\;$ \\
 The golden section search is performed for the
quality function based mu oracle. (Only used if
option "mu\_oracle" is set to "quality-function".) The valid range for this integer option is
$0 \le {\tt quality\_function\_max\_section\_steps } <  {\tt +inf}$
and its default value is $8$.


\paragraph{fixed\_mu\_oracle:} Oracle for the barrier parameter when switching to fixed mode. $\;$ \\
 Determines how the first value of the barrier
parameter should be computed when switching to
the "monotone mode" in the adaptive strategy.
(Only considered if "adaptive" is selected for
option "mu\_strategy".)
The default value for this string option is "average\_compl".
\\ 
Possible values:
\begin{itemize}
   \item probing: Mehrotra's probing heuristic
   \item loqo: LOQO's centrality rule
   \item quality-function: minimize a quality function
   \item average\_compl: base on current average complementarity
\end{itemize}

\paragraph{mu\_init:} Initial value for the barrier parameter. $\;$ \\
 This option determines the initial value for the
barrier parameter (mu).  It is only relevant in
the monotone, Fiacco-McCormick version of the
algorithm. (i.e., if "mu\_strategy" is chosen as
"monotone") The valid range for this real option is 
$0 <  {\tt mu\_init } <  {\tt +inf}$
and its default value is $0.1$.


\paragraph{mu\_max\_fact:} Factor for initialization of maximum value for barrier parameter. $\;$ \\
 This option determines the upper bound on the
barrier parameter.  This upper bound is computed
as the average complementarity at the initial
point times the value of this option. (Only used
if option "mu\_strategy" is chosen as "adaptive".) The valid range for this real option is 
$0 <  {\tt mu\_max\_fact } <  {\tt +inf}$
and its default value is $1000$.


\paragraph{mu\_max:} Maximum value for barrier parameter. $\;$ \\
 This option specifies an upper bound on the
barrier parameter in the adaptive mu selection
mode.  If this option is set, it overwrites the
effect of mu\_max\_fact. (Only used if option
"mu\_strategy" is chosen as "adaptive".) The valid range for this real option is 
$0 <  {\tt mu\_max } <  {\tt +inf}$
and its default value is $100000$.


\paragraph{mu\_min:} Minimum value for barrier parameter. $\;$ \\
 This option specifies the lower bound on the
barrier parameter in the adaptive mu selection
mode. By default, it is set to
min("tol","compl\_inf\_tol")/("barrier\_tol\_fact-
or"+1), which should be a reasonable value. (Only
used if option "mu\_strategy" is chosen as
"adaptive".) The valid range for this real option is 
$0 <  {\tt mu\_min } <  {\tt +inf}$
and its default value is $1 \cdot 10^{-09}$.


\paragraph{mu\_linear\_decrease\_factor:} Determines linear decrease rate of barrier parameter. $\;$ \\
 For the Fiacco-McCormick update procedure the new
barrier parameter mu is obtained by taking the
minimum of mu*"mu\_linear\_decrease\_factor" and
mu\^"superlinear\_decrease\_power".  (This is
kappa\_mu in implementation paper.) This option
is also used in the adaptive mu strategy during
the monotone mode. The valid range for this real option is 
$0 <  {\tt mu\_linear\_decrease\_factor } <  1$
and its default value is $0.2$.


\paragraph{mu\_superlinear\_decrease\_power:} Determines superlinear decrease rate of barrier parameter. $\;$ \\
 For the Fiacco-McCormick update procedure the new
barrier parameter mu is obtained by taking the
minimum of mu*"mu\_linear\_decrease\_factor" and
mu\^"superlinear\_decrease\_power".  (This is
theta\_mu in implementation paper.) This option
is also used in the adaptive mu strategy during
the monotone mode. The valid range for this real option is 
$1 <  {\tt mu\_superlinear\_decrease\_power } <  2$
and its default value is $1.5$.


\paragraph{bound\_frac:} Desired minimum relative distance from the initial point to bound. $\;$ \\
 Determines how much the initial point might have
to be modified in order to be sufficiently inside
the bounds (together with "bound\_push").  (This
is kappa\_2 in Section 3.6 of implementation
paper.) The valid range for this real option is 
$0 <  {\tt bound\_frac } \le 0.5$
and its default value is $0.01$.


\paragraph{bound\_push:} Desired minimum absolute distance from the initial point to bound. $\;$ \\
 Determines how much the initial point might have
to be modified in order to be sufficiently inside
the bounds (together with "bound\_frac").  (This
is kappa\_1 in Section 3.6 of implementation
paper.) The valid range for this real option is 
$0 <  {\tt bound\_push } <  {\tt +inf}$
and its default value is $0.01$.


\paragraph{bound\_mult\_init\_val:} Initial value for the bound multipliers. $\;$ \\
 All dual variables corresponding to bound
constraints are initialized to this value. The valid range for this real option is 
$0 <  {\tt bound\_mult\_init\_val } <  {\tt +inf}$
and its default value is $1$.


\paragraph{constr\_mult\_init\_max:} Maximum allowed least-square guess of constraint multipliers. $\;$ \\
 Determines how large the initial least-square
guesses of the constraint multipliers are allowed
to be (in max-norm). If the guess is larger than
this value, it is discarded and all constraint
multipliers are set to zero.  This options is
also used when initializing the restoration
phase. By default,
"resto.constr\_mult\_init\_max" (the one used in
RestoIterateInitializer) is set to zero. The valid range for this real option is 
$0 \le {\tt constr\_mult\_init\_max } <  {\tt +inf}$
and its default value is $1000$.


\paragraph{bound\_mult\_init\_val:} Initial value for the bound multipliers. $\;$ \\
 All dual variables corresponding to bound
constraints are initialized to this value. The valid range for this real option is 
$0 <  {\tt bound\_mult\_init\_val } <  {\tt +inf}$
and its default value is $1$.


\paragraph{warm\_start\_init\_point:} Warm-start for initial point $\;$ \\
 Indicates whether this optimization should use a
warm start initialization, where values of primal
and dual variables are given (e.g., from a
previous optimization of a related problem.)
The default value for this string option is "no".
\\ 
Possible values:
\begin{itemize}
   \item no: do not use the warm start initialization
   \item yes: use the warm start initialization
\end{itemize}

\paragraph{warm\_start\_bound\_push:} same as bound\_push for the regular initializer. $\;$ \\
 The valid range for this real option is 
$0 <  {\tt warm\_start\_bound\_push } <  {\tt +inf}$
and its default value is $0.001$.


\paragraph{warm\_start\_bound\_frac:} same as bound\_frac for the regular initializer. $\;$ \\
 The valid range for this real option is 
$0 <  {\tt warm\_start\_bound\_frac } \le 0.5$
and its default value is $0.001$.


\paragraph{warm\_start\_mult\_bound\_push:} same as mult\_bound\_push for the regular initializer. $\;$ \\
 The valid range for this real option is 
$0 <  {\tt warm\_start\_mult\_bound\_push } <  {\tt +inf}$
and its default value is $0.001$.


\paragraph{warm\_start\_mult\_init\_max:} Maximum initial value for the equality multipliers. $\;$ \\
 The valid range for this real option is 
${\tt -inf} <  {\tt warm\_start\_mult\_init\_max } <  {\tt +inf}$
and its default value is $1 \cdot 10^{+06}$.


\paragraph{alpha\_for\_y:} Method to determine the step size for constraint multipliers. $\;$ \\
 This option determines how the step size
(alpha\_y) will be calculated when updating the
constraint multipliers.
The default value for this string option is "primal".
\\ 
Possible values:
\begin{itemize}
   \item primal: use primal step size
   \item bound\_mult: use step size for the bound multipliers (good
for LPs)
   \item min: use the min of primal and bound multipliers
   \item max: use the max of primal and bound multipliers
   \item full: take a full step of size one
   \item min\_dual\_infeas: choose step size minimizing new dual
infeasibility
   \item safe\_min\_dual\_infeas: like "min\_dual\_infeas", but safeguarded by
"min" and "max"
\end{itemize}

\paragraph{recalc\_y:} Tells the algorithm to recalculate the equality and inequality multipliers as least square estimates. $\;$ \\
 This asks the algorithm to recompute the
multipliers, whenever the current infeasibility
is less than recalc\_y\_feas\_tol. Choosing yes
might be helpful in the quasi-Newton option. 
However, each recalculation requires an extra
factorization of the linear system.  If a limited
memory quasi-Newton option is chosen, this is
used by default.
The default value for this string option is "no".
\\ 
Possible values:
\begin{itemize}
   \item no: use the Newton step to update the multipliers
   \item yes: use least-square multiplier estimates
\end{itemize}

\paragraph{recalc\_y\_feas\_tol:} Feasibility threshold for recomputation of multipliers. $\;$ \\
 If recalc\_y is chosen and the current
infeasibility is less than this value, then the
multipliers are recomputed. The valid range for this real option is 
$0 <  {\tt recalc\_y\_feas\_tol } <  {\tt +inf}$
and its default value is $1 \cdot 10^{-06}$.


\paragraph{max\_soc:} Maximum number of second order correction trial steps at each iteration. $\;$ \\
 Choosing 0 disables the second order corrections.
(This is p\^{max} of Step A-5.9 of Algorithm A in
implementation paper.) The valid range for this integer option is
$0 \le {\tt max\_soc } <  {\tt +inf}$
and its default value is $4$.


\paragraph{watchdog\_shortened\_iter\_trigger:} Number of shortened iterations that trigger the watchdog. $\;$ \\
 If the number of successive iterations in which
the backtracking line search did not accept the
first trial point exceeds this number, the
watchdog procedure is activated.  Choosing "0"
here disables the watchdog procedure. The valid range for this integer option is
$0 \le {\tt watchdog\_shortened\_iter\_trigger } <  {\tt +inf}$
and its default value is $10$.


\paragraph{watchdog\_trial\_iter\_max:} Maximum number of watchdog iterations. $\;$ \\
 This option determines the number of trial
iterations allowed before the watchdog procedure
is aborted and the algorithm returns to the
stored point. The valid range for this integer option is
$1 \le {\tt watchdog\_trial\_iter\_max } <  {\tt +inf}$
and its default value is $3$.


\paragraph{expect\_infeasible\_problem:} Enable heuristics to quickly detect an infeasible problem. $\;$ \\
 This options is meant to activate heuristics that
may speed up the infeasibility determination if
you expect that there is a good chance for the
problem to be infeasible.  In the filter line
search procedure, the restoration phase is called
more quickly than usually, and more reduction in
the constraint violation is enforced before the
restoration phase is left. If the problem is
square, this option is enabled automatically.
The default value for this string option is "no".
\\ 
Possible values:
\begin{itemize}
   \item no: the problem probably be feasible
   \item yes: the problem has a good chance to be infeasible
\end{itemize}

\paragraph{expect\_infeasible\_problem\_ctol:} Threshold for disabling "expect\_infeasible\_problem" option. $\;$ \\
 If the constraint violation becomes smaller than
this threshold, the "expect\_infeasible\_problem"
heuristics in the filter line search are
disabled. If the problem is square, this options
is set to 0. The valid range for this real option is 
$0 \le {\tt expect\_infeasible\_problem\_ctol } <  {\tt +inf}$
and its default value is $0.001$.


\paragraph{start\_with\_resto:} Tells algorithm to switch to restoration phase in first iteration. $\;$ \\
 Setting this option to "yes" forces the algorithm
to switch to the feasibility restoration phase in
the first iteration. If the initial point is
feasible, the algorithm will abort with a failure.
The default value for this string option is "no".
\\ 
Possible values:
\begin{itemize}
   \item no: don't force start in restoration phase
   \item yes: force start in restoration phase
\end{itemize}

\paragraph{soft\_resto\_pderror\_reduction\_factor:} Required reduction in primal-dual error in the soft restoration phase. $\;$ \\
 The soft restoration phase attempts to reduce the
primal-dual error with regular steps. If the
damped primal-dual step (damped only to satisfy
the fraction-to-the-boundary rule) is not
decreasing the primal-dual error by at least this
factor, then the regular restoration phase is
called. Choosing "0" here disables the soft
restoration phase. The valid range for this real option is 
$0 \le {\tt soft\_resto\_pderror\_reduction\_factor } <  {\tt +inf}$
and its default value is $0.9999$.


\paragraph{required\_infeasibility\_reduction:} Required reduction of infeasibility before leaving restoration phase. $\;$ \\
 The restoration phase algorithm is performed,
until a point is found that is acceptable to the
filter and the infeasibility has been reduced by
at least the fraction given by this option. The valid range for this real option is 
$0 \le {\tt required\_infeasibility\_reduction } <  1$
and its default value is $0.9$.


\paragraph{bound\_mult\_reset\_threshold:} Threshold for resetting bound multipliers after the restoration phase. $\;$ \\
 After returning from the restoration phase, the
bound multipliers are updated with a Newton step
for complementarity.  Here, the change in the
primal variables during the entire restoration
phase is taken to be the corresponding primal
Newton step. However, if after the update the
largest bound multiplier exceeds the threshold
specified by this option, the multipliers are all
reset to 1. The valid range for this real option is 
$0 \le {\tt bound\_mult\_reset\_threshold } <  {\tt +inf}$
and its default value is $1000$.


\paragraph{constr\_mult\_reset\_threshold:} Threshold for resetting equality and inequality multipliers after restoration phase. $\;$ \\
 After returning from the restoration phase, the
constraint multipliers are recomputed by a least
square estimate.  This option triggers when those
least-square estimates should be ignored. The valid range for this real option is 
$0 \le {\tt constr\_mult\_reset\_threshold } <  {\tt +inf}$
and its default value is $0$.


\paragraph{evaluate\_orig\_obj\_at\_resto\_trial:} Determines if the original objective function should be evaluated at restoration phase trial points. $\;$ \\
 Setting this option to "yes" makes the
restoration phase algorithm evaluate the
objective function of the original problem at
every trial point encountered during the
restoration phase, even if this value is not
required.  In this way, it is guaranteed that the
original objective function can be evaluated
without error at all accepted iterates; otherwise
the algorithm might fail at a point where the
restoration phase accepts an iterate that is good
for the restoration phase problem, but not the
original problem.  On the other hand, if the
evaluation of the original objective is
expensive, this might be costly.
The default value for this string option is "yes".
\\ 
Possible values:
\begin{itemize}
   \item no: skip evaluation
   \item yes: evaluate at every trial point
\end{itemize}

\paragraph{linear\_solver:} Linear solver used for step computations. $\;$ \\
 Determines which linear algebra package is to be
used for the solution of the augmented linear
system (for obtaining the search directions).
Note, the code must have been compiled with the
linear solver you want to choose. Depending on
your Ipopt installation, not all options are
available.
The default value for this string option is "ma27".
\\ 
Possible values:
\begin{itemize}
   \item ma27: use the Harwell routine MA27
   \item ma57: use the Harwell routine MA57
   \item pardiso: use the Pardiso package
   \item wsmp: use WSMP package
   \item taucs: use TAUCS package (not yet working)
   \item mumps: use MUMPS package (not yet working)
\end{itemize}

\paragraph{linear\_system\_scaling:} Method for scaling the linear system. $\;$ \\
 Determines the method used to compute symmetric
scaling factors for the augmented system (see
also the "linear\_scaling\_on\_demand" option). 
This scaling is independentof the NLP problem
scaling.  By default, MC19 is only used if MA27
or MA57 are selected as linear solvers. This
option is only available if Ipopt has been
compiled with MC19.
The default value for this string option is "mc19".
\\ 
Possible values:
\begin{itemize}
   \item none: no scaling will be performed
   \item mc19: use the Harwell routine MC19
\end{itemize}

\paragraph{linear\_scaling\_on\_demand:} Flag indicating that linear scaling is only done if it seems required. $\;$ \\
 This option is only important if a linear scaling
method (e.g., mc19) is used.  If you choose "no",
then the scaling factors are computed for every
linear system from the start.  This can be quite
expensive. Choosing "yes" means that the
algorithm will start the scaling method only when
the solutions to the linear system seem not good,
and then use it until the end.
The default value for this string option is "yes".
\\ 
Possible values:
\begin{itemize}
   \item no: Always scale the linear system.
   \item yes: Start using linear system scaling if solutions
seem not good.
\end{itemize}

\paragraph{max\_refinement\_steps:} Maximum number of iterative refinement steps per linear system solve. $\;$ \\
 Iterative refinement (on the full unsymmetric
system) is performed for each right hand side. 
This option determines the maximum number of
iterative refinement steps. The valid range for this integer option is
$0 \le {\tt max\_refinement\_steps } <  {\tt +inf}$
and its default value is $10$.


\paragraph{min\_refinement\_steps:} Minimum number of iterative refinement steps per linear system solve. $\;$ \\
 Iterative refinement (on the full unsymmetric
system) is performed for each right hand side. 
This option determines the minimum number of
iterative refinements (i.e. at least
"min\_refinement\_steps" iterative refinement
steps are enforced per right hand side.) The valid range for this integer option is
$0 \le {\tt min\_refinement\_steps } <  {\tt +inf}$
and its default value is $1$.


\paragraph{max\_hessian\_perturbation:} Maximum value of regularization parameter for handling negative curvature. $\;$ \\
 In order to guarantee that the search directions
are indeed proper descent directions, Ipopt
requires that the inertia of the (augmented)
linear system for the step computation has the
correct number of negative and positive
eigenvalues. The idea is that this guides the
algorithm away from maximizers and makes Ipopt
more likely converge to first order optimal
points that are minimizers. If the inertia is not
correct, a multiple of the identity matrix is
added to the Hessian of the Lagrangian in the
augmented system. This parameter gives the
maximum value of the regularization parameter. If
a regularization of that size is not enough, the
algorithm skips this iteration and goes to the
restoration phase. (This is delta\_w\^max in the
implementation paper.) The valid range for this real option is 
$0 <  {\tt max\_hessian\_perturbation } <  {\tt +inf}$
and its default value is $1 \cdot 10^{+20}$.


\paragraph{min\_hessian\_perturbation:} Smallest perturbation of the Hessian block. $\;$ \\
 The size of the perturbation of the Hessian block
is never selected smaller than this value, unless
no perturbation is necessary. (This is
delta\_w\^min in implementation paper.) The valid range for this real option is 
$0 \le {\tt min\_hessian\_perturbation } <  {\tt +inf}$
and its default value is $1 \cdot 10^{-20}$.


\paragraph{first\_hessian\_perturbation:} Size of first x-s perturbation tried. $\;$ \\
 The first value tried for the x-s perturbation in
the inertia correction scheme.(This is delta\_0
in the implementation paper.) The valid range for this real option is 
$0 <  {\tt first\_hessian\_perturbation } <  {\tt +inf}$
and its default value is $0.0001$.


\paragraph{perturb\_inc\_fact\_first:} Increase factor for x-s perturbation for very first perturbation. $\;$ \\
 The factor by which the perturbation is increased
when a trial value was not sufficient - this
value is used for the computation of the very
first perturbation and allows a different value
for for the first perturbation than that used for
the remaining perturbations. (This is
bar\_kappa\_w\^+ in the implementation paper.) The valid range for this real option is 
$1 <  {\tt perturb\_inc\_fact\_first } <  {\tt +inf}$
and its default value is $100$.


\paragraph{perturb\_inc\_fact:} Increase factor for x-s perturbation. $\;$ \\
 The factor by which the perturbation is increased
when a trial value was not sufficient - this
value is used for the computation of all
perturbations except for the first. (This is
kappa\_w\^+ in the implementation paper.) The valid range for this real option is 
$1 <  {\tt perturb\_inc\_fact } <  {\tt +inf}$
and its default value is $8$.


\paragraph{perturb\_dec\_fact:} Decrease factor for x-s perturbation. $\;$ \\
 The factor by which the perturbation is decreased
when a trial value is deduced from the size of
the most recent successful perturbation. (This is
kappa\_w\^- in the implementation paper.) The valid range for this real option is 
$0 <  {\tt perturb\_dec\_fact } <  1$
and its default value is $0.333333$.


\paragraph{jacobian\_regularization\_value:} Size of the regularization for rank-deficient constraint Jacobians. $\;$ \\
 (This is bar delta\_c in the implementation
paper.) The valid range for this real option is 
$0 \le {\tt jacobian\_regularization\_value } <  {\tt +inf}$
and its default value is $1 \cdot 10^{-08}$.


\paragraph{hessian\_approximation:} Indicates what Hessian information is to be used. $\;$ \\
 This determines which kind of information for the
Hessian of the Lagrangian function is used by the
algorithm.
The default value for this string option is "exact".
\\ 
Possible values:
\begin{itemize}
   \item exact: Use second derivatives provided by the NLP.
   \item limited-memory: Perform a limited-memory quasi-Newton 
approximation
\end{itemize}

\paragraph{limited\_memory\_max\_history:} Maximum size of the history for the limited quasi-Newton Hessian approximation. $\;$ \\
 This option determines the number of most recent
iterations that are taken into account for the
limited-memory quasi-Newton approximation. The valid range for this integer option is
$0 \le {\tt limited\_memory\_max\_history } <  {\tt +inf}$
and its default value is $6$.


\paragraph{limited\_memory\_max\_skipping:} Threshold for successive iterations where update is skipped. $\;$ \\
 If the update is skipped more than this number of
successive iterations, we quasi-Newton
approximation is reset. The valid range for this integer option is
$1 \le {\tt limited\_memory\_max\_skipping } <  {\tt +inf}$
and its default value is $2$.


\paragraph{derivative\_test:} Enable derivative checker $\;$ \\
 If this option is enabled, a (slow) derivative
test will be performed before the optimization. 
The test is performed at the user provided
starting point and marks derivative values that
seem suspicious
The default value for this string option is "none".
\\ 
Possible values:
\begin{itemize}
   \item none: do not perform derivative test
   \item first-order: perform test of first derivatives at starting
point
   \item second-order: perform test of first and second derivatives at
starting point
\end{itemize}

\paragraph{derivative\_test\_perturbation:} Size of the finite difference perturbation in derivative test. $\;$ \\
 This determines the relative perturbation of the
variable entries. The valid range for this real option is 
$0 <  {\tt derivative\_test\_perturbation } <  {\tt +inf}$
and its default value is $1 \cdot 10^{-08}$.


\paragraph{derivative\_test\_tol:} Threshold for indicating wrong derivative. $\;$ \\
 If the relative deviation of the estimated
derivative from the given one is larger than this
value, the corresponding derivative is marked as
wrong. The valid range for this real option is 
$0 <  {\tt derivative\_test\_tol } <  {\tt +inf}$
and its default value is $0.0001$.


\paragraph{derivative\_test\_print\_all:} Indicates whether information for all estimated derivatives should be printed. $\;$ \\
 Determines verbosity of derivative checker.
The default value for this string option is "no".
\\ 
Possible values:
\begin{itemize}
   \item no: Print only suspect derivatives
   \item yes: Print all derivatives
\end{itemize}

\paragraph{ma27\_pivtol:} Pivot tolerance for the linear solver MA27. $\;$ \\
 A smaller number pivots for sparsity, a larger
number pivots for stability.  This option is only
available if Ipopt has been compiled with MA27. The valid range for this real option is 
$0 <  {\tt ma27\_pivtol } <  1$
and its default value is $1 \cdot 10^{-08}$.


\paragraph{ma27\_pivtolmax:} Maximum pivot tolerance for the linear solver MA27. $\;$ \\
 Ipopt may increase pivtol as high as pivtolmax to
get a more accurate solution to the linear
system.  This option is only available if Ipopt
has been compiled with MA27. The valid range for this real option is 
$0 <  {\tt ma27\_pivtolmax } <  1$
and its default value is $0.0001$.


\paragraph{ma27\_liw\_init\_factor:} Integer workspace memory for MA27. $\;$ \\
 The initial integer workspace memory =
liw\_init\_factor * memory required by unfactored
system. Ipopt will increase the workspace size by
meminc\_factor if required.  This option is only
available if Ipopt has been compiled with MA27. The valid range for this real option is 
$1 \le {\tt ma27\_liw\_init\_factor } <  {\tt +inf}$
and its default value is $5$.


\paragraph{ma27\_la\_init\_factor:} Real workspace memory for MA27. $\;$ \\
 The initial real workspace memory =
la\_init\_factor * memory required by unfactored
system. Ipopt will increase the workspace size by
meminc\_factor if required.  This option is only
available if  Ipopt has been compiled with MA27. The valid range for this real option is 
$1 \le {\tt ma27\_la\_init\_factor } <  {\tt +inf}$
and its default value is $5$.


\paragraph{ma27\_meminc\_factor:} Increment factor for workspace size for MA27. $\;$ \\
 If the integer or real workspace is not large
enough, Ipopt will increase its size by this
factor.  This option is only available if Ipopt
has been compiled with MA27. The valid range for this real option is 
$1 \le {\tt ma27\_meminc\_factor } <  {\tt +inf}$
and its default value is $10$.


\paragraph{ma57\_pivtol:} Pivot tolerance for the linear solver MA57. $\;$ \\
 A smaller number pivots for sparsity, a larger
number pivots for stability. This option is only
available if Ipopt has been compiled with MA57. The valid range for this real option is 
$0 <  {\tt ma57\_pivtol } <  1$
and its default value is $1 \cdot 10^{-08}$.


\paragraph{ma57\_pivtolmax:} Maximum pivot tolerance for the linear solver MA57. $\;$ \\
 Ipopt may increase pivtol as high as
ma57\_pivtolmax to get a more accurate solution
to the linear system.  This option is only
available if Ipopt has been compiled with MA57. The valid range for this real option is 
$0 <  {\tt ma57\_pivtolmax } <  1$
and its default value is $0.0001$.


\paragraph{ma57\_pre\_alloc:} Safety factor for work space memory allocation for the linear solver MA57. $\;$ \\
 If 1 is chosen, the suggested amount of work
space is used.  However, choosing a larger number
might avoid reallocation if the suggest values do
not suffice.  This option is only available if
Ipopt has been compiled with MA57. The valid range for this real option is 
$1 \le {\tt ma57\_pre\_alloc } <  {\tt +inf}$
and its default value is $3$.


\paragraph{pardiso\_matching\_strategy:} Matching strategy to be used by Pardiso $\;$ \\
 This is IPAR(13) in Pardiso manual.  This option
is only available if Ipopt has been compiled with
Pardiso.
The default value for this string option is "complete+2x2".
\\ 
Possible values:
\begin{itemize}
   \item complete: Match complete (IPAR(13)=1)
   \item complete+2x2: Match complete+2x2 (IPAR(13)=2)
   \item constraints: Match constraints (IPAR(13)=3)
\end{itemize}

\paragraph{pardiso\_out\_of\_core\_power:} Enables out-of-core variant of Pardiso $\;$ \\
 Setting this option to a positive integer k makes
Pardiso work in the out-of-core variant where the
factor is split in 2\^k subdomains.  This is
IPARM(50) in the Pardiso manual.  This option is
only available if Ipopt has been compiled with
Pardiso. The valid range for this integer option is
$0 \le {\tt pardiso\_out\_of\_core\_power } <  {\tt +inf}$
and its default value is $0$.


\paragraph{wsmp\_num\_threads:} Number of threads to be used in WSMP $\;$ \\
 This determines on how many processors WSMP is
running on.  This option is only available if
Ipopt has been compiled with WSMP. The valid range for this integer option is
$1 \le {\tt wsmp\_num\_threads } <  {\tt +inf}$
and its default value is $1$.


\paragraph{wsmp\_ordering\_option:} Determines how ordering is done in WSMP $\;$ \\
 This corresponds to the value of WSSMP's
IPARM(16).  This option is only available if
Ipopt has been compiled with WSMP. The valid range for this integer option is
$-2 \le {\tt wsmp\_ordering\_option } \le 3$
and its default value is $1$.


\paragraph{wsmp\_pivtol:} Pivot tolerance for the linear solver WSMP. $\;$ \\
 A smaller number pivots for sparsity, a larger
number pivots for stability.  This option is only
available if Ipopt has been compiled with WSMP. The valid range for this real option is 
$0 <  {\tt wsmp\_pivtol } <  1$
and its default value is $0.0001$.


\paragraph{wsmp\_pivtolmax:} Maximum pivot tolerance for the linear solver WSMP. $\;$ \\
 Ipopt may increase pivtol as high as pivtolmax to
get a more accurate solution to the linear
system.  This option is only available if Ipopt
has been compiled with WSMP. The valid range for this real option is 
$0 <  {\tt wsmp\_pivtolmax } <  1$
and its default value is $0.1$.


\paragraph{wsmp\_scaling:} Determines how the matrix is scaled by WSMP. $\;$ \\
 This corresponds to the value of WSSMP's
IPARM(10). This option is only available if Ipopt
has been compiled with WSMP. The valid range for this integer option is
$0 \le {\tt wsmp\_scaling } \le 3$
and its default value is $0$.


\newpage
\section{Detailed Installation Information}\label{ExpertInstall}

The configuration script and Makefiles in the \Ipopt\ distribution
have been created using GNU's {\tt autoconf} and {\tt automake}.  They
attempt to automatically adapt the compiler settings etc.\ to the
system they are running on.  We tested the provided scripts for a
number of different machines, operating systems and compilers, but you
might run into a situation where the default setting does not work, or
where you need to change the settings to fit your particular
environment.

In general, you can see the list of options and variables that can be
set for the {\tt configure} script by typing \verb/configure --help/.
Below a few particular options are discussed:

\begin{itemize}
\item The {\tt configure} script tries to determine automatically, if
  you have BLAS and/or LAPACK already installed on your system (trying
  a few default libraries), and if it does not find them, it makes
  sure that you put the source code in the required place.

  However, you can specify a BLAS library (such as your local ATLAS
  library\footnote{see {\tt http://math-atlas.sourceforge.net/}})
  explicitly, using the \verb/--with-blas/ flag for {\tt configure}.
  For example,

  \verb|./configure --with=blas="-L$HOME/lib -latlas"|

  To tell the configure script to compile and use the downloaded BLAS
  source files even if a BLAS library is found on your system, specify
  \verb|--with-blas=BUILD|.

  Similarly, you can use the \verb/--with-lapack/ switch to specify
  the location of your LAPACK library, or use the keyword {\tt BUILD}
  to force the \Ipopt\ makefiles to compile LAPACK together with
  \Ipopt.

\item Similarly, if you have a precompiled library containing the
  Harwell Subroutines, you can specify its location with the
  \verb|--with-hsl| flag.  And the location of the AMPL solver library
  (with the ASL header files) can be specified with
  \verb|--with-asldir|.
  {\bf TODO Other linear solvers}

\item If you want to specify that you want to use particular
  compilers, you can do so by adding the variables definitions for
  {\tt CXX}, {\tt CC}, and {\tt F77} to the {\tt ./configure} command
  line, to specify the C++, C, and Fortran compiler, respectively.
  For example,

  {\tt ./configure CXX=g++ CC=gcc F77=g77}

  In order to set the compiler flags, you should use the variables
  {\tt CXXFLAGS}, {\tt CFLAGS}, {\tt FFLAGS}.  Note, that the \Ipopt\
  code uses ``{\tt dynamic\_cast}''.  Therefore it is necessary that
  the C++ code is compiled including RTTI (Run-Time Type Information).
  Some compilers need to be given special flags to do that (e.g.,
  ``{\tt -qrtti=dyna}'' for the AIX {\tt xlC} compiler).

\item If you want to link the \Ipopt\ library with a main program
  written in C or Fortran, the C and Fortran compiler doing the
  linking of the executable needs to be told about the C++ runtime
  libraries.  Unfortunately, the current version of {\tt autoconf}
  does not provide the automatic detection of those libraries.  We
  have hard-coded some default values for some systems and compilers,
  but this might not work all the time.

  If you have problems linking your Fortran or C code with the \Ipopt\
  library {\tt libipopt.a} and the linker complains about missing
  symbols from C++ (e.g., the standard template library), you should
  specify the C++ libraries with the {\tt CXXLIBS} variable.  To find out
  what those libraries are, it is probably helpful to link a  simple C++
  program with verbose compiler output.

  For example, for the Intel compilers on a Linux system, you
  might need to specify something like

  {\tt ./configure CC=icc F77=ifort CXX=icpc $\backslash$\\ \hspace*{14ex} CXXLIBS='-L/usr/lib/gcc-lib/i386-redhat-linux/3.2.3 -lstdc++'}

\item Compilation in 64bit mode sometimes requires some special
  consideration.  For example, for compilation of 64bit code on AIX,
  we recommend the following configuration

  {\tt ./configure AR='ar -X64' AR\_X='ar -X64 x' $\backslash$\\
    \hspace*{14ex} CC='xlc -q64' F77='xlf -q64' CXX='xlC
    -q64'$\backslash$\\ \hspace*{14ex} CFLAGS='-O3
    -bmaxdata:0x3f0000000'
    $\backslash$\\ \hspace*{14ex} FFLAGS='-O3 -bmaxdata:0x3f0000000' $\backslash$\\
    \hspace*{14ex} CXXFLAGS='-qrtti=dyna -O3 -bmaxdata:0x3f0000000'}

\item To build library/archive files (with the ending {\tt .a})
  including C++ code in some environments, it is necessary to use the
  C++ compiler instead of {\tt ar} to build the archive.  This is for
  example the case for some older compilers on SGI and SUN.  For this,
  the {\tt configure} variables {\tt AR}, {\tt ARFLAGS}, and {\tt
    AR\_X} are provided.  Here, {\tt AR} specifies the command for the
  archiver for creating an archive, and {\tt ARFLAGS} specifies
  additional flags.  {\tt AR\_X} contains the command for extracting
  all files from an archive.  For example, the default setting for SUN
  compilers for our configure script is

  {\tt AR='CC -xar' ARFLAGS='-o' AR\_X='ar x'}

\item It is possible to compile the \Ipopt\ library in a debug
  configuration, by specifying \verb|--enable-debug|.  Then the
  compilers will use the debug flags (unless the compilation flag
  variables are overwritten in the {\tt configure} command line), and
  additional debug checks are compiled into the code (see {\tt
    IpDebug.hpp}).  This usually leads to a significant slowdown of
  the code, but might be helpful when debugging something.

\item It is not necessary to produce the binary files in the
  directories where the source files are.  If you want to compile the
  code on different systems or with different compilers/options on a
  shared file system, you can keep one single copy of the source files
  in one directory, and the binary files for each configuration in
  separate directories.  For this, simply run the configure script in
  the directory where you want the base directory for the \Ipopt\
  binary files.  For example:

  {\tt \$ mkdir \$HOME/Ipopt-objects}\\
  {\tt \$ cd \$HOME/Ipopt-objects}\\
  {\tt \$ \$HOME/Ipopt/configure}  (or {\tt \$HOME/ipopt-3.1.0/configure})

\end{itemize}

%\bibliographystyle{plain}
%\bibliography{/home/andreasw/tex/andreas}
%% Copyright (C) 2005, 2006 Carnegie Mellon University and others.
%%
%% The first version of this file was contributed to the Ipopt project
%% on Aug 1, 2005, by Yoshiaki Kawajiri
%%                    Department of Chemical Engineering
%%                    Carnegie Mellon University
%%                    Pittsburgh, PA 15213
%%
%% Since then, the content of this file has been updated significantly by
%%     Carl Laird and Andreas Waechter        IBM
%%
%%
%% $Id$
%%
\documentclass[10pt]{article}
\setlength{\textwidth}{6.3in}       % Text width
\setlength{\textheight}{9.4in}      % Text height
\setlength{\oddsidemargin}{0.1in}     % Left margin for even-numbered pages
\setlength{\evensidemargin}{0.1in}    % Left margin for odd-numbered pages
\setlength{\topmargin}{-0.5in}         % Top margin
\renewcommand{\baselinestretch}{1.1}
\usepackage{amsfonts}
\usepackage{amsmath}

\newcommand{\RR}{{\mathbb{R}}}
\newcommand{\Ipopt}{{\sc Ipopt}}


\begin{document}
\title{Introduction to \Ipopt:\\
A tutorial for downloading, installing, and using \Ipopt.}

\author{Revision number of this document: $Revision$}

%\date{\today}
\maketitle

\begin{abstract}
  This document is a guide to using \Ipopt\ 3.1 (the new C++ version
  of \Ipopt).  It includes instructions on how to obtain and compile
  \Ipopt, a description of the interface, user options, etc.,, as
  well as a tutorial on how to solve a nonlinear optimization problem
  with \Ipopt.

  The initial version of this document was created by
  Yoshiaki Kawajir\footnote{Department of Chemical Engineering,
    Carnegie Mellon University, Pittsburgh PA} as a course project for
  \textit{47852 Open Source Software for Optimization}, taught by
  Prof. Fran\c cois Margot at Tepper School of Business, Carnegie
  Mellon University.  The current version is maintained by Carl
  Laird\footnote{Department of Chemical Engineering, Carnegie Mellon
    University, Pittsburgh PA} and Andreas
  W\"achter\footnote{Department of Mathematical Sciences, IBM T.J.\
    Watson Research Center, Yorktown Heights, NY}.
\end{abstract}

\tableofcontents

\vspace{\baselineskip}
\begin{small}
\noindent
The following names used in this document are trademarks or registered
trademarks: AMPL, IBM, Intel, Microsoft, Visual Studio C++, Visual
Studio C++ .NET
\end{small}

\section{Introduction}
\Ipopt\ (\underline{I}nterior \underline{P}oint \underline{Opt}imizer,
pronounced ``I--P--Opt'') is an open source software package for
large-scale nonlinear optimization. It can be used to solve general
nonlinear programming problems of the form
%\begin{subequations}\label{NLP}
\begin{eqnarray}
\min_{x\in\RR^n} &&f(x) \label{eq:obj} \\
\mbox{s.t.} \;  &&g^L \leq g(x) \leq g^U \\
                &&x^L \leq x \leq x^U, \label{eq:bounds}
\end{eqnarray}
%\end{subequations}
where $x \in \RR^n$ are the optimization variables (possibly with
lower and upper bounds, $x^L\in(\RR\cup\{-\infty\})^n$ and
$x^U\in(\RR\cup\{+\infty\})^n$), $f:\RR^n\longrightarrow\RR$ is the
objective function, and $g:\RR^n\longrightarrow \RR^m$ are the general
nonlinear constraints.  The functions $f(x)$ and $g(x)$ can be linear
or nonlinear and convex or non-convex (but should be twice
continuously differentiable). The constraints, $g(x)$, have lower and
upper bounds, $g^L\in(\RR\cup\{-\infty\})^n$ and
$g^U\in(\RR\cup\{+\infty\})^m$. Note that equality constraints of the
form $g_i(x)=\bar g_i$ can be specified by setting
$g^L_{i}=g^U_{i}=\bar g_i$.

\subsection{Mathematical Background}
\Ipopt\ implements an interior point line search filter method that
aims to find a local solution of (\ref{eq:obj})-(\ref{eq:bounds}).  The
mathematical details of the algorithm can be found in several
publications
\cite{NocWaeWal:adaptive,WaechterPhD,WaecBieg06:mp,WaeBie05:filterglobal,WaeBie05:filterlocal}.

\subsection{Availability}
The \Ipopt\ package is available from COIN-OR
(\texttt{www.coin-or.org}) under the CPL (Common Public License)
open-source license and includes the source code for \Ipopt.  This
means, it is available free of charge, also for commercial purposes.
However, if you give away software including \Ipopt\ code (in source
code or binary form) and you made changes to the \Ipopt\ source code,
you are required to make those changes public and to clearly indicate
which modifications you made.  After all, the goal of open source
software is the continuous development and improvement of software.
For details, please refer to the Common Public License.

Also, if you are using \Ipopt\ to obtain results for a publication, we
politely ask you to point out in your paper that you used \Ipopt, and
to cite the publication \cite{WaecBieg06:mp}.  Writing high-quality
numerical software takes a lot of time and effort, and does usually
not translate into a large number of publications, therefore we believe
this request is only fair :).

\subsection{Prerequisites}
In order to build \Ipopt, some third party components are required:
\begin{itemize}
\item BLAS (Basic Linear Algebra Subroutines).  Many vendors of
  compilers and operating systems provide precompiled and optimized
  libraries for these dense linear algebra subroutines.  But you can
  also get the source code from {\tt www.netlib.org} and have the
  \Ipopt\ distribution compile it automatically.
\item LAPACK (Linear Algebra PACKage).  Also for LAPACK, some vendors
  offer precompiled and optimized libraries.  But like with BLAS, you
  can get the source code from {\tt www.netlib.org} and have the
  \Ipopt\ distribution compile it automatically.

  Note that currently LAPACK is only required if you intend to use the
  quasi-Newton options in \Ipopt.  You can compile the code without
  LAPACK, but an error message will then occur if you try to run the
  code with an option that requires LAPACK.  Currently, the LAPACK
  routines that are used by \Ipopt\ are only {\tt DPOTRF}, {\tt
    DPOTRS}, and {\tt DSYEV}.
\item A sparse symmetric indefinite linear solver. The \Ipopt\ needs
  to obtain the solution of sparse, symmetric, indefinite linear
  systems, and for this it relies on third-party code.  

  Currently, the following linear solvers can be used:
  \begin{itemize}
  \item MA27 from the Harwell Subroutine Library\\ (see {\tt
      http://www.cse.clrc.ac.uk/nag/hsl/}).
  \item MA57 from the Harwell Subroutine Library\\ (see {\tt
      http://www.cse.clrc.ac.uk/nag/hsl/}).
  \item The Watson Sparse Matrix Package (WSMP)\\ (see {\tt
      http://www-users.cs.umn.edu/\~agupta/wsmp.html})
  \item The Parallel Sparse Direct Linear Solver (PARDISO)\\ (see {\tt
      http://www.computational.unibas.ch/cs/scicomp/software/pardiso/}).
  \end{itemize}
  You need to include at least one of the linear solvers above in
  order to run \Ipopt.

  Interfaces to other linear solvers might be added in the future; if
  you are interested in contributing such an interface please contact
  us!  Note that \Ipopt\ requires that the linear solver is able to
  provide the inertia (number of positive and negative eigenvalues) of
  the symmetric matrix that is factorized.

\item Furthermore, \Ipopt\ can also use the Harwell Subroutine MC19
  for scaling of the linear systems before they are passed to the
  linear solver.  This may be particularly useful if \Ipopt\ is used
  with MA27 or MA57.  However, it is not required to have MC19 to
  compile \Ipopt; if this routine is missing, the scaling is never
  performed.
\item ASL (AMPL Solver Library).  The source code is available at {\tt
    www.netlib.org}, and the \Ipopt\ makefiles will automatically
  compile it for you if you put the source code into a designated
  space.  NOTE: This is only required if you want to use \Ipopt\ from
  AMPL and want to compile the \Ipopt\ AMPL solver executable.
\end{itemize}
For more information on third-party components and how to obtain them,
see Section~\ref{ExternalCode}.

Since the \Ipopt\ code is written in C++, you will need a C++ compiler
to build the \Ipopt\ library.  We tried very hard to write the code as
platform and compiler independent as possible.

In addition, the configuration script currently also searches for a
Fortran, since some of the dependencies above are written in Fortran.
If all third party dependencies are available as self-contained
libraries, those compilers are in principle not necessary.  Also, it
is possible to use the Fortran-to-C compiler {\tt f2c} from {\tt
  www.netlib.org} to convert Fortran code to C, and compile the
resulting C files with a C compiler and create a library containing
the required third party dependencies.  But so far we have not tested
this ourselves, and currently the configuration script for \Ipopt\
looks for a Fortran compiler.

\subsection{How to use \Ipopt}
If desired, the \Ipopt\ distribution generates an executable for the
modeling environment AMPL. As well, you can link your problem
statement with \Ipopt\ using interfaces for C++, C, or Fortran.
\Ipopt\ can be used with most Linux/Unix environments, and on Windows
using Visual Studio .NET or Cygwin.  Below in
Section~\ref{sec:tutorial-example} this document demonstrates how to
solve problems using \Ipopt. This includes installation and
compilation of \Ipopt\ for use with AMPL as well as linking with your
own code.

Finally, the \Ipopt\ distribution includes an interface for {\tt
  CUTEr}\footnote{see {\tt http://cuter.rl.ac.uk/cuter-www/}}, if you
want to use \Ipopt\ to solve problems modeled in SIF.

The old (Fortran 2.x) version of \Ipopt\ has been interface with
Matlab, and is also available from NEOS, and the new version will be
available through similar means in the future.  Please check the
\Ipopt\ homepage for updates.

\subsection{More Information and Contributions}
More and up-to-date information can be found at the \Ipopt\ homepage,

\begin{center}
\texttt{http://projects.coin-or.org/Ipopt}.
\end{center}

Here, you can find FAQs, some (hopefully useful) hints, a bug report
system etc.  The website is managed with Wiki, which means that every
user can edit the webpages from the regular web browser.  In
particular, we encourage \Ipopt\ users to share their experiences and
usage hints on the ``Success Stories'' and ``Hints and Tricks''
pages\footnote{Since we had some malicious hacker attacks destroying
  the content of the web pages in the past, you are now required to
  enter a user name and password; simply follow the instructions in
  the last paragraph of the Documentation section on the main project
  page.}

\Ipopt\ is an open source project, and we encourage people to
contribute code (such as interfaces to appropriate linear solvers,
modeling environments, or even algorithmic features).  If you are
interested in contributing code, please have a look at the COIN
constributions webpage\footnote{see \tt
  http://www.coin-or.org/contributions.html}, and contact the \Ipopt\
project leader.

There is also a mailing list for \Ipopt, available from the webpage
\begin{center}
\texttt{http://list.coin-or.org/mailman/listinfo/coin-ipopt},
\end{center}
where you can
subscribe to get notified of updates, and to ask general questions
regarding installation and usage. (You might want to look at the
archives before posting a question.)

We try to answer questions posted to the mailing list in a reasonable
manner.  Please understand that we cannot answer all questions in
detail, and because of time constraints, we may not be able to help
you model and debug your particular optimization problem.  However, if
you have a challenging optimization problem and are interested in
consulting services by IBM Research, please contact the \Ipopt\
project leader, Andreas W\"achter.

\subsection{History of \Ipopt}
The original \Ipopt\ (Fortran version) was a product of the dissertation
research of Andreas W\"achter \cite{WaechterPhD}, under Lorenz
T. Biegler at the Chemical Engineering Department at Carnegie Mellon
University. The code was made open source and distributed by the
COIN-OR initiative, which is now a non-profit corporation.  \Ipopt\ has
been actively developed under COIN-OR since 2002.

To continue natural extension of the code and allow easy addition of
new features, IBM Research decided to invest in an open source
re-write of \Ipopt\ in C++.  The new C++ version of the \Ipopt\
optimization code (\Ipopt\ 3.0 and beyond) is currently developed at IBM
Research and remains part of the COIN-OR initiative. Future
development on the Fortran version will cease with the exception of
occasional bug fix releases.

\section{Installing \Ipopt}\label{Installing}

The following sections describe the installation procedures on
UNIX/Linux systems.  For installation instructions on Windows
see Section~\ref{WindowsInstall}.

\subsection{Getting the \Ipopt\ Code}
\Ipopt\ is available from the COIN-OR subversion repository. You can
either download the code using \texttt{svn} (the
\textit{subversion}\footnote{see
  \texttt{http://subversion.tigris.org/}} client similar to CVS) or
simply retrieve a tarball (compressed archive file).  While the
tarball is an easy method to retrieve the code, using the
\textit{subversion} system allows users the benefits of the version
control system, including easy updates and revision control.

\subsubsection{Getting the \Ipopt\ code via subversion}

Of course, the \textit{subversion} client must be installed on your
system if you want to obtain the code this way (the executable is
called \texttt{svn}); it is already installed by default for many
recent Linux distributions.  Information about \textit{subversion} and
how to download it can be found at
\texttt{http://subversion.tigris.org/}.\\

To obtain the \Ipopt\ source code via subversion, change into the
directory in which you want to create a subdirectory {\tt Ipopt} with
the \Ipopt\ source code.  Then follow the steps below:
\begin{enumerate}
\item{Download the code from the repository}\\
{\tt \$ svn co https://www.coin-or.org/svn/Ipopt/trunk Ipopt} \\
Note: The {\tt \$} indicates the command line
prompt, do not type {\tt \$}, only the text following it.
\item Change into the root directory of the \Ipopt\ distribution\\
{\tt \$ cd Ipopt}
\end{enumerate}

In the following, ``\texttt{\$IPOPTDIR}'' will refer to the directory in
which you are right now (output of \texttt{pwd}).

\subsubsection{Getting the \Ipopt\ code as a tarball}

To use the tarball, follow the steps below:
\begin{enumerate}
\item Download the latest tarball from
\texttt{http://www.coin-or.org/Tarballs}.  The file you should look
for has the form \texttt{ipopt-3.x.x.tar.gz} (where
``\texttt{3.x.x.}'' is the version number).  Put this file in a
directory under which you want to put the \Ipopt\ installation.
\item Issue the following commands to unpack the archive file: \\
\texttt{\$ gunzip ipopt-3.x.x.tar.gz} \\
\texttt{\$ tar xvf ipopt-3.x.x.tar} \\
Note: The {\tt \$} indicates the command line
prompt, do not type {\tt \$}, only the text following it.
\item Change into the root directory of the \Ipopt\ distribution\\
{\tt \$ cd ipopt-3.x.x}
\end{enumerate}

In the following, ``\texttt{\$IPOPTDIR}'' will refer to the directory in
which you are right now (output of \texttt{pwd}).

\subsection{Download External Code}\label{ExternalCode}
\Ipopt\ uses a few external packages that are not included in the
\Ipopt\ source code distribution, namely ASL (the AMPL Solver
Library), BLAS, LAPACK.  It also requires a sparse symmetric linear
solver.

Since this third party software released under different licenses than
\Ipopt, we cannot distribute that code together with the \Ipopt\
packages and have to ask you to go through the hassle of obtaining it
yourself (even though we tried to make it as easy for you as we
could).  Keep in mind that it is still your responsibility to ensure
that your downloading and usage if the third party components conforms
with their licenses.

Note that you only need to obtain the ASL if you intend to use \Ipopt\
from AMPL.  It is not required if you want to specify your
optimization problem in a programming language (C++, C, or Fortran).
Also, currently, LAPACK is only required if you intend to use the
quasi-Newton options implemented in \Ipopt.

\subsubsection{Download BLAS, LAPACK and ASL}
If you have the download utility \texttt{wget} installed on your
system, retrieving BLAS, LAPACK, and ASL is straightforward using
scripts included with the ipopt distribution. These scripts download
the required files from the Netlib Repository
(\texttt{www.netlib.org}).\\

\noindent
{\tt \$ cd \$IPOPTDIR/Extern/blas}\\
{\tt \$ ./get.blas}\\
{\tt \$ cd ../lapack}\\
{\tt \$ ./get.lapack}\\
{\tt \$ cd ../ASL}\\
{\tt \$ ./get.ASL}\\

\noindent
If you do not have \texttt{wget} installed on your system, please read
the \texttt{INSTALL.*} files in the \texttt{\$IPOPTDIR/Extern/blas},
\texttt{\$IPOPTDIR/Extern/lapack} and \texttt{\$IPOPTDIR/Extern/ASL}
directories for alternative instructions.

\subsubsection{Download HSL Subroutines}
\Ipopt\ requires a sparse symmetric linear solver.  There are
different possibilities.  In this section we describe how to obtain
the source code for MA27 (and MC19) from the Harwell Subroutine
Library (HSL).  Those routines are freely available for
non-commercial, academic use, but it is your responsibility to
investigate the licensing of all third party code.

The use of alternative linear solvers is described in
Appendix~\ref{ExpertInstall}.  You do not necessarily have to use MA27
as described in this section, but at least one linear solver is
required for \Ipopt\ to function.

\begin{enumerate}
\item Go to {\tt http://hsl.rl.ac.uk/archive/hslarchive.html}
\item Follow the instruction on the website, read the license, and
  submit the registration form.
\item Go to \textit{HSL Archive Programs}, and find the package list.
\item In your browser window, click on \textit{MA27}.
\item Make sure that \textit{Double precision:} is checked. 
  Click \textit{Download package (comments removed)}
\item Save the file as {\tt ma27ad.f} in {\tt \$IPOPTDIR/Extern/HSL/}\\
  Note: Some browsers append a file extension ({\tt .txt}) when you save
  the file, in which case you have to rename it.
\item Go back to the package list using the back button of your browser.
\item In your browser window, click on \textit{MC19}.
\item Make sure \textit{Double precision:} is checked. Click 
  \textit{Download package (comments removed)}
\item Save the file as {\tt mc19ad.f} in {\tt
    \$IPOPTDIR/Extern/HSL/}\\
  Note: Some browsers append a file extension ({\tt .txt}) when you save
  the file, so you may have to rename it.
\end{enumerate}

Note: Whereas currently obtaining MA27 is essential for using \Ipopt,
MC19 could be omitted (with the consequence that you cannot use this
method for scaling the linear systems arising inside the \Ipopt\
algorithm).

Note: If you have the source code for the linear solver MA57
(successor of MA27) in a file called ma57ad.f (including all
dependencies), you can simply put it into the {\tt
  \$IPOPTDIR/Extern/HSL/} directory.  The \Ipopt\ configuration script
will then find this file and compile it into the \Ipopt\ library (just
as is would compile MA27).

\subsection{Compiling and Installing \Ipopt} \label{sec.comp_and_inst}

\Ipopt\ can be easily compiled and installed with the usual {\tt
  configure}, {\tt make}, {\tt make install} commands.  Below are the
basic steps that should work on most systems.  For special
compilations and for some troubleshooting see
Appendix~\ref{ExpertInstall} and consult the \Ipopt\ homepage before
submitting a ticket or sending a message to the mailing list.
\begin{enumerate}
\item Go to the main directory of \Ipopt:\\
  {\tt \$ cd \$IPOPTDIR} 
\item Run the configure script\\
  {\tt \$ ./configure}

  If the last output line of the script reads ``\texttt{configure:
    Configuration successful}'' then everything worked fine.
  Otherwise, look at the screen output, have a look at the
  \texttt{config.log} output file and/or consult
  Appendix~\ref{ExpertInstall}.

  The default configure (without any options) is sufficient for most
  users. If you want to see the configure options, consult
  Appendix~\ref{ExpertInstall}.
\item Build the code \\
{\tt \$ make}
\item Install \Ipopt \\
  {\tt \$ make install}\\
  This installs
  \begin{itemize}
  \item the \Ipopt\ AMPL solver executable (if ASL source was
    downloaded) in \texttt{\$IPOPTDIR/bin},
  \item the \Ipopt\ library (\texttt{libipopt.a}) in
    \texttt{\$IPOPTDIR/lib},
  \item text files {\tt ipopt\_addlibs\_cpp.txt} and {\tt
      ipopt\_addlibs\_f.txt} in \texttt{\$IPOPTDIR/lib} that contain a
    line each with additional linking flags that are required for
    linking code with the ipopt library, for C++ and Fortran main
    programs, respectively. (This is only for convenience if you want
    to find out what additional flags are required, for example, to
    include the Fortran runtime libraries with a C++ compiler.)
  \item the necessary header files in
    \texttt{\$IPOPTDIR/include/ipopt}.
  \end{itemize}
  You can change the default installation directory (here
  \texttt{\$IPOPTDIR}) to something else (such as \texttt{/usr/local})
  by using the \verb|--prefix| switch for \texttt{configure}.
%\item Test the installation \\
%  {\tt \$ make test}\\
%  This should ?...?
\item Install \Ipopt\ for use with {\tt CUTEr}\\
  If you have {\tt CUTEr} already installed on your system and you
  want to use \Ipopt\ as a solver for problems modeled in {\tt SIF},
  type\\
  {\tt \$ make cuter}\\
  This assumes that you have the environment variable {\tt MYCUTER}
  defined according to the {\tt CUTEr} instructions.  After this, you
  can use the script {\tt sdipo} as the {\tt CUTEr} script to solve a
  {\tt SIF} model.
\end{enumerate}

\subsection{Installation on Windows}\label{WindowsInstall}

There are two ways to install \Ipopt\ on Windows systems.  The first
option, described in Section~\ref{CygwinInstall}, is to use Cygwin (see
\texttt{www.cygwin.com}), which offers a UNIX-like environment
on Windows and in which the installation procedure described earlier
in this section can be used.  The \Ipopt\ distribution also includes
projects files for the Microsoft Visual Studio (see
Section~\ref{VisualStudioInstall}).

\subsubsection{Installation with Cygwin}\label{CygwinInstall}

Cygwin is a Linux-like environment for Windows; if you don't know what
it is you might want to have a look at the Cygwin homepage,
\texttt{www.cygwin.com}.

It is possible to build the \Ipopt\ AMPL solver executable in Cygwin
for general use in Windows.  You can also hook up \Ipopt\ to your own
program if you compile it in the Cygwin environment\footnote{It is
  also possible to build an \Ipopt\ DLL that can be used from
  non-cygwin compilers, but this is not (yet?) supported.}.

If you want to compile \Ipopt\ under Cygwin, you first have to install
Cygwin on your Windows system.  This is pretty straight forward; you
simply download the ``setup'' program from
\texttt{www.cygwin.com} and start it.

Then you do the following steps (assuming here that you don't have any
complications with firewall settings etc - in that case you might have
to choose some connection settings differently):

\begin{enumerate}
\item Click next
\item Select ``install from the internet'' (default) and click next
\item Select a directory where Cygwin is to be installed (you can
  leave the default) and choose all other things to your liking, then
  click next
\item Select a temp dir for Cygwin setup to store some files (if you
  put it on your desktop you will later remember to delete it)
\item Select ``direct connection'' (default) and click next
\item Select some mirror site that seems close by to you and click next
\item OK, now comes the complicated part:\\
  You need to select the packages that you want to have installed.  By
  default, there are already selections, but the compilers are usually
  not pre-chosen.  You need to make sure that you select the GNU
  compilers (for Fortran, C, and C++ --- together with the MinGW
  options), the GNU Make, and Subversion.  For this, click on the "Devel"
  branch (which opens a subtree) and select:
  \begin{itemize}
  \item gcc
  \item gcc-core
  \item gcc-g77
  \item gcc-g++
  \item gcc-mingw
  \item gcc-mingw-core
  \item gcc-mingw-g77
  \item gcc-mingw-g++
  \item make
  \item subversion
  \end{itemize}

  Then, in the ``Web'' branch, please select ``wget'' (which will make
  the installation of third party dependencies for \Ipopt\ easier)

  This will automatically also select some other packages.
\item Then you click on next, and Cygwin will be installed (follow the
  rest of the instructions and choose everything else to your liking).
  At a later point you can easily add/remove packages with the setup
  program.

\item Now that you have Cygwin, you can open a Cygwin window, which is
  like a UNIX shell window.

\item Now you just follow the instructions in the beginning of
  Sections~\ref{Installing}:  You download the \Ipopt\ code into
  your Cygwin home directory (from the Windows explorer that is
  usually something like
  \texttt{C:$\backslash$Cygwin$\backslash$home$\backslash$your\_user\_name}).
  After that you obtain the third party code (like on Linux/UNIX),
  type

  \texttt{./configure}

  and

  \texttt{make install}

  in the correct directories, and hopefully that will work.  The
  \Ipopt\ AMPL solver executable will be in the subdirectory
  \texttt{bin} (called ``\texttt{ipopt.exe}'').
\end{enumerate}

\subsubsection{Using Visual Studio}\label{VisualStudioInstall}

The \Ipopt\ distribution includes project files that can be used to
compile the \Ipopt\ library and a Fortran and C++ example within the
Microsoft Visual Studio.  The project files have been created with
Microsoft Visual C++ .NET 2003 Standard, and the Intel Visual Fortran
Compiler 8.1.

In order to use those project files, download the \Ipopt\ source code,
as well as the required third party code (put it into the {\tt
  Extern/blas}, {\tt Extern/lapack}, and {\tt Extern/HSL}
directories---ASL is not required for the Fortran and C
examples). Then open the solution file\\

\texttt{\$IPOPTDIR$\backslash$Windows$\backslash$VisualStudio\_dotNET$\backslash$Ipopt$\backslash$Ipopt.sln}\\

Note: Since the project files were created only with the Standard
edition of the C++ compiler, code optimization might be disabled; for
fast performance make sure you enable code optimization.

\section{Interfacing your NLP to \Ipopt: A tutorial example.}
\label{sec:tutorial-example}

\Ipopt\ has been designed to be flexible for a wide variety of
applications, and there are a number of ways to interface with \Ipopt\
that allow specific data structures and linear solver
techniques. Nevertheless, the authors have included a standard
representation that should meet the needs of most users.

This tutorial will discuss four interfaces to \Ipopt, namely the AMPL
modeling language\cite{FouGayKer:AMPLbook} interface, and the C++, C,
and Fortran code interfaces.  AMPL is a 3rd party modeling language
tool that allows users to write their optimization problem in a syntax
that resembles the way the problem would be written mathematically.
Once the problem has been formulated in AMPL, the problem can be
easily solved using the (already compiled) \Ipopt\ AMPL solver
executable, {\tt ipopt}. Interfacing your problem by directly linking
code requires more effort to write, but can be far more efficient for
large problems.

We will illustrate how to use each of the four interfaces using an
example problem, number 71 from the Hock-Schittkowsky test suite \cite{HS},
%\begin{subequations}\label{HS71}
  \begin{eqnarray}
    \min_{x \in \Re^4} &&x_1 x_4 (x_1 + x_2 + x_3)  +  x_3 \label{eq:ex_obj} \\
    \mbox{s.t.}  &&x_1 x_2 x_3 x_4 \ge 25 \label{eq:ex_ineq} \\
    &&x_1^2 + x_2^2 + x_3^2 + x_4^2  =  40 \label{eq:ex_equ} \\
    &&1 \leq x_1, x_2, x_3, x_4 \leq 5, \label{eq:ex_bounds}
  \end{eqnarray}
%\end{subequations}
with the starting point
\begin{equation}
x_0 = (1, 5, 5, 1) \label{eq:ex_startpt}
\end{equation}
and the optimal solution
\[
x_* = (1.00000000, 4.74299963, 3.82114998, 1.37940829). \nonumber
\]

\subsection{Using \Ipopt\ through AMPL}
Using the AMPL solver executable is by far the easiest way to
solve a problem with \Ipopt. The user must simply formulate the problem
in AMPL syntax, and solve the problem through the AMPL environment.
There are drawbacks, however. AMPL is a 3rd party package and, as
such, must be appropriately licensed (a free student version for
limited problem size is available from the AMPL website,
\texttt{www.ampl.com}). Furthermore, the AMPL environment may be prohibitive
for very large problems. Nevertheless, formulating the problem in AMPL
is straightforward and even for large problems, it is often used as a
prototyping tool before using one of the code interfaces.

This tutorial is not intended as a guide to formulating models in
AMPL. If you are not already familiar with AMPL, please consult
\cite{FouGayKer:AMPLbook}.

The problem presented in equations
(\ref{eq:ex_obj})--(\ref{eq:ex_startpt}) can be solved with \Ipopt\ with
the AMPL model file given in Figure~\ref{fig:HS71}.

\begin{figure}
  \centering
\begin{footnotesize}
\begin{verbatim}
# tell ampl to use the ipopt executable as a solver
# make sure ipopt is in the path!
option solver ipopt;

# declare the variables and their bounds, 
# set notation could be used, but this is straightforward
var x1 >= 1, <= 5; 
var x2 >= 1, <= 5; 
var x3 >= 1, <= 5; 
var x4 >= 1, <= 5;

# specify the objective function
minimize obj:
                x1 * x4 * (x1 + x2 + x3) + x3;
        
# specify the constraints
s.t.
        inequality:
                x1 * x2 * x3 * x4 >= 25;
                
        equality:
                x1^2 + x2^2 + x3^2 +x4^2 = 40;

# specify the starting point            
let x1 := 1;
let x2 := 5;
let x3 := 5;
let x4 := 1;

# solve the problem
solve;

# print the solution
display x1;
display x2;
display x3;
display x4;
\end{verbatim}
\end{footnotesize}
  
  \caption{AMPL model file hs071\_ampl.mod}
  \label{fig:HS71}
\end{figure}

The line, ``{\tt option solver ipopt;}'' tells AMPL to use \Ipopt\ as
the solver. The \Ipopt\ executable (installed in
Section~\ref{sec.comp_and_inst}) must be in the {\tt PATH} for AMPL to
find it. The remaining lines specify the problem in AMPL format. The
problem can now be solved by starting AMPL and loading the mod file:
\begin{verbatim}
$ ampl
> model hs071_ampl.mod;
.
.
.
\end{verbatim}
%$
The problem will be solved using \Ipopt\ and the solution will be
displayed.

At this point, AMPL users may wish to skip the sections about
interfacing with code, but should read Section \ref{sec.options}
concerning \Ipopt\ options, and Section \ref{sec.output} which
explains the output displayed by \Ipopt.

\subsection{Interfacing with \Ipopt\ through code}
In order to solve a problem, \Ipopt\ needs more information than just
the problem definition (for example, the derivative information). If
you are using a modeling language like AMPL, the extra information is
provided by the modeling tool and the \Ipopt\ interface. When
interfacing with \Ipopt\ through your own code, however, you must
provide this additional information.

\begin{figure}
\begin{enumerate}
\item Problem dimensions \label{it.prob_dim}
  \begin{itemize}
  \item number of variables
  \item number of constraints
  \end{itemize}
\item Problem bounds
  \begin{itemize}
  \item variable bounds
  \item constraint bounds
  \end{itemize}
\item Initial starting point
  \begin{itemize}
  \item Initial values for the primal $x$ variables
  \item Initial values for the multipliers (only
    required for a warm start option)
  \end{itemize}
\item Problem Structure \label{it.prob_struct}
  \begin{itemize}
  \item number of nonzeros in the Jacobian of the constraints
  \item number of nonzeros in the Hessian of the Lagrangian function
  \item sparsity structure of the Jacobian of the constraints
  \item sparsity structure of the Hessian of the Lagrangian function
  \end{itemize}
\item Evaluation of Problem Functions \label{it.prob_eval} \\
  Information evaluated using a given point ($x,
  \lambda, \sigma_f$ coming from \Ipopt)
  \begin{itemize}
  \item Objective function, $f(x)$
  \item Gradient of the objective $\nabla f(x)$
  \item Constraint function values, $g(x)$
  \item Jacobian of the constraints, $\nabla g(x)^T$
  \item Hessian of the Lagrangian function, 
    $\sigma_f \nabla^2 f(x) + \sum_{i=1}^m\lambda_i\nabla^2
    g_i(x)$ \\
    (this is not required if a quasi-Newton options is chosen to
    approximate the second derivatives)
  \end{itemize}
\end{enumerate}
\caption{Information required by \Ipopt}
\label{fig.required_info}
\end{figure}
%\vspace{0.1in}
The information required by \Ipopt\ is shown in Figure
\ref{fig.required_info}. The problem dimensions and bounds are
straightforward and come solely from the problem definition. The
initial starting point is used by the algorithm when it begins
iterating to solve the problem. If \Ipopt\ has difficulty converging, or
if it converges to a locally infeasible point, adjusting the starting
point may help.  Depending on the starting point, \Ipopt\ may also
converge to different local solutions.

Providing the sparsity structure of derivative matrices is a bit more
involved. \Ipopt\ is a nonlinear programming solver that is designed
for solving large-scale, sparse problems. While \Ipopt\ can be
customized for a variety of matrix formats, the triplet format is used
for the standard interfaces in this tutorial. For an overview of the
triplet format for sparse matrices, see Appendix~\ref{app.triplet}.
Before solving the problem, \Ipopt\ needs to know the number of
nonzero elements and the sparsity structure (row and column indices of
each of the nonzero entries) of the constraint Jacobian and the
Lagrangian function Hessian. Once defined, this nonzero structure MUST
remain constant for the entire optimization procedure. This means that
the structure needs to include entries for any element that could ever
be nonzero, not only those that are nonzero at the starting point.

As \Ipopt\ iterates, it will need the values for
Item~\ref{it.prob_eval}. in Figure~\ref{fig.required_info} evaluated at
particular points. Before we can begin coding the interface, however,
we need to work out the details of these equations symbolically for
example problem (\ref{eq:ex_obj})-(\ref{eq:ex_bounds}).

The gradient of the objective $f(x)$ is given by
\[%\begin{equation}
\left[
\begin{array}{c}
x_1 x_4 + x_4 (x_1 + x_2 + x_3) \\
x_1 x_4 \\
x_1 x_4 + 1 \\
x_1 (x_1 + x_2 + x_3)
\end{array}
\right],
\]%\end{equation}
and the Jacobian of the constraints $g(x)$ is
\[%\begin{equation}
\left[
\begin{array}{cccc}
x_2 x_3 x_4     & x_1 x_3 x_4   & x_1 x_2 x_4   & x_1 x_2 x_3   \\
2 x_1           & 2 x_2         & 2 x_3         & 2 x_4
\end{array}
\right].
\]%\end{equation}

We also need to determine the Hessian of the Lagrangian\footnote{If a
  quasi-Newton option is chosen to approximate the second derivatives,
  this is not required.  However, if second derivatives can be
  computed, it is often worthwhile to let \Ipopt\ use them, since the
  algorithm is then usually more robust and converges faster.  More on
  the quasi-Newton approximation in Section~\ref{sec:quasiNewton}.}.
The Lagrangian function for the NLP
(\ref{eq:ex_obj})-(\ref{eq:ex_bounds}) is defined as $f(x) + g(x)^T
\lambda$ and the Hessian of the Lagrangian function is, technically, $
\nabla^2 f(x_k) + \sum_{i=1}^m\lambda_i\nabla^2 g_i(x_k)$.  However,
so that \Ipopt\ can ask for the Hessian of the objective or the
constraints independently if required, we introduce a factor
($\sigma_f$) in front of the objective term.
%
For \Ipopt\ then, the symbolic form of the Hessian of the
Lagrangian is
\begin{equation}\label{eq:IpoptLAG}
\sigma_f \nabla^2 f(x_k) + \sum_{i=1}^m\lambda_i\nabla^2 g_i(x_k)
\end{equation}
(with the $\sigma_f$ parameter), and for the example problem this becomes
%\begin{eqnarray}
%{\cal L}(x,\lambda) &{=}& f(x) + c(x)^T \lambda \nonumber \\
%&{=}& \left(x_1 x_4 (x_1 + x_2 + x_3)  +  x_3\right) 
%+ \left(x_1 x_2 x_3 x_4\right) \lambda_1 \nonumber \\
%&& \;\;\;\;\;+ \left(x_1^2 + x_2^2 + x_3^2 + x_4^2\right) \lambda_2 
%- \displaystyle \sum_{i \in 1..4} z^L_i + \sum_{i \in 1..4} z^U_i
%\end{eqnarray}
\[%\begin{equation}
\sigma_f \left[
\begin{array}{cccc}
2 x_4           & x_4           & x_4           & 2 x_1 + x_2 + x_3     \\
x_4             & 0             & 0             & x_1                   \\
x_4             & 0             & 0             & x_1                   \\
2 x_1+x_2+x_3   & x_1           & x_1           & 0
\end{array}
\right]
+
\lambda_1
\left[
\begin{array}{cccc}
0               & x_3 x_4       & x_2 x_4       & x_2 x_3       \\
x_3 x_4         & 0             & x_1 x_4       & x_1 x_3       \\
x_2 x_4         & x_1 x_4       & 0             & x_1 x_2       \\
x_2 x_3         & x_1 x_3       & x_1 x_2       & 0 
\end{array}
\right]
+
\lambda_2
\left[
\begin{array}{cccc}
2       & 0     & 0     & 0     \\
0       & 2     & 0     & 0     \\
0       & 0     & 2     & 0     \\
0       & 0     & 0     & 2
\end{array}
\right]
\]%\end{equation}
where the first term comes from the Hessian of the objective function,
and the second and third term from the Hessian of the constraints
(\ref{eq:ex_ineq}) and (\ref{eq:ex_equ}), respectively. Therefore, the
dual variables $\lambda_1$ and $\lambda_2$ are then the multipliers
for constraints (\ref{eq:ex_ineq}) and (\ref{eq:ex_equ}), respectively.

%C =============================================================================
%C
%C     This is an example for the usage of IPOPT.
%C     It implements problem 71 from the Hock-Schittkowsky test suite:
%C
%C     min   x1*x4*(x1 + x2 + x3)  +  x3
%C     s.t.  x1*x2*x3*x4                   >=  25
%C           x1**2 + x2**2 + x3**2 + x4**2  =  40
%C           1 <=  x1,x2,x3,x4  <= 5
%C
%C     Starting point:
%C        x = (1, 5, 5, 1)
%C
%C     Optimal solution:
%C        x = (1.00000000, 4.74299963, 3.82114998, 1.37940829)
%C
%C =============================================================================
\vspace{\baselineskip}

The remaining sections of the tutorial will lead you through
the coding required to solve example problem
(\ref{eq:ex_obj})--(\ref{eq:ex_bounds}) using, first C++, then C, and finally
Fortran. Completed versions of these examples can be found in {\tt
\$IPOPTDIR/Examples} under {\tt hs071\_cpp}, {\tt hs071\_c}, {\tt
hs071\_f}.

As a user, you are responsible for coding two sections of the program
that solves a problem using \Ipopt: the main executable (e.g., {\tt
  main}) and the problem representation.  Typically, you will write an
executable that prepares the problem, and then passes control over to
\Ipopt\ through an {\tt Optimize} or {\tt Solve} call. In this call,
you will give \Ipopt\ everything that it requires to call back to your
code whenever it needs functions evaluated (like the objective
function, the Jacobian of the constraints, etc.).  In each of the
three sections that follow (C++, C, and Fortran), we will first
discuss how to code the problem representation, and then how to code
the executable.

\subsection{The C++ Interface}
This tutorial assumes that you are familiar with the C++ programming
language, however, we will lead you through each step of the
implementation. For the problem representation, we will create a class
that inherits off of the pure virtual base class, {\tt TNLP} ({\tt
  IpTNLP.hpp}). For the executable (the {\tt main} function) we will
make the call to \Ipopt\ through the {\tt IpoptApplication} class
({\tt IpIpoptApplication.hpp}). In addition, we will also be using the
{\tt SmartPtr} class ({\tt IpSmartPtr.hpp}) which implements a reference
counting pointer that takes care of memory management (object
deletion) for you (for details, see Appendix~\ref{app.smart_ptr}).

After ``\texttt{make install}'' (see Section~\ref{sec.comp_and_inst}),
the header files are installed in \texttt{\$IPOPTDIR/include/ipopt}
(or in \texttt{\$PREFIX/include/ipopt} if the switch
\verb|--prefix=$PREFIX| was used for {\tt configure}).

\subsubsection{Coding the Problem Representation}\label{sec.cpp_problem}
We provide the information required in Figure \ref{fig.required_info}
by coding the {\tt HS071\_NLP} class, a specific implementation of the
{\tt TNLP} base class. In the executable, we will create an instance
of the {\tt HS071\_NLP} class and give this class to \Ipopt\ so it can
evaluate the problem functions through the {\tt TNLP} interface. If
you have any difficulty as the implementation proceeds, have a look at
the completed example in the {\tt Examples/hs071\_cpp} directory.

Start by creating a new directory under Examples, called {\tt
  MyExample} and create the files {\tt hs071\_nlp.hpp} and {\tt
  hs071\_nlp.cpp}. In {\tt hs071\_nlp.hpp}, include {\tt IpTNLP.hpp}
(the base class), tell the compiler that we are using the \Ipopt\
namespace, and create the declaration of the {\tt HS071\_NLP} class,
inheriting off of {\tt TNLP}. Have a look at the {\tt TNLP} class in
{\tt IpTNLP.hpp}; you will see eight pure virtual methods that we must
implement. Declare these methods in the header file.  Implement each
of the methods in {\tt HS071\_NLP.cpp} using the descriptions given
below. In {\tt hs071\_nlp.cpp}, first include the header file for your
class and tell the compiler that you are using the \Ipopt\ namespace.
A full version of these files can be found in the {\tt
  Examples/hs071\_cpp} directory.

It is very easy to make mistakes in the implementation of the function
evaluation methods, in particular regarding the derivatives.  \Ipopt\
has a feature that can help you to debug the derivative code, using
finite differences, see Section~\ref{sec:deriv-checker}.

Note that the return value of any {\tt bool}-valued function should be
{\tt true}, unless an error occured, for example, because the value of
a problem function could not be evaluated at the required point.

\paragraph{Method {\texttt{get\_nlp\_info}}} with prototype
\begin{verbatim}
virtual bool get_nlp_info(Index& n, Index& m, Index& nnz_jac_g,
                          Index& nnz_h_lag, IndexStyleEnum& index_style)
\end{verbatim}
Give \Ipopt\ the information about the size of the problem (and hence,
the size of the arrays that it needs to allocate). 
\begin{itemize}
\item {\tt n}: (out), the number of variables in the problem (dimension of $x$).
\item {\tt m}: (out), the number of constraints in the problem (dimension of $g(x)$).
\item {\tt nnz\_jac\_g}: (out), the number of nonzero entries in the Jacobian.
\item {\tt nnz\_h\_lag}: (out), the number of nonzero entries in the Hessian.
\item {\tt index\_style}: (out), the numbering style used for row/col entries in the sparse matrix
format ({\tt C\_STYLE}: 0-based, {\tt FORTRAN\_STYLE}: 1-based; see
also Appendix~\ref{app.triplet}).
\end{itemize}
\Ipopt\ uses this information when allocating the arrays that
it will later ask you to fill with values. Be careful in this method
since incorrect values will cause memory bugs which may be very
difficult to find.

Our example problem has 4 variables (n), and 2 constraints (m). The
constraint Jacobian for this small problem is actually dense and has 8
nonzeros (we still need to represent this Jacobian using the sparse
matrix triplet format). The Hessian of the Lagrangian has 10
``symmetric'' nonzeros (i.e., nonzeros in the lower left triangular
part.).  Keep in mind that the number of nonzeros is the total number
of elements that may \emph{ever} be nonzero, not just those that are
nonzero at the starting point. This information is set once for the
entire problem.

\begin{footnotesize}
\begin{verbatim}
bool HS071_NLP::get_nlp_info(Index& n, Index& m, Index& nnz_jac_g, 
                             Index& nnz_h_lag, IndexStyleEnum& index_style)
{
  // The problem described in HS071_NLP.hpp has 4 variables, x[0] through x[3]
  n = 4;

  // one equality constraint and one inequality constraint
  m = 2;

  // in this example the Jacobian is dense and contains 8 nonzeros
  nnz_jac_g = 8;

  // the Hessian is also dense and has 16 total nonzeros, but we
  // only need the lower left corner (since it is symmetric)
  nnz_h_lag = 10;

  // use the C style indexing (0-based)
  index_style = TNLP::C_STYLE;

  return true;
}
\end{verbatim}
\end{footnotesize}

\paragraph{Method {\texttt{get\_bounds\_info}}} with prototype
\begin{verbatim}
virtual bool get_bounds_info(Index n, Number* x_l, Number* x_u,
                             Index m, Number* g_l, Number* g_u)
\end{verbatim}
Give \Ipopt\ the value of the bounds on the variables and constraints.
\begin{itemize}
\item {\tt n}: (in), the number of variables in the problem (dimension of $x$). 
\item {\tt x\_l}: (out) the lower bounds $x^L$ for $x$. 
\item {\tt x\_u}: (out) the upper bounds $x^U$ for $x$.
\item {\tt m}: (in), the number of constraints in the problem (dimension of $g(x)$).
\item {\tt g\_l}: (out) the lower bounds $g^L$ for $g(x)$. 
\item {\tt g\_u}: (out) the upper bounds $g^U$ for $g(x)$.
\end{itemize}
The values of {\tt n} and {\tt m} that you specified in {\tt
  get\_nlp\_info} are passed to you for debug checking.  Setting a
lower bound to a value less than or equal to the value of the option
{\tt nlp\_lower\_bound\_inf} will cause \Ipopt\ to assume no lower
bound. Likewise, specifying the upper bound above or equal to the
value of the option {\tt nlp\_upper\_bound\_inf} will cause \Ipopt\ to
assume no upper bound.  These options, {\tt nlp\_lower\_bound\_inf}
and {\tt nlp\_upper\_bound\_inf}, are set to $-10^{19}$ and $10^{19}$,
respectively, by default, but may be modified by changing the options
(see Section \ref{sec.options}).

In our example, the first constraint has a lower bound of $25$ and no upper
bound, so we set the lower bound of constraint {\tt [0]} to $25$ and
the upper bound to some number greater than $10^{19}$. The second
constraint is an equality constraint and we set both bounds to
$40$. \Ipopt\ recognizes this as an equality constraint and does not
treat it as two inequalities.

\begin{footnotesize}
\begin{verbatim}
bool HS071_NLP::get_bounds_info(Index n, Number* x_l, Number* x_u,
                                Index m, Number* g_l, Number* g_u)
{
  // here, the n and m we gave IPOPT in get_nlp_info are passed back to us.
  // If desired, we could assert to make sure they are what we think they are.
  assert(n == 4);
  assert(m == 2);

  // the variables have lower bounds of 1
  for (Index i=0; i<4; i++) {
    x_l[i] = 1.0;
  }

  // the variables have upper bounds of 5
  for (Index i=0; i<4; i++) {
    x_u[i] = 5.0;
  }

  // the first constraint g1 has a lower bound of 25
  g_l[0] = 25;
  // the first constraint g1 has NO upper bound, here we set it to 2e19.
  // Ipopt interprets any number greater than nlp_upper_bound_inf as 
  // infinity. The default value of nlp_upper_bound_inf and nlp_lower_bound_inf
  // is 1e19 and can be changed through ipopt options.
  g_u[0] = 2e19;

  // the second constraint g2 is an equality constraint, so we set the 
  // upper and lower bound to the same value
  g_l[1] = g_u[1] = 40.0;

  return true;
}
\end{verbatim}
\end{footnotesize}

\paragraph{Method {\texttt{get\_starting\_point}}} with prototype
\begin{verbatim}
virtual bool get_starting_point(Index n, bool init_x, Number* x,
                                bool init_z, Number* z_L, Number* z_U,
                                Index m, bool init_lambda, Number* lambda)
\end{verbatim}
Give \Ipopt\ the starting point before it begins iterating.
\begin{itemize}
\item {\tt n}: (in), the number of variables in the problem (dimension of $x$). 
\item {\tt init\_x}: (in), if true, this method must provide an initial value for $x$.
\item {\tt x}: (out), the initial values for the primal variables, $x$.
\item {\tt init\_z}: (in), if true, this method must provide an initial value 
        for the bound multipliers $z^L$ and $z^U$.
\item {\tt z\_L}: (out), the initial values for the bound multipliers, $z^L$.
\item {\tt z\_U}: (out), the initial values for the bound multipliers, $z^U$.
\item {\tt m}: (in), the number of constraints in the problem (dimension of $g(x)$).
\item {\tt init\_lambda}: (in), if true, this method must provide an initial value 
        for the constraint multipliers, $\lambda$.
\item {\tt lambda}: (out), the initial values for the constraint multipliers, $\lambda$.
\end{itemize}

The variables {\tt n} and {\tt m} are passed in for your convenience.
These variables will have the same values you specified in {\tt
  get\_nlp\_info}.

Depending on the options that have been set, \Ipopt\ may or may not
require bounds for the primal variables $x$, the bound multipliers
$z^L$ and $z^U$, and the constraint multipliers $\lambda$. The boolean
flags {\tt init\_x}, {\tt init\_z}, and {\tt init\_lambda} tell you
whether or not you should provide initial values for $x$, $z^L$, $z^U$, or
$\lambda$ respectively. The default options only require an initial
value for the primal variables $x$.  Note, the initial values for
bound multiplier components for ``infinity'' bounds
($x_L^{(i)}=-\infty$ or $x_U^{(i)}=\infty$) are ignored.

In our example, we provide initial values for $x$ as specified in the
example problem. We do not provide any initial values for the dual
variables, but use an assert to immediately let us know if we are ever
asked for them.

\begin{footnotesize}
\begin{verbatim}
bool HS071_NLP::get_starting_point(Index n, bool init_x, Number* x,
                                   bool init_z, Number* z_L, Number* z_U,
                                   Index m, bool init_lambda,
                                   Number* lambda)
{
  // Here, we assume we only have starting values for x, if you code
  // your own NLP, you can provide starting values for the dual variables
  // if you wish to use a warmstart option
  assert(init_x == true);
  assert(init_z == false);
  assert(init_lambda == false);

  // initialize to the given starting point
  x[0] = 1.0;
  x[1] = 5.0;
  x[2] = 5.0;
  x[3] = 1.0;

  return true;
}
\end{verbatim}
\end{footnotesize}

\paragraph{Method {\texttt{eval\_f}}} with prototype
\begin{verbatim}
virtual bool eval_f(Index n, const Number* x, 
                    bool new_x, Number& obj_value)
\end{verbatim}
Return the value of the objective function at the point $x$.
\begin{itemize}
\item {\tt n}: (in), the number of variables in the problem (dimension
  of $x$).
\item {\tt x}: (in), the values for the primal variables, $x$, at which
  $f(x)$ is to be evaluated.
\item {\tt new\_x}: (in), false if any evaluation method was
  previously called with the same values in {\tt x}, true otherwise.
\item {\tt obj\_value}: (out) the value of the objective function
  ($f(x)$).
\end{itemize}

The boolean variable {\tt new\_x} will be false if the last call to
any of the evaluation methods ({\tt eval\_*}) used the same $x$
values. This can be helpful when users have efficient implementations
that calculate multiple outputs at once. \Ipopt\ internally caches
results from the {\tt TNLP} and generally, this flag can be ignored.

The variable {\tt n} is passed in for your convenience. This variable
will have the same value you specified in {\tt get\_nlp\_info}.

For our example, we ignore the {\tt new\_x} flag and calculate the objective.

\begin{footnotesize}
\begin{verbatim}
bool HS071_NLP::eval_f(Index n, const Number* x, bool new_x, Number& obj_value)
{
  assert(n == 4);

  obj_value = x[0] * x[3] * (x[0] + x[1] + x[2]) + x[2];

  return true;
}
\end{verbatim}
\end{footnotesize}

\paragraph{Method {\texttt{eval\_grad\_f}}} with prototype
\begin{verbatim}
virtual bool eval_grad_f(Index n, const Number* x, bool new_x, 
                         Number* grad_f)
\end{verbatim}
Return the gradient of the objective function at the point $x$.
\begin{itemize}
\item {\tt n}: (in), the number of variables in the problem (dimension of $x$). 
\item {\tt x}: (in), the values for the primal variables, $x$, at which
  $\nabla f(x)$ is to be evaluated.
\item {\tt new\_x}: (in), false if any evaluation method was previously called 
        with the same values in {\tt x}, true otherwise.
\item {\tt grad\_f}: (out) the array of values for the gradient of the 
        objective function ($\nabla f(x)$).
\end{itemize}

The gradient array is in the same order as the $x$ variables (i.e., the
gradient of the objective with respect to {\tt x[2]} should be put in
{\tt grad\_f[2]}).

The boolean variable {\tt new\_x} will be false if the last call to
any of the evaluation methods ({\tt eval\_*}) used the same $x$
values. This can be helpful when users have efficient implementations
that calculate multiple outputs at once. \Ipopt\ internally caches
results from the {\tt TNLP} and generally, this flag can be ignored.

The variable {\tt n} is passed in for your convenience. This
variable will have the same value you specified in {\tt
get\_nlp\_info}.

In our example, we ignore the {\tt new\_x} flag and calculate the
values for the gradient of the objective.

\begin{footnotesize}
\begin{verbatim}
bool HS071_NLP::eval_grad_f(Index n, const Number* x, bool new_x, Number* grad_f)
{
  assert(n == 4);

  grad_f[0] = x[0] * x[3] + x[3] * (x[0] + x[1] + x[2]);
  grad_f[1] = x[0] * x[3];
  grad_f[2] = x[0] * x[3] + 1;
  grad_f[3] = x[0] * (x[0] + x[1] + x[2]);

  return true;
}
\end{verbatim}
\end{footnotesize}

\paragraph{Method {\texttt{eval\_g}}} with prototype
\begin{verbatim}
virtual bool eval_g(Index n, const Number* x, 
                    bool new_x, Index m, Number* g)
\end{verbatim}
Return the value of the constraint function at the point $x$.
\begin{itemize}
\item {\tt n}: (in), the number of variables in the problem (dimension of $x$). 
\item {\tt x}: (in), the values for the primal variables, $x$, at
  which the constraint functions,
  $g(x)$, are to be evaluated.
\item {\tt new\_x}: (in), false if any evaluation method was previously called 
        with the same values in {\tt x}, true otherwise.
\item {\tt m}: (in), the number of constraints in the problem (dimension of $g(x)$).
\item {\tt g}: (out) the array of constraint function values, $g(x)$.
\end{itemize}

The values returned in {\tt g} should be only the $g(x)$ values, 
do not add or subtract the bound values $g^L$ or $g^U$.

The boolean variable {\tt new\_x} will be false if the last call to
any of the evaluation methods ({\tt eval\_*}) used the same $x$
values. This can be helpful when users have efficient implementations
that calculate multiple outputs at once. \Ipopt\ internally caches
results from the {\tt TNLP} and generally, this flag can be ignored.

The variables {\tt n} and {\tt m} are passed in for your convenience.
These variables will have the same values you specified in {\tt
  get\_nlp\_info}.

In our example, we ignore the {\tt new\_x} flag and calculate the
values of constraint functions.

\begin{footnotesize}
\begin{verbatim}
bool HS071_NLP::eval_g(Index n, const Number* x, bool new_x, Index m, Number* g)
{
  assert(n == 4);
  assert(m == 2);

  g[0] = x[0] * x[1] * x[2] * x[3];
  g[1] = x[0]*x[0] + x[1]*x[1] + x[2]*x[2] + x[3]*x[3];

  return true;
} 
\end{verbatim}
\end{footnotesize}

\paragraph{Method {\texttt{eval\_jac\_g}}} with prototype
\begin{verbatim}
virtual bool eval_jac_g(Index n, const Number* x, bool new_x,
                        Index m, Index nele_jac, Index* iRow, 
                        Index *jCol, Number* values)
\end{verbatim}
Return either the sparsity structure of the Jacobian of the
constraints, or the values for the Jacobian of the constraints at the
point $x$.
\begin{itemize}
\item {\tt n}: (in), the number of variables in the problem (dimension of $x$). 
\item {\tt x}: (in), the values for the primal variables, $x$, at which
  the constraint Jacobian, $\nabla g(x)^T$, is to be evaluated.
\item {\tt new\_x}: (in), false if any evaluation method was previously called 
        with the same values in {\tt x}, true otherwise.
\item {\tt m}: (in), the number of constraints in the problem (dimension of $g(x)$).
\item {\tt n\_ele\_jac}: (in), the number of nonzero elements in the 
        Jacobian (dimension of {\tt iRow}, {\tt jCol}, and {\tt values}).
\item {\tt iRow}: (out), the row indices of entries in the Jacobian of the constraints.
\item {\tt jCol}: (out), the column indices of entries in the Jacobian of the constraints.
\item {\tt values}: (out), the values of the entries in the Jacobian of the constraints.
\end{itemize}

The Jacobian is the matrix of derivatives where the derivative of
constraint $g^{(i)}$ with respect to variable $x^{(j)}$ is placed in
row $i$ and column $j$. See Appendix \ref{app.triplet} for a
discussion of the sparse matrix format used in this method.

If the {\tt iRow} and {\tt jCol} arguments are not {\tt NULL}, then
\Ipopt\ wants you to fill in the sparsity structure of the Jacobian
(the row and column indices only). At this time, the {\tt x} argument
and the {\tt values} argument will be {\tt NULL}.

If the {\tt x} argument and the {\tt values} argument are not {\tt
  NULL}, then \Ipopt\ wants you to fill in the values of the Jacobian
as calculated from the array {\tt x} (using the same order as you used
when specifying the sparsity structure). At this time, the {\tt iRow}
and {\tt jCol} arguments will be {\tt NULL};

The boolean variable {\tt new\_x} will be false if the last call to
any of the evaluation methods ({\tt eval\_*}) used the same $x$
values. This can be helpful when users have efficient implementations
that calculate multiple outputs at once. \Ipopt\ internally caches
results from the {\tt TNLP} and generally, this flag can be ignored.

The variables {\tt n}, {\tt m}, and {\tt nele\_jac} are passed in for
your convenience. These arguments will have the same values you
specified in {\tt get\_nlp\_info}.

In our example, the Jacobian is actually dense, but we still
specify it using the sparse format.

\begin{footnotesize}
\begin{verbatim}
bool HS071_NLP::eval_jac_g(Index n, const Number* x, bool new_x,
                           Index m, Index nele_jac, Index* iRow, Index *jCol,
                           Number* values)
{
  if (values == NULL) {
    // return the structure of the Jacobian

    // this particular Jacobian is dense
    iRow[0] = 0; jCol[0] = 0;
    iRow[1] = 0; jCol[1] = 1;
    iRow[2] = 0; jCol[2] = 2;
    iRow[3] = 0; jCol[3] = 3;
    iRow[4] = 1; jCol[4] = 0;
    iRow[5] = 1; jCol[5] = 1;
    iRow[6] = 1; jCol[6] = 2;
    iRow[7] = 1; jCol[7] = 3;
  }
  else {
    // return the values of the Jacobian of the constraints
    
    values[0] = x[1]*x[2]*x[3]; // 0,0
    values[1] = x[0]*x[2]*x[3]; // 0,1
    values[2] = x[0]*x[1]*x[3]; // 0,2
    values[3] = x[0]*x[1]*x[2]; // 0,3

    values[4] = 2*x[0]; // 1,0
    values[5] = 2*x[1]; // 1,1
    values[6] = 2*x[2]; // 1,2
    values[7] = 2*x[3]; // 1,3
  }

  return true;
}
\end{verbatim}
\end{footnotesize}

\paragraph{Method {\texttt{eval\_h}}} with prototype
\begin{verbatim}
virtual bool eval_h(Index n, const Number* x, bool new_x,
                    Number obj_factor, Index m, const Number* lambda,
                    bool new_lambda, Index nele_hess, Index* iRow,
                    Index* jCol, Number* values)
\end{verbatim}
Return either the sparsity structure of the Hessian of the Lagrangian, or the values of the 
Hessian of the Lagrangian (\ref{eq:IpoptLAG}) for the given values for $x$,
$\sigma_f$, and $\lambda$.
\begin{itemize}
\item {\tt n}: (in), the number of variables in the problem (dimension
  of $x$).
\item {\tt x}: (in), the values for the primal variables, $x$, at which
  the Hessian is to be evaluated.
\item {\tt new\_x}: (in), false if any evaluation method was previously called 
        with the same values in {\tt x}, true otherwise.
\item {\tt obj\_factor}: (in), factor in front of the objective term
  in the Hessian, $sigma_f$.
\item {\tt m}: (in), the number of constraints in the problem (dimension of $g(x)$).
\item {\tt lambda}: (in), the values for the constraint multipliers,
  $\lambda$, at which the Hessian is to be evaluated.
\item {\tt new\_lambda}: (in), false if any evaluation method was
  previously called with the same values in {\tt lambda}, true
  otherwise.
\item {\tt nele\_hess}: (in), the number of nonzero elements in the
  Hessian (dimension of {\tt iRow}, {\tt jCol}, and {\tt values}).
\item {\tt iRow}: (out), the row indices of entries in the Hessian.
\item {\tt jCol}: (out), the column indices of entries in the Hessian.
\item {\tt values}: (out), the values of the entries in the Hessian.
\end{itemize}

The Hessian matrix that \Ipopt\ uses is defined in
Eq.~\ref(eq:IpoptLAG).  See Appendix \ref{app.triplet} for a
discussion of the sparse symmetric matrix format used in this method.

If the {\tt iRow} and {\tt jCol} arguments are not {\tt NULL}, then
\Ipopt\ wants you to fill in the sparsity structure of the Hessian
(the row and column indices for the lower or upper triangular part
only). In this case, the {\tt x}, {\tt lambda}, and {\tt values}
arrays will be {\tt NULL}.

If the {\tt x}, {\tt lambda}, and {\tt values} arrays are not {\tt
  NULL}, then \Ipopt\ wants you to fill in the values of the Hessian
as calculated using {\tt x} and {\tt lambda} (using the same order as
you used when specifying the sparsity structure). In this case, the
{\tt iRow} and {\tt jCol} arguments will be {\tt NULL}.

The boolean variables {\tt new\_x} and {\tt new\_lambda} will both be
false if the last call to any of the evaluation methods ({\tt
  eval\_*}) used the same values. This can be helpful when users have
efficient implementations that calculate multiple outputs at once.
\Ipopt\ internally caches results from the {\tt TNLP} and generally,
this flag can be ignored.

The variables {\tt n}, {\tt m}, and {\tt nele\_hess} are passed in for
your convenience. These arguments will have the same values you
specified in {\tt get\_nlp\_info}.

In our example, the Hessian is dense, but we still specify it using the
sparse matrix format. Because the Hessian is symmetric, we only need to 
specify the lower left corner.

\begin{footnotesize}
\begin{verbatim}
bool HS071_NLP::eval_h(Index n, const Number* x, bool new_x,
                       Number obj_factor, Index m, const Number* lambda,
                       bool new_lambda, Index nele_hess, Index* iRow,
                       Index* jCol, Number* values)
{
  if (values == NULL) {
    // return the structure. This is a symmetric matrix, fill the lower left
    // triangle only.

    // the Hessian for this problem is actually dense
    Index idx=0;
    for (Index row = 0; row < 4; row++) {
      for (Index col = 0; col <= row; col++) {
        iRow[idx] = row; 
        jCol[idx] = col;
        idx++;
      }
    }
    
    assert(idx == nele_hess);
  }
  else {
    // return the values. This is a symmetric matrix, fill the lower left
    // triangle only

    // fill the objective portion
    values[0] = obj_factor * (2*x[3]); // 0,0

    values[1] = obj_factor * (x[3]);   // 1,0
    values[2] = 0;                     // 1,1

    values[3] = obj_factor * (x[3]);   // 2,0
    values[4] = 0;                     // 2,1
    values[5] = 0;                     // 2,2

    values[6] = obj_factor * (2*x[0] + x[1] + x[2]); // 3,0
    values[7] = obj_factor * (x[0]);                 // 3,1
    values[8] = obj_factor * (x[0]);                 // 3,2
    values[9] = 0;                                   // 3,3


    // add the portion for the first constraint
    values[1] += lambda[0] * (x[2] * x[3]); // 1,0
    
    values[3] += lambda[0] * (x[1] * x[3]); // 2,0
    values[4] += lambda[0] * (x[0] * x[3]); // 2,1

    values[6] += lambda[0] * (x[1] * x[2]); // 3,0
    values[7] += lambda[0] * (x[0] * x[2]); // 3,1
    values[8] += lambda[0] * (x[0] * x[1]); // 3,2

    // add the portion for the second constraint
    values[0] += lambda[1] * 2; // 0,0

    values[2] += lambda[1] * 2; // 1,1

    values[5] += lambda[1] * 2; // 2,2

    values[9] += lambda[1] * 2; // 3,3
  }

  return true;
}
\end{verbatim}
\end{footnotesize}

{\bf TODO: User STOP method here?}

\paragraph{Method \texttt{finalize\_solution}} with prototype
\begin{verbatim}
virtual void finalize_solution(SolverReturn status, Index n,
                               const Number* x, const Number* z_L,
                               const Number* z_U, Index m, const Number* g,
                               const Number* lambda, Number obj_value)
\end{verbatim}
This is the only method that is not mentioned in Figure
\ref{fig.required_info}. This method is called by \Ipopt\ after the
algorithm has finished (successfully or even with most errors).
\begin{itemize}
\item {\tt status}: (in), gives the status of the algorithm as
  specified in {\tt IpAlgTypes.hpp},
  \begin{itemize}
  \item {\tt SUCCESS}: Algorithm terminated successfully at a locally
    optimal point, satisfying the convergence tolerances (can be
    specified by options).
  \item {\tt MAXITER\_EXCEEDED}: Maximum number of iterations exceeded
    (can be specified by an option).
  \item {\tt STOP\_AT\_TINY\_STEP}: Algorithm proceeds with very
    little progress.
  \item {\tt STOP\_AT\_ACCEPTABLE\_POINT}: Algorithm stopped at a
    point that was converged, not to ``desired'' tolerances, but to
    ``acceptable'' tolerances (see the {\tt acceptable-...} options).
  \item {\tt LOCAL\_INFEASIBILITY}: Algorithm converged to a point of
    local infeasibility. Problem may be infeasible.
  \item {\tt DIVERGING\_ITERATES}: It seems that the iterates diverge.
  \item {\tt RESTORATION\_FAILURE}: Restoration phase failed,
    algorithm doesn't know how to proceed.
  \item {\tt
      INTERNAL\_ERROR}: An unknown internal error occurred.  Please
    contact the \Ipopt\ authors through the mailing list.
  \end{itemize}
\item {\tt n}: (in), the number of variables in the problem (dimension
  of $x$).
\item {\tt x}: (in), the final values for the primal variables, $x_*$.
\item {\tt z\_L}: (in), the final values for the lower bound
  multipliers, $z^L_*$.
\item {\tt z\_U}: (in), the final values for the upper bound
  multipliers, $z^U_*$.
\item {\tt m}: (in), the number of constraints in the problem
  (dimension of $g(x)$).
\item {\tt g}: (in), the final value of the constraint function
  values, $g(x_*)$.
\item {\tt lambda}: (in), the final values of the constraint
  multipliers, $\lambda_*$.
\item {\tt obj\_value}: (in), the final value of the objective,
  $f(x_*)$.
\end{itemize}

{\bf TODO: Should be provide IpStatistics here?}

This method gives you the return status of the algorithm
(SolverReturn), and the values of the variables, 
the objective and constraint function values when the algorithm exited.

In our example, we will print the values of some of the variables to 
the screen.

\begin{footnotesize}
\begin{verbatim}
void HS071_NLP::finalize_solution(SolverReturn status,
                                  Index n, const Number* x, const Number* z_L,
                                  const Number* z_U, Index m, const Number* g,
                                  const Number* lambda, Number obj_value)
{
  // here is where we would store the solution to variables, or write to a file, etc
  // so we could use the solution. 

  // For this example, we write the solution to the console
  printf("\n\nSolution of the primal variables, x\n");
  for (Index i=0; i<n; i++) {
    printf("x[%d] = %e\n", i, x[i]); 
  }

  printf("\n\nSolution of the bound multipliers, z_L and z_U\n");
  for (Index i=0; i<n; i++) {
    printf("z_L[%d] = %e\n", i, z_L[i]); 
  }
  for (Index i=0; i<n; i++) {
    printf("z_U[%d] = %e\n", i, z_U[i]); 
  }

  printf("\n\nObjective value\n");
  printf("f(x*) = %e\n", obj_value); 
}
\end{verbatim}
\end{footnotesize}

This is all that is required for our {\tt HS071\_NLP} class and 
the coding of the problem representation.
 
\subsubsection{Coding the Executable (\texttt{main})}
Now that we have a problem representation, the {\tt HS071\_NLP} class,
we need to code the main function that will call \Ipopt\ and ask \Ipopt\
to find a solution.

Here, we must create an instance of our problem ({\tt HS071\_NLP}),
create an instance of the \Ipopt\ solver (\texttt{IpoptApplication}),
and ask the solver to find a solution. We always use the
\texttt{SmartPtr} template class instead of raw C++ pointers when
creating and passing \Ipopt\ objects. To find out more information
about smart pointers and the {\tt SmartPtr} implementation used in
\Ipopt, see Appendix \ref{app.smart_ptr}.

Create the file {\tt MyExample.cpp} in the MyExample directory.
Include {\tt HS071\_NLP.hpp} and {\tt IpIpoptApplication.hpp}, tell
the compiler to use the {\tt Ipopt} namespace, and implement the {\tt
  main} function.

\begin{footnotesize}
\begin{verbatim}
#include "IpIpoptApplication.hpp"
#include "hs071_nlp.hpp"

using namespace Ipopt;

int main(int argv, char* argc[])
{
  // Create a new instance of your nlp 
  //  (use a SmartPtr, not raw)
  SmartPtr<TNLP> mynlp = new HS071_NLP();

  // Create a new instance of IpoptApplication
  //  (use a SmartPtr, not raw)
  SmartPtr<IpoptApplication> app = new IpoptApplication();

  // Change some options
  app->Options()->SetNumericValue("tol", 1e-9);
  app->Options()->SetStringValue("mu_strategy", "adaptive");

  // Ask Ipopt to solve the problem
  ApplicationReturnStatus status = app->OptimizeTNLP(mynlp);

  if (status == Solve_Succeeded) {
    printf("\n\n*** The problem solved!\n");
  }
  else {
    printf("\n\n*** The problem FAILED!\n");
  }

  // As the SmartPtrs go out of scope, the reference count
  // will be decremented and the objects will automatically 
  // be deleted.

  return (int) status;
}
\end{verbatim} 
\end{footnotesize}

The first line of code in {\tt main} creates an instance of {\tt
  HS071\_NLP}. We then create an instance of the \Ipopt\ solver, {\tt
  IpoptApplication}. The call to {\tt app->OptimizeTNLP(...)} will run
\Ipopt\ and try to solve the problem. By default, \Ipopt\ will write
to its progress to the console, and return the {\tt SolverReturn}
status.

\subsubsection{Compiling and Testing the Example}
Our next task is to compile and test the code. If you are familiar
with the compiler and linker used on your system, you can build the
code, including the \Ipopt\ library {\tt libipopt.a} (and other
necessary libraries, as listed in the {\tt ipopt\_addlibs\_cpp.txt}
and {\tt ipopt\_addlibs\_f.txt} files).  If you are using Linux/UNIX,
then a sample makefile exists already that was created by configure.
Copy {\tt Examples/hs071\_cpp/Makefile} into your {\tt MyExample}
directory.  This makefile was created for the {\tt hs071\_cpp} code,
but it can be easily modified for your example problem. Edit the file,
making the following changes,

\begin{itemize}
\item change the {\tt EXE} variable \\
{\tt EXE = my\_example}
\item change the {\tt OBJS} variable \\
{\tt OBJS = HS071\_NLP.o MyExample.o}
\end{itemize}
and the problem should compile easily with, \\
{\tt \$ make} \\
Now run the executable,\\ 
{\tt \$ ./my\_example} \\
and you should see output resembling the following,

\begin{footnotesize}
\begin{verbatim}
Total number of variables............................:        4
                     variables with only lower bounds:        0
                variables with lower and upper bounds:        4
                     variables with only upper bounds:        0
Total number of equality constraints.................:        1
Total number of inequality constraints...............:        1
        inequality constraints with only lower bounds:        1
   inequality constraints with lower and upper bounds:        0
        inequality constraints with only upper bounds:        0
 
 iter     objective    inf_pr   inf_du lg(mu)  ||d||  lg(rg) alpha_du alpha_pr  ls
    0   1.7159878e+01 2.01e-02 5.20e-01  -1.0 0.00e+00    -  0.00e+00 0.00e+00   0 y
    1   1.7146308e+01 1.63e-01 1.47e-01  -1.0 1.15e-01    -  9.86e-01 1.00e+00f  1
    2   1.7065508e+01 3.10e-02 8.47e-02  -1.7 1.99e-01    -  9.54e-01 1.00e+00h  1 Nhj
    3   1.7002626e+01 4.10e-02 4.81e-03  -2.5 5.52e-02    -  1.00e+00 1.00e+00h  1
    4   1.7019082e+01 1.20e-03 1.81e-04  -2.5 1.10e-02    -  1.00e+00 1.00e+00h  1
    5   1.7014253e+01 1.80e-04 4.87e-05  -3.8 4.86e-03    -  1.00e+00 1.00e+00h  1
    6   1.7014020e+01 9.25e-07 2.15e-07  -5.7 2.76e-04    -  1.00e+00 1.00e+00h  1
    7   1.7014017e+01 1.01e-10 2.60e-11  -8.6 3.32e-06    -  1.00e+00 1.00e+00h  1
 
Number of Iterations....: 7
 
                                   (scaled)                 (unscaled)
Objective...............:   1.7014017145177885e+01    1.7014017145177885e+01
Dual infeasibility......:   2.5980210027546616e-11    2.5980210027546616e-11
Constraint violation....:   1.8175683180743363e-11    1.8175683180743363e-11
Complementarity.........:   2.5282956951655172e-09    2.5282956951655172e-09
Overall NLP error.......:   2.5282956951655172e-09    2.5282956951655172e-09
 
 
Number of objective function evaluations             = 8
Number of objective gradient evaluations             = 8
Number of equality constraint evaluations            = 8
Number of inequality constraint evaluations          = 8
Number of equality constraint Jacobian evaluations   = 8
Number of inequality constraint Jacobian evaluations = 8
Number of Lagrangian Hessian evaluations             = 9
 
EXIT: Optimal Solution Found.
 
 
Solution of the primal variables, x
x[0] = 1
x[1] = 4.743
x[2] = 3.82115
x[3] = 1.37941
 
 
Solution of the bound multipliers, z_L and z_U
z_L[0] = 1.08787
z_L[1] = 6.69317e-10
z_L[2] = 8.8877e-10
z_L[3] = 6.57011e-09
z_U[0] = 6.26262e-10
z_U[1] = 9.78906e-09
z_U[2] = 2.12283e-09
z_U[3] = 6.92528e-10
 
 
Objective value
f(x*) = 17.014
 
 
*** The problem solved!
\end{verbatim}
\end{footnotesize}

This completes the basic C++ tutorial, but see Section
\ref{sec.output} which explains the standard console output of \Ipopt
and Section \ref{sec.options} for information about the use of options
to customize the behavior of \Ipopt.

The {\tt Examples/ScalableProblems} directory contains another set
of NLP problems coded in C++.

\subsection{The C Interface}\label{sec.cinterface}
The C interface for \Ipopt\ is declared in the header file {\tt
  IpStdCInterface.h}, which is found in\\
\texttt{\$IPOPTDIR/include/ipopt} (or in
\texttt{\$PREFIX/include/ipopt} if the switch
\verb|--prefix=$PREFIX| was used for {\tt configure}); while
reading this section, it will be helpful to have a look at this file.

In order to solve an optimization problem with the C interface, one
has to create an {\tt IpoptProblem}\footnote{{\tt IpoptProblem} is a
  pointer to a C structure; you should not access this structure
  directly, only through the functions provided in the C interface.}
with the function {\tt CreateIpoptProblem}, which later has to be
passed to the {\tt IpoptSolve} function.

The {\tt IpoptProblem} created by {\tt CreateIpoptProblem} contains
the problem dimensions, the variable and constraint bounds, and the
function pointers for callbacks that will be used to evaluate the NLP
problem functions and their derivatives (see also the discussion of
the C++ methods {\tt get\_nlp\_info} and {\tt get\_bounds\_info} in
Section~\ref{sec.cpp_problem} for information about the arguments of
{\tt CreateIpoptProblem}).

The prototypes for the callback functions, {\tt Eval\_F\_CB}, {\tt
  Eval\_Grad\_F\_CB}, etc., are defined in the header file {\tt
  IpStdCInterface.h}.  Their arguments correspond one-to-one to the
arguments for the C++ methods discussed in
Section~\ref{sec.cpp_problem}; for example, for the meaning of $\tt
n$, $\tt x$, $\tt new\_x$, $\tt obj\_value$ in the declaration of {\tt
  Eval\_F\_CB} see the discussion of ``{\tt eval\_f}''.  The callback
functions should return {\tt TRUE}, unless there was a problem doing
the requested function/derivative evaluation at the given point {\tt
  x} (then it should return {\tt FALSE}).

Note the additional argument of type {\tt UserDataPtr} in the callback
functions.  This pointer argument is available for you to communicate
information between the main program that calls {\tt IpoptSolve} and
any of the callback functions.  This pointer is simply passed
unmodified by \Ipopt\ among those functions.  For example, you can
use this to pass constants that define the optimization problem and
are computed before the optimization in the main C program to the
callback functions.

After an {\tt IpoptProblem} has been created, you can set algorithmic
options for \Ipopt\ (see Section~\ref{sec.options}) using the {\tt
  AddIpopt...Option} functions.  Finally, the \Ipopt\ algorithm is
called with {\tt IpoptSolve}, giving \Ipopt\ the {\tt IpoptProblem},
the starting point, and arrays to store the solution values (primal
and dual variables), if desired.  Finally, after everything is done,
you should call {\tt FreeIpoptProblem} to release internal memory that
is still allocated inside \Ipopt.

In the remainder of this section we discuss how the example problem
(\ref{eq:ex_obj})--(\ref{eq:ex_bounds}) can be solved using the C
interface.  A completed version of this example can be found in {\tt
  Examples/hs071\_c}.

% We first create the necessary callback
% functions for evaluating the NLP. As just discussed, the \Ipopt\ C
% interface required callbacks to evaluate the objective value,
% constraints, gradient of the objective, Jacobian of the constraints,
% and the Hessian of the Lagrangian.  These callbacks are implemented
% using function pointers.  Have a look at the C++ implementation for
% {\tt eval\_f}, {\tt eval\_g}, {\tt eval\_grad\_f}, {\tt eval\_jac\_g},
% and {\tt eval\_h} in Section \ref{sec.cpp_problem}. The C
% implementations have somewhat different prototypes, but are
% implemented almost identically to the C++ code.

\vspace{\baselineskip}

In order to implement the example problem on your own, create a new
directory {\tt MyCExample} and create a new file, {\tt
  hs071\_c.c}.  Here, include the interface header file {\tt
  IpStdCInterface.h}, along with other necessary header files, such as
{\tt stdlib.h} and {\tt assert.h}.  Add the prototypes and
implementations for the five callback functions.  Have a look at the
C++ implementation for {\tt eval\_f}, {\tt eval\_g}, {\tt
  eval\_grad\_f}, {\tt eval\_jac\_g}, and {\tt eval\_h} in Section
\ref{sec.cpp_problem}. The C implementations have somewhat different
prototypes, but are implemented almost identically to the C++ code.
See the completed example in {\tt Examples/hs071\_c/hs071\_c.c} if you
are not sure how to do this.

We now need to implement the {\tt main} function, create the {\tt
  IpoptProblem}, set options, and call {\tt IpoptSolve}. The {\tt
  CreateIpoptProblem} function requires the problem dimensions, the
variable and constraint bounds, and the function pointers to the
callback routines. The {\tt IpoptSolve} function requires the {\tt
  IpoptProblem}, the starting point, and allocated arrays for the
solution.  The {\tt main} function from the example is shown next, and
discussed below.

%in Figure~\ref{fig:cexample-main}.
%\begin{figure}
%  \centering
\begin{footnotesize}
\begin{verbatim}
int main()
{
  Index n=-1;                          /* number of variables */
  Index m=-1;                          /* number of constraints */
  Number* x_L = NULL;                  /* lower bounds on x */
  Number* x_U = NULL;                  /* upper bounds on x */
  Number* g_L = NULL;                  /* lower bounds on g */
  Number* g_U = NULL;                  /* upper bounds on g */
  IpoptProblem nlp = NULL;             /* IpoptProblem */
  enum ApplicationReturnStatus status; /* Solve return code */
  Number* x = NULL;                    /* starting point and solution vector */
  Number* mult_x_L = NULL;             /* lower bound multipliers 
					  at the solution */
  Number* mult_x_U = NULL;             /* upper bound multipliers 
					  at the solution */
  Number obj;                          /* objective value */
  Index i;                             /* generic counter */
  
  /* set the number of variables and allocate space for the bounds */
  n=4;
  x_L = (Number*)malloc(sizeof(Number)*n);
  x_U = (Number*)malloc(sizeof(Number)*n);
  /* set the values for the variable bounds */
  for (i=0; i<n; i++) {
    x_L[i] = 1.0;
    x_U[i] = 5.0;
  }

  /* set the number of constraints and allocate space for the bounds */
  m=2;
  g_L = (Number*)malloc(sizeof(Number)*m);
  g_U = (Number*)malloc(sizeof(Number)*m);
  /* set the values of the constraint bounds */
  g_L[0] = 25; g_U[0] = 2e19;
  g_L[1] = 40; g_U[1] = 40;

  /* create the IpoptProblem */
  nlp = CreateIpoptProblem(n, x_L, x_U, m, g_L, g_U, 8, 10, 0, 
			   &eval_f, &eval_g, &eval_grad_f, 
			   &eval_jac_g, &eval_h);
  
  /* We can free the memory now - the values for the bounds have been
     copied internally in CreateIpoptProblem */
  free(x_L);
  free(x_U);
  free(g_L);
  free(g_U);

  /* set some options */
  AddIpoptNumOption(nlp, "tol", 1e-9);
  AddIpoptStrOption(nlp, "mu_strategy", "adaptive");

  /* allocate space for the initial point and set the values */
  x = (Number*)malloc(sizeof(Number)*n);
  x[0] = 1.0;
  x[1] = 5.0;
  x[2] = 5.0;
  x[3] = 1.0;

  /* allocate space to store the bound multipliers at the solution */
  mult_x_L = (Number*)malloc(sizeof(Number)*n);
  mult_x_U = (Number*)malloc(sizeof(Number)*n);

  /* solve the problem */
  status = IpoptSolve(nlp, x, NULL, &obj, NULL, mult_x_L, mult_x_U, NULL);

  if (status == Solve_Succeeded) {
    printf("\n\nSolution of the primal variables, x\n");
    for (i=0; i<n; i++) {
      printf("x[%d] = %e\n", i, x[i]); 
    }

    printf("\n\nSolution of the bound multipliers, z_L and z_U\n");
    for (i=0; i<n; i++) {
      printf("z_L[%d] = %e\n", i, mult_x_L[i]); 
    }
    for (i=0; i<n; i++) {
      printf("z_U[%d] = %e\n", i, mult_x_U[i]); 
    }

    printf("\n\nObjective value\n");
    printf("f(x*) = %e\n", obj); 
  }
 
  /* free allocated memory */
  FreeIpoptProblem(nlp);
  free(x);
  free(mult_x_L);
  free(mult_x_U);

  return 0;
}
\end{verbatim}
\end{footnotesize}
%  \caption{{\tt main} function for C example}
%  \label{fig:cexample-main}
%\end{figure}

Here, we declare all the necessary variables and set the dimensions of
the problem.  The problem has 4 variables, so we set {\tt n} and
allocate space for the variable bounds (don't forget to call {\tt
  free} for each of your {\tt malloc} calls before the end of the
program). We then set the values for the variable bounds.

The problem has 2 constraints, so we set {\tt m} and allocate space
for the constraint bounds. The first constraint has a lower bound of
$25$ and no upper bound.  Here we set the upper bound to
\texttt{2e19}. \Ipopt\ interprets any number greater than or equal to
\texttt{nlp\_upper\_bound\_inf} as infinity. The default value of
\texttt{nlp\_lower\_bound\_inf} and \texttt{nlp\_upper\_bound\_inf} is
\texttt{-1e19} and \texttt{1e19}, respectively, and can be changed
through \Ipopt\ options.  The second constraint is an equality with
right hand side 40, so we set both the upper and the lower bound to
40.

We next create an instance of the {\tt IpoptProblem} by calling {\tt
CreateIpoptProblem}, giving it the problem dimensions and the variable
and constraint bounds. The arguments {\tt nele\_jac} and {\tt
nele\_hess} are the number of elements in Jacobian and the Hessian,
respectively. See Appendix~\ref{app.triplet} for a description of the
sparse matrix format. The {\tt index\_style} argument specifies whether
we want to use C style indexing for the row and column indices of the
matrices or Fortran style indexing. Here, we set it to {\tt 0} to
indicate C style.  We also include the references to each of our
callback functions. \Ipopt\ uses these function pointers to ask for
evaluation of the NLP when required.

After freeing the bound arrays that are no longer required, the next
two lines illustrate how you can change the value of options through
the interface.  \Ipopt\ options can also be changed by creating a {\tt
PARAMS.DAT} file (see Section~\ref{sec.options}). We next allocate
space for the initial point and set the values as given in the problem
definition.

The call to {\tt IpoptSolve} can provide us with information about the
solution, but most of this is optional. Here, we want values for the
bound multipliers at the solution and we allocate space for these.

We can now make the call to {\tt IpoptSolve} and find the solution of
the problem. We pass in the {\tt IpoptProblem}, the starting point
{\tt x} (\Ipopt\ will use this array to return the solution or final
point as well).  The next 5 arguments are pointers so \Ipopt\ can fill
in values at the solution.  If these pointers are set to {\tt NULL},
\Ipopt\ will ignore that entry.  For example, here, we do not want the
constraint function values at the solution or the constraint
multipliers, so we set those entries to {\tt NULL}. We do want the
value of the objective, and the multipliers for the variable bounds.
The last argument is a {\tt void*} for user data. Any pointer you give
here will also be passed to you in the callback functions.

The return code is an {\tt ApplicationReturnStatus} enumeration, see
the header file {\tt ReturnCodes\_inc.h} which is installed along {\tt
  IpStdCInterface.h} in the \Ipopt\ include directory.

After the optimizer terminates, we check the status and print the
solution if successful. Finally, we free the {\tt IpoptProblem} and
the remaining memory, and return from {\tt main}.

\subsection{The Fortran Interface}

The Fortran interface is essentially a wrapper of the C interface
discussed in Section~\ref{sec.cinterface}.  The way to hook up \Ipopt\
in a Fortran program is very similar to how it is done for the C
interface, and the functions of the Fortran interface correspond
one-to-one to the those of the C and C++ interface, including their
arguments.  You can find an implementation of the example problem
(\ref{eq:ex_obj})--(\ref{eq:ex_bounds}) in {\tt
  \$IPOPTDIR/Examples/hs071\_f}.

The only special things to consider are:
\begin{itemize}
\item The return value of the function {\tt IPCREATE} is of an {\tt
    INTEGER} type that must be large enough to capture a pointer
  on the particular machine.  This means, that you have to declare
  the ``handle'' for the IpoptProblem as {\tt INTEGER*8} if your
  program is compiled in 64-bit mode.  All other {\tt INTEGER}-type
  variables must be of the regular type.
\item For the call of {\tt IPSOLVE} (which is the function that is to
  be called to run \Ipopt), all arrays, including those for the dual
  variables, must be given (in contrast to the C interface).  The
  return value {\tt IERR} of this function indicates the outcome of
  the optimization (see the include file {\tt IpReturnCodes.inc} in
  the \Ipopt\ include directory).
\item The return {\tt IERR} value of the remaining functions has to be
  set to zero, unless there was a problem during execution of the
  function call.
\item The callback functions ({\tt EV\_*} in the example) include the
  arguments {\tt IDAT} and {\tt DAT}, which are {\tt INTEGER} and {\tt
    DOUBLE PRECISION} arrays that are passed unmodified between the
  main program calling {\tt IPSOLVE} and the evaluation subroutines
  {\tt EV\_*} (similarly to {\tt UserDataPtr} arguments in the C
  interface).  These arrays can be used to pass ``private'' data
  between the main program and the user-provided Fortran subroutines.

  The last argument of the {\tt EV\_*} subroutines, {\tt IERR}, is to
  be set to 0 by the user on return, unless there was a problem
  during the evaluation of the optimization problem
  function/derivative for the given point {\tt X} (then it should
  return a non-zero value).
\end{itemize}

\section{Special Features}
\subsection{Derivative Checker}\label{sec:deriv-checker}
\subsection{Quasi-Newton Approximation of Second
  Derivatives}\label{sec:quasiNewton}

\section{\Ipopt\ Options}\label{sec.options}
Ipopt has many (maybe too many) options that can be adjusted for the
algorithm.  Options are all identified by a string name, and their
values can be of one of three types: Number (real), Integer, or
String. Number options are used for things like tolerances, integer
options are used for things like maximum number of iterations, and
string options are used for setting algorithm details, like the NLP
scaling method. Options can be set through code, through the AMPL
interface if you are using AMPL, or by creating a {\tt PARAMS.DAT}
file in the directory you are executing \Ipopt.

The {\tt PARAMS.DAT} file is read line by line and each line should
contain the option name, followed by whitespace, and then the
value. Comments can be included with the {\tt \#} symbol. Don't forget
to ensure you have a newline at the end of the file. For example,
\begin{verbatim}
# This is a comment

# Turn off the NLP scaling
nlp_scaling_method none

# Change the initial barrier parameter
mu_init 1e-2

# Set the max number of iterations
max_iter 500
\end{verbatim}
is a valid {\tt PARAMS.DAT} file.

Options can also be set in code. Have a look at the examples to see
how this is done. 

A subset of \Ipopt\ options are available through AMPL. To set options
through AMPL, use the internal AMPL command {\tt options}.  For
example, \\ 
{\tt options ipopt "nlp\_scaling\_method=none mu\_init=1e-2
max\_iter=500"} \\ 
is a valid options command in AMPL. The most common
options are referenced in Appendix~\ref{app.options_ref}. These are also
the options that are available through AMPL using the {\tt options}
command {\bf TODO: CHECK IF THAT IS CORRECT}. To specify other options when using AMPL, you can always
create {\tt PARAMS.DAT}.  Note, the {\tt PARAMS.DAT} file is given
preference when setting options. This way, you can easily override any
options set in a particular executable or AMPL model by specifying new
values in {\tt PARAMS.DAT}.

For a short list of the valid options, see the Appendix
\ref{app.options_ref}. You can print the documentation for all \Ipopt\
options by adding the option, \\

{\tt print\_options\_documentation yes} \\

and running \Ipopt\ (like the AMPL solver executable, for
instance). This will output all of the options documentation to the
console.

\section{\Ipopt\ Output}\label{sec.output}
This section describes the standard \Ipopt\ console output with the
default setting for {\tt print\_level}. The output is designed to
provide a quick summary of each iteration as \Ipopt\ solves the problem.

Before \Ipopt\ starts to solve the problem, it displays the problem
statistics (number of variables, etc.). Note that if you have fixed
variables (both upper and lower bounds are equal), \Ipopt\ may remove
these variables from the problem internally and not include them in
the problem statistics.

Following the problem statistics, \Ipopt\ will begin to solve the
problem and you will see output resembling the following,
\begin{verbatim}
iter    objective    inf_pr   inf_du lg(mu)  ||d||  lg(rg) alpha_du alpha_pr  ls
   0  1.6109693e+01 1.12e+01 5.28e-01   0.0 0.00e+00    -  0.00e+00 0.00e+00   0
   1  1.8029749e+01 9.90e-01 6.62e+01   0.1 2.05e+00    -  2.14e-01 1.00e+00f  1
   2  1.8719906e+01 1.25e-02 9.04e+00  -2.2 5.94e-02   2.0 8.04e-01 1.00e+00h  1
\end{verbatim}
and the columns of output are defined as,
\begin{description}
\item[{\tt iter}:] The current iteration count. This includes regular
  iterations and iterations while in restoration phase. If the
  algorithm is in the restoration phase, the letter {\tt r'} will be
  appended to the iteration number.
\item[{\tt objective}:] The unscaled objective value at the current
  point. During the restoration phase, this value remains the unscaled
  objective value for the original problem.
\item[{\tt inf\_pr}:] The scaled primal infeasibility at the current
  point. During the restoration phase, this value is the primal
  infeasibility of the original problem at the current point.
\item[{\tt inf\_du}:] The scaled dual infeasibility at the current
  point. During the restoration phase, this is the value of the dual
  infeasibility for the restoration phase problem.
\item[{\tt lg(mu)}:] $\log_{10}$ of the value of the barrier parameter mu.
\item[{\tt ||d||}:] The infinity norm (max) of the primal step (for
  the original variables $x$ and the internal slack variables $s$).
  During the restoration phase, this value includes the values of
  additional variables, $p$ and $n$ (see Eq.~(30) in
  \cite{WaecBieg06:mp}).
\item[{\tt lg(rg)}:] $\log_{10}$ of the value of the regularization
  term for the Hessian of the Lagrangian in the augmented system.
\item[{\tt alpha\_du}:] The stepsize for the dual variables.
\item[{\tt alpha\_pr}:] The stepsize for the primal variables.
\item[{\tt ls}:] The number of backtracking line search steps.
\end{description}

When the algorithm terminates, \Ipopt\ will output a message to the
screen based on the return status of the call to {\tt Optimize}. The following
is a list of the possible return codes, their corresponding output message
to the console, and a brief description.
\begin{description}
\item[{\tt Solve\_Succeeded}:] $\;$ \\
  Console Message: {\tt EXIT: Optimal Solution Found.} \\
  This message indicates that \Ipopt\ found a (locally) optimal point
  within the desired tolerances.
\item[{\tt Solved\_To\_Acceptable\_Level}:]  $\;$ \\
  Console Message: {\tt EXIT: Solved To Acceptable Level.} \\
  This indicates that the algorithm did not converge to the
  ``desired'' tolerances, but that it was able to obtain a point
  satisfying the ``acceptable'' tolerance level as specified by {\tt
    acceptable-*} options. This may happen if the desired tolerances
  are too small for the current problem.
\item[{\tt Infeasible\_Problem\_Detected}:]  $\;$ \\
  Console Message: {\tt EXIT: Converged to a point of
    local infeasibility. Problem may be infeasible.} \\
  The restoration phase converged to a point that is a minimizer for
  the constraint violation (in the $\ell_1$-norm), but is not feasible
  for the original problem. This indicates that the problem may be
  infeasible (or at least that the algorithm is stuck at a locally
  infeasible point).  The returned point (the minimizer of the
  constraint violation) might help you to find which constraint is
  causing the problem.  If you believe that the NLP is feasible,
  it might help to start the optimization from a different point.
\item[{\tt Search\_Direction\_Becomes\_Too\_Small}:]  $\;$ \\
  Console Message: {\tt EXIT: Search Direction is becoming Too Small.} \\
  This indicates that \Ipopt\ is calculating very small step sizes and
  making very little progress.  This could happen if the problem has
  been solved to the best numerical accuracy possible given the
  current scaling.
\item[{\tt Maximum\_Iterations\_Exceeded}:]  $\;$ \\
  Console Message: {\tt EXIT: Maximum Number of Iterations Exceeded.} \\
  This indicates that \Ipopt\ has exceeded the maximum number of
  iterations as specified by the option {\tt max\_iter}.
\item[{\tt Restoration\_Failed}:]  $\;$ \\
  Console Message: {\tt EXIT: Restoration Failed!} \\
  This indicates that the restoration phase failed to find a feasible
  point that was acceptable to the filter line search for the original
  problem. This could happen if the problem is highly degenerate, does
  not satisfy the constraint qualification, or if your NLP code
  provides incorrect derivative information.
\item[{\tt Invalid\_Option}:]  $\;$ \\
  Console Message: (details about the particular error
  will be output to the console) \\
  This indicates that there was some problem specifying the options.
  See the specific message for details.
\item[{\tt Not\_Enough\_Degrees\_Of\_Freedom}:]  $\;$ \\
  Console Message: {\tt EXIT: Problem has too few degrees of freedom.} \\
  This indicates that your problem, as specified, has too few degrees
  of freedom. This can happen if you have too many equality
  constraints, or if you fix too many variables (\Ipopt\ removes fixed
  variables).
\item[{\tt Invalid\_Problem\_Definition}:]  $\;$ \\
  Console Message: (no console message, this is a return code for the
  C and Fortran interfaces only.) \\
  This indicates that there was an exception of some sort when
  building the {\tt IpoptProblem} structure in the C or Fortran
  interface. Likely there is an error in your model or the {\tt main}
  routine.
\item[{\tt Unrecoverable\_Exception}:]  $\;$ \\
  Console Message: (details about the particular error
  will be output to the console) \\
  This indicates that \Ipopt\ has thrown an exception that does not
  have an internal return code. See the specific message for details.
\item[{\tt NonIpopt\_Exception\_Thrown}:]  $\;$ \\
  Console Message: {\tt Unknown Exception caught in Ipopt} \\
  An unknown exception was caught in \Ipopt. This exception could have
  originated from your model or any linked in third party code.
\item[{\tt Insufficient\_Memory}:]  $\;$ \\
  Console Message: {\tt EXIT: Not enough memory.} \\
  An error occurred while trying to allocate memory. The problem may
  be too large for your current memory and swap configuration.
\item[{\tt Internal\_Error}:]  $\;$ \\
  Console Message: {\tt EXIT: INTERNAL ERROR: Unknown SolverReturn
    value - Notify IPOPT Authors.} \\
  An unknown internal error has occurred. Please notify the authors of
  \Ipopt.

\end{description}

\appendix
\newpage
\section{Triplet Format for Sparse Matrices}\label{app.triplet}
\Ipopt\ was designed for optimizing large sparse nonlinear programs.
Because of problem sparsity, the required matrices (like the
constraints Jacobian or Lagrangian Hessian) are not stored as dense
matrices, but rather in a sparse matrix format. For the tutorials in
this document, we use the triplet format.  Consider the matrix
\begin{equation}
\label{eqn.ex_matrix}
\left[
\begin{array}{ccccccc}
1.1     & 0             & 0             & 0             & 0             & 0             & 0.5 \\
0       & 1.9   & 0             & 0             & 0             & 0             & 0.5 \\
0       & 0             & 2.6   & 0             & 0             & 0             & 0.5 \\
0       & 0             & 7.8   & 0.6   & 0             & 0             & 0    \\
0       & 0             & 0             & 1.5   & 2.7   & 0             & 0     \\
1.6     & 0             & 0             & 0             & 0.4   & 0             & 0     \\
0       & 0             & 0             & 0             & 0             & 0.9   & 1.7 \\
\end{array}
\right]
\end{equation}

A standard dense matrix representation would need to store $7 \cdot
7{=} 49$ floating point numbers, where many entries would be zero. In
triplet format, however, only the nonzero entries are stored. The
triplet format records the row number, the column number, and the
value of all nonzero entries in the matrix. For the matrix above, this
means storing $14$ integers for the rows, $14$ integers for the
columns, and $14$ floating point numbers for the values. While this
does not seem like a huge space savings over the $49$ floating point
numbers stored in the dense representation, for larger matrices, the
space savings are very dramatic\footnote{For an $n \times n$ matrix,
the dense representation grows with the the square of $n$, while the
sparse representation grows linearly in the number of nonzeros.}.

The option {\tt index\_style} in {\tt get\_nlp\_info} tells \Ipopt\ if
you prefer to use C style indexing (0-based, i.e., starting the
counting at 0) for the row and column indices or Fortran style
(1-based). Tables \ref{tab.fortran_triplet} and \ref{tab.c_triplet}
below show the triplet format for both indexing styles, using the
example matrix (\ref{eqn.ex_matrix}).

\begin{footnotesize}
\begin{table}[ht]%[!h]
\begin{center}
\begin{tabular}{c c c}
row     		&       col     	&       value 			    \\
\hline
{\tt iRow[0] = 1}       &       {\tt jCol[0] = 1}       & {\tt values[0] = 1.1}     \\
{\tt iRow[1] = 1}       &       {\tt jCol[1] = 7}       & {\tt values[1] = 0.5}     \\
{\tt iRow[2] = 2}       &       {\tt jCol[2] = 2}       & {\tt values[2] = 1.9}     \\
{\tt iRow[3] = 2}       &       {\tt jCol[3] = 7}       & {\tt values[3] = 0.5}     \\
{\tt iRow[4] = 3}       &       {\tt jCol[4] = 3}       & {\tt values[4] = 2.6}     \\
{\tt iRow[5] = 3}       &       {\tt jCol[5] = 7}       & {\tt values[5] = 0.5}     \\
{\tt iRow[6] = 4}       &       {\tt jCol[6] = 3}       & {\tt values[6] = 7.8}     \\
{\tt iRow[7] = 4}       &       {\tt jCol[7] = 4}       & {\tt values[7] = 0.6}     \\
{\tt iRow[8] = 5}       &       {\tt jCol[8] = 4}       & {\tt values[8] = 1.5}     \\
{\tt iRow[9] = 5}       &       {\tt jCol[9] = 5}       & {\tt values[9] = 2.7}     \\
{\tt iRow[10] = 6}      &       {\tt jCol[10] = 1}      & {\tt values[10] = 1.6}     \\
{\tt iRow[11] = 6}      &       {\tt jCol[11] = 5}      & {\tt values[11] = 0.4}     \\
{\tt iRow[12] = 7}      &       {\tt jCol[12] = 6}      & {\tt values[12] = 0.9}     \\
{\tt iRow[13] = 7}      &       {\tt jCol[13] = 7}      & {\tt values[13] = 1.7}
\end{tabular}
\caption{Triplet Format of Matrix (\ref{eqn.ex_matrix}) 
with {\tt index\_style=FORTRAN\_STYLE}}
\label{tab.fortran_triplet}
\end{center}
\end{table}
\begin{table}[ht]%[!h]
\begin{center}
\begin{tabular}{c c c}
row     		&       col     	&       value 			    \\
\hline
{\tt iRow[0] = 0}       &       {\tt jCol[0] = 0}       & {\tt values[0] = 1.1}     \\
{\tt iRow[1] = 0}       &       {\tt jCol[1] = 6}       & {\tt values[1] = 0.5}     \\
{\tt iRow[2] = 1}       &       {\tt jCol[2] = 1}       & {\tt values[2] = 1.9}     \\
{\tt iRow[3] = 1}       &       {\tt jCol[3] = 6}       & {\tt values[3] = 0.5}     \\
{\tt iRow[4] = 2}       &       {\tt jCol[4] = 2}       & {\tt values[4] = 2.6}     \\
{\tt iRow[5] = 2}       &       {\tt jCol[5] = 6}       & {\tt values[5] = 0.5}     \\
{\tt iRow[6] = 3}       &       {\tt jCol[6] = 2}       & {\tt values[6] = 7.8}     \\
{\tt iRow[7] = 3}       &       {\tt jCol[7] = 3}       & {\tt values[7] = 0.6}     \\
{\tt iRow[8] = 4}       &       {\tt jCol[8] = 3}       & {\tt values[8] = 1.5}     \\
{\tt iRow[9] = 4}       &       {\tt jCol[9] = 4}       & {\tt values[9] = 2.7}     \\
{\tt iRow[10] = 5}      &       {\tt jCol[10] = 0}      & {\tt values[10] = 1.6}     \\
{\tt iRow[11] = 5}      &       {\tt jCol[11] = 4}      & {\tt values[11] = 0.4}     \\
{\tt iRow[12] = 6}      &       {\tt jCol[12] = 5}      & {\tt values[12] = 0.9}     \\
{\tt iRow[13] = 6}      &       {\tt jCol[13] = 6}      & {\tt values[13] = 1.7}
\end{tabular}
\caption{Triplet Format of Matrix (\ref{eqn.ex_matrix}) 
with {\tt index\_style=C\_STYLE}}
\label{tab.c_triplet}
\end{center}
\end{table}
\end{footnotesize}
The individual elements of the matrix can be listed in any order, and
if there are multiple items for the same nonzero position, the values
provided for those positions are added.

The Hessian of the Lagrangian is a symmetric matrix. In the case of a
symmetric matrix, you only need to specify the lower left triangual
part (or, alternatively, the upper right triangular part). For
example, given the matrix,
\begin{equation}
\label{eqn.ex_sym_matrix}
\left[
\begin{array}{ccccccc}
1.0	& 0	& 3.0	& 0	& 2.0 	\\
0	& 1.1	& 0	& 0	& 5.0	\\
3.0	& 0	& 1.2	& 6.0	& 0	\\
0	& 0	& 6.0	& 1.3	& 9.0	\\
2.0	& 5.0	& 0	& 9.0	& 1.4
\end{array}
\right]
\end{equation}
the triplet format is shown in Tables \ref{tab.sym_fortran_triplet}
and \ref{tab.sym_c_triplet}.

\begin{footnotesize}
\begin{table}[ht]%[!h]
\begin{center}
\caption{Triplet Format of Matrix (\ref{eqn.ex_matrix}) 
with {\tt index\_style=FORTRAN\_STYLE}}
\label{tab.sym_fortran_triplet}
\begin{tabular}{c c c}
row     		&       col     	&       value 			    \\
\hline
{\tt iRow[0] = 1}       &       {\tt jCol[0] = 1}       & {\tt values[0] = 1.0}     \\
{\tt iRow[1] = 2}       &       {\tt jCol[1] = 1}       & {\tt values[1] = 1.1}     \\
{\tt iRow[2] = 3}       &       {\tt jCol[2] = 1}       & {\tt values[2] = 3.0}     \\
{\tt iRow[3] = 3}       &       {\tt jCol[3] = 3}       & {\tt values[3] = 1.2}     \\
{\tt iRow[4] = 4}       &       {\tt jCol[4] = 3}       & {\tt values[4] = 6.0}     \\
{\tt iRow[5] = 4}       &       {\tt jCol[5] = 4}       & {\tt values[5] = 1.3}     \\
{\tt iRow[6] = 5}       &       {\tt jCol[6] = 1}       & {\tt values[6] = 2.0}     \\
{\tt iRow[7] = 5}       &       {\tt jCol[7] = 2}       & {\tt values[7] = 5.0}     \\
{\tt iRow[8] = 5}       &       {\tt jCol[8] = 4}       & {\tt values[8] = 9.0}     \\
{\tt iRow[9] = 5}       &       {\tt jCol[9] = 5}       & {\tt values[9] = 1.4}
\end{tabular}
\end{center}
\end{table}
\begin{table}[ht]%[!h]
\begin{center}
\caption{Triplet Format of Matrix (\ref{eqn.ex_matrix}) 
with {\tt index\_style=C\_STYLE}}
\label{tab.sym_c_triplet}
\begin{tabular}{c c c}
row     		&       col     	&       value 			    \\
\hline
{\tt iRow[0] = 0}       &       {\tt jCol[0] = 0}       & {\tt values[0] = 1.0}     \\
{\tt iRow[1] = 1}       &       {\tt jCol[1] = 0}       & {\tt values[1] = 1.1}     \\
{\tt iRow[2] = 2}       &       {\tt jCol[2] = 0}       & {\tt values[2] = 3.0}     \\
{\tt iRow[3] = 2}       &       {\tt jCol[3] = 2}       & {\tt values[3] = 1.2}     \\
{\tt iRow[4] = 3}       &       {\tt jCol[4] = 2}       & {\tt values[4] = 6.0}     \\
{\tt iRow[5] = 3}       &       {\tt jCol[5] = 3}       & {\tt values[5] = 1.3}     \\
{\tt iRow[6] = 4}       &       {\tt jCol[6] = 0}       & {\tt values[6] = 2.0}     \\
{\tt iRow[7] = 4}       &       {\tt jCol[7] = 1}       & {\tt values[7] = 5.0}     \\
{\tt iRow[8] = 4}       &       {\tt jCol[8] = 3}       & {\tt values[8] = 9.0}     \\
{\tt iRow[9] = 4}       &       {\tt jCol[9] = 4}       & {\tt values[9] = 1.4}
\end{tabular}
\end{center}
\end{table}
\end{footnotesize}
\newpage
\section{The Smart Pointer Implementation: {\tt SmartPtr<T>}} \label{app.smart_ptr}

The {\tt SmartPtr} class is described in {\tt IpSmartPtr.hpp}. It is a
template class that takes care of deleting objects for us so we need
not be concerned about memory leaks. Instead of pointing to an object
with a raw C++ pointer (e.g. {\tt HS071\_NLP*}), we use a {\tt
  SmartPtr}.  Every time a {\tt SmartPtr} is set to reference an
object, it increments a counter in that object (see the {\tt
  ReferencedObject} base class if you are interested). If a {\tt
  SmartPtr} is done with the object, either by leaving scope or being
set to point to another object, the counter is decremented. When the
count of the object goes to zero, the object is automatically deleted.
{\tt SmartPtr}'s are very simple, just use them as you would a
standard pointer.

It is very important to use {\tt SmartPtr}'s instead of raw pointers
when passing objects to \Ipopt. Internally, \Ipopt\ uses smart
pointers for referencing objects. If you use a raw pointer in your
executable, the object's counter will NOT get incremented. Then, when
\Ipopt\ uses smart pointers inside its own code, the counter will get
incremented. However, before \Ipopt\ returns control to your code, it
will decrement as many times as it incremented earlier, and the
counter will return to zero. Therefore, \Ipopt\ will delete the
object. When control returns to you, you now have a raw pointer that
points to a deleted object.

This might sound difficult to anyone not familiar with the use of
smart pointers, but just follow one simple rule; always use a SmartPtr
when creating or passing an \Ipopt\ object.

\newpage
\section{Options Reference} \label{app.options_ref}
Options can be set using {\tt PARAMS.DAT}, through your own code, or through the 
AMPL {\tt options} command. See Section \ref{sec.options} for an explanation of
how to use these commands.
Shown here is a short list of the most common options for Ipopt. To view
the full list of options, run the ipopt executable with the option,
\begin{verbatim}
print_options_documentation yes
\end{verbatim}

The most common options are:


\paragraph{print\_level:} Output verbosity level. $\;$ \\
 Sets the default verbosity level for console
output. The larger this value the more detailed
is the output. The valid range for this integer option is
$0 \le {\tt print\_level } \le 11$
and its default value is $4$.


\paragraph{print\_user\_options:} Print all options set by the user. $\;$ \\
 If selected, the algorithm will print the list of
all options set by the user including their
values and whether they have been used.
The default value for this string option is "no".
\\ 
Possible values:
\begin{itemize}
   \item no: don't print options
   \item yes: print options
\end{itemize}

\paragraph{print\_options\_documentation:} Switch to print all algorithmic options. $\;$ \\
 If selected, the algorithm will print the list of
all available algorithmic options with some
documentation before solving the optimization
problem.
The default value for this string option is "no".
\\ 
Possible values:
\begin{itemize}
   \item no: don't print list
   \item yes: print list
\end{itemize}

\paragraph{output\_file:} File name of desired output file (leave unset for no file output). $\;$ \\
 NOTE: This option only works when read from the
ipopt.opt options file! An output file with this
name will be written (leave unset for no file
output).  The verbosity level is by default set
to "print\_level", but can be overridden with
"file\_print\_level".  The file name is changed
to use only small letters.
The default value for this string option is "".
\\ 
Possible values:
\begin{itemize}
   \item *: Any acceptable standard file name
\end{itemize}

\paragraph{file\_print\_level:} Verbosity level for output file. $\;$ \\
 NOTE: This option only works when read from the
ipopt.opt options file! Determines the verbosity
level for the file specified by "output\_file". 
By default it is the same as "print\_level". The valid range for this integer option is
$0 \le {\tt file\_print\_level } \le 11$
and its default value is $4$.


\paragraph{tol:} Desired convergence tolerance (relative). $\;$ \\
 Determines the convergence tolerance for the
algorithm.  The algorithm terminates
successfully, if the (scaled) NLP error becomes
smaller than this value, and if the (absolute)
criteria according to "dual\_inf\_tol",
"primal\_inf\_tol", and "cmpl\_inf\_tol" are met.
 (This is epsilon\_tol in Eqn. (6) in
implementation paper).  See also
"acceptable\_tol" as a second termination
criterion.  Note, some other algorithmic features
also use this quantity to determine thresholds
etc. The valid range for this real option is 
$0 <  {\tt tol } <  {\tt +inf}$
and its default value is $1 \cdot 10^{-08}$.


\paragraph{max\_iter:} Maximum number of iterations. $\;$ \\
 The algorithm terminates with an error message if
the number of iterations exceeded this number. The valid range for this integer option is
$0 \le {\tt max\_iter } <  {\tt +inf}$
and its default value is $3000$.


\paragraph{compl\_inf\_tol:} Desired threshold for the complementarity conditions. $\;$ \\
 Absolute tolerance on the complementarity.
Successful termination requires that the max-norm
of the (unscaled) complementarity is less than
this threshold. The valid range for this real option is 
$0 <  {\tt compl\_inf\_tol } <  {\tt +inf}$
and its default value is $0.0001$.


\paragraph{dual\_inf\_tol:} Desired threshold for the dual infeasibility. $\;$ \\
 Absolute tolerance on the dual infeasibility.
Successful termination requires that the max-norm
of the (unscaled) dual infeasibility is less than
this threshold. The valid range for this real option is 
$0 <  {\tt dual\_inf\_tol } <  {\tt +inf}$
and its default value is $0.0001$.


\paragraph{constr\_viol\_tol:} Desired threshold for the constraint violation. $\;$ \\
 Absolute tolerance on the constraint violation.
Successful termination requires that the max-norm
of the (unscaled) constraint violation is less
than this threshold. The valid range for this real option is 
$0 <  {\tt constr\_viol\_tol } <  {\tt +inf}$
and its default value is $0.0001$.


\paragraph{acceptable\_tol:} "Acceptable" convergence tolerance (relative). $\;$ \\
 Determines which (scaled) overall optimality
error is considered to be "acceptable." There are
two levels of termination criteria.  If the usual
"desired" tolerances (see tol, dual\_inf\_tol
etc) are satisfied at an iteration, the algorithm
immediately terminates with a success message. 
On the other hand, if the algorithm encounters
"acceptable\_iter" many iterations in a row that
are considered "acceptable", it will terminate
before the desired convergence tolerance is met.
This is useful in cases where the algorithm might
not be able to achieve the "desired" level of
accuracy. The valid range for this real option is 
$0 <  {\tt acceptable\_tol } <  {\tt +inf}$
and its default value is $1 \cdot 10^{-06}$.


\paragraph{acceptable\_compl\_inf\_tol:} "Acceptance" threshold for the complementarity conditions. $\;$ \\
 Absolute tolerance on the complementarity.
"Acceptable" termination requires that the
max-norm of the (unscaled) complementarity is
less than this threshold; see also
acceptable\_tol. The valid range for this real option is 
$0 <  {\tt acceptable\_compl\_inf\_tol } <  {\tt +inf}$
and its default value is $0.01$.


\paragraph{acceptable\_constr\_viol\_tol:} "Acceptance" threshold for the constraint violation. $\;$ \\
 Absolute tolerance on the constraint violation.
"Acceptable" termination requires that the
max-norm of the (unscaled) constraint violation
is less than this threshold; see also
acceptable\_tol. The valid range for this real option is 
$0 <  {\tt acceptable\_constr\_viol\_tol } <  {\tt +inf}$
and its default value is $0.01$.


\paragraph{acceptable\_dual\_inf\_tol:} "Acceptance" threshold for the dual infeasibility. $\;$ \\
 Absolute tolerance on the dual infeasibility.
"Acceptable" termination requires that the
(max-norm of the unscaled) dual infeasibility is
less than this threshold; see also
acceptable\_tol. The valid range for this real option is 
$0 <  {\tt acceptable\_dual\_inf\_tol } <  {\tt +inf}$
and its default value is $0.01$.


\paragraph{diverging\_iterates\_tol:} Threshold for maximal value of primal iterates. $\;$ \\
 If any component of the primal iterates exceeded
this value (in absolute terms), the optimization
is aborted with the exit message that the
iterates seem to be diverging. The valid range for this real option is 
$0 <  {\tt diverging\_iterates\_tol } <  {\tt +inf}$
and its default value is $1 \cdot 10^{+20}$.


\paragraph{barrier\_tol\_factor:} Factor for mu in barrier stop test. $\;$ \\
 The convergence tolerance for each barrier
problem in the monotone mode is the value of the
barrier parameter times "barrier\_tol\_factor".
This option is also used in the adaptive mu
strategy during the monotone mode. (This is
kappa\_epsilon in implementation paper). The valid range for this real option is 
$0 <  {\tt barrier\_tol\_factor } <  {\tt +inf}$
and its default value is $10$.


\paragraph{obj\_scaling\_factor:} Scaling factor for the objective function. $\;$ \\
 This option sets a scaling factor for the
objective function. The scaling is seen
internally by Ipopt but the unscaled objective is
reported in the console output. If additional
scaling parameters are computed (e.g.
user-scaling or gradient-based), both factors are
multiplied. If this value is chosen to be
negative, Ipopt will maximize the objective
function instead of minimizing it. The valid range for this real option is 
${\tt -inf} <  {\tt obj\_scaling\_factor } <  {\tt +inf}$
and its default value is $1$.


\paragraph{nlp\_scaling\_method:} Select the technique used for scaling the NLP. $\;$ \\
 Selects the technique used for scaling the
problem internally before it is solved. For
user-scaling, the parameters come from the NLP.
If you are using AMPL, they can be specified
through suffixes ("scaling\_factor")
The default value for this string option is "gradient-based".
\\ 
Possible values:
\begin{itemize}
   \item none: no problem scaling will be performed
   \item user-scaling: scaling parameters will come from the user
   \item gradient-based: scale the problem so the maximum gradient at
the starting point is scaling\_max\_gradient
\end{itemize}

\paragraph{nlp\_scaling\_max\_gradient:} Maximum gradient after NLP scaling. $\;$ \\
 This is the gradient scaling cut-off. If the
maximum gradient is above this value, then
gradient based scaling will be performed. Scaling
parameters are calculated to scale the maximum
gradient back to this value. (This is g\_max in
Section 3.8 of the implementation paper.) Note:
This option is only used if
"nlp\_scaling\_method" is chosen as
"gradient-based". The valid range for this real option is 
$0 <  {\tt nlp\_scaling\_max\_gradient } <  {\tt +inf}$
and its default value is $100$.


\paragraph{bound\_relax\_factor:} Factor for initial relaxation of the bounds. $\;$ \\
 Before start of the optimization, the bounds
given by the user are relaxed.  This option sets
the factor for this relaxation.  If it is set to
zero, then then bounds relaxation is disabled.
(See Eqn.(35) in implementation paper.) The valid range for this real option is 
$0 \le {\tt bound\_relax\_factor } <  {\tt +inf}$
and its default value is $1 \cdot 10^{-08}$.


\paragraph{honor\_original\_bounds:} Indicates whether final points should be projected into original bounds. $\;$ \\
 Ipopt might relax the bounds during the
optimization (see, e.g., option
"bound\_relax\_factor").  This option determines
whether the final point should be projected back
into the user-provide original bounds after the
optimization.
The default value for this string option is "yes".
\\ 
Possible values:
\begin{itemize}
   \item no: Leave final point unchanged
   \item yes: Project final point back into original bounds
\end{itemize}

\paragraph{check\_derivatives\_for\_naninf:} Indicates whether it is desired to check for Nan/Inf in derivative matrices $\;$ \\
 Activating this option will cause an error if an
invalid number is detected in the constraint
Jacobians or the Lagrangian Hessian.  If this is
not activated, the test is skipped, and the
algorithm might proceed with invalid numbers and
fail.
The default value for this string option is "no".
\\ 
Possible values:
\begin{itemize}
   \item no: Don't check (faster).
   \item yes: Check Jacobians and Hessian for Nan and Inf.
\end{itemize}

\paragraph{mu\_strategy:} Update strategy for barrier parameter. $\;$ \\
 Determines which barrier parameter update
strategy is to be used.
The default value for this string option is "monotone".
\\ 
Possible values:
\begin{itemize}
   \item monotone: use the monotone (Fiacco-McCormick) strategy
   \item adaptive: use the adaptive update strategy
\end{itemize}

\paragraph{mu\_oracle:} Oracle for a new barrier parameter in the adaptive strategy. $\;$ \\
 Determines how a new barrier parameter is
computed in each "free-mode" iteration of the
adaptive barrier parameter strategy. (Only
considered if "adaptive" is selected for option
"mu\_strategy").
The default value for this string option is "quality-function".
\\ 
Possible values:
\begin{itemize}
   \item probing: Mehrotra's probing heuristic
   \item loqo: LOQO's centrality rule
   \item quality-function: minimize a quality function
\end{itemize}

\paragraph{quality\_function\_max\_section\_steps:} Maximum number of search steps during direct search procedure determining the optimal centering parameter. $\;$ \\
 The golden section search is performed for the
quality function based mu oracle. (Only used if
option "mu\_oracle" is set to "quality-function".) The valid range for this integer option is
$0 \le {\tt quality\_function\_max\_section\_steps } <  {\tt +inf}$
and its default value is $8$.


\paragraph{fixed\_mu\_oracle:} Oracle for the barrier parameter when switching to fixed mode. $\;$ \\
 Determines how the first value of the barrier
parameter should be computed when switching to
the "monotone mode" in the adaptive strategy.
(Only considered if "adaptive" is selected for
option "mu\_strategy".)
The default value for this string option is "average\_compl".
\\ 
Possible values:
\begin{itemize}
   \item probing: Mehrotra's probing heuristic
   \item loqo: LOQO's centrality rule
   \item quality-function: minimize a quality function
   \item average\_compl: base on current average complementarity
\end{itemize}

\paragraph{mu\_init:} Initial value for the barrier parameter. $\;$ \\
 This option determines the initial value for the
barrier parameter (mu).  It is only relevant in
the monotone, Fiacco-McCormick version of the
algorithm. (i.e., if "mu\_strategy" is chosen as
"monotone") The valid range for this real option is 
$0 <  {\tt mu\_init } <  {\tt +inf}$
and its default value is $0.1$.


\paragraph{mu\_max\_fact:} Factor for initialization of maximum value for barrier parameter. $\;$ \\
 This option determines the upper bound on the
barrier parameter.  This upper bound is computed
as the average complementarity at the initial
point times the value of this option. (Only used
if option "mu\_strategy" is chosen as "adaptive".) The valid range for this real option is 
$0 <  {\tt mu\_max\_fact } <  {\tt +inf}$
and its default value is $1000$.


\paragraph{mu\_max:} Maximum value for barrier parameter. $\;$ \\
 This option specifies an upper bound on the
barrier parameter in the adaptive mu selection
mode.  If this option is set, it overwrites the
effect of mu\_max\_fact. (Only used if option
"mu\_strategy" is chosen as "adaptive".) The valid range for this real option is 
$0 <  {\tt mu\_max } <  {\tt +inf}$
and its default value is $100000$.


\paragraph{mu\_min:} Minimum value for barrier parameter. $\;$ \\
 This option specifies the lower bound on the
barrier parameter in the adaptive mu selection
mode. By default, it is set to
min("tol","compl\_inf\_tol")/("barrier\_tol\_fact-
or"+1), which should be a reasonable value. (Only
used if option "mu\_strategy" is chosen as
"adaptive".) The valid range for this real option is 
$0 <  {\tt mu\_min } <  {\tt +inf}$
and its default value is $1 \cdot 10^{-09}$.


\paragraph{mu\_linear\_decrease\_factor:} Determines linear decrease rate of barrier parameter. $\;$ \\
 For the Fiacco-McCormick update procedure the new
barrier parameter mu is obtained by taking the
minimum of mu*"mu\_linear\_decrease\_factor" and
mu\^"superlinear\_decrease\_power".  (This is
kappa\_mu in implementation paper.) This option
is also used in the adaptive mu strategy during
the monotone mode. The valid range for this real option is 
$0 <  {\tt mu\_linear\_decrease\_factor } <  1$
and its default value is $0.2$.


\paragraph{mu\_superlinear\_decrease\_power:} Determines superlinear decrease rate of barrier parameter. $\;$ \\
 For the Fiacco-McCormick update procedure the new
barrier parameter mu is obtained by taking the
minimum of mu*"mu\_linear\_decrease\_factor" and
mu\^"superlinear\_decrease\_power".  (This is
theta\_mu in implementation paper.) This option
is also used in the adaptive mu strategy during
the monotone mode. The valid range for this real option is 
$1 <  {\tt mu\_superlinear\_decrease\_power } <  2$
and its default value is $1.5$.


\paragraph{bound\_frac:} Desired minimum relative distance from the initial point to bound. $\;$ \\
 Determines how much the initial point might have
to be modified in order to be sufficiently inside
the bounds (together with "bound\_push").  (This
is kappa\_2 in Section 3.6 of implementation
paper.) The valid range for this real option is 
$0 <  {\tt bound\_frac } \le 0.5$
and its default value is $0.01$.


\paragraph{bound\_push:} Desired minimum absolute distance from the initial point to bound. $\;$ \\
 Determines how much the initial point might have
to be modified in order to be sufficiently inside
the bounds (together with "bound\_frac").  (This
is kappa\_1 in Section 3.6 of implementation
paper.) The valid range for this real option is 
$0 <  {\tt bound\_push } <  {\tt +inf}$
and its default value is $0.01$.


\paragraph{bound\_mult\_init\_val:} Initial value for the bound multipliers. $\;$ \\
 All dual variables corresponding to bound
constraints are initialized to this value. The valid range for this real option is 
$0 <  {\tt bound\_mult\_init\_val } <  {\tt +inf}$
and its default value is $1$.


\paragraph{constr\_mult\_init\_max:} Maximum allowed least-square guess of constraint multipliers. $\;$ \\
 Determines how large the initial least-square
guesses of the constraint multipliers are allowed
to be (in max-norm). If the guess is larger than
this value, it is discarded and all constraint
multipliers are set to zero.  This options is
also used when initializing the restoration
phase. By default,
"resto.constr\_mult\_init\_max" (the one used in
RestoIterateInitializer) is set to zero. The valid range for this real option is 
$0 \le {\tt constr\_mult\_init\_max } <  {\tt +inf}$
and its default value is $1000$.


\paragraph{bound\_mult\_init\_val:} Initial value for the bound multipliers. $\;$ \\
 All dual variables corresponding to bound
constraints are initialized to this value. The valid range for this real option is 
$0 <  {\tt bound\_mult\_init\_val } <  {\tt +inf}$
and its default value is $1$.


\paragraph{warm\_start\_init\_point:} Warm-start for initial point $\;$ \\
 Indicates whether this optimization should use a
warm start initialization, where values of primal
and dual variables are given (e.g., from a
previous optimization of a related problem.)
The default value for this string option is "no".
\\ 
Possible values:
\begin{itemize}
   \item no: do not use the warm start initialization
   \item yes: use the warm start initialization
\end{itemize}

\paragraph{warm\_start\_bound\_push:} same as bound\_push for the regular initializer. $\;$ \\
 The valid range for this real option is 
$0 <  {\tt warm\_start\_bound\_push } <  {\tt +inf}$
and its default value is $0.001$.


\paragraph{warm\_start\_bound\_frac:} same as bound\_frac for the regular initializer. $\;$ \\
 The valid range for this real option is 
$0 <  {\tt warm\_start\_bound\_frac } \le 0.5$
and its default value is $0.001$.


\paragraph{warm\_start\_mult\_bound\_push:} same as mult\_bound\_push for the regular initializer. $\;$ \\
 The valid range for this real option is 
$0 <  {\tt warm\_start\_mult\_bound\_push } <  {\tt +inf}$
and its default value is $0.001$.


\paragraph{warm\_start\_mult\_init\_max:} Maximum initial value for the equality multipliers. $\;$ \\
 The valid range for this real option is 
${\tt -inf} <  {\tt warm\_start\_mult\_init\_max } <  {\tt +inf}$
and its default value is $1 \cdot 10^{+06}$.


\paragraph{alpha\_for\_y:} Method to determine the step size for constraint multipliers. $\;$ \\
 This option determines how the step size
(alpha\_y) will be calculated when updating the
constraint multipliers.
The default value for this string option is "primal".
\\ 
Possible values:
\begin{itemize}
   \item primal: use primal step size
   \item bound\_mult: use step size for the bound multipliers (good
for LPs)
   \item min: use the min of primal and bound multipliers
   \item max: use the max of primal and bound multipliers
   \item full: take a full step of size one
   \item min\_dual\_infeas: choose step size minimizing new dual
infeasibility
   \item safe\_min\_dual\_infeas: like "min\_dual\_infeas", but safeguarded by
"min" and "max"
\end{itemize}

\paragraph{recalc\_y:} Tells the algorithm to recalculate the equality and inequality multipliers as least square estimates. $\;$ \\
 This asks the algorithm to recompute the
multipliers, whenever the current infeasibility
is less than recalc\_y\_feas\_tol. Choosing yes
might be helpful in the quasi-Newton option. 
However, each recalculation requires an extra
factorization of the linear system.  If a limited
memory quasi-Newton option is chosen, this is
used by default.
The default value for this string option is "no".
\\ 
Possible values:
\begin{itemize}
   \item no: use the Newton step to update the multipliers
   \item yes: use least-square multiplier estimates
\end{itemize}

\paragraph{recalc\_y\_feas\_tol:} Feasibility threshold for recomputation of multipliers. $\;$ \\
 If recalc\_y is chosen and the current
infeasibility is less than this value, then the
multipliers are recomputed. The valid range for this real option is 
$0 <  {\tt recalc\_y\_feas\_tol } <  {\tt +inf}$
and its default value is $1 \cdot 10^{-06}$.


\paragraph{max\_soc:} Maximum number of second order correction trial steps at each iteration. $\;$ \\
 Choosing 0 disables the second order corrections.
(This is p\^{max} of Step A-5.9 of Algorithm A in
implementation paper.) The valid range for this integer option is
$0 \le {\tt max\_soc } <  {\tt +inf}$
and its default value is $4$.


\paragraph{watchdog\_shortened\_iter\_trigger:} Number of shortened iterations that trigger the watchdog. $\;$ \\
 If the number of successive iterations in which
the backtracking line search did not accept the
first trial point exceeds this number, the
watchdog procedure is activated.  Choosing "0"
here disables the watchdog procedure. The valid range for this integer option is
$0 \le {\tt watchdog\_shortened\_iter\_trigger } <  {\tt +inf}$
and its default value is $10$.


\paragraph{watchdog\_trial\_iter\_max:} Maximum number of watchdog iterations. $\;$ \\
 This option determines the number of trial
iterations allowed before the watchdog procedure
is aborted and the algorithm returns to the
stored point. The valid range for this integer option is
$1 \le {\tt watchdog\_trial\_iter\_max } <  {\tt +inf}$
and its default value is $3$.


\paragraph{expect\_infeasible\_problem:} Enable heuristics to quickly detect an infeasible problem. $\;$ \\
 This options is meant to activate heuristics that
may speed up the infeasibility determination if
you expect that there is a good chance for the
problem to be infeasible.  In the filter line
search procedure, the restoration phase is called
more quickly than usually, and more reduction in
the constraint violation is enforced before the
restoration phase is left. If the problem is
square, this option is enabled automatically.
The default value for this string option is "no".
\\ 
Possible values:
\begin{itemize}
   \item no: the problem probably be feasible
   \item yes: the problem has a good chance to be infeasible
\end{itemize}

\paragraph{expect\_infeasible\_problem\_ctol:} Threshold for disabling "expect\_infeasible\_problem" option. $\;$ \\
 If the constraint violation becomes smaller than
this threshold, the "expect\_infeasible\_problem"
heuristics in the filter line search are
disabled. If the problem is square, this options
is set to 0. The valid range for this real option is 
$0 \le {\tt expect\_infeasible\_problem\_ctol } <  {\tt +inf}$
and its default value is $0.001$.


\paragraph{start\_with\_resto:} Tells algorithm to switch to restoration phase in first iteration. $\;$ \\
 Setting this option to "yes" forces the algorithm
to switch to the feasibility restoration phase in
the first iteration. If the initial point is
feasible, the algorithm will abort with a failure.
The default value for this string option is "no".
\\ 
Possible values:
\begin{itemize}
   \item no: don't force start in restoration phase
   \item yes: force start in restoration phase
\end{itemize}

\paragraph{soft\_resto\_pderror\_reduction\_factor:} Required reduction in primal-dual error in the soft restoration phase. $\;$ \\
 The soft restoration phase attempts to reduce the
primal-dual error with regular steps. If the
damped primal-dual step (damped only to satisfy
the fraction-to-the-boundary rule) is not
decreasing the primal-dual error by at least this
factor, then the regular restoration phase is
called. Choosing "0" here disables the soft
restoration phase. The valid range for this real option is 
$0 \le {\tt soft\_resto\_pderror\_reduction\_factor } <  {\tt +inf}$
and its default value is $0.9999$.


\paragraph{required\_infeasibility\_reduction:} Required reduction of infeasibility before leaving restoration phase. $\;$ \\
 The restoration phase algorithm is performed,
until a point is found that is acceptable to the
filter and the infeasibility has been reduced by
at least the fraction given by this option. The valid range for this real option is 
$0 \le {\tt required\_infeasibility\_reduction } <  1$
and its default value is $0.9$.


\paragraph{bound\_mult\_reset\_threshold:} Threshold for resetting bound multipliers after the restoration phase. $\;$ \\
 After returning from the restoration phase, the
bound multipliers are updated with a Newton step
for complementarity.  Here, the change in the
primal variables during the entire restoration
phase is taken to be the corresponding primal
Newton step. However, if after the update the
largest bound multiplier exceeds the threshold
specified by this option, the multipliers are all
reset to 1. The valid range for this real option is 
$0 \le {\tt bound\_mult\_reset\_threshold } <  {\tt +inf}$
and its default value is $1000$.


\paragraph{constr\_mult\_reset\_threshold:} Threshold for resetting equality and inequality multipliers after restoration phase. $\;$ \\
 After returning from the restoration phase, the
constraint multipliers are recomputed by a least
square estimate.  This option triggers when those
least-square estimates should be ignored. The valid range for this real option is 
$0 \le {\tt constr\_mult\_reset\_threshold } <  {\tt +inf}$
and its default value is $0$.


\paragraph{evaluate\_orig\_obj\_at\_resto\_trial:} Determines if the original objective function should be evaluated at restoration phase trial points. $\;$ \\
 Setting this option to "yes" makes the
restoration phase algorithm evaluate the
objective function of the original problem at
every trial point encountered during the
restoration phase, even if this value is not
required.  In this way, it is guaranteed that the
original objective function can be evaluated
without error at all accepted iterates; otherwise
the algorithm might fail at a point where the
restoration phase accepts an iterate that is good
for the restoration phase problem, but not the
original problem.  On the other hand, if the
evaluation of the original objective is
expensive, this might be costly.
The default value for this string option is "yes".
\\ 
Possible values:
\begin{itemize}
   \item no: skip evaluation
   \item yes: evaluate at every trial point
\end{itemize}

\paragraph{linear\_solver:} Linear solver used for step computations. $\;$ \\
 Determines which linear algebra package is to be
used for the solution of the augmented linear
system (for obtaining the search directions).
Note, the code must have been compiled with the
linear solver you want to choose. Depending on
your Ipopt installation, not all options are
available.
The default value for this string option is "ma27".
\\ 
Possible values:
\begin{itemize}
   \item ma27: use the Harwell routine MA27
   \item ma57: use the Harwell routine MA57
   \item pardiso: use the Pardiso package
   \item wsmp: use WSMP package
   \item taucs: use TAUCS package (not yet working)
   \item mumps: use MUMPS package (not yet working)
\end{itemize}

\paragraph{linear\_system\_scaling:} Method for scaling the linear system. $\;$ \\
 Determines the method used to compute symmetric
scaling factors for the augmented system (see
also the "linear\_scaling\_on\_demand" option). 
This scaling is independentof the NLP problem
scaling.  By default, MC19 is only used if MA27
or MA57 are selected as linear solvers. This
option is only available if Ipopt has been
compiled with MC19.
The default value for this string option is "mc19".
\\ 
Possible values:
\begin{itemize}
   \item none: no scaling will be performed
   \item mc19: use the Harwell routine MC19
\end{itemize}

\paragraph{linear\_scaling\_on\_demand:} Flag indicating that linear scaling is only done if it seems required. $\;$ \\
 This option is only important if a linear scaling
method (e.g., mc19) is used.  If you choose "no",
then the scaling factors are computed for every
linear system from the start.  This can be quite
expensive. Choosing "yes" means that the
algorithm will start the scaling method only when
the solutions to the linear system seem not good,
and then use it until the end.
The default value for this string option is "yes".
\\ 
Possible values:
\begin{itemize}
   \item no: Always scale the linear system.
   \item yes: Start using linear system scaling if solutions
seem not good.
\end{itemize}

\paragraph{max\_refinement\_steps:} Maximum number of iterative refinement steps per linear system solve. $\;$ \\
 Iterative refinement (on the full unsymmetric
system) is performed for each right hand side. 
This option determines the maximum number of
iterative refinement steps. The valid range for this integer option is
$0 \le {\tt max\_refinement\_steps } <  {\tt +inf}$
and its default value is $10$.


\paragraph{min\_refinement\_steps:} Minimum number of iterative refinement steps per linear system solve. $\;$ \\
 Iterative refinement (on the full unsymmetric
system) is performed for each right hand side. 
This option determines the minimum number of
iterative refinements (i.e. at least
"min\_refinement\_steps" iterative refinement
steps are enforced per right hand side.) The valid range for this integer option is
$0 \le {\tt min\_refinement\_steps } <  {\tt +inf}$
and its default value is $1$.


\paragraph{max\_hessian\_perturbation:} Maximum value of regularization parameter for handling negative curvature. $\;$ \\
 In order to guarantee that the search directions
are indeed proper descent directions, Ipopt
requires that the inertia of the (augmented)
linear system for the step computation has the
correct number of negative and positive
eigenvalues. The idea is that this guides the
algorithm away from maximizers and makes Ipopt
more likely converge to first order optimal
points that are minimizers. If the inertia is not
correct, a multiple of the identity matrix is
added to the Hessian of the Lagrangian in the
augmented system. This parameter gives the
maximum value of the regularization parameter. If
a regularization of that size is not enough, the
algorithm skips this iteration and goes to the
restoration phase. (This is delta\_w\^max in the
implementation paper.) The valid range for this real option is 
$0 <  {\tt max\_hessian\_perturbation } <  {\tt +inf}$
and its default value is $1 \cdot 10^{+20}$.


\paragraph{min\_hessian\_perturbation:} Smallest perturbation of the Hessian block. $\;$ \\
 The size of the perturbation of the Hessian block
is never selected smaller than this value, unless
no perturbation is necessary. (This is
delta\_w\^min in implementation paper.) The valid range for this real option is 
$0 \le {\tt min\_hessian\_perturbation } <  {\tt +inf}$
and its default value is $1 \cdot 10^{-20}$.


\paragraph{first\_hessian\_perturbation:} Size of first x-s perturbation tried. $\;$ \\
 The first value tried for the x-s perturbation in
the inertia correction scheme.(This is delta\_0
in the implementation paper.) The valid range for this real option is 
$0 <  {\tt first\_hessian\_perturbation } <  {\tt +inf}$
and its default value is $0.0001$.


\paragraph{perturb\_inc\_fact\_first:} Increase factor for x-s perturbation for very first perturbation. $\;$ \\
 The factor by which the perturbation is increased
when a trial value was not sufficient - this
value is used for the computation of the very
first perturbation and allows a different value
for for the first perturbation than that used for
the remaining perturbations. (This is
bar\_kappa\_w\^+ in the implementation paper.) The valid range for this real option is 
$1 <  {\tt perturb\_inc\_fact\_first } <  {\tt +inf}$
and its default value is $100$.


\paragraph{perturb\_inc\_fact:} Increase factor for x-s perturbation. $\;$ \\
 The factor by which the perturbation is increased
when a trial value was not sufficient - this
value is used for the computation of all
perturbations except for the first. (This is
kappa\_w\^+ in the implementation paper.) The valid range for this real option is 
$1 <  {\tt perturb\_inc\_fact } <  {\tt +inf}$
and its default value is $8$.


\paragraph{perturb\_dec\_fact:} Decrease factor for x-s perturbation. $\;$ \\
 The factor by which the perturbation is decreased
when a trial value is deduced from the size of
the most recent successful perturbation. (This is
kappa\_w\^- in the implementation paper.) The valid range for this real option is 
$0 <  {\tt perturb\_dec\_fact } <  1$
and its default value is $0.333333$.


\paragraph{jacobian\_regularization\_value:} Size of the regularization for rank-deficient constraint Jacobians. $\;$ \\
 (This is bar delta\_c in the implementation
paper.) The valid range for this real option is 
$0 \le {\tt jacobian\_regularization\_value } <  {\tt +inf}$
and its default value is $1 \cdot 10^{-08}$.


\paragraph{hessian\_approximation:} Indicates what Hessian information is to be used. $\;$ \\
 This determines which kind of information for the
Hessian of the Lagrangian function is used by the
algorithm.
The default value for this string option is "exact".
\\ 
Possible values:
\begin{itemize}
   \item exact: Use second derivatives provided by the NLP.
   \item limited-memory: Perform a limited-memory quasi-Newton 
approximation
\end{itemize}

\paragraph{limited\_memory\_max\_history:} Maximum size of the history for the limited quasi-Newton Hessian approximation. $\;$ \\
 This option determines the number of most recent
iterations that are taken into account for the
limited-memory quasi-Newton approximation. The valid range for this integer option is
$0 \le {\tt limited\_memory\_max\_history } <  {\tt +inf}$
and its default value is $6$.


\paragraph{limited\_memory\_max\_skipping:} Threshold for successive iterations where update is skipped. $\;$ \\
 If the update is skipped more than this number of
successive iterations, we quasi-Newton
approximation is reset. The valid range for this integer option is
$1 \le {\tt limited\_memory\_max\_skipping } <  {\tt +inf}$
and its default value is $2$.


\paragraph{derivative\_test:} Enable derivative checker $\;$ \\
 If this option is enabled, a (slow) derivative
test will be performed before the optimization. 
The test is performed at the user provided
starting point and marks derivative values that
seem suspicious
The default value for this string option is "none".
\\ 
Possible values:
\begin{itemize}
   \item none: do not perform derivative test
   \item first-order: perform test of first derivatives at starting
point
   \item second-order: perform test of first and second derivatives at
starting point
\end{itemize}

\paragraph{derivative\_test\_perturbation:} Size of the finite difference perturbation in derivative test. $\;$ \\
 This determines the relative perturbation of the
variable entries. The valid range for this real option is 
$0 <  {\tt derivative\_test\_perturbation } <  {\tt +inf}$
and its default value is $1 \cdot 10^{-08}$.


\paragraph{derivative\_test\_tol:} Threshold for indicating wrong derivative. $\;$ \\
 If the relative deviation of the estimated
derivative from the given one is larger than this
value, the corresponding derivative is marked as
wrong. The valid range for this real option is 
$0 <  {\tt derivative\_test\_tol } <  {\tt +inf}$
and its default value is $0.0001$.


\paragraph{derivative\_test\_print\_all:} Indicates whether information for all estimated derivatives should be printed. $\;$ \\
 Determines verbosity of derivative checker.
The default value for this string option is "no".
\\ 
Possible values:
\begin{itemize}
   \item no: Print only suspect derivatives
   \item yes: Print all derivatives
\end{itemize}

\paragraph{ma27\_pivtol:} Pivot tolerance for the linear solver MA27. $\;$ \\
 A smaller number pivots for sparsity, a larger
number pivots for stability.  This option is only
available if Ipopt has been compiled with MA27. The valid range for this real option is 
$0 <  {\tt ma27\_pivtol } <  1$
and its default value is $1 \cdot 10^{-08}$.


\paragraph{ma27\_pivtolmax:} Maximum pivot tolerance for the linear solver MA27. $\;$ \\
 Ipopt may increase pivtol as high as pivtolmax to
get a more accurate solution to the linear
system.  This option is only available if Ipopt
has been compiled with MA27. The valid range for this real option is 
$0 <  {\tt ma27\_pivtolmax } <  1$
and its default value is $0.0001$.


\paragraph{ma27\_liw\_init\_factor:} Integer workspace memory for MA27. $\;$ \\
 The initial integer workspace memory =
liw\_init\_factor * memory required by unfactored
system. Ipopt will increase the workspace size by
meminc\_factor if required.  This option is only
available if Ipopt has been compiled with MA27. The valid range for this real option is 
$1 \le {\tt ma27\_liw\_init\_factor } <  {\tt +inf}$
and its default value is $5$.


\paragraph{ma27\_la\_init\_factor:} Real workspace memory for MA27. $\;$ \\
 The initial real workspace memory =
la\_init\_factor * memory required by unfactored
system. Ipopt will increase the workspace size by
meminc\_factor if required.  This option is only
available if  Ipopt has been compiled with MA27. The valid range for this real option is 
$1 \le {\tt ma27\_la\_init\_factor } <  {\tt +inf}$
and its default value is $5$.


\paragraph{ma27\_meminc\_factor:} Increment factor for workspace size for MA27. $\;$ \\
 If the integer or real workspace is not large
enough, Ipopt will increase its size by this
factor.  This option is only available if Ipopt
has been compiled with MA27. The valid range for this real option is 
$1 \le {\tt ma27\_meminc\_factor } <  {\tt +inf}$
and its default value is $10$.


\paragraph{ma57\_pivtol:} Pivot tolerance for the linear solver MA57. $\;$ \\
 A smaller number pivots for sparsity, a larger
number pivots for stability. This option is only
available if Ipopt has been compiled with MA57. The valid range for this real option is 
$0 <  {\tt ma57\_pivtol } <  1$
and its default value is $1 \cdot 10^{-08}$.


\paragraph{ma57\_pivtolmax:} Maximum pivot tolerance for the linear solver MA57. $\;$ \\
 Ipopt may increase pivtol as high as
ma57\_pivtolmax to get a more accurate solution
to the linear system.  This option is only
available if Ipopt has been compiled with MA57. The valid range for this real option is 
$0 <  {\tt ma57\_pivtolmax } <  1$
and its default value is $0.0001$.


\paragraph{ma57\_pre\_alloc:} Safety factor for work space memory allocation for the linear solver MA57. $\;$ \\
 If 1 is chosen, the suggested amount of work
space is used.  However, choosing a larger number
might avoid reallocation if the suggest values do
not suffice.  This option is only available if
Ipopt has been compiled with MA57. The valid range for this real option is 
$1 \le {\tt ma57\_pre\_alloc } <  {\tt +inf}$
and its default value is $3$.


\paragraph{pardiso\_matching\_strategy:} Matching strategy to be used by Pardiso $\;$ \\
 This is IPAR(13) in Pardiso manual.  This option
is only available if Ipopt has been compiled with
Pardiso.
The default value for this string option is "complete+2x2".
\\ 
Possible values:
\begin{itemize}
   \item complete: Match complete (IPAR(13)=1)
   \item complete+2x2: Match complete+2x2 (IPAR(13)=2)
   \item constraints: Match constraints (IPAR(13)=3)
\end{itemize}

\paragraph{pardiso\_out\_of\_core\_power:} Enables out-of-core variant of Pardiso $\;$ \\
 Setting this option to a positive integer k makes
Pardiso work in the out-of-core variant where the
factor is split in 2\^k subdomains.  This is
IPARM(50) in the Pardiso manual.  This option is
only available if Ipopt has been compiled with
Pardiso. The valid range for this integer option is
$0 \le {\tt pardiso\_out\_of\_core\_power } <  {\tt +inf}$
and its default value is $0$.


\paragraph{wsmp\_num\_threads:} Number of threads to be used in WSMP $\;$ \\
 This determines on how many processors WSMP is
running on.  This option is only available if
Ipopt has been compiled with WSMP. The valid range for this integer option is
$1 \le {\tt wsmp\_num\_threads } <  {\tt +inf}$
and its default value is $1$.


\paragraph{wsmp\_ordering\_option:} Determines how ordering is done in WSMP $\;$ \\
 This corresponds to the value of WSSMP's
IPARM(16).  This option is only available if
Ipopt has been compiled with WSMP. The valid range for this integer option is
$-2 \le {\tt wsmp\_ordering\_option } \le 3$
and its default value is $1$.


\paragraph{wsmp\_pivtol:} Pivot tolerance for the linear solver WSMP. $\;$ \\
 A smaller number pivots for sparsity, a larger
number pivots for stability.  This option is only
available if Ipopt has been compiled with WSMP. The valid range for this real option is 
$0 <  {\tt wsmp\_pivtol } <  1$
and its default value is $0.0001$.


\paragraph{wsmp\_pivtolmax:} Maximum pivot tolerance for the linear solver WSMP. $\;$ \\
 Ipopt may increase pivtol as high as pivtolmax to
get a more accurate solution to the linear
system.  This option is only available if Ipopt
has been compiled with WSMP. The valid range for this real option is 
$0 <  {\tt wsmp\_pivtolmax } <  1$
and its default value is $0.1$.


\paragraph{wsmp\_scaling:} Determines how the matrix is scaled by WSMP. $\;$ \\
 This corresponds to the value of WSSMP's
IPARM(10). This option is only available if Ipopt
has been compiled with WSMP. The valid range for this integer option is
$0 \le {\tt wsmp\_scaling } \le 3$
and its default value is $0$.


\newpage
\section{Detailed Installation Information}\label{ExpertInstall}

The configuration script and Makefiles in the \Ipopt\ distribution
have been created using GNU's {\tt autoconf} and {\tt automake}.  They
attempt to automatically adapt the compiler settings etc.\ to the
system they are running on.  We tested the provided scripts for a
number of different machines, operating systems and compilers, but you
might run into a situation where the default setting does not work, or
where you need to change the settings to fit your particular
environment.

In general, you can see the list of options and variables that can be
set for the {\tt configure} script by typing \verb/configure --help/.
Below a few particular options are discussed:

\begin{itemize}
\item The {\tt configure} script tries to determine automatically, if
  you have BLAS and/or LAPACK already installed on your system (trying
  a few default libraries), and if it does not find them, it makes
  sure that you put the source code in the required place.

  However, you can specify a BLAS library (such as your local ATLAS
  library\footnote{see {\tt http://math-atlas.sourceforge.net/}})
  explicitly, using the \verb/--with-blas/ flag for {\tt configure}.
  For example,

  \verb|./configure --with=blas="-L$HOME/lib -latlas"|

  To tell the configure script to compile and use the downloaded BLAS
  source files even if a BLAS library is found on your system, specify
  \verb|--with-blas=BUILD|.

  Similarly, you can use the \verb/--with-lapack/ switch to specify
  the location of your LAPACK library, or use the keyword {\tt BUILD}
  to force the \Ipopt\ makefiles to compile LAPACK together with
  \Ipopt.

\item Similarly, if you have a precompiled library containing the
  Harwell Subroutines, you can specify its location with the
  \verb|--with-hsl| flag.  And the location of the AMPL solver library
  (with the ASL header files) can be specified with
  \verb|--with-asldir|.
  {\bf TODO Other linear solvers}

\item If you want to specify that you want to use particular
  compilers, you can do so by adding the variables definitions for
  {\tt CXX}, {\tt CC}, and {\tt F77} to the {\tt ./configure} command
  line, to specify the C++, C, and Fortran compiler, respectively.
  For example,

  {\tt ./configure CXX=g++ CC=gcc F77=g77}

  In order to set the compiler flags, you should use the variables
  {\tt CXXFLAGS}, {\tt CFLAGS}, {\tt FFLAGS}.  Note, that the \Ipopt\
  code uses ``{\tt dynamic\_cast}''.  Therefore it is necessary that
  the C++ code is compiled including RTTI (Run-Time Type Information).
  Some compilers need to be given special flags to do that (e.g.,
  ``{\tt -qrtti=dyna}'' for the AIX {\tt xlC} compiler).

\item If you want to link the \Ipopt\ library with a main program
  written in C or Fortran, the C and Fortran compiler doing the
  linking of the executable needs to be told about the C++ runtime
  libraries.  Unfortunately, the current version of {\tt autoconf}
  does not provide the automatic detection of those libraries.  We
  have hard-coded some default values for some systems and compilers,
  but this might not work all the time.

  If you have problems linking your Fortran or C code with the \Ipopt\
  library {\tt libipopt.a} and the linker complains about missing
  symbols from C++ (e.g., the standard template library), you should
  specify the C++ libraries with the {\tt CXXLIBS} variable.  To find out
  what those libraries are, it is probably helpful to link a  simple C++
  program with verbose compiler output.

  For example, for the Intel compilers on a Linux system, you
  might need to specify something like

  {\tt ./configure CC=icc F77=ifort CXX=icpc $\backslash$\\ \hspace*{14ex} CXXLIBS='-L/usr/lib/gcc-lib/i386-redhat-linux/3.2.3 -lstdc++'}

\item Compilation in 64bit mode sometimes requires some special
  consideration.  For example, for compilation of 64bit code on AIX,
  we recommend the following configuration

  {\tt ./configure AR='ar -X64' AR\_X='ar -X64 x' $\backslash$\\
    \hspace*{14ex} CC='xlc -q64' F77='xlf -q64' CXX='xlC
    -q64'$\backslash$\\ \hspace*{14ex} CFLAGS='-O3
    -bmaxdata:0x3f0000000'
    $\backslash$\\ \hspace*{14ex} FFLAGS='-O3 -bmaxdata:0x3f0000000' $\backslash$\\
    \hspace*{14ex} CXXFLAGS='-qrtti=dyna -O3 -bmaxdata:0x3f0000000'}

\item To build library/archive files (with the ending {\tt .a})
  including C++ code in some environments, it is necessary to use the
  C++ compiler instead of {\tt ar} to build the archive.  This is for
  example the case for some older compilers on SGI and SUN.  For this,
  the {\tt configure} variables {\tt AR}, {\tt ARFLAGS}, and {\tt
    AR\_X} are provided.  Here, {\tt AR} specifies the command for the
  archiver for creating an archive, and {\tt ARFLAGS} specifies
  additional flags.  {\tt AR\_X} contains the command for extracting
  all files from an archive.  For example, the default setting for SUN
  compilers for our configure script is

  {\tt AR='CC -xar' ARFLAGS='-o' AR\_X='ar x'}

\item It is possible to compile the \Ipopt\ library in a debug
  configuration, by specifying \verb|--enable-debug|.  Then the
  compilers will use the debug flags (unless the compilation flag
  variables are overwritten in the {\tt configure} command line), and
  additional debug checks are compiled into the code (see {\tt
    IpDebug.hpp}).  This usually leads to a significant slowdown of
  the code, but might be helpful when debugging something.

\item It is not necessary to produce the binary files in the
  directories where the source files are.  If you want to compile the
  code on different systems or with different compilers/options on a
  shared file system, you can keep one single copy of the source files
  in one directory, and the binary files for each configuration in
  separate directories.  For this, simply run the configure script in
  the directory where you want the base directory for the \Ipopt\
  binary files.  For example:

  {\tt \$ mkdir \$HOME/Ipopt-objects}\\
  {\tt \$ cd \$HOME/Ipopt-objects}\\
  {\tt \$ \$HOME/Ipopt/configure}  (or {\tt \$HOME/ipopt-3.1.0/configure})

\end{itemize}

%\bibliographystyle{plain}
%\bibliography{/home/andreasw/tex/andreas}
%% Copyright (C) 2005, 2006 Carnegie Mellon University and others.
%%
%% The first version of this file was contributed to the Ipopt project
%% on Aug 1, 2005, by Yoshiaki Kawajiri
%%                    Department of Chemical Engineering
%%                    Carnegie Mellon University
%%                    Pittsburgh, PA 15213
%%
%% Since then, the content of this file has been updated significantly by
%%     Carl Laird and Andreas Waechter        IBM
%%
%%
%% $Id$
%%
\documentclass[10pt]{article}
\setlength{\textwidth}{6.3in}       % Text width
\setlength{\textheight}{9.4in}      % Text height
\setlength{\oddsidemargin}{0.1in}     % Left margin for even-numbered pages
\setlength{\evensidemargin}{0.1in}    % Left margin for odd-numbered pages
\setlength{\topmargin}{-0.5in}         % Top margin
\renewcommand{\baselinestretch}{1.1}
\usepackage{amsfonts}
\usepackage{amsmath}

\newcommand{\RR}{{\mathbb{R}}}
\newcommand{\Ipopt}{{\sc Ipopt}}


\begin{document}
\title{Introduction to \Ipopt:\\
A tutorial for downloading, installing, and using \Ipopt.}

\author{Revision number of this document: $Revision$}

%\date{\today}
\maketitle

\begin{abstract}
  This document is a guide to using \Ipopt\ 3.1 (the new C++ version
  of \Ipopt).  It includes instructions on how to obtain and compile
  \Ipopt, a description of the interface, user options, etc.,, as
  well as a tutorial on how to solve a nonlinear optimization problem
  with \Ipopt.

  The initial version of this document was created by
  Yoshiaki Kawajir\footnote{Department of Chemical Engineering,
    Carnegie Mellon University, Pittsburgh PA} as a course project for
  \textit{47852 Open Source Software for Optimization}, taught by
  Prof. Fran\c cois Margot at Tepper School of Business, Carnegie
  Mellon University.  The current version is maintained by Carl
  Laird\footnote{Department of Chemical Engineering, Carnegie Mellon
    University, Pittsburgh PA} and Andreas
  W\"achter\footnote{Department of Mathematical Sciences, IBM T.J.\
    Watson Research Center, Yorktown Heights, NY}.
\end{abstract}

\tableofcontents

\vspace{\baselineskip}
\begin{small}
\noindent
The following names used in this document are trademarks or registered
trademarks: AMPL, IBM, Intel, Microsoft, Visual Studio C++, Visual
Studio C++ .NET
\end{small}

\section{Introduction}
\Ipopt\ (\underline{I}nterior \underline{P}oint \underline{Opt}imizer,
pronounced ``I--P--Opt'') is an open source software package for
large-scale nonlinear optimization. It can be used to solve general
nonlinear programming problems of the form
%\begin{subequations}\label{NLP}
\begin{eqnarray}
\min_{x\in\RR^n} &&f(x) \label{eq:obj} \\
\mbox{s.t.} \;  &&g^L \leq g(x) \leq g^U \\
                &&x^L \leq x \leq x^U, \label{eq:bounds}
\end{eqnarray}
%\end{subequations}
where $x \in \RR^n$ are the optimization variables (possibly with
lower and upper bounds, $x^L\in(\RR\cup\{-\infty\})^n$ and
$x^U\in(\RR\cup\{+\infty\})^n$), $f:\RR^n\longrightarrow\RR$ is the
objective function, and $g:\RR^n\longrightarrow \RR^m$ are the general
nonlinear constraints.  The functions $f(x)$ and $g(x)$ can be linear
or nonlinear and convex or non-convex (but should be twice
continuously differentiable). The constraints, $g(x)$, have lower and
upper bounds, $g^L\in(\RR\cup\{-\infty\})^n$ and
$g^U\in(\RR\cup\{+\infty\})^m$. Note that equality constraints of the
form $g_i(x)=\bar g_i$ can be specified by setting
$g^L_{i}=g^U_{i}=\bar g_i$.

\subsection{Mathematical Background}
\Ipopt\ implements an interior point line search filter method that
aims to find a local solution of (\ref{eq:obj})-(\ref{eq:bounds}).  The
mathematical details of the algorithm can be found in several
publications
\cite{NocWaeWal:adaptive,WaechterPhD,WaecBieg06:mp,WaeBie05:filterglobal,WaeBie05:filterlocal}.

\subsection{Availability}
The \Ipopt\ package is available from COIN-OR
(\texttt{www.coin-or.org}) under the CPL (Common Public License)
open-source license and includes the source code for \Ipopt.  This
means, it is available free of charge, also for commercial purposes.
However, if you give away software including \Ipopt\ code (in source
code or binary form) and you made changes to the \Ipopt\ source code,
you are required to make those changes public and to clearly indicate
which modifications you made.  After all, the goal of open source
software is the continuous development and improvement of software.
For details, please refer to the Common Public License.

Also, if you are using \Ipopt\ to obtain results for a publication, we
politely ask you to point out in your paper that you used \Ipopt, and
to cite the publication \cite{WaecBieg06:mp}.  Writing high-quality
numerical software takes a lot of time and effort, and does usually
not translate into a large number of publications, therefore we believe
this request is only fair :).

\subsection{Prerequisites}
In order to build \Ipopt, some third party components are required:
\begin{itemize}
\item BLAS (Basic Linear Algebra Subroutines).  Many vendors of
  compilers and operating systems provide precompiled and optimized
  libraries for these dense linear algebra subroutines.  But you can
  also get the source code from {\tt www.netlib.org} and have the
  \Ipopt\ distribution compile it automatically.
\item LAPACK (Linear Algebra PACKage).  Also for LAPACK, some vendors
  offer precompiled and optimized libraries.  But like with BLAS, you
  can get the source code from {\tt www.netlib.org} and have the
  \Ipopt\ distribution compile it automatically.

  Note that currently LAPACK is only required if you intend to use the
  quasi-Newton options in \Ipopt.  You can compile the code without
  LAPACK, but an error message will then occur if you try to run the
  code with an option that requires LAPACK.  Currently, the LAPACK
  routines that are used by \Ipopt\ are only {\tt DPOTRF}, {\tt
    DPOTRS}, and {\tt DSYEV}.
\item A sparse symmetric indefinite linear solver. The \Ipopt\ needs
  to obtain the solution of sparse, symmetric, indefinite linear
  systems, and for this it relies on third-party code.  

  Currently, the following linear solvers can be used:
  \begin{itemize}
  \item MA27 from the Harwell Subroutine Library\\ (see {\tt
      http://www.cse.clrc.ac.uk/nag/hsl/}).
  \item MA57 from the Harwell Subroutine Library\\ (see {\tt
      http://www.cse.clrc.ac.uk/nag/hsl/}).
  \item The Watson Sparse Matrix Package (WSMP)\\ (see {\tt
      http://www-users.cs.umn.edu/\~agupta/wsmp.html})
  \item The Parallel Sparse Direct Linear Solver (PARDISO)\\ (see {\tt
      http://www.computational.unibas.ch/cs/scicomp/software/pardiso/}).
  \end{itemize}
  You need to include at least one of the linear solvers above in
  order to run \Ipopt.

  Interfaces to other linear solvers might be added in the future; if
  you are interested in contributing such an interface please contact
  us!  Note that \Ipopt\ requires that the linear solver is able to
  provide the inertia (number of positive and negative eigenvalues) of
  the symmetric matrix that is factorized.

\item Furthermore, \Ipopt\ can also use the Harwell Subroutine MC19
  for scaling of the linear systems before they are passed to the
  linear solver.  This may be particularly useful if \Ipopt\ is used
  with MA27 or MA57.  However, it is not required to have MC19 to
  compile \Ipopt; if this routine is missing, the scaling is never
  performed.
\item ASL (AMPL Solver Library).  The source code is available at {\tt
    www.netlib.org}, and the \Ipopt\ makefiles will automatically
  compile it for you if you put the source code into a designated
  space.  NOTE: This is only required if you want to use \Ipopt\ from
  AMPL and want to compile the \Ipopt\ AMPL solver executable.
\end{itemize}
For more information on third-party components and how to obtain them,
see Section~\ref{ExternalCode}.

Since the \Ipopt\ code is written in C++, you will need a C++ compiler
to build the \Ipopt\ library.  We tried very hard to write the code as
platform and compiler independent as possible.

In addition, the configuration script currently also searches for a
Fortran, since some of the dependencies above are written in Fortran.
If all third party dependencies are available as self-contained
libraries, those compilers are in principle not necessary.  Also, it
is possible to use the Fortran-to-C compiler {\tt f2c} from {\tt
  www.netlib.org} to convert Fortran code to C, and compile the
resulting C files with a C compiler and create a library containing
the required third party dependencies.  But so far we have not tested
this ourselves, and currently the configuration script for \Ipopt\
looks for a Fortran compiler.

\subsection{How to use \Ipopt}
If desired, the \Ipopt\ distribution generates an executable for the
modeling environment AMPL. As well, you can link your problem
statement with \Ipopt\ using interfaces for C++, C, or Fortran.
\Ipopt\ can be used with most Linux/Unix environments, and on Windows
using Visual Studio .NET or Cygwin.  Below in
Section~\ref{sec:tutorial-example} this document demonstrates how to
solve problems using \Ipopt. This includes installation and
compilation of \Ipopt\ for use with AMPL as well as linking with your
own code.

Finally, the \Ipopt\ distribution includes an interface for {\tt
  CUTEr}\footnote{see {\tt http://cuter.rl.ac.uk/cuter-www/}}, if you
want to use \Ipopt\ to solve problems modeled in SIF.

The old (Fortran 2.x) version of \Ipopt\ has been interface with
Matlab, and is also available from NEOS, and the new version will be
available through similar means in the future.  Please check the
\Ipopt\ homepage for updates.

\subsection{More Information and Contributions}
More and up-to-date information can be found at the \Ipopt\ homepage,

\begin{center}
\texttt{http://projects.coin-or.org/Ipopt}.
\end{center}

Here, you can find FAQs, some (hopefully useful) hints, a bug report
system etc.  The website is managed with Wiki, which means that every
user can edit the webpages from the regular web browser.  In
particular, we encourage \Ipopt\ users to share their experiences and
usage hints on the ``Success Stories'' and ``Hints and Tricks''
pages\footnote{Since we had some malicious hacker attacks destroying
  the content of the web pages in the past, you are now required to
  enter a user name and password; simply follow the instructions in
  the last paragraph of the Documentation section on the main project
  page.}

\Ipopt\ is an open source project, and we encourage people to
contribute code (such as interfaces to appropriate linear solvers,
modeling environments, or even algorithmic features).  If you are
interested in contributing code, please have a look at the COIN
constributions webpage\footnote{see \tt
  http://www.coin-or.org/contributions.html}, and contact the \Ipopt\
project leader.

There is also a mailing list for \Ipopt, available from the webpage
\begin{center}
\texttt{http://list.coin-or.org/mailman/listinfo/coin-ipopt},
\end{center}
where you can
subscribe to get notified of updates, and to ask general questions
regarding installation and usage. (You might want to look at the
archives before posting a question.)

We try to answer questions posted to the mailing list in a reasonable
manner.  Please understand that we cannot answer all questions in
detail, and because of time constraints, we may not be able to help
you model and debug your particular optimization problem.  However, if
you have a challenging optimization problem and are interested in
consulting services by IBM Research, please contact the \Ipopt\
project leader, Andreas W\"achter.

\subsection{History of \Ipopt}
The original \Ipopt\ (Fortran version) was a product of the dissertation
research of Andreas W\"achter \cite{WaechterPhD}, under Lorenz
T. Biegler at the Chemical Engineering Department at Carnegie Mellon
University. The code was made open source and distributed by the
COIN-OR initiative, which is now a non-profit corporation.  \Ipopt\ has
been actively developed under COIN-OR since 2002.

To continue natural extension of the code and allow easy addition of
new features, IBM Research decided to invest in an open source
re-write of \Ipopt\ in C++.  The new C++ version of the \Ipopt\
optimization code (\Ipopt\ 3.0 and beyond) is currently developed at IBM
Research and remains part of the COIN-OR initiative. Future
development on the Fortran version will cease with the exception of
occasional bug fix releases.

\section{Installing \Ipopt}\label{Installing}

The following sections describe the installation procedures on
UNIX/Linux systems.  For installation instructions on Windows
see Section~\ref{WindowsInstall}.

\subsection{Getting the \Ipopt\ Code}
\Ipopt\ is available from the COIN-OR subversion repository. You can
either download the code using \texttt{svn} (the
\textit{subversion}\footnote{see
  \texttt{http://subversion.tigris.org/}} client similar to CVS) or
simply retrieve a tarball (compressed archive file).  While the
tarball is an easy method to retrieve the code, using the
\textit{subversion} system allows users the benefits of the version
control system, including easy updates and revision control.

\subsubsection{Getting the \Ipopt\ code via subversion}

Of course, the \textit{subversion} client must be installed on your
system if you want to obtain the code this way (the executable is
called \texttt{svn}); it is already installed by default for many
recent Linux distributions.  Information about \textit{subversion} and
how to download it can be found at
\texttt{http://subversion.tigris.org/}.\\

To obtain the \Ipopt\ source code via subversion, change into the
directory in which you want to create a subdirectory {\tt Ipopt} with
the \Ipopt\ source code.  Then follow the steps below:
\begin{enumerate}
\item{Download the code from the repository}\\
{\tt \$ svn co https://www.coin-or.org/svn/Ipopt/trunk Ipopt} \\
Note: The {\tt \$} indicates the command line
prompt, do not type {\tt \$}, only the text following it.
\item Change into the root directory of the \Ipopt\ distribution\\
{\tt \$ cd Ipopt}
\end{enumerate}

In the following, ``\texttt{\$IPOPTDIR}'' will refer to the directory in
which you are right now (output of \texttt{pwd}).

\subsubsection{Getting the \Ipopt\ code as a tarball}

To use the tarball, follow the steps below:
\begin{enumerate}
\item Download the latest tarball from
\texttt{http://www.coin-or.org/Tarballs}.  The file you should look
for has the form \texttt{ipopt-3.x.x.tar.gz} (where
``\texttt{3.x.x.}'' is the version number).  Put this file in a
directory under which you want to put the \Ipopt\ installation.
\item Issue the following commands to unpack the archive file: \\
\texttt{\$ gunzip ipopt-3.x.x.tar.gz} \\
\texttt{\$ tar xvf ipopt-3.x.x.tar} \\
Note: The {\tt \$} indicates the command line
prompt, do not type {\tt \$}, only the text following it.
\item Change into the root directory of the \Ipopt\ distribution\\
{\tt \$ cd ipopt-3.x.x}
\end{enumerate}

In the following, ``\texttt{\$IPOPTDIR}'' will refer to the directory in
which you are right now (output of \texttt{pwd}).

\subsection{Download External Code}\label{ExternalCode}
\Ipopt\ uses a few external packages that are not included in the
\Ipopt\ source code distribution, namely ASL (the AMPL Solver
Library), BLAS, LAPACK.  It also requires a sparse symmetric linear
solver.

Since this third party software released under different licenses than
\Ipopt, we cannot distribute that code together with the \Ipopt\
packages and have to ask you to go through the hassle of obtaining it
yourself (even though we tried to make it as easy for you as we
could).  Keep in mind that it is still your responsibility to ensure
that your downloading and usage if the third party components conforms
with their licenses.

Note that you only need to obtain the ASL if you intend to use \Ipopt\
from AMPL.  It is not required if you want to specify your
optimization problem in a programming language (C++, C, or Fortran).
Also, currently, LAPACK is only required if you intend to use the
quasi-Newton options implemented in \Ipopt.

\subsubsection{Download BLAS, LAPACK and ASL}
If you have the download utility \texttt{wget} installed on your
system, retrieving BLAS, LAPACK, and ASL is straightforward using
scripts included with the ipopt distribution. These scripts download
the required files from the Netlib Repository
(\texttt{www.netlib.org}).\\

\noindent
{\tt \$ cd \$IPOPTDIR/Extern/blas}\\
{\tt \$ ./get.blas}\\
{\tt \$ cd ../lapack}\\
{\tt \$ ./get.lapack}\\
{\tt \$ cd ../ASL}\\
{\tt \$ ./get.ASL}\\

\noindent
If you do not have \texttt{wget} installed on your system, please read
the \texttt{INSTALL.*} files in the \texttt{\$IPOPTDIR/Extern/blas},
\texttt{\$IPOPTDIR/Extern/lapack} and \texttt{\$IPOPTDIR/Extern/ASL}
directories for alternative instructions.

\subsubsection{Download HSL Subroutines}
\Ipopt\ requires a sparse symmetric linear solver.  There are
different possibilities.  In this section we describe how to obtain
the source code for MA27 (and MC19) from the Harwell Subroutine
Library (HSL).  Those routines are freely available for
non-commercial, academic use, but it is your responsibility to
investigate the licensing of all third party code.

The use of alternative linear solvers is described in
Appendix~\ref{ExpertInstall}.  You do not necessarily have to use MA27
as described in this section, but at least one linear solver is
required for \Ipopt\ to function.

\begin{enumerate}
\item Go to {\tt http://hsl.rl.ac.uk/archive/hslarchive.html}
\item Follow the instruction on the website, read the license, and
  submit the registration form.
\item Go to \textit{HSL Archive Programs}, and find the package list.
\item In your browser window, click on \textit{MA27}.
\item Make sure that \textit{Double precision:} is checked. 
  Click \textit{Download package (comments removed)}
\item Save the file as {\tt ma27ad.f} in {\tt \$IPOPTDIR/Extern/HSL/}\\
  Note: Some browsers append a file extension ({\tt .txt}) when you save
  the file, in which case you have to rename it.
\item Go back to the package list using the back button of your browser.
\item In your browser window, click on \textit{MC19}.
\item Make sure \textit{Double precision:} is checked. Click 
  \textit{Download package (comments removed)}
\item Save the file as {\tt mc19ad.f} in {\tt
    \$IPOPTDIR/Extern/HSL/}\\
  Note: Some browsers append a file extension ({\tt .txt}) when you save
  the file, so you may have to rename it.
\end{enumerate}

Note: Whereas currently obtaining MA27 is essential for using \Ipopt,
MC19 could be omitted (with the consequence that you cannot use this
method for scaling the linear systems arising inside the \Ipopt\
algorithm).

Note: If you have the source code for the linear solver MA57
(successor of MA27) in a file called ma57ad.f (including all
dependencies), you can simply put it into the {\tt
  \$IPOPTDIR/Extern/HSL/} directory.  The \Ipopt\ configuration script
will then find this file and compile it into the \Ipopt\ library (just
as is would compile MA27).

\subsection{Compiling and Installing \Ipopt} \label{sec.comp_and_inst}

\Ipopt\ can be easily compiled and installed with the usual {\tt
  configure}, {\tt make}, {\tt make install} commands.  Below are the
basic steps that should work on most systems.  For special
compilations and for some troubleshooting see
Appendix~\ref{ExpertInstall} and consult the \Ipopt\ homepage before
submitting a ticket or sending a message to the mailing list.
\begin{enumerate}
\item Go to the main directory of \Ipopt:\\
  {\tt \$ cd \$IPOPTDIR} 
\item Run the configure script\\
  {\tt \$ ./configure}

  If the last output line of the script reads ``\texttt{configure:
    Configuration successful}'' then everything worked fine.
  Otherwise, look at the screen output, have a look at the
  \texttt{config.log} output file and/or consult
  Appendix~\ref{ExpertInstall}.

  The default configure (without any options) is sufficient for most
  users. If you want to see the configure options, consult
  Appendix~\ref{ExpertInstall}.
\item Build the code \\
{\tt \$ make}
\item Install \Ipopt \\
  {\tt \$ make install}\\
  This installs
  \begin{itemize}
  \item the \Ipopt\ AMPL solver executable (if ASL source was
    downloaded) in \texttt{\$IPOPTDIR/bin},
  \item the \Ipopt\ library (\texttt{libipopt.a}) in
    \texttt{\$IPOPTDIR/lib},
  \item text files {\tt ipopt\_addlibs\_cpp.txt} and {\tt
      ipopt\_addlibs\_f.txt} in \texttt{\$IPOPTDIR/lib} that contain a
    line each with additional linking flags that are required for
    linking code with the ipopt library, for C++ and Fortran main
    programs, respectively. (This is only for convenience if you want
    to find out what additional flags are required, for example, to
    include the Fortran runtime libraries with a C++ compiler.)
  \item the necessary header files in
    \texttt{\$IPOPTDIR/include/ipopt}.
  \end{itemize}
  You can change the default installation directory (here
  \texttt{\$IPOPTDIR}) to something else (such as \texttt{/usr/local})
  by using the \verb|--prefix| switch for \texttt{configure}.
%\item Test the installation \\
%  {\tt \$ make test}\\
%  This should ?...?
\item Install \Ipopt\ for use with {\tt CUTEr}\\
  If you have {\tt CUTEr} already installed on your system and you
  want to use \Ipopt\ as a solver for problems modeled in {\tt SIF},
  type\\
  {\tt \$ make cuter}\\
  This assumes that you have the environment variable {\tt MYCUTER}
  defined according to the {\tt CUTEr} instructions.  After this, you
  can use the script {\tt sdipo} as the {\tt CUTEr} script to solve a
  {\tt SIF} model.
\end{enumerate}

\subsection{Installation on Windows}\label{WindowsInstall}

There are two ways to install \Ipopt\ on Windows systems.  The first
option, described in Section~\ref{CygwinInstall}, is to use Cygwin (see
\texttt{www.cygwin.com}), which offers a UNIX-like environment
on Windows and in which the installation procedure described earlier
in this section can be used.  The \Ipopt\ distribution also includes
projects files for the Microsoft Visual Studio (see
Section~\ref{VisualStudioInstall}).

\subsubsection{Installation with Cygwin}\label{CygwinInstall}

Cygwin is a Linux-like environment for Windows; if you don't know what
it is you might want to have a look at the Cygwin homepage,
\texttt{www.cygwin.com}.

It is possible to build the \Ipopt\ AMPL solver executable in Cygwin
for general use in Windows.  You can also hook up \Ipopt\ to your own
program if you compile it in the Cygwin environment\footnote{It is
  also possible to build an \Ipopt\ DLL that can be used from
  non-cygwin compilers, but this is not (yet?) supported.}.

If you want to compile \Ipopt\ under Cygwin, you first have to install
Cygwin on your Windows system.  This is pretty straight forward; you
simply download the ``setup'' program from
\texttt{www.cygwin.com} and start it.

Then you do the following steps (assuming here that you don't have any
complications with firewall settings etc - in that case you might have
to choose some connection settings differently):

\begin{enumerate}
\item Click next
\item Select ``install from the internet'' (default) and click next
\item Select a directory where Cygwin is to be installed (you can
  leave the default) and choose all other things to your liking, then
  click next
\item Select a temp dir for Cygwin setup to store some files (if you
  put it on your desktop you will later remember to delete it)
\item Select ``direct connection'' (default) and click next
\item Select some mirror site that seems close by to you and click next
\item OK, now comes the complicated part:\\
  You need to select the packages that you want to have installed.  By
  default, there are already selections, but the compilers are usually
  not pre-chosen.  You need to make sure that you select the GNU
  compilers (for Fortran, C, and C++ --- together with the MinGW
  options), the GNU Make, and Subversion.  For this, click on the "Devel"
  branch (which opens a subtree) and select:
  \begin{itemize}
  \item gcc
  \item gcc-core
  \item gcc-g77
  \item gcc-g++
  \item gcc-mingw
  \item gcc-mingw-core
  \item gcc-mingw-g77
  \item gcc-mingw-g++
  \item make
  \item subversion
  \end{itemize}

  Then, in the ``Web'' branch, please select ``wget'' (which will make
  the installation of third party dependencies for \Ipopt\ easier)

  This will automatically also select some other packages.
\item Then you click on next, and Cygwin will be installed (follow the
  rest of the instructions and choose everything else to your liking).
  At a later point you can easily add/remove packages with the setup
  program.

\item Now that you have Cygwin, you can open a Cygwin window, which is
  like a UNIX shell window.

\item Now you just follow the instructions in the beginning of
  Sections~\ref{Installing}:  You download the \Ipopt\ code into
  your Cygwin home directory (from the Windows explorer that is
  usually something like
  \texttt{C:$\backslash$Cygwin$\backslash$home$\backslash$your\_user\_name}).
  After that you obtain the third party code (like on Linux/UNIX),
  type

  \texttt{./configure}

  and

  \texttt{make install}

  in the correct directories, and hopefully that will work.  The
  \Ipopt\ AMPL solver executable will be in the subdirectory
  \texttt{bin} (called ``\texttt{ipopt.exe}'').
\end{enumerate}

\subsubsection{Using Visual Studio}\label{VisualStudioInstall}

The \Ipopt\ distribution includes project files that can be used to
compile the \Ipopt\ library and a Fortran and C++ example within the
Microsoft Visual Studio.  The project files have been created with
Microsoft Visual C++ .NET 2003 Standard, and the Intel Visual Fortran
Compiler 8.1.

In order to use those project files, download the \Ipopt\ source code,
as well as the required third party code (put it into the {\tt
  Extern/blas}, {\tt Extern/lapack}, and {\tt Extern/HSL}
directories---ASL is not required for the Fortran and C
examples). Then open the solution file\\

\texttt{\$IPOPTDIR$\backslash$Windows$\backslash$VisualStudio\_dotNET$\backslash$Ipopt$\backslash$Ipopt.sln}\\

Note: Since the project files were created only with the Standard
edition of the C++ compiler, code optimization might be disabled; for
fast performance make sure you enable code optimization.

\section{Interfacing your NLP to \Ipopt: A tutorial example.}
\label{sec:tutorial-example}

\Ipopt\ has been designed to be flexible for a wide variety of
applications, and there are a number of ways to interface with \Ipopt\
that allow specific data structures and linear solver
techniques. Nevertheless, the authors have included a standard
representation that should meet the needs of most users.

This tutorial will discuss four interfaces to \Ipopt, namely the AMPL
modeling language\cite{FouGayKer:AMPLbook} interface, and the C++, C,
and Fortran code interfaces.  AMPL is a 3rd party modeling language
tool that allows users to write their optimization problem in a syntax
that resembles the way the problem would be written mathematically.
Once the problem has been formulated in AMPL, the problem can be
easily solved using the (already compiled) \Ipopt\ AMPL solver
executable, {\tt ipopt}. Interfacing your problem by directly linking
code requires more effort to write, but can be far more efficient for
large problems.

We will illustrate how to use each of the four interfaces using an
example problem, number 71 from the Hock-Schittkowsky test suite \cite{HS},
%\begin{subequations}\label{HS71}
  \begin{eqnarray}
    \min_{x \in \Re^4} &&x_1 x_4 (x_1 + x_2 + x_3)  +  x_3 \label{eq:ex_obj} \\
    \mbox{s.t.}  &&x_1 x_2 x_3 x_4 \ge 25 \label{eq:ex_ineq} \\
    &&x_1^2 + x_2^2 + x_3^2 + x_4^2  =  40 \label{eq:ex_equ} \\
    &&1 \leq x_1, x_2, x_3, x_4 \leq 5, \label{eq:ex_bounds}
  \end{eqnarray}
%\end{subequations}
with the starting point
\begin{equation}
x_0 = (1, 5, 5, 1) \label{eq:ex_startpt}
\end{equation}
and the optimal solution
\[
x_* = (1.00000000, 4.74299963, 3.82114998, 1.37940829). \nonumber
\]

\subsection{Using \Ipopt\ through AMPL}
Using the AMPL solver executable is by far the easiest way to
solve a problem with \Ipopt. The user must simply formulate the problem
in AMPL syntax, and solve the problem through the AMPL environment.
There are drawbacks, however. AMPL is a 3rd party package and, as
such, must be appropriately licensed (a free student version for
limited problem size is available from the AMPL website,
\texttt{www.ampl.com}). Furthermore, the AMPL environment may be prohibitive
for very large problems. Nevertheless, formulating the problem in AMPL
is straightforward and even for large problems, it is often used as a
prototyping tool before using one of the code interfaces.

This tutorial is not intended as a guide to formulating models in
AMPL. If you are not already familiar with AMPL, please consult
\cite{FouGayKer:AMPLbook}.

The problem presented in equations
(\ref{eq:ex_obj})--(\ref{eq:ex_startpt}) can be solved with \Ipopt\ with
the AMPL model file given in Figure~\ref{fig:HS71}.

\begin{figure}
  \centering
\begin{footnotesize}
\begin{verbatim}
# tell ampl to use the ipopt executable as a solver
# make sure ipopt is in the path!
option solver ipopt;

# declare the variables and their bounds, 
# set notation could be used, but this is straightforward
var x1 >= 1, <= 5; 
var x2 >= 1, <= 5; 
var x3 >= 1, <= 5; 
var x4 >= 1, <= 5;

# specify the objective function
minimize obj:
                x1 * x4 * (x1 + x2 + x3) + x3;
        
# specify the constraints
s.t.
        inequality:
                x1 * x2 * x3 * x4 >= 25;
                
        equality:
                x1^2 + x2^2 + x3^2 +x4^2 = 40;

# specify the starting point            
let x1 := 1;
let x2 := 5;
let x3 := 5;
let x4 := 1;

# solve the problem
solve;

# print the solution
display x1;
display x2;
display x3;
display x4;
\end{verbatim}
\end{footnotesize}
  
  \caption{AMPL model file hs071\_ampl.mod}
  \label{fig:HS71}
\end{figure}

The line, ``{\tt option solver ipopt;}'' tells AMPL to use \Ipopt\ as
the solver. The \Ipopt\ executable (installed in
Section~\ref{sec.comp_and_inst}) must be in the {\tt PATH} for AMPL to
find it. The remaining lines specify the problem in AMPL format. The
problem can now be solved by starting AMPL and loading the mod file:
\begin{verbatim}
$ ampl
> model hs071_ampl.mod;
.
.
.
\end{verbatim}
%$
The problem will be solved using \Ipopt\ and the solution will be
displayed.

At this point, AMPL users may wish to skip the sections about
interfacing with code, but should read Section \ref{sec.options}
concerning \Ipopt\ options, and Section \ref{sec.output} which
explains the output displayed by \Ipopt.

\subsection{Interfacing with \Ipopt\ through code}
In order to solve a problem, \Ipopt\ needs more information than just
the problem definition (for example, the derivative information). If
you are using a modeling language like AMPL, the extra information is
provided by the modeling tool and the \Ipopt\ interface. When
interfacing with \Ipopt\ through your own code, however, you must
provide this additional information.

\begin{figure}
\begin{enumerate}
\item Problem dimensions \label{it.prob_dim}
  \begin{itemize}
  \item number of variables
  \item number of constraints
  \end{itemize}
\item Problem bounds
  \begin{itemize}
  \item variable bounds
  \item constraint bounds
  \end{itemize}
\item Initial starting point
  \begin{itemize}
  \item Initial values for the primal $x$ variables
  \item Initial values for the multipliers (only
    required for a warm start option)
  \end{itemize}
\item Problem Structure \label{it.prob_struct}
  \begin{itemize}
  \item number of nonzeros in the Jacobian of the constraints
  \item number of nonzeros in the Hessian of the Lagrangian function
  \item sparsity structure of the Jacobian of the constraints
  \item sparsity structure of the Hessian of the Lagrangian function
  \end{itemize}
\item Evaluation of Problem Functions \label{it.prob_eval} \\
  Information evaluated using a given point ($x,
  \lambda, \sigma_f$ coming from \Ipopt)
  \begin{itemize}
  \item Objective function, $f(x)$
  \item Gradient of the objective $\nabla f(x)$
  \item Constraint function values, $g(x)$
  \item Jacobian of the constraints, $\nabla g(x)^T$
  \item Hessian of the Lagrangian function, 
    $\sigma_f \nabla^2 f(x) + \sum_{i=1}^m\lambda_i\nabla^2
    g_i(x)$ \\
    (this is not required if a quasi-Newton options is chosen to
    approximate the second derivatives)
  \end{itemize}
\end{enumerate}
\caption{Information required by \Ipopt}
\label{fig.required_info}
\end{figure}
%\vspace{0.1in}
The information required by \Ipopt\ is shown in Figure
\ref{fig.required_info}. The problem dimensions and bounds are
straightforward and come solely from the problem definition. The
initial starting point is used by the algorithm when it begins
iterating to solve the problem. If \Ipopt\ has difficulty converging, or
if it converges to a locally infeasible point, adjusting the starting
point may help.  Depending on the starting point, \Ipopt\ may also
converge to different local solutions.

Providing the sparsity structure of derivative matrices is a bit more
involved. \Ipopt\ is a nonlinear programming solver that is designed
for solving large-scale, sparse problems. While \Ipopt\ can be
customized for a variety of matrix formats, the triplet format is used
for the standard interfaces in this tutorial. For an overview of the
triplet format for sparse matrices, see Appendix~\ref{app.triplet}.
Before solving the problem, \Ipopt\ needs to know the number of
nonzero elements and the sparsity structure (row and column indices of
each of the nonzero entries) of the constraint Jacobian and the
Lagrangian function Hessian. Once defined, this nonzero structure MUST
remain constant for the entire optimization procedure. This means that
the structure needs to include entries for any element that could ever
be nonzero, not only those that are nonzero at the starting point.

As \Ipopt\ iterates, it will need the values for
Item~\ref{it.prob_eval}. in Figure~\ref{fig.required_info} evaluated at
particular points. Before we can begin coding the interface, however,
we need to work out the details of these equations symbolically for
example problem (\ref{eq:ex_obj})-(\ref{eq:ex_bounds}).

The gradient of the objective $f(x)$ is given by
\[%\begin{equation}
\left[
\begin{array}{c}
x_1 x_4 + x_4 (x_1 + x_2 + x_3) \\
x_1 x_4 \\
x_1 x_4 + 1 \\
x_1 (x_1 + x_2 + x_3)
\end{array}
\right],
\]%\end{equation}
and the Jacobian of the constraints $g(x)$ is
\[%\begin{equation}
\left[
\begin{array}{cccc}
x_2 x_3 x_4     & x_1 x_3 x_4   & x_1 x_2 x_4   & x_1 x_2 x_3   \\
2 x_1           & 2 x_2         & 2 x_3         & 2 x_4
\end{array}
\right].
\]%\end{equation}

We also need to determine the Hessian of the Lagrangian\footnote{If a
  quasi-Newton option is chosen to approximate the second derivatives,
  this is not required.  However, if second derivatives can be
  computed, it is often worthwhile to let \Ipopt\ use them, since the
  algorithm is then usually more robust and converges faster.  More on
  the quasi-Newton approximation in Section~\ref{sec:quasiNewton}.}.
The Lagrangian function for the NLP
(\ref{eq:ex_obj})-(\ref{eq:ex_bounds}) is defined as $f(x) + g(x)^T
\lambda$ and the Hessian of the Lagrangian function is, technically, $
\nabla^2 f(x_k) + \sum_{i=1}^m\lambda_i\nabla^2 g_i(x_k)$.  However,
so that \Ipopt\ can ask for the Hessian of the objective or the
constraints independently if required, we introduce a factor
($\sigma_f$) in front of the objective term.
%
For \Ipopt\ then, the symbolic form of the Hessian of the
Lagrangian is
\begin{equation}\label{eq:IpoptLAG}
\sigma_f \nabla^2 f(x_k) + \sum_{i=1}^m\lambda_i\nabla^2 g_i(x_k)
\end{equation}
(with the $\sigma_f$ parameter), and for the example problem this becomes
%\begin{eqnarray}
%{\cal L}(x,\lambda) &{=}& f(x) + c(x)^T \lambda \nonumber \\
%&{=}& \left(x_1 x_4 (x_1 + x_2 + x_3)  +  x_3\right) 
%+ \left(x_1 x_2 x_3 x_4\right) \lambda_1 \nonumber \\
%&& \;\;\;\;\;+ \left(x_1^2 + x_2^2 + x_3^2 + x_4^2\right) \lambda_2 
%- \displaystyle \sum_{i \in 1..4} z^L_i + \sum_{i \in 1..4} z^U_i
%\end{eqnarray}
\[%\begin{equation}
\sigma_f \left[
\begin{array}{cccc}
2 x_4           & x_4           & x_4           & 2 x_1 + x_2 + x_3     \\
x_4             & 0             & 0             & x_1                   \\
x_4             & 0             & 0             & x_1                   \\
2 x_1+x_2+x_3   & x_1           & x_1           & 0
\end{array}
\right]
+
\lambda_1
\left[
\begin{array}{cccc}
0               & x_3 x_4       & x_2 x_4       & x_2 x_3       \\
x_3 x_4         & 0             & x_1 x_4       & x_1 x_3       \\
x_2 x_4         & x_1 x_4       & 0             & x_1 x_2       \\
x_2 x_3         & x_1 x_3       & x_1 x_2       & 0 
\end{array}
\right]
+
\lambda_2
\left[
\begin{array}{cccc}
2       & 0     & 0     & 0     \\
0       & 2     & 0     & 0     \\
0       & 0     & 2     & 0     \\
0       & 0     & 0     & 2
\end{array}
\right]
\]%\end{equation}
where the first term comes from the Hessian of the objective function,
and the second and third term from the Hessian of the constraints
(\ref{eq:ex_ineq}) and (\ref{eq:ex_equ}), respectively. Therefore, the
dual variables $\lambda_1$ and $\lambda_2$ are then the multipliers
for constraints (\ref{eq:ex_ineq}) and (\ref{eq:ex_equ}), respectively.

%C =============================================================================
%C
%C     This is an example for the usage of IPOPT.
%C     It implements problem 71 from the Hock-Schittkowsky test suite:
%C
%C     min   x1*x4*(x1 + x2 + x3)  +  x3
%C     s.t.  x1*x2*x3*x4                   >=  25
%C           x1**2 + x2**2 + x3**2 + x4**2  =  40
%C           1 <=  x1,x2,x3,x4  <= 5
%C
%C     Starting point:
%C        x = (1, 5, 5, 1)
%C
%C     Optimal solution:
%C        x = (1.00000000, 4.74299963, 3.82114998, 1.37940829)
%C
%C =============================================================================
\vspace{\baselineskip}

The remaining sections of the tutorial will lead you through
the coding required to solve example problem
(\ref{eq:ex_obj})--(\ref{eq:ex_bounds}) using, first C++, then C, and finally
Fortran. Completed versions of these examples can be found in {\tt
\$IPOPTDIR/Examples} under {\tt hs071\_cpp}, {\tt hs071\_c}, {\tt
hs071\_f}.

As a user, you are responsible for coding two sections of the program
that solves a problem using \Ipopt: the main executable (e.g., {\tt
  main}) and the problem representation.  Typically, you will write an
executable that prepares the problem, and then passes control over to
\Ipopt\ through an {\tt Optimize} or {\tt Solve} call. In this call,
you will give \Ipopt\ everything that it requires to call back to your
code whenever it needs functions evaluated (like the objective
function, the Jacobian of the constraints, etc.).  In each of the
three sections that follow (C++, C, and Fortran), we will first
discuss how to code the problem representation, and then how to code
the executable.

\subsection{The C++ Interface}
This tutorial assumes that you are familiar with the C++ programming
language, however, we will lead you through each step of the
implementation. For the problem representation, we will create a class
that inherits off of the pure virtual base class, {\tt TNLP} ({\tt
  IpTNLP.hpp}). For the executable (the {\tt main} function) we will
make the call to \Ipopt\ through the {\tt IpoptApplication} class
({\tt IpIpoptApplication.hpp}). In addition, we will also be using the
{\tt SmartPtr} class ({\tt IpSmartPtr.hpp}) which implements a reference
counting pointer that takes care of memory management (object
deletion) for you (for details, see Appendix~\ref{app.smart_ptr}).

After ``\texttt{make install}'' (see Section~\ref{sec.comp_and_inst}),
the header files are installed in \texttt{\$IPOPTDIR/include/ipopt}
(or in \texttt{\$PREFIX/include/ipopt} if the switch
\verb|--prefix=$PREFIX| was used for {\tt configure}).

\subsubsection{Coding the Problem Representation}\label{sec.cpp_problem}
We provide the information required in Figure \ref{fig.required_info}
by coding the {\tt HS071\_NLP} class, a specific implementation of the
{\tt TNLP} base class. In the executable, we will create an instance
of the {\tt HS071\_NLP} class and give this class to \Ipopt\ so it can
evaluate the problem functions through the {\tt TNLP} interface. If
you have any difficulty as the implementation proceeds, have a look at
the completed example in the {\tt Examples/hs071\_cpp} directory.

Start by creating a new directory under Examples, called {\tt
  MyExample} and create the files {\tt hs071\_nlp.hpp} and {\tt
  hs071\_nlp.cpp}. In {\tt hs071\_nlp.hpp}, include {\tt IpTNLP.hpp}
(the base class), tell the compiler that we are using the \Ipopt\
namespace, and create the declaration of the {\tt HS071\_NLP} class,
inheriting off of {\tt TNLP}. Have a look at the {\tt TNLP} class in
{\tt IpTNLP.hpp}; you will see eight pure virtual methods that we must
implement. Declare these methods in the header file.  Implement each
of the methods in {\tt HS071\_NLP.cpp} using the descriptions given
below. In {\tt hs071\_nlp.cpp}, first include the header file for your
class and tell the compiler that you are using the \Ipopt\ namespace.
A full version of these files can be found in the {\tt
  Examples/hs071\_cpp} directory.

It is very easy to make mistakes in the implementation of the function
evaluation methods, in particular regarding the derivatives.  \Ipopt\
has a feature that can help you to debug the derivative code, using
finite differences, see Section~\ref{sec:deriv-checker}.

Note that the return value of any {\tt bool}-valued function should be
{\tt true}, unless an error occured, for example, because the value of
a problem function could not be evaluated at the required point.

\paragraph{Method {\texttt{get\_nlp\_info}}} with prototype
\begin{verbatim}
virtual bool get_nlp_info(Index& n, Index& m, Index& nnz_jac_g,
                          Index& nnz_h_lag, IndexStyleEnum& index_style)
\end{verbatim}
Give \Ipopt\ the information about the size of the problem (and hence,
the size of the arrays that it needs to allocate). 
\begin{itemize}
\item {\tt n}: (out), the number of variables in the problem (dimension of $x$).
\item {\tt m}: (out), the number of constraints in the problem (dimension of $g(x)$).
\item {\tt nnz\_jac\_g}: (out), the number of nonzero entries in the Jacobian.
\item {\tt nnz\_h\_lag}: (out), the number of nonzero entries in the Hessian.
\item {\tt index\_style}: (out), the numbering style used for row/col entries in the sparse matrix
format ({\tt C\_STYLE}: 0-based, {\tt FORTRAN\_STYLE}: 1-based; see
also Appendix~\ref{app.triplet}).
\end{itemize}
\Ipopt\ uses this information when allocating the arrays that
it will later ask you to fill with values. Be careful in this method
since incorrect values will cause memory bugs which may be very
difficult to find.

Our example problem has 4 variables (n), and 2 constraints (m). The
constraint Jacobian for this small problem is actually dense and has 8
nonzeros (we still need to represent this Jacobian using the sparse
matrix triplet format). The Hessian of the Lagrangian has 10
``symmetric'' nonzeros (i.e., nonzeros in the lower left triangular
part.).  Keep in mind that the number of nonzeros is the total number
of elements that may \emph{ever} be nonzero, not just those that are
nonzero at the starting point. This information is set once for the
entire problem.

\begin{footnotesize}
\begin{verbatim}
bool HS071_NLP::get_nlp_info(Index& n, Index& m, Index& nnz_jac_g, 
                             Index& nnz_h_lag, IndexStyleEnum& index_style)
{
  // The problem described in HS071_NLP.hpp has 4 variables, x[0] through x[3]
  n = 4;

  // one equality constraint and one inequality constraint
  m = 2;

  // in this example the Jacobian is dense and contains 8 nonzeros
  nnz_jac_g = 8;

  // the Hessian is also dense and has 16 total nonzeros, but we
  // only need the lower left corner (since it is symmetric)
  nnz_h_lag = 10;

  // use the C style indexing (0-based)
  index_style = TNLP::C_STYLE;

  return true;
}
\end{verbatim}
\end{footnotesize}

\paragraph{Method {\texttt{get\_bounds\_info}}} with prototype
\begin{verbatim}
virtual bool get_bounds_info(Index n, Number* x_l, Number* x_u,
                             Index m, Number* g_l, Number* g_u)
\end{verbatim}
Give \Ipopt\ the value of the bounds on the variables and constraints.
\begin{itemize}
\item {\tt n}: (in), the number of variables in the problem (dimension of $x$). 
\item {\tt x\_l}: (out) the lower bounds $x^L$ for $x$. 
\item {\tt x\_u}: (out) the upper bounds $x^U$ for $x$.
\item {\tt m}: (in), the number of constraints in the problem (dimension of $g(x)$).
\item {\tt g\_l}: (out) the lower bounds $g^L$ for $g(x)$. 
\item {\tt g\_u}: (out) the upper bounds $g^U$ for $g(x)$.
\end{itemize}
The values of {\tt n} and {\tt m} that you specified in {\tt
  get\_nlp\_info} are passed to you for debug checking.  Setting a
lower bound to a value less than or equal to the value of the option
{\tt nlp\_lower\_bound\_inf} will cause \Ipopt\ to assume no lower
bound. Likewise, specifying the upper bound above or equal to the
value of the option {\tt nlp\_upper\_bound\_inf} will cause \Ipopt\ to
assume no upper bound.  These options, {\tt nlp\_lower\_bound\_inf}
and {\tt nlp\_upper\_bound\_inf}, are set to $-10^{19}$ and $10^{19}$,
respectively, by default, but may be modified by changing the options
(see Section \ref{sec.options}).

In our example, the first constraint has a lower bound of $25$ and no upper
bound, so we set the lower bound of constraint {\tt [0]} to $25$ and
the upper bound to some number greater than $10^{19}$. The second
constraint is an equality constraint and we set both bounds to
$40$. \Ipopt\ recognizes this as an equality constraint and does not
treat it as two inequalities.

\begin{footnotesize}
\begin{verbatim}
bool HS071_NLP::get_bounds_info(Index n, Number* x_l, Number* x_u,
                                Index m, Number* g_l, Number* g_u)
{
  // here, the n and m we gave IPOPT in get_nlp_info are passed back to us.
  // If desired, we could assert to make sure they are what we think they are.
  assert(n == 4);
  assert(m == 2);

  // the variables have lower bounds of 1
  for (Index i=0; i<4; i++) {
    x_l[i] = 1.0;
  }

  // the variables have upper bounds of 5
  for (Index i=0; i<4; i++) {
    x_u[i] = 5.0;
  }

  // the first constraint g1 has a lower bound of 25
  g_l[0] = 25;
  // the first constraint g1 has NO upper bound, here we set it to 2e19.
  // Ipopt interprets any number greater than nlp_upper_bound_inf as 
  // infinity. The default value of nlp_upper_bound_inf and nlp_lower_bound_inf
  // is 1e19 and can be changed through ipopt options.
  g_u[0] = 2e19;

  // the second constraint g2 is an equality constraint, so we set the 
  // upper and lower bound to the same value
  g_l[1] = g_u[1] = 40.0;

  return true;
}
\end{verbatim}
\end{footnotesize}

\paragraph{Method {\texttt{get\_starting\_point}}} with prototype
\begin{verbatim}
virtual bool get_starting_point(Index n, bool init_x, Number* x,
                                bool init_z, Number* z_L, Number* z_U,
                                Index m, bool init_lambda, Number* lambda)
\end{verbatim}
Give \Ipopt\ the starting point before it begins iterating.
\begin{itemize}
\item {\tt n}: (in), the number of variables in the problem (dimension of $x$). 
\item {\tt init\_x}: (in), if true, this method must provide an initial value for $x$.
\item {\tt x}: (out), the initial values for the primal variables, $x$.
\item {\tt init\_z}: (in), if true, this method must provide an initial value 
        for the bound multipliers $z^L$ and $z^U$.
\item {\tt z\_L}: (out), the initial values for the bound multipliers, $z^L$.
\item {\tt z\_U}: (out), the initial values for the bound multipliers, $z^U$.
\item {\tt m}: (in), the number of constraints in the problem (dimension of $g(x)$).
\item {\tt init\_lambda}: (in), if true, this method must provide an initial value 
        for the constraint multipliers, $\lambda$.
\item {\tt lambda}: (out), the initial values for the constraint multipliers, $\lambda$.
\end{itemize}

The variables {\tt n} and {\tt m} are passed in for your convenience.
These variables will have the same values you specified in {\tt
  get\_nlp\_info}.

Depending on the options that have been set, \Ipopt\ may or may not
require bounds for the primal variables $x$, the bound multipliers
$z^L$ and $z^U$, and the constraint multipliers $\lambda$. The boolean
flags {\tt init\_x}, {\tt init\_z}, and {\tt init\_lambda} tell you
whether or not you should provide initial values for $x$, $z^L$, $z^U$, or
$\lambda$ respectively. The default options only require an initial
value for the primal variables $x$.  Note, the initial values for
bound multiplier components for ``infinity'' bounds
($x_L^{(i)}=-\infty$ or $x_U^{(i)}=\infty$) are ignored.

In our example, we provide initial values for $x$ as specified in the
example problem. We do not provide any initial values for the dual
variables, but use an assert to immediately let us know if we are ever
asked for them.

\begin{footnotesize}
\begin{verbatim}
bool HS071_NLP::get_starting_point(Index n, bool init_x, Number* x,
                                   bool init_z, Number* z_L, Number* z_U,
                                   Index m, bool init_lambda,
                                   Number* lambda)
{
  // Here, we assume we only have starting values for x, if you code
  // your own NLP, you can provide starting values for the dual variables
  // if you wish to use a warmstart option
  assert(init_x == true);
  assert(init_z == false);
  assert(init_lambda == false);

  // initialize to the given starting point
  x[0] = 1.0;
  x[1] = 5.0;
  x[2] = 5.0;
  x[3] = 1.0;

  return true;
}
\end{verbatim}
\end{footnotesize}

\paragraph{Method {\texttt{eval\_f}}} with prototype
\begin{verbatim}
virtual bool eval_f(Index n, const Number* x, 
                    bool new_x, Number& obj_value)
\end{verbatim}
Return the value of the objective function at the point $x$.
\begin{itemize}
\item {\tt n}: (in), the number of variables in the problem (dimension
  of $x$).
\item {\tt x}: (in), the values for the primal variables, $x$, at which
  $f(x)$ is to be evaluated.
\item {\tt new\_x}: (in), false if any evaluation method was
  previously called with the same values in {\tt x}, true otherwise.
\item {\tt obj\_value}: (out) the value of the objective function
  ($f(x)$).
\end{itemize}

The boolean variable {\tt new\_x} will be false if the last call to
any of the evaluation methods ({\tt eval\_*}) used the same $x$
values. This can be helpful when users have efficient implementations
that calculate multiple outputs at once. \Ipopt\ internally caches
results from the {\tt TNLP} and generally, this flag can be ignored.

The variable {\tt n} is passed in for your convenience. This variable
will have the same value you specified in {\tt get\_nlp\_info}.

For our example, we ignore the {\tt new\_x} flag and calculate the objective.

\begin{footnotesize}
\begin{verbatim}
bool HS071_NLP::eval_f(Index n, const Number* x, bool new_x, Number& obj_value)
{
  assert(n == 4);

  obj_value = x[0] * x[3] * (x[0] + x[1] + x[2]) + x[2];

  return true;
}
\end{verbatim}
\end{footnotesize}

\paragraph{Method {\texttt{eval\_grad\_f}}} with prototype
\begin{verbatim}
virtual bool eval_grad_f(Index n, const Number* x, bool new_x, 
                         Number* grad_f)
\end{verbatim}
Return the gradient of the objective function at the point $x$.
\begin{itemize}
\item {\tt n}: (in), the number of variables in the problem (dimension of $x$). 
\item {\tt x}: (in), the values for the primal variables, $x$, at which
  $\nabla f(x)$ is to be evaluated.
\item {\tt new\_x}: (in), false if any evaluation method was previously called 
        with the same values in {\tt x}, true otherwise.
\item {\tt grad\_f}: (out) the array of values for the gradient of the 
        objective function ($\nabla f(x)$).
\end{itemize}

The gradient array is in the same order as the $x$ variables (i.e., the
gradient of the objective with respect to {\tt x[2]} should be put in
{\tt grad\_f[2]}).

The boolean variable {\tt new\_x} will be false if the last call to
any of the evaluation methods ({\tt eval\_*}) used the same $x$
values. This can be helpful when users have efficient implementations
that calculate multiple outputs at once. \Ipopt\ internally caches
results from the {\tt TNLP} and generally, this flag can be ignored.

The variable {\tt n} is passed in for your convenience. This
variable will have the same value you specified in {\tt
get\_nlp\_info}.

In our example, we ignore the {\tt new\_x} flag and calculate the
values for the gradient of the objective.

\begin{footnotesize}
\begin{verbatim}
bool HS071_NLP::eval_grad_f(Index n, const Number* x, bool new_x, Number* grad_f)
{
  assert(n == 4);

  grad_f[0] = x[0] * x[3] + x[3] * (x[0] + x[1] + x[2]);
  grad_f[1] = x[0] * x[3];
  grad_f[2] = x[0] * x[3] + 1;
  grad_f[3] = x[0] * (x[0] + x[1] + x[2]);

  return true;
}
\end{verbatim}
\end{footnotesize}

\paragraph{Method {\texttt{eval\_g}}} with prototype
\begin{verbatim}
virtual bool eval_g(Index n, const Number* x, 
                    bool new_x, Index m, Number* g)
\end{verbatim}
Return the value of the constraint function at the point $x$.
\begin{itemize}
\item {\tt n}: (in), the number of variables in the problem (dimension of $x$). 
\item {\tt x}: (in), the values for the primal variables, $x$, at
  which the constraint functions,
  $g(x)$, are to be evaluated.
\item {\tt new\_x}: (in), false if any evaluation method was previously called 
        with the same values in {\tt x}, true otherwise.
\item {\tt m}: (in), the number of constraints in the problem (dimension of $g(x)$).
\item {\tt g}: (out) the array of constraint function values, $g(x)$.
\end{itemize}

The values returned in {\tt g} should be only the $g(x)$ values, 
do not add or subtract the bound values $g^L$ or $g^U$.

The boolean variable {\tt new\_x} will be false if the last call to
any of the evaluation methods ({\tt eval\_*}) used the same $x$
values. This can be helpful when users have efficient implementations
that calculate multiple outputs at once. \Ipopt\ internally caches
results from the {\tt TNLP} and generally, this flag can be ignored.

The variables {\tt n} and {\tt m} are passed in for your convenience.
These variables will have the same values you specified in {\tt
  get\_nlp\_info}.

In our example, we ignore the {\tt new\_x} flag and calculate the
values of constraint functions.

\begin{footnotesize}
\begin{verbatim}
bool HS071_NLP::eval_g(Index n, const Number* x, bool new_x, Index m, Number* g)
{
  assert(n == 4);
  assert(m == 2);

  g[0] = x[0] * x[1] * x[2] * x[3];
  g[1] = x[0]*x[0] + x[1]*x[1] + x[2]*x[2] + x[3]*x[3];

  return true;
} 
\end{verbatim}
\end{footnotesize}

\paragraph{Method {\texttt{eval\_jac\_g}}} with prototype
\begin{verbatim}
virtual bool eval_jac_g(Index n, const Number* x, bool new_x,
                        Index m, Index nele_jac, Index* iRow, 
                        Index *jCol, Number* values)
\end{verbatim}
Return either the sparsity structure of the Jacobian of the
constraints, or the values for the Jacobian of the constraints at the
point $x$.
\begin{itemize}
\item {\tt n}: (in), the number of variables in the problem (dimension of $x$). 
\item {\tt x}: (in), the values for the primal variables, $x$, at which
  the constraint Jacobian, $\nabla g(x)^T$, is to be evaluated.
\item {\tt new\_x}: (in), false if any evaluation method was previously called 
        with the same values in {\tt x}, true otherwise.
\item {\tt m}: (in), the number of constraints in the problem (dimension of $g(x)$).
\item {\tt n\_ele\_jac}: (in), the number of nonzero elements in the 
        Jacobian (dimension of {\tt iRow}, {\tt jCol}, and {\tt values}).
\item {\tt iRow}: (out), the row indices of entries in the Jacobian of the constraints.
\item {\tt jCol}: (out), the column indices of entries in the Jacobian of the constraints.
\item {\tt values}: (out), the values of the entries in the Jacobian of the constraints.
\end{itemize}

The Jacobian is the matrix of derivatives where the derivative of
constraint $g^{(i)}$ with respect to variable $x^{(j)}$ is placed in
row $i$ and column $j$. See Appendix \ref{app.triplet} for a
discussion of the sparse matrix format used in this method.

If the {\tt iRow} and {\tt jCol} arguments are not {\tt NULL}, then
\Ipopt\ wants you to fill in the sparsity structure of the Jacobian
(the row and column indices only). At this time, the {\tt x} argument
and the {\tt values} argument will be {\tt NULL}.

If the {\tt x} argument and the {\tt values} argument are not {\tt
  NULL}, then \Ipopt\ wants you to fill in the values of the Jacobian
as calculated from the array {\tt x} (using the same order as you used
when specifying the sparsity structure). At this time, the {\tt iRow}
and {\tt jCol} arguments will be {\tt NULL};

The boolean variable {\tt new\_x} will be false if the last call to
any of the evaluation methods ({\tt eval\_*}) used the same $x$
values. This can be helpful when users have efficient implementations
that calculate multiple outputs at once. \Ipopt\ internally caches
results from the {\tt TNLP} and generally, this flag can be ignored.

The variables {\tt n}, {\tt m}, and {\tt nele\_jac} are passed in for
your convenience. These arguments will have the same values you
specified in {\tt get\_nlp\_info}.

In our example, the Jacobian is actually dense, but we still
specify it using the sparse format.

\begin{footnotesize}
\begin{verbatim}
bool HS071_NLP::eval_jac_g(Index n, const Number* x, bool new_x,
                           Index m, Index nele_jac, Index* iRow, Index *jCol,
                           Number* values)
{
  if (values == NULL) {
    // return the structure of the Jacobian

    // this particular Jacobian is dense
    iRow[0] = 0; jCol[0] = 0;
    iRow[1] = 0; jCol[1] = 1;
    iRow[2] = 0; jCol[2] = 2;
    iRow[3] = 0; jCol[3] = 3;
    iRow[4] = 1; jCol[4] = 0;
    iRow[5] = 1; jCol[5] = 1;
    iRow[6] = 1; jCol[6] = 2;
    iRow[7] = 1; jCol[7] = 3;
  }
  else {
    // return the values of the Jacobian of the constraints
    
    values[0] = x[1]*x[2]*x[3]; // 0,0
    values[1] = x[0]*x[2]*x[3]; // 0,1
    values[2] = x[0]*x[1]*x[3]; // 0,2
    values[3] = x[0]*x[1]*x[2]; // 0,3

    values[4] = 2*x[0]; // 1,0
    values[5] = 2*x[1]; // 1,1
    values[6] = 2*x[2]; // 1,2
    values[7] = 2*x[3]; // 1,3
  }

  return true;
}
\end{verbatim}
\end{footnotesize}

\paragraph{Method {\texttt{eval\_h}}} with prototype
\begin{verbatim}
virtual bool eval_h(Index n, const Number* x, bool new_x,
                    Number obj_factor, Index m, const Number* lambda,
                    bool new_lambda, Index nele_hess, Index* iRow,
                    Index* jCol, Number* values)
\end{verbatim}
Return either the sparsity structure of the Hessian of the Lagrangian, or the values of the 
Hessian of the Lagrangian (\ref{eq:IpoptLAG}) for the given values for $x$,
$\sigma_f$, and $\lambda$.
\begin{itemize}
\item {\tt n}: (in), the number of variables in the problem (dimension
  of $x$).
\item {\tt x}: (in), the values for the primal variables, $x$, at which
  the Hessian is to be evaluated.
\item {\tt new\_x}: (in), false if any evaluation method was previously called 
        with the same values in {\tt x}, true otherwise.
\item {\tt obj\_factor}: (in), factor in front of the objective term
  in the Hessian, $sigma_f$.
\item {\tt m}: (in), the number of constraints in the problem (dimension of $g(x)$).
\item {\tt lambda}: (in), the values for the constraint multipliers,
  $\lambda$, at which the Hessian is to be evaluated.
\item {\tt new\_lambda}: (in), false if any evaluation method was
  previously called with the same values in {\tt lambda}, true
  otherwise.
\item {\tt nele\_hess}: (in), the number of nonzero elements in the
  Hessian (dimension of {\tt iRow}, {\tt jCol}, and {\tt values}).
\item {\tt iRow}: (out), the row indices of entries in the Hessian.
\item {\tt jCol}: (out), the column indices of entries in the Hessian.
\item {\tt values}: (out), the values of the entries in the Hessian.
\end{itemize}

The Hessian matrix that \Ipopt\ uses is defined in
Eq.~\ref(eq:IpoptLAG).  See Appendix \ref{app.triplet} for a
discussion of the sparse symmetric matrix format used in this method.

If the {\tt iRow} and {\tt jCol} arguments are not {\tt NULL}, then
\Ipopt\ wants you to fill in the sparsity structure of the Hessian
(the row and column indices for the lower or upper triangular part
only). In this case, the {\tt x}, {\tt lambda}, and {\tt values}
arrays will be {\tt NULL}.

If the {\tt x}, {\tt lambda}, and {\tt values} arrays are not {\tt
  NULL}, then \Ipopt\ wants you to fill in the values of the Hessian
as calculated using {\tt x} and {\tt lambda} (using the same order as
you used when specifying the sparsity structure). In this case, the
{\tt iRow} and {\tt jCol} arguments will be {\tt NULL}.

The boolean variables {\tt new\_x} and {\tt new\_lambda} will both be
false if the last call to any of the evaluation methods ({\tt
  eval\_*}) used the same values. This can be helpful when users have
efficient implementations that calculate multiple outputs at once.
\Ipopt\ internally caches results from the {\tt TNLP} and generally,
this flag can be ignored.

The variables {\tt n}, {\tt m}, and {\tt nele\_hess} are passed in for
your convenience. These arguments will have the same values you
specified in {\tt get\_nlp\_info}.

In our example, the Hessian is dense, but we still specify it using the
sparse matrix format. Because the Hessian is symmetric, we only need to 
specify the lower left corner.

\begin{footnotesize}
\begin{verbatim}
bool HS071_NLP::eval_h(Index n, const Number* x, bool new_x,
                       Number obj_factor, Index m, const Number* lambda,
                       bool new_lambda, Index nele_hess, Index* iRow,
                       Index* jCol, Number* values)
{
  if (values == NULL) {
    // return the structure. This is a symmetric matrix, fill the lower left
    // triangle only.

    // the Hessian for this problem is actually dense
    Index idx=0;
    for (Index row = 0; row < 4; row++) {
      for (Index col = 0; col <= row; col++) {
        iRow[idx] = row; 
        jCol[idx] = col;
        idx++;
      }
    }
    
    assert(idx == nele_hess);
  }
  else {
    // return the values. This is a symmetric matrix, fill the lower left
    // triangle only

    // fill the objective portion
    values[0] = obj_factor * (2*x[3]); // 0,0

    values[1] = obj_factor * (x[3]);   // 1,0
    values[2] = 0;                     // 1,1

    values[3] = obj_factor * (x[3]);   // 2,0
    values[4] = 0;                     // 2,1
    values[5] = 0;                     // 2,2

    values[6] = obj_factor * (2*x[0] + x[1] + x[2]); // 3,0
    values[7] = obj_factor * (x[0]);                 // 3,1
    values[8] = obj_factor * (x[0]);                 // 3,2
    values[9] = 0;                                   // 3,3


    // add the portion for the first constraint
    values[1] += lambda[0] * (x[2] * x[3]); // 1,0
    
    values[3] += lambda[0] * (x[1] * x[3]); // 2,0
    values[4] += lambda[0] * (x[0] * x[3]); // 2,1

    values[6] += lambda[0] * (x[1] * x[2]); // 3,0
    values[7] += lambda[0] * (x[0] * x[2]); // 3,1
    values[8] += lambda[0] * (x[0] * x[1]); // 3,2

    // add the portion for the second constraint
    values[0] += lambda[1] * 2; // 0,0

    values[2] += lambda[1] * 2; // 1,1

    values[5] += lambda[1] * 2; // 2,2

    values[9] += lambda[1] * 2; // 3,3
  }

  return true;
}
\end{verbatim}
\end{footnotesize}

{\bf TODO: User STOP method here?}

\paragraph{Method \texttt{finalize\_solution}} with prototype
\begin{verbatim}
virtual void finalize_solution(SolverReturn status, Index n,
                               const Number* x, const Number* z_L,
                               const Number* z_U, Index m, const Number* g,
                               const Number* lambda, Number obj_value)
\end{verbatim}
This is the only method that is not mentioned in Figure
\ref{fig.required_info}. This method is called by \Ipopt\ after the
algorithm has finished (successfully or even with most errors).
\begin{itemize}
\item {\tt status}: (in), gives the status of the algorithm as
  specified in {\tt IpAlgTypes.hpp},
  \begin{itemize}
  \item {\tt SUCCESS}: Algorithm terminated successfully at a locally
    optimal point, satisfying the convergence tolerances (can be
    specified by options).
  \item {\tt MAXITER\_EXCEEDED}: Maximum number of iterations exceeded
    (can be specified by an option).
  \item {\tt STOP\_AT\_TINY\_STEP}: Algorithm proceeds with very
    little progress.
  \item {\tt STOP\_AT\_ACCEPTABLE\_POINT}: Algorithm stopped at a
    point that was converged, not to ``desired'' tolerances, but to
    ``acceptable'' tolerances (see the {\tt acceptable-...} options).
  \item {\tt LOCAL\_INFEASIBILITY}: Algorithm converged to a point of
    local infeasibility. Problem may be infeasible.
  \item {\tt DIVERGING\_ITERATES}: It seems that the iterates diverge.
  \item {\tt RESTORATION\_FAILURE}: Restoration phase failed,
    algorithm doesn't know how to proceed.
  \item {\tt
      INTERNAL\_ERROR}: An unknown internal error occurred.  Please
    contact the \Ipopt\ authors through the mailing list.
  \end{itemize}
\item {\tt n}: (in), the number of variables in the problem (dimension
  of $x$).
\item {\tt x}: (in), the final values for the primal variables, $x_*$.
\item {\tt z\_L}: (in), the final values for the lower bound
  multipliers, $z^L_*$.
\item {\tt z\_U}: (in), the final values for the upper bound
  multipliers, $z^U_*$.
\item {\tt m}: (in), the number of constraints in the problem
  (dimension of $g(x)$).
\item {\tt g}: (in), the final value of the constraint function
  values, $g(x_*)$.
\item {\tt lambda}: (in), the final values of the constraint
  multipliers, $\lambda_*$.
\item {\tt obj\_value}: (in), the final value of the objective,
  $f(x_*)$.
\end{itemize}

{\bf TODO: Should be provide IpStatistics here?}

This method gives you the return status of the algorithm
(SolverReturn), and the values of the variables, 
the objective and constraint function values when the algorithm exited.

In our example, we will print the values of some of the variables to 
the screen.

\begin{footnotesize}
\begin{verbatim}
void HS071_NLP::finalize_solution(SolverReturn status,
                                  Index n, const Number* x, const Number* z_L,
                                  const Number* z_U, Index m, const Number* g,
                                  const Number* lambda, Number obj_value)
{
  // here is where we would store the solution to variables, or write to a file, etc
  // so we could use the solution. 

  // For this example, we write the solution to the console
  printf("\n\nSolution of the primal variables, x\n");
  for (Index i=0; i<n; i++) {
    printf("x[%d] = %e\n", i, x[i]); 
  }

  printf("\n\nSolution of the bound multipliers, z_L and z_U\n");
  for (Index i=0; i<n; i++) {
    printf("z_L[%d] = %e\n", i, z_L[i]); 
  }
  for (Index i=0; i<n; i++) {
    printf("z_U[%d] = %e\n", i, z_U[i]); 
  }

  printf("\n\nObjective value\n");
  printf("f(x*) = %e\n", obj_value); 
}
\end{verbatim}
\end{footnotesize}

This is all that is required for our {\tt HS071\_NLP} class and 
the coding of the problem representation.
 
\subsubsection{Coding the Executable (\texttt{main})}
Now that we have a problem representation, the {\tt HS071\_NLP} class,
we need to code the main function that will call \Ipopt\ and ask \Ipopt\
to find a solution.

Here, we must create an instance of our problem ({\tt HS071\_NLP}),
create an instance of the \Ipopt\ solver (\texttt{IpoptApplication}),
and ask the solver to find a solution. We always use the
\texttt{SmartPtr} template class instead of raw C++ pointers when
creating and passing \Ipopt\ objects. To find out more information
about smart pointers and the {\tt SmartPtr} implementation used in
\Ipopt, see Appendix \ref{app.smart_ptr}.

Create the file {\tt MyExample.cpp} in the MyExample directory.
Include {\tt HS071\_NLP.hpp} and {\tt IpIpoptApplication.hpp}, tell
the compiler to use the {\tt Ipopt} namespace, and implement the {\tt
  main} function.

\begin{footnotesize}
\begin{verbatim}
#include "IpIpoptApplication.hpp"
#include "hs071_nlp.hpp"

using namespace Ipopt;

int main(int argv, char* argc[])
{
  // Create a new instance of your nlp 
  //  (use a SmartPtr, not raw)
  SmartPtr<TNLP> mynlp = new HS071_NLP();

  // Create a new instance of IpoptApplication
  //  (use a SmartPtr, not raw)
  SmartPtr<IpoptApplication> app = new IpoptApplication();

  // Change some options
  app->Options()->SetNumericValue("tol", 1e-9);
  app->Options()->SetStringValue("mu_strategy", "adaptive");

  // Ask Ipopt to solve the problem
  ApplicationReturnStatus status = app->OptimizeTNLP(mynlp);

  if (status == Solve_Succeeded) {
    printf("\n\n*** The problem solved!\n");
  }
  else {
    printf("\n\n*** The problem FAILED!\n");
  }

  // As the SmartPtrs go out of scope, the reference count
  // will be decremented and the objects will automatically 
  // be deleted.

  return (int) status;
}
\end{verbatim} 
\end{footnotesize}

The first line of code in {\tt main} creates an instance of {\tt
  HS071\_NLP}. We then create an instance of the \Ipopt\ solver, {\tt
  IpoptApplication}. The call to {\tt app->OptimizeTNLP(...)} will run
\Ipopt\ and try to solve the problem. By default, \Ipopt\ will write
to its progress to the console, and return the {\tt SolverReturn}
status.

\subsubsection{Compiling and Testing the Example}
Our next task is to compile and test the code. If you are familiar
with the compiler and linker used on your system, you can build the
code, including the \Ipopt\ library {\tt libipopt.a} (and other
necessary libraries, as listed in the {\tt ipopt\_addlibs\_cpp.txt}
and {\tt ipopt\_addlibs\_f.txt} files).  If you are using Linux/UNIX,
then a sample makefile exists already that was created by configure.
Copy {\tt Examples/hs071\_cpp/Makefile} into your {\tt MyExample}
directory.  This makefile was created for the {\tt hs071\_cpp} code,
but it can be easily modified for your example problem. Edit the file,
making the following changes,

\begin{itemize}
\item change the {\tt EXE} variable \\
{\tt EXE = my\_example}
\item change the {\tt OBJS} variable \\
{\tt OBJS = HS071\_NLP.o MyExample.o}
\end{itemize}
and the problem should compile easily with, \\
{\tt \$ make} \\
Now run the executable,\\ 
{\tt \$ ./my\_example} \\
and you should see output resembling the following,

\begin{footnotesize}
\begin{verbatim}
Total number of variables............................:        4
                     variables with only lower bounds:        0
                variables with lower and upper bounds:        4
                     variables with only upper bounds:        0
Total number of equality constraints.................:        1
Total number of inequality constraints...............:        1
        inequality constraints with only lower bounds:        1
   inequality constraints with lower and upper bounds:        0
        inequality constraints with only upper bounds:        0
 
 iter     objective    inf_pr   inf_du lg(mu)  ||d||  lg(rg) alpha_du alpha_pr  ls
    0   1.7159878e+01 2.01e-02 5.20e-01  -1.0 0.00e+00    -  0.00e+00 0.00e+00   0 y
    1   1.7146308e+01 1.63e-01 1.47e-01  -1.0 1.15e-01    -  9.86e-01 1.00e+00f  1
    2   1.7065508e+01 3.10e-02 8.47e-02  -1.7 1.99e-01    -  9.54e-01 1.00e+00h  1 Nhj
    3   1.7002626e+01 4.10e-02 4.81e-03  -2.5 5.52e-02    -  1.00e+00 1.00e+00h  1
    4   1.7019082e+01 1.20e-03 1.81e-04  -2.5 1.10e-02    -  1.00e+00 1.00e+00h  1
    5   1.7014253e+01 1.80e-04 4.87e-05  -3.8 4.86e-03    -  1.00e+00 1.00e+00h  1
    6   1.7014020e+01 9.25e-07 2.15e-07  -5.7 2.76e-04    -  1.00e+00 1.00e+00h  1
    7   1.7014017e+01 1.01e-10 2.60e-11  -8.6 3.32e-06    -  1.00e+00 1.00e+00h  1
 
Number of Iterations....: 7
 
                                   (scaled)                 (unscaled)
Objective...............:   1.7014017145177885e+01    1.7014017145177885e+01
Dual infeasibility......:   2.5980210027546616e-11    2.5980210027546616e-11
Constraint violation....:   1.8175683180743363e-11    1.8175683180743363e-11
Complementarity.........:   2.5282956951655172e-09    2.5282956951655172e-09
Overall NLP error.......:   2.5282956951655172e-09    2.5282956951655172e-09
 
 
Number of objective function evaluations             = 8
Number of objective gradient evaluations             = 8
Number of equality constraint evaluations            = 8
Number of inequality constraint evaluations          = 8
Number of equality constraint Jacobian evaluations   = 8
Number of inequality constraint Jacobian evaluations = 8
Number of Lagrangian Hessian evaluations             = 9
 
EXIT: Optimal Solution Found.
 
 
Solution of the primal variables, x
x[0] = 1
x[1] = 4.743
x[2] = 3.82115
x[3] = 1.37941
 
 
Solution of the bound multipliers, z_L and z_U
z_L[0] = 1.08787
z_L[1] = 6.69317e-10
z_L[2] = 8.8877e-10
z_L[3] = 6.57011e-09
z_U[0] = 6.26262e-10
z_U[1] = 9.78906e-09
z_U[2] = 2.12283e-09
z_U[3] = 6.92528e-10
 
 
Objective value
f(x*) = 17.014
 
 
*** The problem solved!
\end{verbatim}
\end{footnotesize}

This completes the basic C++ tutorial, but see Section
\ref{sec.output} which explains the standard console output of \Ipopt
and Section \ref{sec.options} for information about the use of options
to customize the behavior of \Ipopt.

The {\tt Examples/ScalableProblems} directory contains another set
of NLP problems coded in C++.

\subsection{The C Interface}\label{sec.cinterface}
The C interface for \Ipopt\ is declared in the header file {\tt
  IpStdCInterface.h}, which is found in\\
\texttt{\$IPOPTDIR/include/ipopt} (or in
\texttt{\$PREFIX/include/ipopt} if the switch
\verb|--prefix=$PREFIX| was used for {\tt configure}); while
reading this section, it will be helpful to have a look at this file.

In order to solve an optimization problem with the C interface, one
has to create an {\tt IpoptProblem}\footnote{{\tt IpoptProblem} is a
  pointer to a C structure; you should not access this structure
  directly, only through the functions provided in the C interface.}
with the function {\tt CreateIpoptProblem}, which later has to be
passed to the {\tt IpoptSolve} function.

The {\tt IpoptProblem} created by {\tt CreateIpoptProblem} contains
the problem dimensions, the variable and constraint bounds, and the
function pointers for callbacks that will be used to evaluate the NLP
problem functions and their derivatives (see also the discussion of
the C++ methods {\tt get\_nlp\_info} and {\tt get\_bounds\_info} in
Section~\ref{sec.cpp_problem} for information about the arguments of
{\tt CreateIpoptProblem}).

The prototypes for the callback functions, {\tt Eval\_F\_CB}, {\tt
  Eval\_Grad\_F\_CB}, etc., are defined in the header file {\tt
  IpStdCInterface.h}.  Their arguments correspond one-to-one to the
arguments for the C++ methods discussed in
Section~\ref{sec.cpp_problem}; for example, for the meaning of $\tt
n$, $\tt x$, $\tt new\_x$, $\tt obj\_value$ in the declaration of {\tt
  Eval\_F\_CB} see the discussion of ``{\tt eval\_f}''.  The callback
functions should return {\tt TRUE}, unless there was a problem doing
the requested function/derivative evaluation at the given point {\tt
  x} (then it should return {\tt FALSE}).

Note the additional argument of type {\tt UserDataPtr} in the callback
functions.  This pointer argument is available for you to communicate
information between the main program that calls {\tt IpoptSolve} and
any of the callback functions.  This pointer is simply passed
unmodified by \Ipopt\ among those functions.  For example, you can
use this to pass constants that define the optimization problem and
are computed before the optimization in the main C program to the
callback functions.

After an {\tt IpoptProblem} has been created, you can set algorithmic
options for \Ipopt\ (see Section~\ref{sec.options}) using the {\tt
  AddIpopt...Option} functions.  Finally, the \Ipopt\ algorithm is
called with {\tt IpoptSolve}, giving \Ipopt\ the {\tt IpoptProblem},
the starting point, and arrays to store the solution values (primal
and dual variables), if desired.  Finally, after everything is done,
you should call {\tt FreeIpoptProblem} to release internal memory that
is still allocated inside \Ipopt.

In the remainder of this section we discuss how the example problem
(\ref{eq:ex_obj})--(\ref{eq:ex_bounds}) can be solved using the C
interface.  A completed version of this example can be found in {\tt
  Examples/hs071\_c}.

% We first create the necessary callback
% functions for evaluating the NLP. As just discussed, the \Ipopt\ C
% interface required callbacks to evaluate the objective value,
% constraints, gradient of the objective, Jacobian of the constraints,
% and the Hessian of the Lagrangian.  These callbacks are implemented
% using function pointers.  Have a look at the C++ implementation for
% {\tt eval\_f}, {\tt eval\_g}, {\tt eval\_grad\_f}, {\tt eval\_jac\_g},
% and {\tt eval\_h} in Section \ref{sec.cpp_problem}. The C
% implementations have somewhat different prototypes, but are
% implemented almost identically to the C++ code.

\vspace{\baselineskip}

In order to implement the example problem on your own, create a new
directory {\tt MyCExample} and create a new file, {\tt
  hs071\_c.c}.  Here, include the interface header file {\tt
  IpStdCInterface.h}, along with other necessary header files, such as
{\tt stdlib.h} and {\tt assert.h}.  Add the prototypes and
implementations for the five callback functions.  Have a look at the
C++ implementation for {\tt eval\_f}, {\tt eval\_g}, {\tt
  eval\_grad\_f}, {\tt eval\_jac\_g}, and {\tt eval\_h} in Section
\ref{sec.cpp_problem}. The C implementations have somewhat different
prototypes, but are implemented almost identically to the C++ code.
See the completed example in {\tt Examples/hs071\_c/hs071\_c.c} if you
are not sure how to do this.

We now need to implement the {\tt main} function, create the {\tt
  IpoptProblem}, set options, and call {\tt IpoptSolve}. The {\tt
  CreateIpoptProblem} function requires the problem dimensions, the
variable and constraint bounds, and the function pointers to the
callback routines. The {\tt IpoptSolve} function requires the {\tt
  IpoptProblem}, the starting point, and allocated arrays for the
solution.  The {\tt main} function from the example is shown next, and
discussed below.

%in Figure~\ref{fig:cexample-main}.
%\begin{figure}
%  \centering
\begin{footnotesize}
\begin{verbatim}
int main()
{
  Index n=-1;                          /* number of variables */
  Index m=-1;                          /* number of constraints */
  Number* x_L = NULL;                  /* lower bounds on x */
  Number* x_U = NULL;                  /* upper bounds on x */
  Number* g_L = NULL;                  /* lower bounds on g */
  Number* g_U = NULL;                  /* upper bounds on g */
  IpoptProblem nlp = NULL;             /* IpoptProblem */
  enum ApplicationReturnStatus status; /* Solve return code */
  Number* x = NULL;                    /* starting point and solution vector */
  Number* mult_x_L = NULL;             /* lower bound multipliers 
					  at the solution */
  Number* mult_x_U = NULL;             /* upper bound multipliers 
					  at the solution */
  Number obj;                          /* objective value */
  Index i;                             /* generic counter */
  
  /* set the number of variables and allocate space for the bounds */
  n=4;
  x_L = (Number*)malloc(sizeof(Number)*n);
  x_U = (Number*)malloc(sizeof(Number)*n);
  /* set the values for the variable bounds */
  for (i=0; i<n; i++) {
    x_L[i] = 1.0;
    x_U[i] = 5.0;
  }

  /* set the number of constraints and allocate space for the bounds */
  m=2;
  g_L = (Number*)malloc(sizeof(Number)*m);
  g_U = (Number*)malloc(sizeof(Number)*m);
  /* set the values of the constraint bounds */
  g_L[0] = 25; g_U[0] = 2e19;
  g_L[1] = 40; g_U[1] = 40;

  /* create the IpoptProblem */
  nlp = CreateIpoptProblem(n, x_L, x_U, m, g_L, g_U, 8, 10, 0, 
			   &eval_f, &eval_g, &eval_grad_f, 
			   &eval_jac_g, &eval_h);
  
  /* We can free the memory now - the values for the bounds have been
     copied internally in CreateIpoptProblem */
  free(x_L);
  free(x_U);
  free(g_L);
  free(g_U);

  /* set some options */
  AddIpoptNumOption(nlp, "tol", 1e-9);
  AddIpoptStrOption(nlp, "mu_strategy", "adaptive");

  /* allocate space for the initial point and set the values */
  x = (Number*)malloc(sizeof(Number)*n);
  x[0] = 1.0;
  x[1] = 5.0;
  x[2] = 5.0;
  x[3] = 1.0;

  /* allocate space to store the bound multipliers at the solution */
  mult_x_L = (Number*)malloc(sizeof(Number)*n);
  mult_x_U = (Number*)malloc(sizeof(Number)*n);

  /* solve the problem */
  status = IpoptSolve(nlp, x, NULL, &obj, NULL, mult_x_L, mult_x_U, NULL);

  if (status == Solve_Succeeded) {
    printf("\n\nSolution of the primal variables, x\n");
    for (i=0; i<n; i++) {
      printf("x[%d] = %e\n", i, x[i]); 
    }

    printf("\n\nSolution of the bound multipliers, z_L and z_U\n");
    for (i=0; i<n; i++) {
      printf("z_L[%d] = %e\n", i, mult_x_L[i]); 
    }
    for (i=0; i<n; i++) {
      printf("z_U[%d] = %e\n", i, mult_x_U[i]); 
    }

    printf("\n\nObjective value\n");
    printf("f(x*) = %e\n", obj); 
  }
 
  /* free allocated memory */
  FreeIpoptProblem(nlp);
  free(x);
  free(mult_x_L);
  free(mult_x_U);

  return 0;
}
\end{verbatim}
\end{footnotesize}
%  \caption{{\tt main} function for C example}
%  \label{fig:cexample-main}
%\end{figure}

Here, we declare all the necessary variables and set the dimensions of
the problem.  The problem has 4 variables, so we set {\tt n} and
allocate space for the variable bounds (don't forget to call {\tt
  free} for each of your {\tt malloc} calls before the end of the
program). We then set the values for the variable bounds.

The problem has 2 constraints, so we set {\tt m} and allocate space
for the constraint bounds. The first constraint has a lower bound of
$25$ and no upper bound.  Here we set the upper bound to
\texttt{2e19}. \Ipopt\ interprets any number greater than or equal to
\texttt{nlp\_upper\_bound\_inf} as infinity. The default value of
\texttt{nlp\_lower\_bound\_inf} and \texttt{nlp\_upper\_bound\_inf} is
\texttt{-1e19} and \texttt{1e19}, respectively, and can be changed
through \Ipopt\ options.  The second constraint is an equality with
right hand side 40, so we set both the upper and the lower bound to
40.

We next create an instance of the {\tt IpoptProblem} by calling {\tt
CreateIpoptProblem}, giving it the problem dimensions and the variable
and constraint bounds. The arguments {\tt nele\_jac} and {\tt
nele\_hess} are the number of elements in Jacobian and the Hessian,
respectively. See Appendix~\ref{app.triplet} for a description of the
sparse matrix format. The {\tt index\_style} argument specifies whether
we want to use C style indexing for the row and column indices of the
matrices or Fortran style indexing. Here, we set it to {\tt 0} to
indicate C style.  We also include the references to each of our
callback functions. \Ipopt\ uses these function pointers to ask for
evaluation of the NLP when required.

After freeing the bound arrays that are no longer required, the next
two lines illustrate how you can change the value of options through
the interface.  \Ipopt\ options can also be changed by creating a {\tt
PARAMS.DAT} file (see Section~\ref{sec.options}). We next allocate
space for the initial point and set the values as given in the problem
definition.

The call to {\tt IpoptSolve} can provide us with information about the
solution, but most of this is optional. Here, we want values for the
bound multipliers at the solution and we allocate space for these.

We can now make the call to {\tt IpoptSolve} and find the solution of
the problem. We pass in the {\tt IpoptProblem}, the starting point
{\tt x} (\Ipopt\ will use this array to return the solution or final
point as well).  The next 5 arguments are pointers so \Ipopt\ can fill
in values at the solution.  If these pointers are set to {\tt NULL},
\Ipopt\ will ignore that entry.  For example, here, we do not want the
constraint function values at the solution or the constraint
multipliers, so we set those entries to {\tt NULL}. We do want the
value of the objective, and the multipliers for the variable bounds.
The last argument is a {\tt void*} for user data. Any pointer you give
here will also be passed to you in the callback functions.

The return code is an {\tt ApplicationReturnStatus} enumeration, see
the header file {\tt ReturnCodes\_inc.h} which is installed along {\tt
  IpStdCInterface.h} in the \Ipopt\ include directory.

After the optimizer terminates, we check the status and print the
solution if successful. Finally, we free the {\tt IpoptProblem} and
the remaining memory, and return from {\tt main}.

\subsection{The Fortran Interface}

The Fortran interface is essentially a wrapper of the C interface
discussed in Section~\ref{sec.cinterface}.  The way to hook up \Ipopt\
in a Fortran program is very similar to how it is done for the C
interface, and the functions of the Fortran interface correspond
one-to-one to the those of the C and C++ interface, including their
arguments.  You can find an implementation of the example problem
(\ref{eq:ex_obj})--(\ref{eq:ex_bounds}) in {\tt
  \$IPOPTDIR/Examples/hs071\_f}.

The only special things to consider are:
\begin{itemize}
\item The return value of the function {\tt IPCREATE} is of an {\tt
    INTEGER} type that must be large enough to capture a pointer
  on the particular machine.  This means, that you have to declare
  the ``handle'' for the IpoptProblem as {\tt INTEGER*8} if your
  program is compiled in 64-bit mode.  All other {\tt INTEGER}-type
  variables must be of the regular type.
\item For the call of {\tt IPSOLVE} (which is the function that is to
  be called to run \Ipopt), all arrays, including those for the dual
  variables, must be given (in contrast to the C interface).  The
  return value {\tt IERR} of this function indicates the outcome of
  the optimization (see the include file {\tt IpReturnCodes.inc} in
  the \Ipopt\ include directory).
\item The return {\tt IERR} value of the remaining functions has to be
  set to zero, unless there was a problem during execution of the
  function call.
\item The callback functions ({\tt EV\_*} in the example) include the
  arguments {\tt IDAT} and {\tt DAT}, which are {\tt INTEGER} and {\tt
    DOUBLE PRECISION} arrays that are passed unmodified between the
  main program calling {\tt IPSOLVE} and the evaluation subroutines
  {\tt EV\_*} (similarly to {\tt UserDataPtr} arguments in the C
  interface).  These arrays can be used to pass ``private'' data
  between the main program and the user-provided Fortran subroutines.

  The last argument of the {\tt EV\_*} subroutines, {\tt IERR}, is to
  be set to 0 by the user on return, unless there was a problem
  during the evaluation of the optimization problem
  function/derivative for the given point {\tt X} (then it should
  return a non-zero value).
\end{itemize}

\section{Special Features}
\subsection{Derivative Checker}\label{sec:deriv-checker}
\subsection{Quasi-Newton Approximation of Second
  Derivatives}\label{sec:quasiNewton}

\section{\Ipopt\ Options}\label{sec.options}
Ipopt has many (maybe too many) options that can be adjusted for the
algorithm.  Options are all identified by a string name, and their
values can be of one of three types: Number (real), Integer, or
String. Number options are used for things like tolerances, integer
options are used for things like maximum number of iterations, and
string options are used for setting algorithm details, like the NLP
scaling method. Options can be set through code, through the AMPL
interface if you are using AMPL, or by creating a {\tt PARAMS.DAT}
file in the directory you are executing \Ipopt.

The {\tt PARAMS.DAT} file is read line by line and each line should
contain the option name, followed by whitespace, and then the
value. Comments can be included with the {\tt \#} symbol. Don't forget
to ensure you have a newline at the end of the file. For example,
\begin{verbatim}
# This is a comment

# Turn off the NLP scaling
nlp_scaling_method none

# Change the initial barrier parameter
mu_init 1e-2

# Set the max number of iterations
max_iter 500
\end{verbatim}
is a valid {\tt PARAMS.DAT} file.

Options can also be set in code. Have a look at the examples to see
how this is done. 

A subset of \Ipopt\ options are available through AMPL. To set options
through AMPL, use the internal AMPL command {\tt options}.  For
example, \\ 
{\tt options ipopt "nlp\_scaling\_method=none mu\_init=1e-2
max\_iter=500"} \\ 
is a valid options command in AMPL. The most common
options are referenced in Appendix~\ref{app.options_ref}. These are also
the options that are available through AMPL using the {\tt options}
command {\bf TODO: CHECK IF THAT IS CORRECT}. To specify other options when using AMPL, you can always
create {\tt PARAMS.DAT}.  Note, the {\tt PARAMS.DAT} file is given
preference when setting options. This way, you can easily override any
options set in a particular executable or AMPL model by specifying new
values in {\tt PARAMS.DAT}.

For a short list of the valid options, see the Appendix
\ref{app.options_ref}. You can print the documentation for all \Ipopt\
options by adding the option, \\

{\tt print\_options\_documentation yes} \\

and running \Ipopt\ (like the AMPL solver executable, for
instance). This will output all of the options documentation to the
console.

\section{\Ipopt\ Output}\label{sec.output}
This section describes the standard \Ipopt\ console output with the
default setting for {\tt print\_level}. The output is designed to
provide a quick summary of each iteration as \Ipopt\ solves the problem.

Before \Ipopt\ starts to solve the problem, it displays the problem
statistics (number of variables, etc.). Note that if you have fixed
variables (both upper and lower bounds are equal), \Ipopt\ may remove
these variables from the problem internally and not include them in
the problem statistics.

Following the problem statistics, \Ipopt\ will begin to solve the
problem and you will see output resembling the following,
\begin{verbatim}
iter    objective    inf_pr   inf_du lg(mu)  ||d||  lg(rg) alpha_du alpha_pr  ls
   0  1.6109693e+01 1.12e+01 5.28e-01   0.0 0.00e+00    -  0.00e+00 0.00e+00   0
   1  1.8029749e+01 9.90e-01 6.62e+01   0.1 2.05e+00    -  2.14e-01 1.00e+00f  1
   2  1.8719906e+01 1.25e-02 9.04e+00  -2.2 5.94e-02   2.0 8.04e-01 1.00e+00h  1
\end{verbatim}
and the columns of output are defined as,
\begin{description}
\item[{\tt iter}:] The current iteration count. This includes regular
  iterations and iterations while in restoration phase. If the
  algorithm is in the restoration phase, the letter {\tt r'} will be
  appended to the iteration number.
\item[{\tt objective}:] The unscaled objective value at the current
  point. During the restoration phase, this value remains the unscaled
  objective value for the original problem.
\item[{\tt inf\_pr}:] The scaled primal infeasibility at the current
  point. During the restoration phase, this value is the primal
  infeasibility of the original problem at the current point.
\item[{\tt inf\_du}:] The scaled dual infeasibility at the current
  point. During the restoration phase, this is the value of the dual
  infeasibility for the restoration phase problem.
\item[{\tt lg(mu)}:] $\log_{10}$ of the value of the barrier parameter mu.
\item[{\tt ||d||}:] The infinity norm (max) of the primal step (for
  the original variables $x$ and the internal slack variables $s$).
  During the restoration phase, this value includes the values of
  additional variables, $p$ and $n$ (see Eq.~(30) in
  \cite{WaecBieg06:mp}).
\item[{\tt lg(rg)}:] $\log_{10}$ of the value of the regularization
  term for the Hessian of the Lagrangian in the augmented system.
\item[{\tt alpha\_du}:] The stepsize for the dual variables.
\item[{\tt alpha\_pr}:] The stepsize for the primal variables.
\item[{\tt ls}:] The number of backtracking line search steps.
\end{description}

When the algorithm terminates, \Ipopt\ will output a message to the
screen based on the return status of the call to {\tt Optimize}. The following
is a list of the possible return codes, their corresponding output message
to the console, and a brief description.
\begin{description}
\item[{\tt Solve\_Succeeded}:] $\;$ \\
  Console Message: {\tt EXIT: Optimal Solution Found.} \\
  This message indicates that \Ipopt\ found a (locally) optimal point
  within the desired tolerances.
\item[{\tt Solved\_To\_Acceptable\_Level}:]  $\;$ \\
  Console Message: {\tt EXIT: Solved To Acceptable Level.} \\
  This indicates that the algorithm did not converge to the
  ``desired'' tolerances, but that it was able to obtain a point
  satisfying the ``acceptable'' tolerance level as specified by {\tt
    acceptable-*} options. This may happen if the desired tolerances
  are too small for the current problem.
\item[{\tt Infeasible\_Problem\_Detected}:]  $\;$ \\
  Console Message: {\tt EXIT: Converged to a point of
    local infeasibility. Problem may be infeasible.} \\
  The restoration phase converged to a point that is a minimizer for
  the constraint violation (in the $\ell_1$-norm), but is not feasible
  for the original problem. This indicates that the problem may be
  infeasible (or at least that the algorithm is stuck at a locally
  infeasible point).  The returned point (the minimizer of the
  constraint violation) might help you to find which constraint is
  causing the problem.  If you believe that the NLP is feasible,
  it might help to start the optimization from a different point.
\item[{\tt Search\_Direction\_Becomes\_Too\_Small}:]  $\;$ \\
  Console Message: {\tt EXIT: Search Direction is becoming Too Small.} \\
  This indicates that \Ipopt\ is calculating very small step sizes and
  making very little progress.  This could happen if the problem has
  been solved to the best numerical accuracy possible given the
  current scaling.
\item[{\tt Maximum\_Iterations\_Exceeded}:]  $\;$ \\
  Console Message: {\tt EXIT: Maximum Number of Iterations Exceeded.} \\
  This indicates that \Ipopt\ has exceeded the maximum number of
  iterations as specified by the option {\tt max\_iter}.
\item[{\tt Restoration\_Failed}:]  $\;$ \\
  Console Message: {\tt EXIT: Restoration Failed!} \\
  This indicates that the restoration phase failed to find a feasible
  point that was acceptable to the filter line search for the original
  problem. This could happen if the problem is highly degenerate, does
  not satisfy the constraint qualification, or if your NLP code
  provides incorrect derivative information.
\item[{\tt Invalid\_Option}:]  $\;$ \\
  Console Message: (details about the particular error
  will be output to the console) \\
  This indicates that there was some problem specifying the options.
  See the specific message for details.
\item[{\tt Not\_Enough\_Degrees\_Of\_Freedom}:]  $\;$ \\
  Console Message: {\tt EXIT: Problem has too few degrees of freedom.} \\
  This indicates that your problem, as specified, has too few degrees
  of freedom. This can happen if you have too many equality
  constraints, or if you fix too many variables (\Ipopt\ removes fixed
  variables).
\item[{\tt Invalid\_Problem\_Definition}:]  $\;$ \\
  Console Message: (no console message, this is a return code for the
  C and Fortran interfaces only.) \\
  This indicates that there was an exception of some sort when
  building the {\tt IpoptProblem} structure in the C or Fortran
  interface. Likely there is an error in your model or the {\tt main}
  routine.
\item[{\tt Unrecoverable\_Exception}:]  $\;$ \\
  Console Message: (details about the particular error
  will be output to the console) \\
  This indicates that \Ipopt\ has thrown an exception that does not
  have an internal return code. See the specific message for details.
\item[{\tt NonIpopt\_Exception\_Thrown}:]  $\;$ \\
  Console Message: {\tt Unknown Exception caught in Ipopt} \\
  An unknown exception was caught in \Ipopt. This exception could have
  originated from your model or any linked in third party code.
\item[{\tt Insufficient\_Memory}:]  $\;$ \\
  Console Message: {\tt EXIT: Not enough memory.} \\
  An error occurred while trying to allocate memory. The problem may
  be too large for your current memory and swap configuration.
\item[{\tt Internal\_Error}:]  $\;$ \\
  Console Message: {\tt EXIT: INTERNAL ERROR: Unknown SolverReturn
    value - Notify IPOPT Authors.} \\
  An unknown internal error has occurred. Please notify the authors of
  \Ipopt.

\end{description}

\appendix
\newpage
\section{Triplet Format for Sparse Matrices}\label{app.triplet}
\Ipopt\ was designed for optimizing large sparse nonlinear programs.
Because of problem sparsity, the required matrices (like the
constraints Jacobian or Lagrangian Hessian) are not stored as dense
matrices, but rather in a sparse matrix format. For the tutorials in
this document, we use the triplet format.  Consider the matrix
\begin{equation}
\label{eqn.ex_matrix}
\left[
\begin{array}{ccccccc}
1.1     & 0             & 0             & 0             & 0             & 0             & 0.5 \\
0       & 1.9   & 0             & 0             & 0             & 0             & 0.5 \\
0       & 0             & 2.6   & 0             & 0             & 0             & 0.5 \\
0       & 0             & 7.8   & 0.6   & 0             & 0             & 0    \\
0       & 0             & 0             & 1.5   & 2.7   & 0             & 0     \\
1.6     & 0             & 0             & 0             & 0.4   & 0             & 0     \\
0       & 0             & 0             & 0             & 0             & 0.9   & 1.7 \\
\end{array}
\right]
\end{equation}

A standard dense matrix representation would need to store $7 \cdot
7{=} 49$ floating point numbers, where many entries would be zero. In
triplet format, however, only the nonzero entries are stored. The
triplet format records the row number, the column number, and the
value of all nonzero entries in the matrix. For the matrix above, this
means storing $14$ integers for the rows, $14$ integers for the
columns, and $14$ floating point numbers for the values. While this
does not seem like a huge space savings over the $49$ floating point
numbers stored in the dense representation, for larger matrices, the
space savings are very dramatic\footnote{For an $n \times n$ matrix,
the dense representation grows with the the square of $n$, while the
sparse representation grows linearly in the number of nonzeros.}.

The option {\tt index\_style} in {\tt get\_nlp\_info} tells \Ipopt\ if
you prefer to use C style indexing (0-based, i.e., starting the
counting at 0) for the row and column indices or Fortran style
(1-based). Tables \ref{tab.fortran_triplet} and \ref{tab.c_triplet}
below show the triplet format for both indexing styles, using the
example matrix (\ref{eqn.ex_matrix}).

\begin{footnotesize}
\begin{table}[ht]%[!h]
\begin{center}
\begin{tabular}{c c c}
row     		&       col     	&       value 			    \\
\hline
{\tt iRow[0] = 1}       &       {\tt jCol[0] = 1}       & {\tt values[0] = 1.1}     \\
{\tt iRow[1] = 1}       &       {\tt jCol[1] = 7}       & {\tt values[1] = 0.5}     \\
{\tt iRow[2] = 2}       &       {\tt jCol[2] = 2}       & {\tt values[2] = 1.9}     \\
{\tt iRow[3] = 2}       &       {\tt jCol[3] = 7}       & {\tt values[3] = 0.5}     \\
{\tt iRow[4] = 3}       &       {\tt jCol[4] = 3}       & {\tt values[4] = 2.6}     \\
{\tt iRow[5] = 3}       &       {\tt jCol[5] = 7}       & {\tt values[5] = 0.5}     \\
{\tt iRow[6] = 4}       &       {\tt jCol[6] = 3}       & {\tt values[6] = 7.8}     \\
{\tt iRow[7] = 4}       &       {\tt jCol[7] = 4}       & {\tt values[7] = 0.6}     \\
{\tt iRow[8] = 5}       &       {\tt jCol[8] = 4}       & {\tt values[8] = 1.5}     \\
{\tt iRow[9] = 5}       &       {\tt jCol[9] = 5}       & {\tt values[9] = 2.7}     \\
{\tt iRow[10] = 6}      &       {\tt jCol[10] = 1}      & {\tt values[10] = 1.6}     \\
{\tt iRow[11] = 6}      &       {\tt jCol[11] = 5}      & {\tt values[11] = 0.4}     \\
{\tt iRow[12] = 7}      &       {\tt jCol[12] = 6}      & {\tt values[12] = 0.9}     \\
{\tt iRow[13] = 7}      &       {\tt jCol[13] = 7}      & {\tt values[13] = 1.7}
\end{tabular}
\caption{Triplet Format of Matrix (\ref{eqn.ex_matrix}) 
with {\tt index\_style=FORTRAN\_STYLE}}
\label{tab.fortran_triplet}
\end{center}
\end{table}
\begin{table}[ht]%[!h]
\begin{center}
\begin{tabular}{c c c}
row     		&       col     	&       value 			    \\
\hline
{\tt iRow[0] = 0}       &       {\tt jCol[0] = 0}       & {\tt values[0] = 1.1}     \\
{\tt iRow[1] = 0}       &       {\tt jCol[1] = 6}       & {\tt values[1] = 0.5}     \\
{\tt iRow[2] = 1}       &       {\tt jCol[2] = 1}       & {\tt values[2] = 1.9}     \\
{\tt iRow[3] = 1}       &       {\tt jCol[3] = 6}       & {\tt values[3] = 0.5}     \\
{\tt iRow[4] = 2}       &       {\tt jCol[4] = 2}       & {\tt values[4] = 2.6}     \\
{\tt iRow[5] = 2}       &       {\tt jCol[5] = 6}       & {\tt values[5] = 0.5}     \\
{\tt iRow[6] = 3}       &       {\tt jCol[6] = 2}       & {\tt values[6] = 7.8}     \\
{\tt iRow[7] = 3}       &       {\tt jCol[7] = 3}       & {\tt values[7] = 0.6}     \\
{\tt iRow[8] = 4}       &       {\tt jCol[8] = 3}       & {\tt values[8] = 1.5}     \\
{\tt iRow[9] = 4}       &       {\tt jCol[9] = 4}       & {\tt values[9] = 2.7}     \\
{\tt iRow[10] = 5}      &       {\tt jCol[10] = 0}      & {\tt values[10] = 1.6}     \\
{\tt iRow[11] = 5}      &       {\tt jCol[11] = 4}      & {\tt values[11] = 0.4}     \\
{\tt iRow[12] = 6}      &       {\tt jCol[12] = 5}      & {\tt values[12] = 0.9}     \\
{\tt iRow[13] = 6}      &       {\tt jCol[13] = 6}      & {\tt values[13] = 1.7}
\end{tabular}
\caption{Triplet Format of Matrix (\ref{eqn.ex_matrix}) 
with {\tt index\_style=C\_STYLE}}
\label{tab.c_triplet}
\end{center}
\end{table}
\end{footnotesize}
The individual elements of the matrix can be listed in any order, and
if there are multiple items for the same nonzero position, the values
provided for those positions are added.

The Hessian of the Lagrangian is a symmetric matrix. In the case of a
symmetric matrix, you only need to specify the lower left triangual
part (or, alternatively, the upper right triangular part). For
example, given the matrix,
\begin{equation}
\label{eqn.ex_sym_matrix}
\left[
\begin{array}{ccccccc}
1.0	& 0	& 3.0	& 0	& 2.0 	\\
0	& 1.1	& 0	& 0	& 5.0	\\
3.0	& 0	& 1.2	& 6.0	& 0	\\
0	& 0	& 6.0	& 1.3	& 9.0	\\
2.0	& 5.0	& 0	& 9.0	& 1.4
\end{array}
\right]
\end{equation}
the triplet format is shown in Tables \ref{tab.sym_fortran_triplet}
and \ref{tab.sym_c_triplet}.

\begin{footnotesize}
\begin{table}[ht]%[!h]
\begin{center}
\caption{Triplet Format of Matrix (\ref{eqn.ex_matrix}) 
with {\tt index\_style=FORTRAN\_STYLE}}
\label{tab.sym_fortran_triplet}
\begin{tabular}{c c c}
row     		&       col     	&       value 			    \\
\hline
{\tt iRow[0] = 1}       &       {\tt jCol[0] = 1}       & {\tt values[0] = 1.0}     \\
{\tt iRow[1] = 2}       &       {\tt jCol[1] = 1}       & {\tt values[1] = 1.1}     \\
{\tt iRow[2] = 3}       &       {\tt jCol[2] = 1}       & {\tt values[2] = 3.0}     \\
{\tt iRow[3] = 3}       &       {\tt jCol[3] = 3}       & {\tt values[3] = 1.2}     \\
{\tt iRow[4] = 4}       &       {\tt jCol[4] = 3}       & {\tt values[4] = 6.0}     \\
{\tt iRow[5] = 4}       &       {\tt jCol[5] = 4}       & {\tt values[5] = 1.3}     \\
{\tt iRow[6] = 5}       &       {\tt jCol[6] = 1}       & {\tt values[6] = 2.0}     \\
{\tt iRow[7] = 5}       &       {\tt jCol[7] = 2}       & {\tt values[7] = 5.0}     \\
{\tt iRow[8] = 5}       &       {\tt jCol[8] = 4}       & {\tt values[8] = 9.0}     \\
{\tt iRow[9] = 5}       &       {\tt jCol[9] = 5}       & {\tt values[9] = 1.4}
\end{tabular}
\end{center}
\end{table}
\begin{table}[ht]%[!h]
\begin{center}
\caption{Triplet Format of Matrix (\ref{eqn.ex_matrix}) 
with {\tt index\_style=C\_STYLE}}
\label{tab.sym_c_triplet}
\begin{tabular}{c c c}
row     		&       col     	&       value 			    \\
\hline
{\tt iRow[0] = 0}       &       {\tt jCol[0] = 0}       & {\tt values[0] = 1.0}     \\
{\tt iRow[1] = 1}       &       {\tt jCol[1] = 0}       & {\tt values[1] = 1.1}     \\
{\tt iRow[2] = 2}       &       {\tt jCol[2] = 0}       & {\tt values[2] = 3.0}     \\
{\tt iRow[3] = 2}       &       {\tt jCol[3] = 2}       & {\tt values[3] = 1.2}     \\
{\tt iRow[4] = 3}       &       {\tt jCol[4] = 2}       & {\tt values[4] = 6.0}     \\
{\tt iRow[5] = 3}       &       {\tt jCol[5] = 3}       & {\tt values[5] = 1.3}     \\
{\tt iRow[6] = 4}       &       {\tt jCol[6] = 0}       & {\tt values[6] = 2.0}     \\
{\tt iRow[7] = 4}       &       {\tt jCol[7] = 1}       & {\tt values[7] = 5.0}     \\
{\tt iRow[8] = 4}       &       {\tt jCol[8] = 3}       & {\tt values[8] = 9.0}     \\
{\tt iRow[9] = 4}       &       {\tt jCol[9] = 4}       & {\tt values[9] = 1.4}
\end{tabular}
\end{center}
\end{table}
\end{footnotesize}
\newpage
\section{The Smart Pointer Implementation: {\tt SmartPtr<T>}} \label{app.smart_ptr}

The {\tt SmartPtr} class is described in {\tt IpSmartPtr.hpp}. It is a
template class that takes care of deleting objects for us so we need
not be concerned about memory leaks. Instead of pointing to an object
with a raw C++ pointer (e.g. {\tt HS071\_NLP*}), we use a {\tt
  SmartPtr}.  Every time a {\tt SmartPtr} is set to reference an
object, it increments a counter in that object (see the {\tt
  ReferencedObject} base class if you are interested). If a {\tt
  SmartPtr} is done with the object, either by leaving scope or being
set to point to another object, the counter is decremented. When the
count of the object goes to zero, the object is automatically deleted.
{\tt SmartPtr}'s are very simple, just use them as you would a
standard pointer.

It is very important to use {\tt SmartPtr}'s instead of raw pointers
when passing objects to \Ipopt. Internally, \Ipopt\ uses smart
pointers for referencing objects. If you use a raw pointer in your
executable, the object's counter will NOT get incremented. Then, when
\Ipopt\ uses smart pointers inside its own code, the counter will get
incremented. However, before \Ipopt\ returns control to your code, it
will decrement as many times as it incremented earlier, and the
counter will return to zero. Therefore, \Ipopt\ will delete the
object. When control returns to you, you now have a raw pointer that
points to a deleted object.

This might sound difficult to anyone not familiar with the use of
smart pointers, but just follow one simple rule; always use a SmartPtr
when creating or passing an \Ipopt\ object.

\newpage
\section{Options Reference} \label{app.options_ref}
Options can be set using {\tt PARAMS.DAT}, through your own code, or through the 
AMPL {\tt options} command. See Section \ref{sec.options} for an explanation of
how to use these commands.
Shown here is a short list of the most common options for Ipopt. To view
the full list of options, run the ipopt executable with the option,
\begin{verbatim}
print_options_documentation yes
\end{verbatim}

The most common options are:

\input{options.tex}

\newpage
\section{Detailed Installation Information}\label{ExpertInstall}

The configuration script and Makefiles in the \Ipopt\ distribution
have been created using GNU's {\tt autoconf} and {\tt automake}.  They
attempt to automatically adapt the compiler settings etc.\ to the
system they are running on.  We tested the provided scripts for a
number of different machines, operating systems and compilers, but you
might run into a situation where the default setting does not work, or
where you need to change the settings to fit your particular
environment.

In general, you can see the list of options and variables that can be
set for the {\tt configure} script by typing \verb/configure --help/.
Below a few particular options are discussed:

\begin{itemize}
\item The {\tt configure} script tries to determine automatically, if
  you have BLAS and/or LAPACK already installed on your system (trying
  a few default libraries), and if it does not find them, it makes
  sure that you put the source code in the required place.

  However, you can specify a BLAS library (such as your local ATLAS
  library\footnote{see {\tt http://math-atlas.sourceforge.net/}})
  explicitly, using the \verb/--with-blas/ flag for {\tt configure}.
  For example,

  \verb|./configure --with=blas="-L$HOME/lib -latlas"|

  To tell the configure script to compile and use the downloaded BLAS
  source files even if a BLAS library is found on your system, specify
  \verb|--with-blas=BUILD|.

  Similarly, you can use the \verb/--with-lapack/ switch to specify
  the location of your LAPACK library, or use the keyword {\tt BUILD}
  to force the \Ipopt\ makefiles to compile LAPACK together with
  \Ipopt.

\item Similarly, if you have a precompiled library containing the
  Harwell Subroutines, you can specify its location with the
  \verb|--with-hsl| flag.  And the location of the AMPL solver library
  (with the ASL header files) can be specified with
  \verb|--with-asldir|.
  {\bf TODO Other linear solvers}

\item If you want to specify that you want to use particular
  compilers, you can do so by adding the variables definitions for
  {\tt CXX}, {\tt CC}, and {\tt F77} to the {\tt ./configure} command
  line, to specify the C++, C, and Fortran compiler, respectively.
  For example,

  {\tt ./configure CXX=g++ CC=gcc F77=g77}

  In order to set the compiler flags, you should use the variables
  {\tt CXXFLAGS}, {\tt CFLAGS}, {\tt FFLAGS}.  Note, that the \Ipopt\
  code uses ``{\tt dynamic\_cast}''.  Therefore it is necessary that
  the C++ code is compiled including RTTI (Run-Time Type Information).
  Some compilers need to be given special flags to do that (e.g.,
  ``{\tt -qrtti=dyna}'' for the AIX {\tt xlC} compiler).

\item If you want to link the \Ipopt\ library with a main program
  written in C or Fortran, the C and Fortran compiler doing the
  linking of the executable needs to be told about the C++ runtime
  libraries.  Unfortunately, the current version of {\tt autoconf}
  does not provide the automatic detection of those libraries.  We
  have hard-coded some default values for some systems and compilers,
  but this might not work all the time.

  If you have problems linking your Fortran or C code with the \Ipopt\
  library {\tt libipopt.a} and the linker complains about missing
  symbols from C++ (e.g., the standard template library), you should
  specify the C++ libraries with the {\tt CXXLIBS} variable.  To find out
  what those libraries are, it is probably helpful to link a  simple C++
  program with verbose compiler output.

  For example, for the Intel compilers on a Linux system, you
  might need to specify something like

  {\tt ./configure CC=icc F77=ifort CXX=icpc $\backslash$\\ \hspace*{14ex} CXXLIBS='-L/usr/lib/gcc-lib/i386-redhat-linux/3.2.3 -lstdc++'}

\item Compilation in 64bit mode sometimes requires some special
  consideration.  For example, for compilation of 64bit code on AIX,
  we recommend the following configuration

  {\tt ./configure AR='ar -X64' AR\_X='ar -X64 x' $\backslash$\\
    \hspace*{14ex} CC='xlc -q64' F77='xlf -q64' CXX='xlC
    -q64'$\backslash$\\ \hspace*{14ex} CFLAGS='-O3
    -bmaxdata:0x3f0000000'
    $\backslash$\\ \hspace*{14ex} FFLAGS='-O3 -bmaxdata:0x3f0000000' $\backslash$\\
    \hspace*{14ex} CXXFLAGS='-qrtti=dyna -O3 -bmaxdata:0x3f0000000'}

\item To build library/archive files (with the ending {\tt .a})
  including C++ code in some environments, it is necessary to use the
  C++ compiler instead of {\tt ar} to build the archive.  This is for
  example the case for some older compilers on SGI and SUN.  For this,
  the {\tt configure} variables {\tt AR}, {\tt ARFLAGS}, and {\tt
    AR\_X} are provided.  Here, {\tt AR} specifies the command for the
  archiver for creating an archive, and {\tt ARFLAGS} specifies
  additional flags.  {\tt AR\_X} contains the command for extracting
  all files from an archive.  For example, the default setting for SUN
  compilers for our configure script is

  {\tt AR='CC -xar' ARFLAGS='-o' AR\_X='ar x'}

\item It is possible to compile the \Ipopt\ library in a debug
  configuration, by specifying \verb|--enable-debug|.  Then the
  compilers will use the debug flags (unless the compilation flag
  variables are overwritten in the {\tt configure} command line), and
  additional debug checks are compiled into the code (see {\tt
    IpDebug.hpp}).  This usually leads to a significant slowdown of
  the code, but might be helpful when debugging something.

\item It is not necessary to produce the binary files in the
  directories where the source files are.  If you want to compile the
  code on different systems or with different compilers/options on a
  shared file system, you can keep one single copy of the source files
  in one directory, and the binary files for each configuration in
  separate directories.  For this, simply run the configure script in
  the directory where you want the base directory for the \Ipopt\
  binary files.  For example:

  {\tt \$ mkdir \$HOME/Ipopt-objects}\\
  {\tt \$ cd \$HOME/Ipopt-objects}\\
  {\tt \$ \$HOME/Ipopt/configure}  (or {\tt \$HOME/ipopt-3.1.0/configure})

\end{itemize}

%\bibliographystyle{plain}
%\bibliography{/home/andreasw/tex/andreas}
\input{documentation.bbl}

\end{document}


\end{document}


\end{document}


\end{document}
