\paragraph{print\_level:} Output verbosity level. $\;$ \\
 Sets the default verbosity level for console
output. The larger this value the more detailed
is the output. The valid range for this integer option is
$0 \le {\tt print\_level } \le 10$
and its default value is $3$.


\paragraph{pivtol:} Pivot tolerance for the linear solver MA27. $\;$ \\
 A smaller number pivots for sparsity, a larger
number pivots for stability. The valid range for this real option is 
$0 <  {\tt pivtol } <  1$
and its default value is $1 \cdot 10^{-08}$.


\paragraph{pivtolmax:} Maximum pivot tolerance. $\;$ \\
 Ipopt may increase pivtol as high as pivtolmax to
get a more accurate solution to the linear system. The valid range for this real option is 
$0 <  {\tt pivtolmax } <  1$
and its default value is $0.0001$.


\paragraph{tol:} Desired convergence tolerance (relative). $\;$ \\
 Determines the convergence tolerance for the
algorithm.  The algorithm terminates
successfully, if the (scaled) NLP error becomes
smaller than this value, and if the (absolute)
criteria according to "dual\_inf\_tol",
"primal\_inf\_tol", and "cmpl\_inf\_tol" are met.
 (This is epsilon\_tol in Eqn. (6) in
implementation paper).  See also
"acceptable\_tol" as a second termination
criterion.  Note, some other algorithmic features
also use this quantity. The valid range for this real option is 
$0 <  {\tt tol } <  {\tt +inf}$
and its default value is $1 \cdot 10^{-08}$.


\paragraph{compl\_inf\_tol:} Desired threshold for the complementarity conditions. $\;$ \\
 Absolute tolerance on the complementarity.
Successful termination requires that the
(unscaled) complementarity is less than this
threshold. The valid range for this real option is 
$0 <  {\tt compl\_inf\_tol } <  {\tt +inf}$
and its default value is $0.0001$.


\paragraph{dual\_inf\_tol:} Desired threshold for the dual infeasibility. $\;$ \\
 Absolute tolerance on the dual infesaibility.
Successful termination requires that the
(unscaled) dual infeasibility is less than this
threshold. The valid range for this real option is 
$0 <  {\tt dual\_inf\_tol } <  {\tt +inf}$
and its default value is $0.0001$.


\paragraph{constr\_mult\_init\_max:} Maximum allowed least-square guess of constraint multipliers. $\;$ \\
 Determines how large the initial least-square
guesses of the contraint multipliers are allowed
to be (in max-norm). If the guess is larger than
this value, it is discarded and all constraint
multipliers are set to zero.  This options is
also used when initializing the restoration
phase. By default,
"resto.constr\_mult\_init\_max" (the one used in
RestoIterateInitializer) is set to zero. The valid range for this real option is 
$0 \le {\tt constr\_mult\_init\_max } <  {\tt +inf}$
and its default value is $1000$.


\paragraph{constr\_viol\_tol:} Desired threshold for the constraint violation. $\;$ \\
 Absolute tolerance on the constraint violation.
Successful termination requires that the
(unscaled) constraint violation is less than this
threshold. The valid range for this real option is 
$0 <  {\tt constr\_viol\_tol } <  {\tt +inf}$
and its default value is $0.0001$.


\paragraph{mu\_strategy:} Update strategy for barrier parameter. $\;$ \\
 Determines which barrier parameter update
strategy is to be used.
The default value for this string option is "monotone".
\\ 
Possible values:
\begin{itemize}
   \item monotone: use the monotone (Fiacco-McCormick) strategy
   \item adaptive: use the adaptive update strategy
\end{itemize}

\paragraph{mu\_init:} Initial value for the barrier parameter. $\;$ \\
 This option determines the initial value for the
barrier parameter (mu).  It is only relevant in
the monotone, Fiacco-McCormick version of the
algorithm. (i.e., if "mu\_strategy" is chosen as
"monotone") The valid range for this real option is 
$0 <  {\tt mu\_init } <  {\tt +inf}$
and its default value is $0.1$.


\paragraph{mu\_oracle:} Oracle for a new barrier parameter in the adaptive strategy. $\;$ \\
 Determines how a new barrier parameter is
computed in each "free-mode" iteration of the
adaptive barrier parameter strategy. (Only
considered if "adaptive" is selected for option
"mu\_strategy").
The default value for this string option is "probing".
\\ 
Possible values:
\begin{itemize}
   \item probing: Mehrotra's probing heuristic
   \item loqo: LOQO's centrality rule
   \item quality\_function: minimize a quality function
\end{itemize}

\paragraph{corrector\_type:} The type of corrector steps that should be taken. $\;$ \\
 If "mu\_strategy" is "adaptive", this option
determines what kind of corrector steps should be
tried.
The default value for this string option is "none".
\\ 
Possible values:
\begin{itemize}
   \item none: no corrector
   \item affine: corrector step towards mu=0
   \item primal-dual: corrector step towards current mu
\end{itemize}

\paragraph{obj\_scaling\_factor:} Scaling factor for the objective function. $\;$ \\
 This option sets a scaling factor for the
objective function. The scaling is seen
internally by Ipopt but the unscaled objective is
reported in the console output. If additional
scaling parameters are computed (e.g.
user-scaling or gradient-based), both factors are
multiplied. If this value is chosen to be
negative, Ipopt will maximize the objective
function instead of minimizing it. The valid range for this real option is 
${\tt -inf} <  {\tt obj\_scaling\_factor } <  {\tt +inf}$
and its default value is $1$.


\paragraph{nlp\_scaling\_method:} Select the technique used for scaling the NLP $\;$ \\
 Selects the technique used for scaling the
problem before it is solved. For user-scaling,
the parameters come from the NLP. If you are
using AMPL, they can be specified through
suffixes (scaling\_factor)
The default value for this string option is "gradient\_based".
\\ 
Possible values:
\begin{itemize}
   \item none: no problem scaling will be performed
   \item user\_scaling: scaling parameters will come from the user
   \item gradient\_based: scale the problem so the maximum gradient at
the starting point is scaling\_max\_gradient
\end{itemize}

\paragraph{nlp\_scaling\_max\_gradient:} Maximum gradient after NLP scaling. $\;$ \\
 This is the gradient scaling cut-off. If the
maximum gradient is above this value, then
gradient based scaling will be performed. Scaling
parameters are calculated to scale the maximum
gradient back to this value. (This is g\_max in
Section 3.8 of the implementation paper.) Note:
This option is only used if
"nlp\_scaling\_method" is chosen as
"gradient\_based". The valid range for this real option is 
$0 <  {\tt nlp\_scaling\_max\_gradient } <  {\tt +inf}$
and its default value is $100$.


\paragraph{bound\_frac:} Desired minimum relative distance from the initial point to bound. $\;$ \\
 Determines how much the initial point might have
to be modified in order to be sufficiently inside
the bounds (together with "bound\_push").  (This
is kappa\_2 in Section 3.6 of implementation
paper.) The valid range for this real option is 
$0 <  {\tt bound\_frac } \le 0.5$
and its default value is $0.01$.


\paragraph{bound\_mult\_init\_val:} Initial value for the bound multipliers. $\;$ \\
 All dual variables corresponding to bound
constraints are initialized to this value. The valid range for this real option is 
$0 <  {\tt bound\_mult\_init\_val } <  {\tt +inf}$
and its default value is $1$.


\paragraph{bound\_push:} Desired minimum absolute distance from the initial point to bound. $\;$ \\
 Determines how much the initial point might have
to be modified in order to be sufficiently inside
the bounds (together with "bound\_frac").  (This
is kappa\_1 in Section 3.6 of implementation
paper.) The valid range for this real option is 
$0 <  {\tt bound\_push } <  {\tt +inf}$
and its default value is $0.01$.


\paragraph{bound\_relax\_factor:} Factor for initial relaxation of the bounds. $\;$ \\
 Before start of the optimization, the bounds
given by the user are relaxed.  This option sets
the factor for this relaxation.  If it is set to
zero, then then bounds relaxation is disabled.
(See Eqn.(35) in implmentation paper.) The valid range for this real option is 
$0 \le {\tt bound\_relax\_factor } <  {\tt +inf}$
and its default value is $1 \cdot 10^{-08}$.


\paragraph{acceptable\_compl\_inf\_tol:} Acceptance threshold for the complementarity conditions. $\;$ \\
 Absolute tolerance on the complementarity.
Acceptable termination requires that the
(unscaled) complementarity is less than this
threshold; see also acceptable\_tol. The valid range for this real option is 
$0 <  {\tt acceptable\_compl\_inf\_tol } <  {\tt +inf}$
and its default value is $0.01$.


\paragraph{acceptable\_constr\_viol\_tol:} Acceptance threshold for the constraint violation. $\;$ \\
 Absolute tolerance on the constraint violation.
Acceptable termination requires that the
(unscaled) constraint violation is less than this
threshold; see also acceptable\_tol. The valid range for this real option is 
$0 <  {\tt acceptable\_constr\_viol\_tol } <  {\tt +inf}$
and its default value is $0.01$.


\paragraph{acceptable\_dual\_inf\_tol:} Acceptance threshold for the dual infeasibility. $\;$ \\
 Absolute tolerance on the dual infesaibility.
Acceptable termination requires that the
(unscaled) dual infeasibility is less than this
threshold; see also acceptable\_tol. The valid range for this real option is 
$0 <  {\tt acceptable\_dual\_inf\_tol } <  {\tt +inf}$
and its default value is $0.01$.


\paragraph{acceptable\_tol:} Acceptable convergence tolerance (relative). $\;$ \\
 Determines which (scaled) overall optimality
error is considered to be "acceptable." There are
two levels of termination criteria.  If the usual
"desired" tolerances (see tol, dual\_inf\_tol
etc) are satisfied at an iteration, the algorithm
immediately terminates with a success message. 
On the other hand, if the algorithm encounters
"acceptable\_iter" many iterations in a row that
are considered "acceptable", it will terminate
before the desired convergence tolerance is met.
This is useful in cases where the algorithm might
not be able to achieve the "desired" level of
accuracy. The valid range for this real option is 
$0 <  {\tt acceptable\_tol } <  {\tt +inf}$
and its default value is $1 \cdot 10^{-06}$.


\paragraph{alpha\_for\_y:} Method to determine the step size for constraint multipliers. $\;$ \\
 This option determines how the step size
(alpha\_y) will be calculated when updating the
constraint multipliers.
The default value for this string option is "primal".
\\ 
Possible values:
\begin{itemize}
   \item primal: use primal step size
   \item bound\_mult: use step size for the bound multipliers
   \item min: use the min of primal and bound multipliers
   \item max: use the max of primal and bound multipliers
   \item full: take a full step of size one
   \item min\_dual\_infeas: choose step size minimizing new dual
infeasibility
   \item safe\_min\_dual\_infeas: like "min\_dual\_infeas", but safeguarded by
"min" and "max"
\end{itemize}

\paragraph{expect\_infeasible\_problem:} Enable heuristics to quickly detect an infeasible problem. $\;$ \\
 This options is meant to activate heuristics that
may speed up the infeasibility determination if
you expect that there is a good chance for the
problem to be infeasible.  In the filter line
search procedure, the restoration phase is called
more qucikly than usually, and more reduction in
the constraint violation is enforced. If the
problem is square, this option is enabled
automatically.
The default value for this string option is "no".
\\ 
Possible values:
\begin{itemize}
   \item no: the problem probably be feasible
   \item yes: the problem has a good chance to be infeasible
\end{itemize}

\paragraph{max\_iter:} Maximum number of iterations. $\;$ \\
 The algorithm terminates with an error message if
the number of iterations exceeded this number.
This option is also used in the restoration phase. The valid range for this integer option is
$0 \le {\tt max\_iter } <  {\tt +inf}$
and its default value is $3000$.


\paragraph{max\_refinement\_steps:} Maximum number of iterative refinement steps per linear system solve. $\;$ \\
 Iterative refinement (on the full unsymmetric
system) is performed for each right hand side. 
This option determines the maximum number of
iterative refinement steps. The valid range for this integer option is
$0 \le {\tt max\_refinement\_steps } <  {\tt +inf}$
and its default value is $10$.


\paragraph{max\_soc:} Maximum number of second order correction trial steps at each iteration. $\;$ \\
 Choosing 0 disables the second order corrections.
(This is p\^{max} of Step A-5.9 of Algorithm A in
implementation paper.) The valid range for this integer option is
$0 \le {\tt max\_soc } <  {\tt +inf}$
and its default value is $4$.


\paragraph{min\_refinement\_steps:} Minimum number of iterative refinement steps per linear system solve. $\;$ \\
 Iterative refinement (on the full unsymmetric
system) is performed for each right hand side. 
This option determines the minimum number of
iterative refinements (i.e. at least
"min\_refinement\_steps" iterative refinement
steps are enforced per right hand side.) The valid range for this integer option is
$0 \le {\tt min\_refinement\_steps } <  {\tt +inf}$
and its default value is $1$.


\paragraph{output\_file:} File name of desired output file (leave unset for no file output). $\;$ \\
 NOTE: This option only works when read from the
PARAMS.DAT options file! An output file with this
name will be written (leave unset for no file
output).  The verbosity level is by default set
to "print\_level", or but can be overridden with
"file\_print\_level".
The default value for this string option is "".
\\ 
Possible values:
\begin{itemize}
   \item *: Any acceptable standard file name
\end{itemize}

\paragraph{file\_print\_level:} Verbosity level for output file. $\;$ \\
 NOTE: This option only works when read from the
PARAMS.DAT options file! Determines the verbosity
level for the file specified by "output\_file". 
By default it is the same as "print\_level". The valid range for this integer option is
$0 \le {\tt file\_print\_level } \le 10$
and its default value is $3$.


